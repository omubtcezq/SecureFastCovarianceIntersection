\documentclass[a4paper]{scrartcl}

%\usepackage{showframe}
\usepackage[margin=2cm,footskip=.7cm]{geometry}
\usepackage{enumitem}
% \usepackage{fourier}
\usepackage{xcolor}
% \usepackage{abkuerzungen}
\usepackage{hyperref}
\usepackage{amsmath}
\usepackage{ISASmacros/isasmathmacros}

\usepackage{pdfpages}


\newcommand{\sA}{\ensuremath{\mathsf{A}}}
\newcommand{\sAB}{\ensuremath{\mathsf{AB}}}
\newcommand{\sB}{\ensuremath{\mathsf{B}}}
\newcommand{\sBA}{\ensuremath{\mathsf{BA}}}
\newcommand{\sC}{\ensuremath{\mathsf{C}}}

\newcommand{\cest}{\ensuremath{\cvec{\gamma}}}
\newcommand{\cerest}{\ensuremath{\ervec{\gamma}}}
\newcommand{\cmat}{\ensuremath{\mat{\Gamma}}}
\newcommand{\tmat}{\ensuremath{\widetilde{\mat{\Gamma}}}}

\newcommand{\gainA}{\ensuremath{\mat{K}}}
\newcommand{\gainB}{\ensuremath{\mat{L}}}

% \newcommand{\fus}{\ensuremath{\op{fus}}}

% \newcommand{\CI}{\op{CI}\xspace}
% \newcommand{\EI}{\op{EI}\xspace}
% \newcommand{\ICI}{\op{ICI}\xspace}
% \newcommand{\BC}{\op{B\!/\!C}\xspace}
% \newcommand{\ind}{\op{s}\xspace}
\newcommand{\ind}{\op{in}\xspace}
\newcommand{\opt}{\cmat}


\newcommand{\excmat}{\ensuremath{\mat{\Gamma}'}}
\newcommand{\excest}{\ensuremath{\cest'}}
% \newcommand{\exoptmat}{\ensuremath{\mat{C}_{\excmat}}}
\newcommand{\exoptmat}{\ensuremath{\mat{C}'_\EI}}

%\RequirePackage[mathscr]{euscript}
%\RequirePackage{bbding}
%\RequirePackage{scalefnt}
%\RequirePackage{mathtools}
% This was enabled but makes the text look ugly
%\RequirePackage[T1]{fontenc}


%%% COLOR DEFINITIONS

% KIT Colors
\definecolor{kitgreenex}{RGB}{0,152,131}
\definecolor{kitblueex}{RGB}{52,115,186}
\definecolor{kitmaygreen}{RGB}{119,184,38}
\definecolor{kityellow}{RGB}{255,228,0}
\definecolor{kitorange}{RGB}{247,154,0}
\definecolor{kitbrown}{RGB}{182,130,28}
\definecolor{kitred}{RGB}{187,25,23}
\definecolor{kitpurple}{RGB}{190,0,126}
\definecolor{kitcyanblue}{RGB}{0,167,227}
% Own Definitions
\definecolor{grey}{RGB}{150,150,150}


\definecolor{nblue}{RGB}{54,95,145}

%%% FONTS
%\setkomafont{pageheadfoot}{\small\color{darkgray}}
%\setkomafont{pagefoot}{\normalfont\color{darkgray}}
%\setkomafont{pagenumber}{\color{darkgray}}
%\setkomafont{captionlabel}{\small\bfseries\color{darkgray}}
\setkomafont{disposition}{\bfseries}
\setkomafont{section}{\normalfont\large\bfseries}
\setkomafont{subsection}{\normalfont\bfseries}
\setkomafont{author}{\normalfont}
\setkomafont{date}{\normalfont}


%%% PARAGRAPH LAYOUT
\setlength{\parindent}{0mm}
\setlength{\parskip}{6pt}


%%% REBUTTAL COMMANDS
\newenvironment{rebuttal}{\begin{enumerate}[label={\color{grey}\thesection.\arabic{enumi}},leftmargin=0pt,ref=\thesection.\arabic{enumi}]}{\end{enumerate}}
\newcommand{\reviewtext}[1]{{\color{nblue} #1}}
\newcommand{\papertext}[1]{\emph{``#1''}}

%%% HYPERREF SETUP
\hypersetup{
        colorlinks = true,
        linkcolor = kitgreenex
}

%%%%%%%%%%%%%%%%%%%%%%%%%%%%%%%%%%%%%%%%%%%%%%%%%%%%%%%%%%%%%%%%%%%%%%%%

\title{\boldmath Secure Fast Covariance Intersection Using Partially Homomorphic and Order Revealing Encryption Schemes}
\subtitle{Response to Reviewers' Comments - Submission L-CSS 20-0282}
\author{Marko Ristic\and Benjamin Noack\and Uwe D. Hanebeck}

%       .d8888b.  888                     888
%      d88P  Y88b 888                     888
%      Y88b.      888                     888
%       "Y888b.   888888  8888b.  888d888 888888
%          "Y88b. 888        "88b 888P"   888
%            "888 888    .d888888 888     888
%      Y88b  d88P Y88b.  888  888 888     Y88b.
%       "Y8888P"   "Y888 "Y888888 888      "Y888



\begin{document}

\maketitle

Dear Dr. Giovanni Cherubini,\\
Dear Reviewers,

We would like to thank you all for your thorough and encouraging reviews. In this letter, we explain how the reviewer comments, questions, and suggestions have been addressed. Throughout this response, reviewers' comments are typed in \reviewtext{blue}. 

Sincerely,\\
Marko Ristic, Benjamin Noack, and Uwe D. Hanebeck

%      8888888888     888 d8b 888
%      888            888 Y8P 888
%      888            888     888
%      8888888    .d88888 888 888888 .d88b.  888d888
%      888       d88" 888 888 888   d88""88b 888P"
%      888       888  888 888 888   888  888 888
%      888       Y88b 888 888 Y88b. Y88..88P 888
%      8888888888 "Y88888 888  "Y888 "Y88P"  888



\section*{Response to the Editor's Report}
\def\thesection{E}
\begin{rebuttal} %\setcounter{enumi}{-1}
\item \reviewtext{Three reports have been collected for this submission. The reviewers
agree that the topic is interesting. However, they also raise a number
of concern that should be carefully addressed in a revised version. 

I also have some specific comments:

1. The problem motivation given in the Introduction is a bit vague.
Specifically, while it is known that encryption plays an important role
in certain network (control) systems, the need for securing
confidentiality of data in sensor networks is less evident. The authors
could discuss some concrete applications related to sensor networks
where data confidentiality is important.}

We are sorry that the role of encryption in sensor networks was not made clearer in our introduction, and agree with the reviewer that a stronger motivation for our work was lacking. We have changed the introduction to include a discussion on when encryption may be desired in sensor networks specifically, and have added Section I.A describing the scenario motivating our work.

\item \reviewtext{2. On a similar vein, the paper lacks a clear problem statement
defining what is the sensitive information (and why). This has been
pointed out also by Reviewer 1.}

We agree that a problem statement was missing, and have in Section I.A given the scenario motivating our work, as well as a stronger definition of our problem. Additionally, we specify which information is treated as sensitive and which collusions we consider.

\item \reviewtext{3. It is known that encryption introduces delays, thus affecting
performance. The authors could add a discussion on the computation
complexity of their method and its trade-off with accuracy. All the
reviewers have similar comments (Reviewer 3 is especially critical on
this point).}

You're right; the manuscript was lacking any run times or complexity analysis of the fusion algorithm, as well as any comparison between methods. We have added the complexity run times of the new algorithm and contrasted them with the unencrypted FCI algorithm, as well as noting the comparative differences to homomorphic operation complexities of some current FHE schemes. These changes are included in the new Section V.A. Additionally, in Section VI we have added a discussion and derivation of the accuracy of the method in terms of algorithms parameters.

\item \reviewtext{4. The authors could summarize the main result in a theorem. This would
help to highlight the method and its properties. As it stands, the
analysis of Section V is rather fuzzy. This has been pointed out also
by Reviewer 2.}

We regret that Section V and our results were not made clearer, and have added Section V.A and updated Section VI to include an explicit complexity analysis of the novel method, and give an upper bound on the accuracy reduction of SecFCI from the FCI algorithm it approximates. We have not stated any explicit theorems as their proofs are beyond the scope of this short manuscript, but have aimed to summarize our results with the added security comments, complexity analysis, and accuracy bound in Sections V and VI.

\end{rebuttal}

%      8888888b.                         d888
%      888   Y88b                       d8888
%      888    888                         888
%      888   d88P .d88b.  888  888        888
%      8888888P" d8P  Y8b 888  888        888
%      888 T88b  88888888 Y88  88P        888
%      888  T88b Y8b.      Y8bd8P         888
%      888   T88b "Y8888    Y88P        8888888



\section*{Response to the Comments of Reviewer 1 (23899)}
\def\thesection{R1}
\begin{rebuttal}
\item \reviewtext{- Equations (9), (11) and (12) are incorrect. Paillier is a
probabilistic cryptosystem which means when you encrypt the same
message two times it will yield different ciphertexts due to the usage
of randomness. Thus, you have to equate the decryption of both sides of
all equations. }

Thank you for pointing this out. We have corrected the manuscript accordingly.

\item \reviewtext{- The paper lacks a clear problem statement and threat model. What are
the sensitive information estimates, measurement or both?. A portion of
the threat model is mentioned very late with the proposed solution in
section IV. The author should mention that collusion between any sensor
and the aggregator is not considered as the sensor holds the ORE key. }

We thank the reviewer for their comment, and agree that a problem statement was lacking in detail, and missing earlier in the manuscript. We have added Section I.A that states our problem explicitly, as well as the information we treat sensitive and the collusions we consider.

\item \reviewtext{- According to (18) and (19), the weights would be public information.
A comment should be made on this privacy leakage.}

We have now stated that weights are leaked in the problem statement, and when explaining SecFCI. Thank you for pointing this out.

\item \reviewtext{- Section IV: The author mentioned, "Since analytical solutions (6)
require division, they cannot be computed exactly with the given PHE
encryptions of sensor information vectors and information matrices."I
disagree with this statement. The Author used an inappropriate encoding
mechanism. Check the one used in [1]. The author should mention why
they avoid using such an encoding mechanism to solve the problem
without going into the proposed direction. }

We are sorry for having overlooked the existing literature, and appreciate your comment and reference. We agree with your statement and have updated Section III.C and our references to motivate and justify our choice of encoding.

\item \reviewtext{- I expected to see complexity analysis for n sensor case or show the
required number of comparisons and the execution time of each one. }

We apologies for having not included any run times or a complexity analysis in the manuscript. We have added the new Section V.A which gives both homomorphic operation complexities and the complexity of the complete fusion algorithm. We have also now noted the number of required comparisons in Section V.

\item \reviewtext{- Section I.A: The author mentioned 
"Epk(a) and EORE;k(a) denote the public-key pk and ORE key k
encryptions of a"I suggest to change it to 
Epk(a) and EORE;k(a) denote the encryption of a using the public-key pk
and ORE key k, respectively

---
[1] "CryptoImg: Privacy-preserving processing over encrypted images" M
Tarek Ibn Ziad, Amr Alanwar, Moustafa Alzantot, Mani Srivastava}

Thank you for the suggestion, we have made the change accordingly.

\end{rebuttal}

%      8888888b.                         .d8888b.
%      888   Y88b                       d88P  Y88b
%      888    888                              888
%      888   d88P .d88b.  888  888           .d88P
%      8888888P" d8P  Y8b 888  888       .od888P"
%      888 T88b  88888888 Y88  88P      d88P"
%      888  T88b Y8b.      Y8bd8P       888"
%      888   T88b "Y8888    Y88P        888888888



\section*{Response to the Comments of Reviewer 2 (23901)}
\def\thesection{R2}
\begin{rebuttal}
\item \reviewtext{This paper applis the combination of two Encryption strategies, named
the Partially Homomorphic Encryption (PHE) 
and Order Revealing Encryption (ORE), to the Fast Covariance
Intersection (FCI) so as to perform secure state estimation. In the
proposed method, the algebraic operations of PHE and ORE are utilized
to find the fusion weight for FCI. Though the topic is for surely
interesting, I think the authors should address the following problems
before the subject paper being considered to be published:

1, Though the topic is certainly interesting, the main idea of this
paper seems to simply combine two existing strategies together. I think
such idea should be motivated more, e.g., why choosing the mixed PHE
and ORE strategies? What I can find is that it can reduce computational
load, but I think this is not enough. Maybe you can say something
concerning the accuracy preservation? }

We thank the reviewer for their comment, and regret that our reasoning was not made clearer. We have updated the introduction to more accurately motivate our work. FHE is a very computationally expensive method of homomorphic computation, and is the primary reason for the development of a private fusion method which does not rely on it. We have made this clearer with a complexity discussion in the added Section V.A as well. PHE and ORE are used to achieve this as their combined properties were suited to defining the new SecFCI fusion method.

\item \reviewtext{2, In the Introduction, the authors should add a short summary of the
technical novelties of this paper. Currently I cannot find anything.}

We apologies for not stating the novelty clearly, and have updated the introduction accordingly.

\item \reviewtext{3, In simulation the authors only considered FCI and SecFCI. Concerning
the fact that authors declared that another encryption scheme, named
Fully Homomorphic Encryption (FHE), will lead to infeasible
computational load. Furthermore, does the proposed method outperforms
the encryption scheme in which only PHE or ORE is adopted? I think it
would be nice to involve all the aforementioned strategies in
Simulation and report also the computational load of them.}

We agree that using different encryption schemes can greatly affect the computational load of the algorithm and apologize for having left out the comparisons of different methods. We also regret that the intention of the simulation was not made clearer, in that it was primarily intended to discuss accuracy. We have now added Section V.A where we give the computational complexity of SecFCI, and compare it to the unencrypted FCI algorithm, as well as making comparisons to the complexity of some current FHE scheme operations. We note that although FHE could be used to implement an encrypted homomorphic FCI, PHE or ORE alone has to the best of our knowledge, no known method to do so. We have for this reason not considered methods adopting only one of these schemes in the complexity analysis or simulation. We have also updated Section VI to make the simulation's purpose clearer and to give a more in-depth discussion on the accuracy of the SecFCI method. Run-times have not been given as an efficient implementation of the required encryption schemes is beyond the scope of this paper, but has been addressed as intended future work in Section VII.


\end{rebuttal}

%      8888888b.                         .d8888b.
%      888   Y88b                       d88P  Y88b
%      888    888                            .d88P
%      888   d88P .d88b.  888  888          8888"
%      8888888P" d8P  Y8b 888  888           "Y8b.
%      888 T88b  88888888 Y88  88P      888    888
%      888  T88b Y8b.      Y8bd8P       Y88b  d88P
%      888   T88b "Y8888    Y88P         "Y8888P"



\section*{Response to the Comments of Reviewer 3 (24111)}
\def\thesection{R3}
\begin{rebuttal}
\item \reviewtext{The paper present a secure algorithm for covariance intersection
implementation. Order revealing encryption is used to find the optimal
parameters in the algorithm. Subsequently, the Paillier's encryption, a
semi-homomorphic encryption method, to implement the algorithm. 

There is a large literature on control, estimation, and optimization
with homomorphic encryption. The authors have not cited any of those
studies. This is particularly important as they have dealt with some of
the issues raised in this paper. For instance, in Section III.C, the
authors discuss encoding real numbers when dealing with encryptions
that are implemented on integer rings. This has been investigated and
discussed at length in those papers, e.g., see secure and private
control using semi-homomorphic encryption in Control Engineering
Practice.}

We apologies for having missed references in homomorphic signal processing, and thank the reviewer for providing a reference. In particular, we agree that our encoding discussion in Section III.C was heavily lacking in references on encoding for homomorphic encryption. The introduction has been updated to include more references regarding homomorphic and partially homomorphic encryption in signal processing, and Section III.C has been updated to include relevant references and justifications for the encoding scheme used.

\item \reviewtext{One drawback of encrypted methods is the added overhead to the
computational complexity. I think the paper benefits more if the
authors discuss the added complexity. If this algorithm is to be
implemented in real-time, would we be able to run all the necessary
computations? Would the increased complexity results in slower
implementation?}

We agree that encrypted methods typically have the drawback of added complexity and apologize for having left this out of the manuscript. We have added a new Section V.A which discusses the added complexity of the fusion algorithm when compared to the unencrypted FCI algorithm. While we have implemented a simulation of the algorithm, its primary purpose was to demonstrate the accuracy of the method, which we have now made clearer in Section VI. Run-times have not been given as an efficient implementation of the required encryption schemes is beyond the scope of this paper, but has been addressed as intended future work in Section VII.

\end{rebuttal}

\includepdf[pages=-]{diff}

\end{document}