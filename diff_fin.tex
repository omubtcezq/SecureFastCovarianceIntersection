%%%%%%%%%%%%%%%%%%%%%%%%%%%%%%%%%%%%%%%%%%%%%%%%%%%%%%%%%%%%%%%%%%%%%%%%%%%%%%%%
%DIF LATEXDIFF DIFFERENCE FILE
%DIF DEL main_submitted_2.tex   Fri May 15 15:41:16 2020
%DIF ADD main.tex               Thu Jun  4 14:59:51 2020
%2345678901234567890123456789012345678901234567890123456789012345678901234567890
%        1         2         3         4         5         6         7         8

%DIF 5c5
%DIF < \documentclass[letterpaper, 10 pt, conference]{ieeeconf}  % Comment this line out if you need a4paper
%DIF -------
\documentclass[letterpaper, 10 pt, journal, twoside]{ieeetran}  % Comment this line out if you need a4paper %DIF > 
%DIF -------

%\documentclass[a4paper, 10pt, conference]{ieeeconf}      % Use this line for a4 paper

\IEEEoverridecommandlockouts                              % This command is only needed if 
                                                          % you want to use the \thanks command

%DIF 12c12
%DIF < \overrideIEEEmargins                                      % Needed to meet printer requirements.
%DIF -------
%\overrideIEEEmargins                                      % Needed to meet printer requirements. %DIF > 
%DIF -------

%In case you encounter the following error:
%Error 1010 The PDF file may be corrupt (unable to open PDF file) OR
%Error 1000 An error occurred while parsing a contents stream. Unable to analyze the PDF file.
%This is a known problem with pdfLaTeX conversion filter. The file cannot be opened with acrobat reader
%Please use one of the alternatives below to circumvent this error by uncommenting one or the other
%\pdfobjcompresslevel=0
%\pdfminorversion=4

% See the \addtolength command later in the file to balance the column lengths
% on the last page of the document

% The following packages can be found on http:\\www.ctan.org
%\usepackage{graphics} % for pdf, bitmapped graphics files
%\usepackage{epsfig} % for postscript graphics files
%\usepackage{mathptmx} % assumes new font selection scheme installed
%\usepackage{times} % assumes new font selection scheme installed
%\usepackage{amsmath} % assumes amsmath package installed
%\usepackage{amssymb}  % assumes amsmath package installed

% Groups consecutive citations together nicely
\usepackage{cite}

% ISAS conventions. Underlined vectors, random vectors, etc.
\usepackage{ISASmacros/isasmathmacros}
%\usepackage{ISASmacros/isastextmacros} % errors?

% Needed to display images generated with Matplotlib nicely
%DIF 41c41
%DIF < \usepackage{pgf}
%DIF -------
\usepackage{graphicx} %DIF > 
%DIF -------

% Used to make subfigures and fit 3 figures across the page. Command lets sub references be in parentheses
\usepackage[labelformat=simple]{subcaption}
\renewcommand\thesubfigure{(\alph{subfigure})}

% Keep the figure captions small
\usepackage[font=small]{caption}

% Used to make the last page have even height columns
\usepackage{flushend}

% Condenses everything and also makes it prettier
\usepackage{microtype}

%DIF 56-58d56
%DIF < % Dodgy method for saving more space
%DIF < %\usepackage[subtle]{savetrees}
%DIF < 
%DIF -------
\AtBeginDocument{%
%DIF 60-66c57-65
%DIF <    \abovedisplayskip=2pt plus 3pt
%DIF <    \belowdisplayskip=2pt plus 3pt
%DIF <    \abovedisplayshortskip=2pt plus 3pt
%DIF <    \belowdisplayshortskip=2pt plus 3pt
%DIF < 
%DIF <    % Use the template Bibtex features
%DIF <    %\bstctlcite{IEEEexample:BSTcontrol}
%DIF -------
   % defaults %DIF > 
   % \abovedisplayskip=10pt plus 2pt minus 5pt %DIF > 
   % \belowdisplayskip=10pt plus 2pt minus 5pt %DIF > 
   % \abovedisplayshortskip=0pt plus 3pt %DIF > 
   % \belowdisplayshortskip=6pt plus 3pt minus 3pt %DIF > 
   \abovedisplayskip=6pt plus 2pt minus 5pt %DIF > 
   \belowdisplayskip=6pt plus 2pt minus 5pt %DIF > 
   \abovedisplayshortskip=0pt plus 3pt %DIF > 
   \belowdisplayshortskip=6pt plus 3pt minus 3pt %DIF > 
%DIF -------
}

\title{\DIFdelbegin %DIFDELCMD < \LARGE \bf
%DIFDELCMD < %%%
\DIFdelend 
Secure Fast Covariance Intersection Using Partially Homomorphic and Order Revealing Encryption Schemes
}

\author{Marko Ristic, Benjamin Noack, and Uwe D. Hanebeck% <-this % stops a space
\thanks{Marko Ristic, Benjamin Noack, and Uwe D. Hanebeck are with the Intelligent Sensor-Actuator-Systems Laboratory (ISAS), Institute for Anthropomatics, Karlsruhe Institute of Technology (KIT), Germany.\newline
{\tt\small \{marko.ristic,noack,uwe.hanebeck\}@kit.edu}%
}% <-this % stops a space
}
%DIF PREAMBLE EXTENSION ADDED BY LATEXDIFF
%DIF UNDERLINE PREAMBLE %DIF PREAMBLE
\RequirePackage[normalem]{ulem} %DIF PREAMBLE
\RequirePackage{color}\definecolor{RED}{rgb}{1,0,0}\definecolor{BLUE}{rgb}{0,0,1} %DIF PREAMBLE
\providecommand{\DIFadd}[1]{{\protect\color{blue}\uwave{#1}}} %DIF PREAMBLE
\providecommand{\DIFdel}[1]{{\protect\color{red}\sout{#1}}}                      %DIF PREAMBLE
%DIF SAFE PREAMBLE %DIF PREAMBLE
\providecommand{\DIFaddbegin}{} %DIF PREAMBLE
\providecommand{\DIFaddend}{} %DIF PREAMBLE
\providecommand{\DIFdelbegin}{} %DIF PREAMBLE
\providecommand{\DIFdelend}{} %DIF PREAMBLE
\providecommand{\DIFmodbegin}{} %DIF PREAMBLE
\providecommand{\DIFmodend}{} %DIF PREAMBLE
%DIF FLOATSAFE PREAMBLE %DIF PREAMBLE
\providecommand{\DIFaddFL}[1]{\DIFadd{#1}} %DIF PREAMBLE
\providecommand{\DIFdelFL}[1]{\DIFdel{#1}} %DIF PREAMBLE
\providecommand{\DIFaddbeginFL}{} %DIF PREAMBLE
\providecommand{\DIFaddendFL}{} %DIF PREAMBLE
\providecommand{\DIFdelbeginFL}{} %DIF PREAMBLE
\providecommand{\DIFdelendFL}{} %DIF PREAMBLE
\newcommand{\DIFscaledelfig}{0.5}
%DIF HIGHLIGHTGRAPHICS PREAMBLE %DIF PREAMBLE
\RequirePackage{settobox} %DIF PREAMBLE
\RequirePackage{letltxmacro} %DIF PREAMBLE
\newsavebox{\DIFdelgraphicsbox} %DIF PREAMBLE
\newlength{\DIFdelgraphicswidth} %DIF PREAMBLE
\newlength{\DIFdelgraphicsheight} %DIF PREAMBLE
% store original definition of \includegraphics %DIF PREAMBLE
\LetLtxMacro{\DIFOincludegraphics}{\includegraphics} %DIF PREAMBLE
\newcommand{\DIFaddincludegraphics}[2][]{{\color{blue}\fbox{\DIFOincludegraphics[#1]{#2}}}} %DIF PREAMBLE
\newcommand{\DIFdelincludegraphics}[2][]{% %DIF PREAMBLE
\sbox{\DIFdelgraphicsbox}{\DIFOincludegraphics[#1]{#2}}% %DIF PREAMBLE
\settoboxwidth{\DIFdelgraphicswidth}{\DIFdelgraphicsbox} %DIF PREAMBLE
\settoboxtotalheight{\DIFdelgraphicsheight}{\DIFdelgraphicsbox} %DIF PREAMBLE
\scalebox{\DIFscaledelfig}{% %DIF PREAMBLE
\parbox[b]{\DIFdelgraphicswidth}{\usebox{\DIFdelgraphicsbox}\\[-\baselineskip] \rule{\DIFdelgraphicswidth}{0em}}\llap{\resizebox{\DIFdelgraphicswidth}{\DIFdelgraphicsheight}{% %DIF PREAMBLE
\setlength{\unitlength}{\DIFdelgraphicswidth}% %DIF PREAMBLE
\begin{picture}(1,1)% %DIF PREAMBLE
\thicklines\linethickness{2pt} %DIF PREAMBLE
{\color[rgb]{1,0,0}\put(0,0){\framebox(1,1){}}}% %DIF PREAMBLE
{\color[rgb]{1,0,0}\put(0,0){\line( 1,1){1}}}% %DIF PREAMBLE
{\color[rgb]{1,0,0}\put(0,1){\line(1,-1){1}}}% %DIF PREAMBLE
\end{picture}% %DIF PREAMBLE
}\hspace*{3pt}}} %DIF PREAMBLE
} %DIF PREAMBLE
\LetLtxMacro{\DIFOaddbegin}{\DIFaddbegin} %DIF PREAMBLE
\LetLtxMacro{\DIFOaddend}{\DIFaddend} %DIF PREAMBLE
\LetLtxMacro{\DIFOdelbegin}{\DIFdelbegin} %DIF PREAMBLE
\LetLtxMacro{\DIFOdelend}{\DIFdelend} %DIF PREAMBLE
\DeclareRobustCommand{\DIFaddbegin}{\DIFOaddbegin \let\includegraphics\DIFaddincludegraphics} %DIF PREAMBLE
\DeclareRobustCommand{\DIFaddend}{\DIFOaddend \let\includegraphics\DIFOincludegraphics} %DIF PREAMBLE
\DeclareRobustCommand{\DIFdelbegin}{\DIFOdelbegin \let\includegraphics\DIFdelincludegraphics} %DIF PREAMBLE
\DeclareRobustCommand{\DIFdelend}{\DIFOaddend \let\includegraphics\DIFOincludegraphics} %DIF PREAMBLE
\LetLtxMacro{\DIFOaddbeginFL}{\DIFaddbeginFL} %DIF PREAMBLE
\LetLtxMacro{\DIFOaddendFL}{\DIFaddendFL} %DIF PREAMBLE
\LetLtxMacro{\DIFOdelbeginFL}{\DIFdelbeginFL} %DIF PREAMBLE
\LetLtxMacro{\DIFOdelendFL}{\DIFdelendFL} %DIF PREAMBLE
\DeclareRobustCommand{\DIFaddbeginFL}{\DIFOaddbeginFL \let\includegraphics\DIFaddincludegraphics} %DIF PREAMBLE
\DeclareRobustCommand{\DIFaddendFL}{\DIFOaddendFL \let\includegraphics\DIFOincludegraphics} %DIF PREAMBLE
\DeclareRobustCommand{\DIFdelbeginFL}{\DIFOdelbeginFL \let\includegraphics\DIFdelincludegraphics} %DIF PREAMBLE
\DeclareRobustCommand{\DIFdelendFL}{\DIFOaddendFL \let\includegraphics\DIFOincludegraphics} %DIF PREAMBLE
%DIF LISTINGS PREAMBLE %DIF PREAMBLE
\RequirePackage{listings} %DIF PREAMBLE
\RequirePackage{color} %DIF PREAMBLE
\lstdefinelanguage{DIFcode}{ %DIF PREAMBLE
%DIF DIFCODE_UNDERLINE %DIF PREAMBLE
  moredelim=[il][\color{red}\sout]{\%DIF\ <\ }, %DIF PREAMBLE
  moredelim=[il][\color{blue}\uwave]{\%DIF\ >\ } %DIF PREAMBLE
} %DIF PREAMBLE
\lstdefinestyle{DIFverbatimstyle}{ %DIF PREAMBLE
	language=DIFcode, %DIF PREAMBLE
	basicstyle=\ttfamily, %DIF PREAMBLE
	columns=fullflexible, %DIF PREAMBLE
	keepspaces=true %DIF PREAMBLE
} %DIF PREAMBLE
\lstnewenvironment{DIFverbatim}{\lstset{style=DIFverbatimstyle}}{} %DIF PREAMBLE
\lstnewenvironment{DIFverbatim*}{\lstset{style=DIFverbatimstyle,showspaces=true}}{} %DIF PREAMBLE
%DIF END PREAMBLE EXTENSION ADDED BY LATEXDIFF

\begin{document}

\maketitle
\DIFdelbegin %DIFDELCMD < \thispagestyle{empty}
%DIFDELCMD < %%%
\DIFdelend \pagestyle{empty}
\DIFaddbegin \thispagestyle{empty}
\DIFaddend 

%        d8888 888               888                             888
%       d88888 888               888                             888
%      d88P888 888               888                             888
%     d88P 888 88888b.  .d8888b  888888 888d888 8888b.   .d8888b 888888
%    d88P  888 888 "88b 88K      888    888P"      "88b d88P"    888
%   d88P   888 888  888 "Y8888b. 888    888    .d888888 888      888
%  d8888888888 888 d88P      X88 Y88b.  888    888  888 Y88b.    Y88b.
% d88P     888 88888P"   88888P'  "Y888 888    "Y888888  "Y8888P  "Y888



%%%%%%%%%%%%%%%%%%%%%%%%%%%%%%%%%%%%%%%%%%%%%%%%%%%%%%%%%%%%%%%%%%%%%%%%%%%%%%%%
\begin{abstract}

Fast covariance intersection is a widespread technique for state estimate fusion in sensor networks when cross-correlations are not known and fast computations are desired. The common requirement of sending estimates from one party to another during fusion \DIFdelbegin \DIFdel{means they do not remain locally private}\DIFdelend \DIFaddbegin \DIFadd{forfeits local privacy}\DIFaddend . Current secure fusion algorithms rely on encryption schemes that do not provide sufficient flexibility\DIFdelbegin \DIFdel{and as a resultrequire, often undesired, }\DIFdelend \DIFaddbegin \DIFadd{. As a result, }\DIFaddend excess communication between estimate producers \DIFaddbegin \DIFadd{is required, which is often undesirable}\DIFaddend . We propose a novel method of homomorphically computing the fast covariance intersection algorithm on estimates encrypted with a combination of encryption schemes. Using order revealing encryption, we show how an approximate solution to the fast covariance intersection weights can be computed and combined with partially homomorphic encryptions of estimates, to calculate an encryption of the fused result. The described approach allows secure fusion of any number of private estimates, making third-party cloud processing a viable option when working with sensitive state estimates or when performing estimation over untrusted networks.

\end{abstract}

\DIFaddbegin \begin{IEEEkeywords}
\DIFadd{Sensor fusion, Secure estimation, Homomorphic encryption, Covariance intersection
}\end{IEEEkeywords}


\DIFaddend %%%%%%%%%%%%%%%%%%%%%%%%%%%%%%%%%%%%%%%%%%%%%%%%%%%%%%%%%%%%%%%%%%%%%%%%%%%%%%%%

% 8888888          888
%   888            888
%   888            888
%   888   88888b.  888888 888d888 .d88b.
%   888   888 "88b 888    888P"  d88""88b
%   888   888  888 888    888    888  888
%   888   888  888 Y88b.  888    Y88..88P
% 8888888 888  888  "Y888 888     "Y88P"



\section{Introduction}
\DIFdelbegin \DIFdel{Sensor }\DIFdelend \DIFaddbegin \IEEEPARstart{S}{ensor} \DIFaddend data processing and state estimation have been increasingly prevalent in networked systems \cite{ligginsDistributedDataFusion2012}. Bayesian state estimation has become a particularly common application since the beginning of Kalman estimation theory and has led to a large interest in the field of state estimation fusion \cite{willnerKalmanFilterAlgorithms1976,hashemipourDecentralizedStructuresParallel1988,chongFortyYearsDistributed2017}. Challenges of estimation fusion are closely tied to the handling and merging of estimation error statistics \cite{noackTreatmentDependentInformation2017}. Cross-correlations between estimation errors characterize dependencies between local estimates and must be considered when performing consistent or optimal fusion \cite{bar-shalomTracktotrackCorrelationProblem1981,sunMultisensorOptimalInformation2004}. Methods that keep track of these cross-correlations may require repeated reconstruction \cite{steinbringOptimalSamplebasedFusion2016} and typically add local computational complexity. An alternative strategy sees the approximation of error cross-correlation based on conservative suboptimal strategies, and has been implemented in a variety of methods \cite{julierNondivergentEstimationAlgorithm1997,noackDecentralizedDataFusion2017,niehsenInformationFusionBased2002}. Covariance Intersection (CI) \cite{julierNondivergentEstimationAlgorithm1997} provides one such popular \DIFdelbegin \DIFdel{conservative }\DIFdelend strategy, from which a less computationally expensive method, the Fast Covariance Intersection (FCI) \cite{niehsenInformationFusionBased2002} has been derived. CI is particularly well paired with the information form of the Kalman filter \cite{mutambaraDecentralizedEstimationControl1998,pfaffInformationFormDistributed2017}. This algebraically equivalent form of the standard Kalman filter requires the persistent storing of the information vector and information matrix instead of the usual state estimate and estimate covariance, and reduces fusion operations to simple summations. 

As advancements in distributed algorithms and cloud computing develop, the requirements for privacy and security in such systems have become more apparent \cite{renSecurityChallengesPublic2012,brennerSecretProgramExecution2011}. \DIFdelbegin \DIFdel{In particular }\DIFdelend \DIFaddbegin \DIFadd{This is particularly pertinent }\DIFaddend for sensor networks, \DIFaddbegin \DIFadd{where }\DIFaddend the desire for sensitive hardware information or estimation methodology to remain private may require the privacy of local measurements and estimates as well, and is a non-trivial problem in networks containing eavesdroppers or untrusted parties. Encryption has until recently been primarily used to secure information transfer between communicating parties\DIFdelbegin \DIFdel{. Common }\DIFdelend \DIFaddbegin \DIFadd{, relying on }\DIFaddend symmetric-key encryption schemes such as AES \cite{gueronIntelAdvancedEncryption2010} \DIFdelbegin \DIFdel{are used }\DIFdelend to encrypt sent information, and public-key schemes such as RSA \cite{rivestMethodObtainingDigital1978} to distribute symmetric keys. However, recent developments in public-key Homomorphic Encryption (HE) schemes \cite{gentryFullyHomomorphicEncryption2009,stehleFasterFullyHomomorphic2010,elgamalPublicKeyCryptosystem1985,paillierPublicKeyCryptosystemsBased1999}, which allow algebraic operations to be performed on encryptions, are leading to novel secure applications for signal processing in distributed and cloud computing environments \cite{lagendijkEncryptedSignalProcessing2012,aristovEncryptedMultisensorInformation2018,farokhiSecurePrivateControl2017,alexandruEncryptedCooperativeControl2019,kogisoCyberSecurityEnhancementNetworked2015,kerschbaumOutsourcedPrivateSet2012}. Fully Homomorphic Encryption (FHE) schemes \cite{gentryFullyHomomorphicEncryption2009,stehleFasterFullyHomomorphic2010} \DIFdelbegin \DIFdel{provide all algebraic operations over }\DIFdelend \DIFaddbegin \DIFadd{allow algebraic operations to be performed on }\DIFaddend encryptions, and are often theoretically suitable for secure processing in distributed environments\DIFdelbegin \DIFdel{. However, }\DIFdelend \DIFaddbegin \DIFadd{, but }\DIFaddend current implementations are still computationally infeasible for large-scale or real-time processing \cite{acarSurveyHomomorphicEncryption2018}. Partially Homomorphic Encryption (PHE) schemes \cite{elgamalPublicKeyCryptosystem1985,paillierPublicKeyCryptosystemsBased1999}, providing only a subset of these operations, have been a focus for such tasks due to their reduced computational requirements. \cite{aristovEncryptedMultisensorInformation2018} use PHE to run a private distributed Information Filter, \cite{alexandruEncryptedCooperativeControl2019,farokhiSecurePrivateControl2017} to compute private distributed control aggregation, \cite{kogisoCyberSecurityEnhancementNetworked2015} for private matrix multiplication, and \cite{kerschbaumOutsourcedPrivateSet2012} for private set intersection\DIFdelbegin \DIFdel{. These works are, however }\DIFdelend , \DIFaddbegin \DIFadd{however these works are relatively restricted in application }\DIFaddend due to the limited operations provided by PHE\DIFdelbegin \DIFdel{, relatively restricted in application}\DIFdelend . Recent developments in new encryption schemes, such as Order Revealing Encryption (ORE) \cite{chenettePracticalOrderRevealingEncryption2016,lewiOrderRevealingEncryptionNew2016,bogatovComparativeEvaluationOrderPreserving2018}, are now providing new light on the possible complexity of securely computable algorithms. In this paper, we develop a method for secure FCI fusion, such that local sensor information is kept private, using a combination of ORE and PHE schemes only, which has to the best of our knowledge not been achieved without the reliance on computationally expensive FHE schemes.

\subsection{Problem Formulation} \label{subsec:problem_formulation}
Our paper is motivated by a key step in multi-sensor fusion, the requirement of transmitting local sensor state estimates and covariance information over a network for the computation of their fused result. In particular, we consider centralized FCI fusion, where a party responsible for many networked sensors capable of computing their local state estimates, wishes to have their fused state estimate and covariance computed securely on an untrusted cloud. The same party may query the cloud fusion center for the fused result at any time. To preserve the privacy of local sensor measurements and state estimates, we aim to provide a secure FCI algorithm such that the fusion center does not learn individual sensor measurements, state estimates, or covariances. This will be achieved by encrypted homomorphic fusion, whereby the untrusted cloud learns only the FCI aggregation weights, \DIFdelbegin \DIFdel{as }\DIFdelend \DIFaddbegin \DIFadd{which }\DIFaddend will be shown in section \ref{sec:secfci}.

As we assume the querying party is the owner of all individual sensors, \DIFdelbegin \DIFdel{we consider }\DIFdelend the threat model \DIFaddbegin \DIFadd{to be considered is that }\DIFaddend of network eavesdroppers and a malicious fusion center, with no possible collusion between sensors and the fusion center.

\subsection{Notation}
Throughout this paper we will use the following notation. Lowercase characters represent scalars, and underlined characters, $\vec{x}$, represent vectors. Uppercase bold characters, $\mM$, are for matrices, where $\mM^{-1}$ denotes the matrix inverse, and $\tr(\cdot)$ the trace function. Covariance matrices will be represented by $\mP$. $\mathcal{E}_{pk}(a)$ and $\mathcal{E}_{ORE,k}(a)$ denote the encryption of $a$ using the public-key $pk$ and ORE key $k$, respectively, and similarly with the decryption functions $\mathcal{D}_{sk}(\cdot)$ and $\mathcal{D}_{ORE,k}(\cdot)$ with secret key $sk$, where any required real-number encodings of the number $a$ are assumed to be performed. $\mathcal{E}(a)$ and $\mathcal{E}_{ORE}(a)$ may be used for brevity when the encryption keys can be inferred from context. All encryption of vectors and matrices are defined element-wise, with elements given by $\mathcal{E}(\mP_{i,j}) = \mathcal{E}(\mP)_{i,j}$. Sets are represented as $\{\cdot\}$ and ordered lists with $[\cdot]$.

%  .d8888b. 8888888
% d88P  Y88b  888
% 888    888  888
% 888         888
% 888         888
% 888    888  888
% Y88b  d88P  888
%  "Y8888P" 8888888



\section{Covariance Intersection and Approximations} \label{sec:ci}
Covariance Intersection (CI), introduced in \cite{julierNondivergentEstimationAlgorithm1997}, provides a consistent state estimate fusion algorithm when cross-correlations are not known. The resulting fused estimate $\mean{\vec{x}}$ and covariance $\mP$ can be easily derived from its equations
\begin{equation}
   \mP^{-1}=\sum_{i=1}^{n}\omega_i \mP_i^{-1},\ \mP^{-1} \mean{\vec{x}} =\sum_{i=1}^{n}\omega_i \mP_i^{-1} \mean{\vec{x}}_i\DIFdelbegin \DIFdel{\,}\DIFdelend \DIFaddbegin \enspace\DIFaddend . \label{eqn:ci_cov_estimate}
\end{equation}
Note that \eqref{eqn:ci_cov_estimate} computes the fusion of the information vectors and information matrices defined in \cite{niehsenInformationFusionBased2002} and reduces the fusion to a weighted sum. Values for weights $\omega_i$ must satisfy
\begin{equation}
   \omega_1 + \omega_2 + \cdots + \omega_n = 1,\ 0 \leq \omega_i \leq 1\DIFdelbegin \DIFdel{\,}\DIFdelend \DIFaddbegin \enspace\DIFaddend , \label{eqn:ci_omega_sum_bound}
\end{equation}
which guarantees consistency of the fused estimates. They are chosen in a way to speed up convergence and minimize error \DIFdelbegin \DIFdel{, }\DIFdelend by minimizing a certain specified property of the resulting fused estimate covariance. One such property, the covariance trace, requires the solution to
\begin{equation}
   \argmin_{\omega_1,\dots,\omega_n} \{\tr(\mP)\}\! =\! \argmin_{\omega_1,\dots,\omega_n} \left\{\tr\left(\!\left(\sum_{i=1}^{n}\omega_i \mP_i^{-1}\!\right)^{-1}\right)\!\right\} \DIFdelbegin \DIFdel{\!. }\DIFdelend \label{eqn:ci_trace_min}
\end{equation}
\DIFaddbegin \DIFadd{for computing weights $\omega_i$. }\DIFaddend However, minimizing this non-linear cost function can be very computationally costly and has led to the development of faster approximation techniques.


\subsection{Fast Covariance intersection}
The Fast Covariance Intersection (FCI) algorithm from \cite{niehsenInformationFusionBased2002} is a non-iterative method for approximating the solution to \eqref{eqn:ci_trace_min} without the loss of guaranteed consistency. It is computed by defining a new constraint
\begin{equation}
   \omega_i \tr(\mP_i) - \omega_j \tr(\mP_j) = 0,\ i,j=1,2,\dots,n \label{eqn:fci_eq_big}
\end{equation}
on $\omega_i$ and solving the resulting equations instead. In the two sensor case, this results in the solving of
\begin{equation}
   \omega_1 \tr(\mP_1) - \omega_2 \tr(\mP_2) = 0,\ \omega_1 + \omega_2 = 1\DIFdelbegin \DIFdel{\,}\DIFdelend \DIFaddbegin \enspace\DIFaddend . \label{eqn:fci_2sen_omega_sum_eq}
\end{equation}
When computed for $n$ sensors, the highly redundant \eqref{eqn:fci_eq_big} can have its largest linearly independent subset represented by
\begin{equation}
   \omega_i \tr(\mP_i) - \omega_{i+1} \tr(\mP_{i+1}) = 0,\ i=1,2,\dots,n-1\DIFdelbegin \DIFdel{\,}\DIFdelend \DIFaddbegin \enspace\DIFaddend , \label{eqn:fci_eq}
\end{equation}
and requires the solution to the linear problem
\begin{equation}
   \begingroup
   \setlength\arraycolsep{2pt}
   \begin{bmatrix}
      \mathcal{P}_1 & -\mathcal{P}_2 & 0 & \cdots & 0 \\
      \cdots & \cdots & \cdots & \cdots & \cdots & \\
      0 & \cdots & 0 & \mathcal{P}_{n-1} & -\mathcal{P}_{n} \\
      1 & \cdots & 1 & 1 & 1 &
   \end{bmatrix}
   \begin{bmatrix}
      \omega_1 \\
      \vdots \\
      \omega_{n-1} \\
      \omega_{n}
   \end{bmatrix}
   =
   \begin{bmatrix}
      0 \\
      \vdots \\
      0 \\
      1
   \end{bmatrix}\DIFdelbegin \DIFdel{\,}\DIFdelend \DIFaddbegin \enspace\DIFaddend , \label{eqn:fci_eq_sys}
   \endgroup
\end{equation}
where we let $\mathcal{P}_i = \tr(\mP_i)$.

Our proposed filter aims to solve FCI fusion, namely \eqref{eqn:ci_cov_estimate} and \eqref{eqn:fci_eq_sys}, using only encrypted values from each sensor $i$, and leaking only the weight values $\omega_1,\dots,\omega_n$.

% 8888888b.  888    888 8888888888       .d8888b.            .d88888b.  8888888b.  8888888888
% 888   Y88b 888    888 888             d88P  "88b          d88P" "Y88b 888   Y88b 888
% 888    888 888    888 888             Y88b. d88P          888     888 888    888 888
% 888   d88P 8888888888 8888888          "Y8888P"           888     888 888   d88P 8888888
% 8888888P"  888    888 888             .d88P88K.d88P       888     888 8888888P"  888
% 888        888    888 888             888"  Y888P"        888     888 888 T88b   888
% 888        888    888 888             Y88b .d8888b        Y88b. .d88P 888  T88b  888
% 888        888    888 8888888888       "Y8888P" Y88b       "Y88888P"  888   T88b 8888888888



\section{Homomorphic and Order Revealing Encryption} \label{sec:encryption}
To achieve a secure solution to the FCI fusion problem, we make use of two types of function-providing encryption schemes\DIFdelbegin \DIFdel{. The }\DIFdelend \DIFaddbegin \DIFadd{: the }\DIFaddend Paillier additive PHE scheme \cite{paillierPublicKeyCryptosystemsBased1999} \DIFdelbegin \DIFdel{providing }\DIFdelend \DIFaddbegin \DIFadd{that provides }\DIFaddend a single homomorphic addition operation \DIFdelbegin \DIFdel{, }\DIFdelend and the Lewi ORE scheme \cite{lewiOrderRevealingEncryptionNew2016} \DIFdelbegin \DIFdel{providing }\DIFdelend \DIFaddbegin \DIFadd{that provides }\DIFaddend a secure comparison function.

The formal security of \DIFdelbegin \DIFdel{encryption schemes }\DIFdelend \DIFaddbegin \DIFadd{an encryption scheme }\DIFaddend consists of a security goal and a formal threat model \cite{katzIntroductionModernCryptography2008}. Indistinguishability of ciphertexts under the adaptive chosen ciphertext attack model (IND-CCA2) is commonly considered the strongest security guarantee, however \DIFdelbegin \DIFdel{, }\DIFdelend no homomorphic encryption scheme provides security against IND-CCA2 due to their apparent ability to create valid cyphertexts via homomorphic operations. Instead, PHE schemes aim to protect against the weaker assumption of the chosen plaintext attack model (IND-CPA) \cite{chaseSecurityHomomorphicEncryption2017}. Similarly, ORE schemes aim to protect against simulation-based security defined in \cite{chenettePracticalOrderRevealingEncryption2016} or the harder to achieve ordered chosen-plaintext attack model (IND-OCPA).

\subsection{Paillier Partially Homomorphic Encryption Scheme} \label{subsec:paillier}
We use the Paillier additive PHE scheme due to its implementation simplicity, and computational speed. The Paillier scheme provides two homomorphic operations on encrypted data, namely
\begin{equation}
   \mathcal{D}_{sk}(\mathcal{E}_{pk}(a)\mathcal{E}_{pk}(b) \!\!\pmod{N^{2}}) = a + b \!\!\pmod{N} \label{eqn:paillier_add}
\end{equation}
and
\begin{equation}
   \mathcal{D}_{sk}(\mathcal{E}_{pk}(a)^c \!\!\pmod{N^{2}}) = c\cdot a \!\!\pmod{N},\ c \in \mathbb{Z}_N\DIFdelbegin \DIFdel{\,}\DIFdelend \DIFaddbegin \enspace\DIFaddend , \label{eqn:paillier_mult}
\end{equation}
where the modulus $N$ is computed as the product of two large random primes chosen at key-generation. The public and secret keys are shown as $pk$ and $sk$ respectively, and plaintext messages $a,b \in \mathbb{Z}_N$. The Paillier encryption scheme successfully provides security against the IND-CPA model.

\subsection{Lewi Left-Right Order Revealing Encryption} \label{subsec:lewi}
For ORE, we use the Lewi symmetric-key Left-Right ORE scheme as it has the added property of only allowing certain comparisons between cyphertexts. This property can be used to decide which values may not be compared\DIFdelbegin \DIFdel{as }\DIFdelend \DIFaddbegin \DIFadd{, which }\DIFaddend will be shown in section \ref{sec:secfci}\DIFdelbegin \DIFdel{and }\DIFdelend \DIFaddbegin \DIFadd{. It }\DIFaddend is described as follows\DIFdelbegin \DIFdel{. Two }\DIFdelend \DIFaddbegin \DIFadd{: two }\DIFaddend encryption functions allow integers to be encrypted as either a ``Left'' ($L$) or ``Right'' ($R$) encryption by
\begin{equation}
   \DIFdelbegin %DIFDELCMD < \begin{aligned} \label{eqn:lewi_l_r}
%DIFDELCMD <       &\operatorfont{enc}^L_{ORE}(k, x) = \mathcal{E}^L_{ORE,k}(x) \\
%DIFDELCMD <       &\operatorfont{enc}^R_{ORE}(k, y) = \mathcal{E}^R_{ORE,k}(y)
%DIFDELCMD <    \end{aligned}%%%
\DIFdelend \DIFaddbegin \begin{aligned} \label{eqn:lewi_l_r}
      &\operatorfont{enc}^L_{ORE}(k, x) = \mathcal{E}^L_{ORE,k}(x)\enspace, \\
      &\operatorfont{enc}^R_{ORE}(k, y) = \mathcal{E}^R_{ORE,k}(y)\enspace,
   \end{aligned}\DIFaddend 
\DIFdelbegin \DIFdel{\,,
}\DIFdelend \end{equation}
and only comparisons between an $L$ and an $R$ encryption are possible, by
\begin{equation}
   \operatorfont{cmp}_{ORE}(\mathcal{E}^L_{ORE}(x),\ \mathcal{E}^R_{ORE}(y)) = \operatorfont{cmp}(x, y)\DIFdelbegin \DIFdel{\,}\DIFdelend \DIFaddbegin \enspace\DIFaddend . \label{eqn:lewi_cmp}
\end{equation}
Note that no decryption function is provided as only encryptions are required to provide a secure comparison. The Lewi ORE encryption scheme provides security against the simulation-based security model \cite{chenettePracticalOrderRevealingEncryption2016} but is not secure against the IND-OCPA model.

\subsection{Real Number Encoding for Homomorphic Encryption} \label{subsec:encoding}
Both encryption schemes in sections \ref{subsec:paillier} and \ref{subsec:lewi} are defined over positive integers, and the Paillier scheme bounds the largest encryptable integer by $N-1$. Due to the prevalence of real numbers in estimation theory, integer encoding of real numbers is an active field of research that \DIFdelbegin \DIFdel{acommpanies }\DIFdelend \DIFaddbegin \DIFadd{accompanies }\DIFaddend encrypted processing \cite{ziadCryptoImgPrivacyPreserving2016,farokhiSecurePrivateControl2017,cheonHomomorphicEncryptionArithmetic2017}, and a requirement for our estimate fusion algorithm. While some encoding schemes for additive homomorphic encryption provide additional operations such as homomorphic division \cite{ziadCryptoImgPrivacyPreserving2016}, they typically complicate the homomorphic operations, and in \cite{ziadCryptoImgPrivacyPreserving2016} leak exponent information of the encrypted real number. We have instead relied on the simpler encoding in \cite{farokhiSecurePrivateControl2017}. 

We consider encoding real numbers representable as rational fixed-point numbers of $b$ bits, consisting of a single sign bit, $i$ integer bits, and $f$ fractional bits. Thus, each encodable rational number is defined by its $b=1+i+f$ bits. As in \cite{farokhiSecurePrivateControl2017}, encoding is performed to allow for multiplication, which requires an operation modulus of $b+2f$ to avoid the requirement for comparisons. Conversion of any real number $a$ to an encoded rational fixed-point number is given by
\begin{equation}
   e = \floor{2^f a}\pmod{2^{b + 2f}}\DIFdelbegin \DIFdel{\,}\DIFdelend \DIFaddbegin \enspace\DIFaddend . \label{eqn:qmn}
\end{equation}
Multiplication of such encoded numbers requires a factor of $1/2^f$ to be removed. As shown in \cite{farokhiSecurePrivateControl2017}, cases of encoded multiplication can be computed exactly when using Paillier encryption, however, FCI guarantees only one homomorphic multiplication which we handle when decoding for simplicity. Decoding is defined by
\begin{equation}
   a=
   \begin{cases}
      2^{-2f}\left(e \!\!\pmod{2^{b+2f}}\right) & e<2^{b+2f-1} \\
      2^{-2f}\left((e \!\!\pmod{2^{b+2f}}) - 2^{b+2f}\right) & e\geq 2^{b+2f-1}
   \end{cases} \DIFdelbegin \DIFdel{\,. }\DIFdelend \label{eqn:qmn_mult_decode}
\end{equation}
and will be correct given only a single encoded multiplication has occurred.

Since the largest encryptable integer is given by $N-1$, the largest encodable real number must account for this. Thus, the integer bits $i$ and fractional bits $f$ must be chosen such that
\begin{equation}\label{eqn:qmn_max}
   \DIFdelbegin %DIFDELCMD < \begin{aligned}
%DIFDELCMD <       N &\geq 2^{b+2f}\\
%DIFDELCMD <       &\geq 2^{1+i+3f}
%DIFDELCMD <    \end{aligned}%%%
\DIFdelend \DIFaddbegin \begin{aligned}
      N &\geq 2^{b+2f}\\
      &\geq 2^{1+i+3f}\enspace.
   \end{aligned}\DIFaddend 
\DIFdelbegin \DIFdel{\,.
}\DIFdelend \end{equation}

%  .d8888b.                    8888888888 .d8888b. 8888888       .d8888b.
% d88P  Y88b                   888       d88P  Y88b  888        d88P  Y88b
% Y88b.                        888       888    888  888               888
%  "Y888b.    .d88b.   .d8888b 8888888   888         888             .d88P        .d8888b   .d88b.  88888b.
%     "Y88b. d8P  Y8b d88P"    888       888         888         .od888P"         88K      d8P  Y8b 888 "88b
%       "888 88888888 888      888       888    888  888        d88P"      888888 "Y8888b. 88888888 888  888
% Y88b  d88P Y8b.     Y88b.    888       Y88b  d88P  888        888"                   X88 Y8b.     888  888 d8b
%  "Y8888P"   "Y8888   "Y8888P 888        "Y8888P" 8888888      888888888          88888P'  "Y8888  888  888 Y8P



\section{Two-sensor Secure Fast Covariance Intersection} \label{sec:secfci}
In this section, we will introduce the Secure FCI (SecFCI) fusion algorithm for the two sensor case, before extending it to the $n$ sensor case in section \ref{sec:multi_secfci}. The network model we consider is described in section \ref{subsec:problem_formulation}, where sensors are capable of running local estimators, as well as the PHE and ORE encryption schemes from section \ref{sec:encryption}. Each sensor $i$ computes its state estimate $\mean{\vec{x}}_i$ and covariance matrix $\mP_i$ and sends relevant encrypted information to an untrusted cloud fusion center. The querying party is the key holding party and generates the PHE public key $pk$, secret key $sk$, and ORE symmetric key $k$. $pk$ is made available to all parties in the network, and $k$ is made available to the sensors only, via any standard public-key scheme such as RSA \cite{rivestMethodObtainingDigital1978}. When encrypting with ORE key $k$, individual sensors are limited to using only $L$ or $R$ ORE encryption to reduce local information leakage. Thus, consecutive ORE encryptions from any sensor cannot be used to infer local information directly, and can only be compared to encryptions from sensors using the alternate ORE encryption.

From \eqref{eqn:ci_cov_estimate}, we can see that both CI fusion equations can be computed on PHE encryptions of sensor information vectors and information matrices, given valid unencrypted values for each $\omega_i$. For this reason, we allow the leakage of all weights $\omega_i$. Thus, in the two sensor case, homomorphic fusion is computed by
\begin{equation}
   \mathcal{E}(\mP^{-1}) = \mathcal{E}(\mP^{-1}_1)^{\omega_1}\mathcal{E}(\mP^{-1}_2)^{(1 -\omega_1)} \label{eqn:paillier_ci_cov}
\end{equation}
and
\begin{equation}
   \mathcal{E}(\mP^{-1}\mean{\vec{x}}) = \mathcal{E}(\mP^{-1}_1\mean{\vec{x}}_1)^{\omega_1}\mathcal{E}(\mP^{-1}_2\mean{\vec{x}}_2)^{(1 - \omega_1)}\DIFdelbegin \DIFdel{\,}\DIFdelend \DIFaddbegin \enspace\DIFaddend , \label{eqn:paillier_ci_estimate}
\end{equation}
where we note that $\omega_2=1-\omega_1$ due to the CI requirement \eqref{eqn:ci_omega_sum_bound}. We also note that in \eqref{eqn:paillier_ci_cov} and \eqref{eqn:paillier_ci_estimate}, each resulting value will have exactly one encoding multiplication factor to remove, and can be decoded exactly by using \eqref{eqn:qmn_mult_decode}.

All that remains for computing CI homomorphically, in the two sensor case, is the calculation of parameter $\omega_1$. For this, we approximate the solution to FCI. Since our encoding scheme in section \ref{subsec:encoding} does not \DIFdelbegin \DIFdel{provide }\DIFdelend \DIFaddbegin \DIFadd{allow }\DIFaddend division, the exact result of \eqref{eqn:fci_2sen_omega_sum_eq} is approximated. This is accomplished by discretizing $\omega_i$ by step-size $s$, such that $s<1$ and $p=1/s \in \mathbb{Z}$, and approximating \eqref{eqn:fci_2sen_omega_sum_eq} with ORE. An ordered discretization of values $\omega^{(x)}$ is defined by
\begin{equation}
   [\omega^{(1)},\dots,\omega^{(p)}] = [0,s,\dots,1-s,1]\DIFdelbegin \DIFdel{\,}\DIFdelend \DIFaddbegin \enspace\DIFaddend ,
\end{equation}
and computed by each sensor $i$. Each $\omega^{(x)}$ is multiplied by $\tr(\mP_i)$ and encrypted with ORE key $k$. Sensor $1$'s list is defined by 
\begin{equation}
   [\mathcal{E}^L_{ORE}(\omega^{(1)}\tr(\mP_1)),\dots,\mathcal{E}^L_{ORE}(\omega^{(p)}\tr(\mP_1))]\DIFdelbegin \DIFdel{\,}\DIFdelend \DIFaddbegin \enspace\DIFaddend , \label{eqn:sen1_ore_list}
\end{equation}
and similarly sensor $2$'s by
\begin{equation}
   [\mathcal{E}^R_{ORE}(\omega^{(1)}\tr(\mP_2)),\dots,\mathcal{E}^R_{ORE}(\omega^{(p)}\tr(\mP_2))]\DIFdelbegin \DIFdel{\,}\DIFdelend \DIFaddbegin \enspace\DIFaddend . \label{eqn:sen2_ore_list}
\end{equation}
Note that Sensor $1$ uses only $L$ ORE while sensor $2$ uses only $R$ ORE\DIFaddbegin \DIFadd{, }\DIFaddend and that both lists are ordered. Lists \eqref{eqn:sen1_ore_list} and \eqref{eqn:sen2_ore_list} are sent alongside PHE encryptions of local information vector and information matrix estimates to the fusion center which uses them to estimate the FCI values of $\omega_1$ and $\omega_2$.

From \eqref{eqn:fci_2sen_omega_sum_eq} we know that $\omega_1$ must satisfy
\begin{equation}
   \omega_1 \tr(\mP_1) = (1-\omega_1)\tr(\mP_2)\DIFdelbegin \DIFdel{\,}\DIFdelend \DIFaddbegin \enspace\DIFaddend . \label{eqn:secfci_2sen_intersect}
\end{equation}
If we reverse \eqref{eqn:sen2_ore_list}, we obtain a list equivalent to one with values $\mathcal{E}^R_{ORE}((1-\omega^{(x)})\tr(\mP_2))$ for each discretization step $x$. When the reversed list is decrypted and plotted over \eqref{eqn:sen1_ore_list} the intersection gives the solution to \eqref{eqn:secfci_2sen_intersect} and therefore, \eqref{eqn:fci_2sen_omega_sum_eq}. However, \eqref{eqn:sen1_ore_list} and reversed \eqref{eqn:sen2_ore_list} consist of $L$ and $R$ ORE encryptions respectively, and the intersection must be approximated by locating consecutive $\omega^{(x)}$ \DIFdelbegin \DIFdel{discretisations }\DIFdelend \DIFaddbegin \DIFadd{discretizations }\DIFaddend where the sign of comparisons changes. This can be seen in Fig. \ref{fig:2_sensor_sol}, and can be performed in $O(\log{p})$ ORE comparisons using a binary search.
\begin{figure}[tb]
   \vspace{-5pt}
   \begin{center}
      \DIFdelbeginFL %DIFDELCMD < %% Creator: Matplotlib, PGF backend
%%
%% To include the figure in your LaTeX document, write
%%   \input{<filename>.pgf}
%%
%% Make sure the required packages are loaded in your preamble
%%   \usepackage{pgf}
%%
%% Figures using additional raster images can only be included by \input if
%% they are in the same directory as the main LaTeX file. For loading figures
%% from other directories you can use the `import` package
%%   \usepackage{import}
%% and then include the figures with
%%   \import{<path to file>}{<filename>.pgf}
%%
%% Matplotlib used the following preamble
%%
\begingroup%
\makeatletter%
\begin{pgfpicture}%
\pgfpathrectangle{\pgfpointorigin}{\pgfqpoint{3.200000in}{1.400000in}}%
\pgfusepath{use as bounding box, clip}%
\begin{pgfscope}%
\pgfsetbuttcap%
\pgfsetmiterjoin%
\definecolor{currentfill}{rgb}{1.000000,1.000000,1.000000}%
\pgfsetfillcolor{currentfill}%
\pgfsetlinewidth{0.000000pt}%
\definecolor{currentstroke}{rgb}{1.000000,1.000000,1.000000}%
\pgfsetstrokecolor{currentstroke}%
\pgfsetdash{}{0pt}%
\pgfpathmoveto{\pgfqpoint{0.000000in}{0.000000in}}%
\pgfpathlineto{\pgfqpoint{3.200000in}{0.000000in}}%
\pgfpathlineto{\pgfqpoint{3.200000in}{1.400000in}}%
\pgfpathlineto{\pgfqpoint{0.000000in}{1.400000in}}%
\pgfpathclose%
\pgfusepath{fill}%
\end{pgfscope}%
\begin{pgfscope}%
\pgfsetbuttcap%
\pgfsetmiterjoin%
\definecolor{currentfill}{rgb}{1.000000,1.000000,1.000000}%
\pgfsetfillcolor{currentfill}%
\pgfsetlinewidth{0.000000pt}%
\definecolor{currentstroke}{rgb}{0.000000,0.000000,0.000000}%
\pgfsetstrokecolor{currentstroke}%
\pgfsetstrokeopacity{0.000000}%
\pgfsetdash{}{0pt}%
\pgfpathmoveto{\pgfqpoint{0.398073in}{0.510201in}}%
\pgfpathlineto{\pgfqpoint{3.050000in}{0.510201in}}%
\pgfpathlineto{\pgfqpoint{3.050000in}{1.250000in}}%
\pgfpathlineto{\pgfqpoint{0.398073in}{1.250000in}}%
\pgfpathclose%
\pgfusepath{fill}%
\end{pgfscope}%
\begin{pgfscope}%
\pgfpathrectangle{\pgfqpoint{0.398073in}{0.510201in}}{\pgfqpoint{2.651927in}{0.739799in}}%
\pgfusepath{clip}%
\pgfsetbuttcap%
\pgfsetroundjoin%
\definecolor{currentfill}{rgb}{1.000000,0.000000,0.000000}%
\pgfsetfillcolor{currentfill}%
\pgfsetlinewidth{1.505625pt}%
\definecolor{currentstroke}{rgb}{1.000000,0.000000,0.000000}%
\pgfsetstrokecolor{currentstroke}%
\pgfsetdash{}{0pt}%
\pgfpathmoveto{\pgfqpoint{1.079659in}{0.522731in}}%
\pgfpathlineto{\pgfqpoint{1.162993in}{0.606064in}}%
\pgfpathmoveto{\pgfqpoint{1.079659in}{0.606064in}}%
\pgfpathlineto{\pgfqpoint{1.162993in}{0.522731in}}%
\pgfusepath{stroke,fill}%
\end{pgfscope}%
\begin{pgfscope}%
\pgfsetbuttcap%
\pgfsetroundjoin%
\definecolor{currentfill}{rgb}{0.000000,0.000000,0.000000}%
\pgfsetfillcolor{currentfill}%
\pgfsetlinewidth{0.803000pt}%
\definecolor{currentstroke}{rgb}{0.000000,0.000000,0.000000}%
\pgfsetstrokecolor{currentstroke}%
\pgfsetdash{}{0pt}%
\pgfsys@defobject{currentmarker}{\pgfqpoint{0.000000in}{-0.048611in}}{\pgfqpoint{0.000000in}{0.000000in}}{%
\pgfpathmoveto{\pgfqpoint{0.000000in}{0.000000in}}%
\pgfpathlineto{\pgfqpoint{0.000000in}{-0.048611in}}%
\pgfusepath{stroke,fill}%
}%
\begin{pgfscope}%
\pgfsys@transformshift{0.518615in}{0.510201in}%
\pgfsys@useobject{currentmarker}{}%
\end{pgfscope}%
\end{pgfscope}%
\begin{pgfscope}%
\definecolor{textcolor}{rgb}{0.000000,0.000000,0.000000}%
\pgfsetstrokecolor{textcolor}%
\pgfsetfillcolor{textcolor}%
\pgftext[x=0.518615in,y=0.412978in,,top]{\color{textcolor}\rmfamily\fontsize{8.330000}{9.996000}\selectfont \(\displaystyle 0.0\)}%
\end{pgfscope}%
\begin{pgfscope}%
\pgfsetbuttcap%
\pgfsetroundjoin%
\definecolor{currentfill}{rgb}{0.000000,0.000000,0.000000}%
\pgfsetfillcolor{currentfill}%
\pgfsetlinewidth{0.803000pt}%
\definecolor{currentstroke}{rgb}{0.000000,0.000000,0.000000}%
\pgfsetstrokecolor{currentstroke}%
\pgfsetdash{}{0pt}%
\pgfsys@defobject{currentmarker}{\pgfqpoint{0.000000in}{-0.048611in}}{\pgfqpoint{0.000000in}{0.000000in}}{%
\pgfpathmoveto{\pgfqpoint{0.000000in}{0.000000in}}%
\pgfpathlineto{\pgfqpoint{0.000000in}{-0.048611in}}%
\pgfusepath{stroke,fill}%
}%
\begin{pgfscope}%
\pgfsys@transformshift{1.000784in}{0.510201in}%
\pgfsys@useobject{currentmarker}{}%
\end{pgfscope}%
\end{pgfscope}%
\begin{pgfscope}%
\definecolor{textcolor}{rgb}{0.000000,0.000000,0.000000}%
\pgfsetstrokecolor{textcolor}%
\pgfsetfillcolor{textcolor}%
\pgftext[x=1.000784in,y=0.412978in,,top]{\color{textcolor}\rmfamily\fontsize{8.330000}{9.996000}\selectfont \(\displaystyle 0.2\)}%
\end{pgfscope}%
\begin{pgfscope}%
\pgfsetbuttcap%
\pgfsetroundjoin%
\definecolor{currentfill}{rgb}{0.000000,0.000000,0.000000}%
\pgfsetfillcolor{currentfill}%
\pgfsetlinewidth{0.803000pt}%
\definecolor{currentstroke}{rgb}{0.000000,0.000000,0.000000}%
\pgfsetstrokecolor{currentstroke}%
\pgfsetdash{}{0pt}%
\pgfsys@defobject{currentmarker}{\pgfqpoint{0.000000in}{-0.048611in}}{\pgfqpoint{0.000000in}{0.000000in}}{%
\pgfpathmoveto{\pgfqpoint{0.000000in}{0.000000in}}%
\pgfpathlineto{\pgfqpoint{0.000000in}{-0.048611in}}%
\pgfusepath{stroke,fill}%
}%
\begin{pgfscope}%
\pgfsys@transformshift{1.482952in}{0.510201in}%
\pgfsys@useobject{currentmarker}{}%
\end{pgfscope}%
\end{pgfscope}%
\begin{pgfscope}%
\definecolor{textcolor}{rgb}{0.000000,0.000000,0.000000}%
\pgfsetstrokecolor{textcolor}%
\pgfsetfillcolor{textcolor}%
\pgftext[x=1.482952in,y=0.412978in,,top]{\color{textcolor}\rmfamily\fontsize{8.330000}{9.996000}\selectfont \(\displaystyle 0.4\)}%
\end{pgfscope}%
\begin{pgfscope}%
\pgfsetbuttcap%
\pgfsetroundjoin%
\definecolor{currentfill}{rgb}{0.000000,0.000000,0.000000}%
\pgfsetfillcolor{currentfill}%
\pgfsetlinewidth{0.803000pt}%
\definecolor{currentstroke}{rgb}{0.000000,0.000000,0.000000}%
\pgfsetstrokecolor{currentstroke}%
\pgfsetdash{}{0pt}%
\pgfsys@defobject{currentmarker}{\pgfqpoint{0.000000in}{-0.048611in}}{\pgfqpoint{0.000000in}{0.000000in}}{%
\pgfpathmoveto{\pgfqpoint{0.000000in}{0.000000in}}%
\pgfpathlineto{\pgfqpoint{0.000000in}{-0.048611in}}%
\pgfusepath{stroke,fill}%
}%
\begin{pgfscope}%
\pgfsys@transformshift{1.965121in}{0.510201in}%
\pgfsys@useobject{currentmarker}{}%
\end{pgfscope}%
\end{pgfscope}%
\begin{pgfscope}%
\definecolor{textcolor}{rgb}{0.000000,0.000000,0.000000}%
\pgfsetstrokecolor{textcolor}%
\pgfsetfillcolor{textcolor}%
\pgftext[x=1.965121in,y=0.412978in,,top]{\color{textcolor}\rmfamily\fontsize{8.330000}{9.996000}\selectfont \(\displaystyle 0.6\)}%
\end{pgfscope}%
\begin{pgfscope}%
\pgfsetbuttcap%
\pgfsetroundjoin%
\definecolor{currentfill}{rgb}{0.000000,0.000000,0.000000}%
\pgfsetfillcolor{currentfill}%
\pgfsetlinewidth{0.803000pt}%
\definecolor{currentstroke}{rgb}{0.000000,0.000000,0.000000}%
\pgfsetstrokecolor{currentstroke}%
\pgfsetdash{}{0pt}%
\pgfsys@defobject{currentmarker}{\pgfqpoint{0.000000in}{-0.048611in}}{\pgfqpoint{0.000000in}{0.000000in}}{%
\pgfpathmoveto{\pgfqpoint{0.000000in}{0.000000in}}%
\pgfpathlineto{\pgfqpoint{0.000000in}{-0.048611in}}%
\pgfusepath{stroke,fill}%
}%
\begin{pgfscope}%
\pgfsys@transformshift{2.447289in}{0.510201in}%
\pgfsys@useobject{currentmarker}{}%
\end{pgfscope}%
\end{pgfscope}%
\begin{pgfscope}%
\definecolor{textcolor}{rgb}{0.000000,0.000000,0.000000}%
\pgfsetstrokecolor{textcolor}%
\pgfsetfillcolor{textcolor}%
\pgftext[x=2.447289in,y=0.412978in,,top]{\color{textcolor}\rmfamily\fontsize{8.330000}{9.996000}\selectfont \(\displaystyle 0.8\)}%
\end{pgfscope}%
\begin{pgfscope}%
\pgfsetbuttcap%
\pgfsetroundjoin%
\definecolor{currentfill}{rgb}{0.000000,0.000000,0.000000}%
\pgfsetfillcolor{currentfill}%
\pgfsetlinewidth{0.803000pt}%
\definecolor{currentstroke}{rgb}{0.000000,0.000000,0.000000}%
\pgfsetstrokecolor{currentstroke}%
\pgfsetdash{}{0pt}%
\pgfsys@defobject{currentmarker}{\pgfqpoint{0.000000in}{-0.048611in}}{\pgfqpoint{0.000000in}{0.000000in}}{%
\pgfpathmoveto{\pgfqpoint{0.000000in}{0.000000in}}%
\pgfpathlineto{\pgfqpoint{0.000000in}{-0.048611in}}%
\pgfusepath{stroke,fill}%
}%
\begin{pgfscope}%
\pgfsys@transformshift{2.929458in}{0.510201in}%
\pgfsys@useobject{currentmarker}{}%
\end{pgfscope}%
\end{pgfscope}%
\begin{pgfscope}%
\definecolor{textcolor}{rgb}{0.000000,0.000000,0.000000}%
\pgfsetstrokecolor{textcolor}%
\pgfsetfillcolor{textcolor}%
\pgftext[x=2.929458in,y=0.412978in,,top]{\color{textcolor}\rmfamily\fontsize{8.330000}{9.996000}\selectfont \(\displaystyle 1.0\)}%
\end{pgfscope}%
\begin{pgfscope}%
\definecolor{textcolor}{rgb}{0.000000,0.000000,0.000000}%
\pgfsetstrokecolor{textcolor}%
\pgfsetfillcolor{textcolor}%
\pgftext[x=1.724037in,y=0.258657in,,top]{\color{textcolor}\rmfamily\fontsize{8.330000}{9.996000}\selectfont \(\displaystyle \omega^{(x)}\)}%
\end{pgfscope}%
\begin{pgfscope}%
\pgfsetbuttcap%
\pgfsetroundjoin%
\definecolor{currentfill}{rgb}{0.000000,0.000000,0.000000}%
\pgfsetfillcolor{currentfill}%
\pgfsetlinewidth{0.803000pt}%
\definecolor{currentstroke}{rgb}{0.000000,0.000000,0.000000}%
\pgfsetstrokecolor{currentstroke}%
\pgfsetdash{}{0pt}%
\pgfsys@defobject{currentmarker}{\pgfqpoint{-0.048611in}{0.000000in}}{\pgfqpoint{0.000000in}{0.000000in}}{%
\pgfpathmoveto{\pgfqpoint{0.000000in}{0.000000in}}%
\pgfpathlineto{\pgfqpoint{-0.048611in}{0.000000in}}%
\pgfusepath{stroke,fill}%
}%
\begin{pgfscope}%
\pgfsys@transformshift{0.398073in}{0.564397in}%
\pgfsys@useobject{currentmarker}{}%
\end{pgfscope}%
\end{pgfscope}%
\begin{pgfscope}%
\definecolor{textcolor}{rgb}{0.000000,0.000000,0.000000}%
\pgfsetstrokecolor{textcolor}%
\pgfsetfillcolor{textcolor}%
\pgftext[x=0.241822in,y=0.525817in,left,base]{\color{textcolor}\rmfamily\fontsize{8.330000}{9.996000}\selectfont \(\displaystyle 0\)}%
\end{pgfscope}%
\begin{pgfscope}%
\pgfsetbuttcap%
\pgfsetroundjoin%
\definecolor{currentfill}{rgb}{0.000000,0.000000,0.000000}%
\pgfsetfillcolor{currentfill}%
\pgfsetlinewidth{0.803000pt}%
\definecolor{currentstroke}{rgb}{0.000000,0.000000,0.000000}%
\pgfsetstrokecolor{currentstroke}%
\pgfsetdash{}{0pt}%
\pgfsys@defobject{currentmarker}{\pgfqpoint{-0.048611in}{0.000000in}}{\pgfqpoint{0.000000in}{0.000000in}}{%
\pgfpathmoveto{\pgfqpoint{0.000000in}{0.000000in}}%
\pgfpathlineto{\pgfqpoint{-0.048611in}{0.000000in}}%
\pgfusepath{stroke,fill}%
}%
\begin{pgfscope}%
\pgfsys@transformshift{0.398073in}{0.993329in}%
\pgfsys@useobject{currentmarker}{}%
\end{pgfscope}%
\end{pgfscope}%
\begin{pgfscope}%
\definecolor{textcolor}{rgb}{0.000000,0.000000,0.000000}%
\pgfsetstrokecolor{textcolor}%
\pgfsetfillcolor{textcolor}%
\pgftext[x=0.241822in,y=0.954748in,left,base]{\color{textcolor}\rmfamily\fontsize{8.330000}{9.996000}\selectfont \(\displaystyle 5\)}%
\end{pgfscope}%
\begin{pgfscope}%
\pgfpathrectangle{\pgfqpoint{0.398073in}{0.510201in}}{\pgfqpoint{2.651927in}{0.739799in}}%
\pgfusepath{clip}%
\pgfsetbuttcap%
\pgfsetroundjoin%
\definecolor{currentfill}{rgb}{0.501961,0.501961,0.501961}%
\pgfsetfillcolor{currentfill}%
\pgfsetlinewidth{1.505625pt}%
\definecolor{currentstroke}{rgb}{0.501961,0.501961,0.501961}%
\pgfsetstrokecolor{currentstroke}%
\pgfsetdash{}{0pt}%
\pgfpathmoveto{\pgfqpoint{0.959117in}{0.522731in}}%
\pgfpathlineto{\pgfqpoint{1.042450in}{0.606064in}}%
\pgfpathmoveto{\pgfqpoint{0.959117in}{0.606064in}}%
\pgfpathlineto{\pgfqpoint{1.042450in}{0.522731in}}%
\pgfusepath{stroke,fill}%
\end{pgfscope}%
\begin{pgfscope}%
\pgfpathrectangle{\pgfqpoint{0.398073in}{0.510201in}}{\pgfqpoint{2.651927in}{0.739799in}}%
\pgfusepath{clip}%
\pgfsetbuttcap%
\pgfsetroundjoin%
\definecolor{currentfill}{rgb}{0.501961,0.501961,0.501961}%
\pgfsetfillcolor{currentfill}%
\pgfsetlinewidth{1.505625pt}%
\definecolor{currentstroke}{rgb}{0.501961,0.501961,0.501961}%
\pgfsetstrokecolor{currentstroke}%
\pgfsetdash{}{0pt}%
\pgfpathmoveto{\pgfqpoint{1.200201in}{0.522731in}}%
\pgfpathlineto{\pgfqpoint{1.283535in}{0.606064in}}%
\pgfpathmoveto{\pgfqpoint{1.200201in}{0.606064in}}%
\pgfpathlineto{\pgfqpoint{1.283535in}{0.522731in}}%
\pgfusepath{stroke,fill}%
\end{pgfscope}%
\begin{pgfscope}%
\pgfpathrectangle{\pgfqpoint{0.398073in}{0.510201in}}{\pgfqpoint{2.651927in}{0.739799in}}%
\pgfusepath{clip}%
\pgfsetbuttcap%
\pgfsetroundjoin%
\pgfsetlinewidth{1.505625pt}%
\definecolor{currentstroke}{rgb}{0.501961,0.501961,0.501961}%
\pgfsetstrokecolor{currentstroke}%
\pgfsetdash{{5.550000pt}{2.400000pt}}{0.000000pt}%
\pgfpathmoveto{\pgfqpoint{1.000784in}{0.564397in}}%
\pgfpathlineto{\pgfqpoint{1.000784in}{0.729107in}}%
\pgfusepath{stroke}%
\end{pgfscope}%
\begin{pgfscope}%
\pgfpathrectangle{\pgfqpoint{0.398073in}{0.510201in}}{\pgfqpoint{2.651927in}{0.739799in}}%
\pgfusepath{clip}%
\pgfsetbuttcap%
\pgfsetroundjoin%
\pgfsetlinewidth{1.505625pt}%
\definecolor{currentstroke}{rgb}{0.501961,0.501961,0.501961}%
\pgfsetstrokecolor{currentstroke}%
\pgfsetdash{{5.550000pt}{2.400000pt}}{0.000000pt}%
\pgfpathmoveto{\pgfqpoint{1.241868in}{0.564397in}}%
\pgfpathlineto{\pgfqpoint{1.241868in}{0.759990in}}%
\pgfusepath{stroke}%
\end{pgfscope}%
\begin{pgfscope}%
\pgfsetrectcap%
\pgfsetmiterjoin%
\pgfsetlinewidth{0.803000pt}%
\definecolor{currentstroke}{rgb}{0.000000,0.000000,0.000000}%
\pgfsetstrokecolor{currentstroke}%
\pgfsetdash{}{0pt}%
\pgfpathmoveto{\pgfqpoint{0.398073in}{0.510201in}}%
\pgfpathlineto{\pgfqpoint{0.398073in}{1.250000in}}%
\pgfusepath{stroke}%
\end{pgfscope}%
\begin{pgfscope}%
\pgfsetrectcap%
\pgfsetmiterjoin%
\pgfsetlinewidth{0.803000pt}%
\definecolor{currentstroke}{rgb}{0.000000,0.000000,0.000000}%
\pgfsetstrokecolor{currentstroke}%
\pgfsetdash{}{0pt}%
\pgfpathmoveto{\pgfqpoint{3.050000in}{0.510201in}}%
\pgfpathlineto{\pgfqpoint{3.050000in}{1.250000in}}%
\pgfusepath{stroke}%
\end{pgfscope}%
\begin{pgfscope}%
\pgfsetrectcap%
\pgfsetmiterjoin%
\pgfsetlinewidth{0.803000pt}%
\definecolor{currentstroke}{rgb}{0.000000,0.000000,0.000000}%
\pgfsetstrokecolor{currentstroke}%
\pgfsetdash{}{0pt}%
\pgfpathmoveto{\pgfqpoint{0.398073in}{0.510201in}}%
\pgfpathlineto{\pgfqpoint{3.050000in}{0.510201in}}%
\pgfusepath{stroke}%
\end{pgfscope}%
\begin{pgfscope}%
\pgfsetrectcap%
\pgfsetmiterjoin%
\pgfsetlinewidth{0.803000pt}%
\definecolor{currentstroke}{rgb}{0.000000,0.000000,0.000000}%
\pgfsetstrokecolor{currentstroke}%
\pgfsetdash{}{0pt}%
\pgfpathmoveto{\pgfqpoint{0.398073in}{1.250000in}}%
\pgfpathlineto{\pgfqpoint{3.050000in}{1.250000in}}%
\pgfusepath{stroke}%
\end{pgfscope}%
\begin{pgfscope}%
\pgfpathrectangle{\pgfqpoint{0.398073in}{0.510201in}}{\pgfqpoint{2.651927in}{0.739799in}}%
\pgfusepath{clip}%
\pgfsetrectcap%
\pgfsetroundjoin%
\pgfsetlinewidth{1.505625pt}%
\definecolor{currentstroke}{rgb}{0.000000,0.500000,0.000000}%
\pgfsetstrokecolor{currentstroke}%
\pgfsetdash{}{0pt}%
\pgfpathmoveto{\pgfqpoint{0.518615in}{0.564397in}}%
\pgfpathlineto{\pgfqpoint{0.759700in}{0.629595in}}%
\pgfpathlineto{\pgfqpoint{1.000784in}{0.694792in}}%
\pgfpathlineto{\pgfqpoint{1.241868in}{0.759990in}}%
\pgfpathlineto{\pgfqpoint{1.482952in}{0.825188in}}%
\pgfpathlineto{\pgfqpoint{1.724037in}{0.890385in}}%
\pgfpathlineto{\pgfqpoint{1.965121in}{0.955583in}}%
\pgfpathlineto{\pgfqpoint{2.206205in}{1.020780in}}%
\pgfpathlineto{\pgfqpoint{2.447289in}{1.085978in}}%
\pgfpathlineto{\pgfqpoint{2.688374in}{1.151175in}}%
\pgfpathlineto{\pgfqpoint{2.929458in}{1.216373in}}%
\pgfusepath{stroke}%
\end{pgfscope}%
\begin{pgfscope}%
\pgfpathrectangle{\pgfqpoint{0.398073in}{0.510201in}}{\pgfqpoint{2.651927in}{0.739799in}}%
\pgfusepath{clip}%
\pgfsetbuttcap%
\pgfsetroundjoin%
\definecolor{currentfill}{rgb}{0.000000,0.500000,0.000000}%
\pgfsetfillcolor{currentfill}%
\pgfsetlinewidth{1.003750pt}%
\definecolor{currentstroke}{rgb}{0.000000,0.500000,0.000000}%
\pgfsetstrokecolor{currentstroke}%
\pgfsetdash{}{0pt}%
\pgfsys@defobject{currentmarker}{\pgfqpoint{-0.020833in}{-0.020833in}}{\pgfqpoint{0.020833in}{0.020833in}}{%
\pgfpathmoveto{\pgfqpoint{0.000000in}{-0.020833in}}%
\pgfpathcurveto{\pgfqpoint{0.005525in}{-0.020833in}}{\pgfqpoint{0.010825in}{-0.018638in}}{\pgfqpoint{0.014731in}{-0.014731in}}%
\pgfpathcurveto{\pgfqpoint{0.018638in}{-0.010825in}}{\pgfqpoint{0.020833in}{-0.005525in}}{\pgfqpoint{0.020833in}{0.000000in}}%
\pgfpathcurveto{\pgfqpoint{0.020833in}{0.005525in}}{\pgfqpoint{0.018638in}{0.010825in}}{\pgfqpoint{0.014731in}{0.014731in}}%
\pgfpathcurveto{\pgfqpoint{0.010825in}{0.018638in}}{\pgfqpoint{0.005525in}{0.020833in}}{\pgfqpoint{0.000000in}{0.020833in}}%
\pgfpathcurveto{\pgfqpoint{-0.005525in}{0.020833in}}{\pgfqpoint{-0.010825in}{0.018638in}}{\pgfqpoint{-0.014731in}{0.014731in}}%
\pgfpathcurveto{\pgfqpoint{-0.018638in}{0.010825in}}{\pgfqpoint{-0.020833in}{0.005525in}}{\pgfqpoint{-0.020833in}{0.000000in}}%
\pgfpathcurveto{\pgfqpoint{-0.020833in}{-0.005525in}}{\pgfqpoint{-0.018638in}{-0.010825in}}{\pgfqpoint{-0.014731in}{-0.014731in}}%
\pgfpathcurveto{\pgfqpoint{-0.010825in}{-0.018638in}}{\pgfqpoint{-0.005525in}{-0.020833in}}{\pgfqpoint{0.000000in}{-0.020833in}}%
\pgfpathclose%
\pgfusepath{stroke,fill}%
}%
\begin{pgfscope}%
\pgfsys@transformshift{0.518615in}{0.564397in}%
\pgfsys@useobject{currentmarker}{}%
\end{pgfscope}%
\begin{pgfscope}%
\pgfsys@transformshift{0.759700in}{0.629595in}%
\pgfsys@useobject{currentmarker}{}%
\end{pgfscope}%
\begin{pgfscope}%
\pgfsys@transformshift{1.000784in}{0.694792in}%
\pgfsys@useobject{currentmarker}{}%
\end{pgfscope}%
\begin{pgfscope}%
\pgfsys@transformshift{1.241868in}{0.759990in}%
\pgfsys@useobject{currentmarker}{}%
\end{pgfscope}%
\begin{pgfscope}%
\pgfsys@transformshift{1.482952in}{0.825188in}%
\pgfsys@useobject{currentmarker}{}%
\end{pgfscope}%
\begin{pgfscope}%
\pgfsys@transformshift{1.724037in}{0.890385in}%
\pgfsys@useobject{currentmarker}{}%
\end{pgfscope}%
\begin{pgfscope}%
\pgfsys@transformshift{1.965121in}{0.955583in}%
\pgfsys@useobject{currentmarker}{}%
\end{pgfscope}%
\begin{pgfscope}%
\pgfsys@transformshift{2.206205in}{1.020780in}%
\pgfsys@useobject{currentmarker}{}%
\end{pgfscope}%
\begin{pgfscope}%
\pgfsys@transformshift{2.447289in}{1.085978in}%
\pgfsys@useobject{currentmarker}{}%
\end{pgfscope}%
\begin{pgfscope}%
\pgfsys@transformshift{2.688374in}{1.151175in}%
\pgfsys@useobject{currentmarker}{}%
\end{pgfscope}%
\begin{pgfscope}%
\pgfsys@transformshift{2.929458in}{1.216373in}%
\pgfsys@useobject{currentmarker}{}%
\end{pgfscope}%
\end{pgfscope}%
\begin{pgfscope}%
\pgfpathrectangle{\pgfqpoint{0.398073in}{0.510201in}}{\pgfqpoint{2.651927in}{0.739799in}}%
\pgfusepath{clip}%
\pgfsetrectcap%
\pgfsetroundjoin%
\pgfsetlinewidth{1.505625pt}%
\definecolor{currentstroke}{rgb}{0.000000,0.000000,1.000000}%
\pgfsetstrokecolor{currentstroke}%
\pgfsetdash{}{0pt}%
\pgfpathmoveto{\pgfqpoint{0.518615in}{0.770284in}}%
\pgfpathlineto{\pgfqpoint{0.759700in}{0.749696in}}%
\pgfpathlineto{\pgfqpoint{1.000784in}{0.729107in}}%
\pgfpathlineto{\pgfqpoint{1.241868in}{0.708518in}}%
\pgfpathlineto{\pgfqpoint{1.482952in}{0.687930in}}%
\pgfpathlineto{\pgfqpoint{1.724037in}{0.667341in}}%
\pgfpathlineto{\pgfqpoint{1.965121in}{0.646752in}}%
\pgfpathlineto{\pgfqpoint{2.206205in}{0.626163in}}%
\pgfpathlineto{\pgfqpoint{2.447289in}{0.605575in}}%
\pgfpathlineto{\pgfqpoint{2.688374in}{0.584986in}}%
\pgfpathlineto{\pgfqpoint{2.929458in}{0.564397in}}%
\pgfusepath{stroke}%
\end{pgfscope}%
\begin{pgfscope}%
\pgfpathrectangle{\pgfqpoint{0.398073in}{0.510201in}}{\pgfqpoint{2.651927in}{0.739799in}}%
\pgfusepath{clip}%
\pgfsetbuttcap%
\pgfsetroundjoin%
\definecolor{currentfill}{rgb}{0.000000,0.000000,1.000000}%
\pgfsetfillcolor{currentfill}%
\pgfsetlinewidth{1.003750pt}%
\definecolor{currentstroke}{rgb}{0.000000,0.000000,1.000000}%
\pgfsetstrokecolor{currentstroke}%
\pgfsetdash{}{0pt}%
\pgfsys@defobject{currentmarker}{\pgfqpoint{-0.020833in}{-0.020833in}}{\pgfqpoint{0.020833in}{0.020833in}}{%
\pgfpathmoveto{\pgfqpoint{0.000000in}{-0.020833in}}%
\pgfpathcurveto{\pgfqpoint{0.005525in}{-0.020833in}}{\pgfqpoint{0.010825in}{-0.018638in}}{\pgfqpoint{0.014731in}{-0.014731in}}%
\pgfpathcurveto{\pgfqpoint{0.018638in}{-0.010825in}}{\pgfqpoint{0.020833in}{-0.005525in}}{\pgfqpoint{0.020833in}{0.000000in}}%
\pgfpathcurveto{\pgfqpoint{0.020833in}{0.005525in}}{\pgfqpoint{0.018638in}{0.010825in}}{\pgfqpoint{0.014731in}{0.014731in}}%
\pgfpathcurveto{\pgfqpoint{0.010825in}{0.018638in}}{\pgfqpoint{0.005525in}{0.020833in}}{\pgfqpoint{0.000000in}{0.020833in}}%
\pgfpathcurveto{\pgfqpoint{-0.005525in}{0.020833in}}{\pgfqpoint{-0.010825in}{0.018638in}}{\pgfqpoint{-0.014731in}{0.014731in}}%
\pgfpathcurveto{\pgfqpoint{-0.018638in}{0.010825in}}{\pgfqpoint{-0.020833in}{0.005525in}}{\pgfqpoint{-0.020833in}{0.000000in}}%
\pgfpathcurveto{\pgfqpoint{-0.020833in}{-0.005525in}}{\pgfqpoint{-0.018638in}{-0.010825in}}{\pgfqpoint{-0.014731in}{-0.014731in}}%
\pgfpathcurveto{\pgfqpoint{-0.010825in}{-0.018638in}}{\pgfqpoint{-0.005525in}{-0.020833in}}{\pgfqpoint{0.000000in}{-0.020833in}}%
\pgfpathclose%
\pgfusepath{stroke,fill}%
}%
\begin{pgfscope}%
\pgfsys@transformshift{0.518615in}{0.770284in}%
\pgfsys@useobject{currentmarker}{}%
\end{pgfscope}%
\begin{pgfscope}%
\pgfsys@transformshift{0.759700in}{0.749696in}%
\pgfsys@useobject{currentmarker}{}%
\end{pgfscope}%
\begin{pgfscope}%
\pgfsys@transformshift{1.000784in}{0.729107in}%
\pgfsys@useobject{currentmarker}{}%
\end{pgfscope}%
\begin{pgfscope}%
\pgfsys@transformshift{1.241868in}{0.708518in}%
\pgfsys@useobject{currentmarker}{}%
\end{pgfscope}%
\begin{pgfscope}%
\pgfsys@transformshift{1.482952in}{0.687930in}%
\pgfsys@useobject{currentmarker}{}%
\end{pgfscope}%
\begin{pgfscope}%
\pgfsys@transformshift{1.724037in}{0.667341in}%
\pgfsys@useobject{currentmarker}{}%
\end{pgfscope}%
\begin{pgfscope}%
\pgfsys@transformshift{1.965121in}{0.646752in}%
\pgfsys@useobject{currentmarker}{}%
\end{pgfscope}%
\begin{pgfscope}%
\pgfsys@transformshift{2.206205in}{0.626163in}%
\pgfsys@useobject{currentmarker}{}%
\end{pgfscope}%
\begin{pgfscope}%
\pgfsys@transformshift{2.447289in}{0.605575in}%
\pgfsys@useobject{currentmarker}{}%
\end{pgfscope}%
\begin{pgfscope}%
\pgfsys@transformshift{2.688374in}{0.584986in}%
\pgfsys@useobject{currentmarker}{}%
\end{pgfscope}%
\begin{pgfscope}%
\pgfsys@transformshift{2.929458in}{0.564397in}%
\pgfsys@useobject{currentmarker}{}%
\end{pgfscope}%
\end{pgfscope}%
\begin{pgfscope}%
\pgfsetbuttcap%
\pgfsetmiterjoin%
\definecolor{currentfill}{rgb}{1.000000,1.000000,1.000000}%
\pgfsetfillcolor{currentfill}%
\pgfsetfillopacity{0.800000}%
\pgfsetlinewidth{1.003750pt}%
\definecolor{currentstroke}{rgb}{0.800000,0.800000,0.800000}%
\pgfsetstrokecolor{currentstroke}%
\pgfsetstrokeopacity{0.800000}%
\pgfsetdash{}{0pt}%
\pgfpathmoveto{\pgfqpoint{1.850758in}{0.556778in}}%
\pgfpathlineto{\pgfqpoint{2.982528in}{0.556778in}}%
\pgfpathquadraticcurveto{\pgfqpoint{3.001806in}{0.556778in}}{\pgfqpoint{3.001806in}{0.576056in}}%
\pgfpathlineto{\pgfqpoint{3.001806in}{1.182528in}}%
\pgfpathquadraticcurveto{\pgfqpoint{3.001806in}{1.201806in}}{\pgfqpoint{2.982528in}{1.201806in}}%
\pgfpathlineto{\pgfqpoint{1.850758in}{1.201806in}}%
\pgfpathquadraticcurveto{\pgfqpoint{1.831480in}{1.201806in}}{\pgfqpoint{1.831480in}{1.182528in}}%
\pgfpathlineto{\pgfqpoint{1.831480in}{0.576056in}}%
\pgfpathquadraticcurveto{\pgfqpoint{1.831480in}{0.556778in}}{\pgfqpoint{1.850758in}{0.556778in}}%
\pgfpathclose%
\pgfusepath{stroke,fill}%
\end{pgfscope}%
\begin{pgfscope}%
\pgfsetrectcap%
\pgfsetroundjoin%
\pgfsetlinewidth{1.505625pt}%
\definecolor{currentstroke}{rgb}{0.000000,0.500000,0.000000}%
\pgfsetstrokecolor{currentstroke}%
\pgfsetdash{}{0pt}%
\pgfpathmoveto{\pgfqpoint{1.870035in}{1.096045in}}%
\pgfpathlineto{\pgfqpoint{2.062813in}{1.096045in}}%
\pgfusepath{stroke}%
\end{pgfscope}%
\begin{pgfscope}%
\pgfsetbuttcap%
\pgfsetroundjoin%
\definecolor{currentfill}{rgb}{0.000000,0.500000,0.000000}%
\pgfsetfillcolor{currentfill}%
\pgfsetlinewidth{1.003750pt}%
\definecolor{currentstroke}{rgb}{0.000000,0.500000,0.000000}%
\pgfsetstrokecolor{currentstroke}%
\pgfsetdash{}{0pt}%
\pgfsys@defobject{currentmarker}{\pgfqpoint{-0.020833in}{-0.020833in}}{\pgfqpoint{0.020833in}{0.020833in}}{%
\pgfpathmoveto{\pgfqpoint{0.000000in}{-0.020833in}}%
\pgfpathcurveto{\pgfqpoint{0.005525in}{-0.020833in}}{\pgfqpoint{0.010825in}{-0.018638in}}{\pgfqpoint{0.014731in}{-0.014731in}}%
\pgfpathcurveto{\pgfqpoint{0.018638in}{-0.010825in}}{\pgfqpoint{0.020833in}{-0.005525in}}{\pgfqpoint{0.020833in}{0.000000in}}%
\pgfpathcurveto{\pgfqpoint{0.020833in}{0.005525in}}{\pgfqpoint{0.018638in}{0.010825in}}{\pgfqpoint{0.014731in}{0.014731in}}%
\pgfpathcurveto{\pgfqpoint{0.010825in}{0.018638in}}{\pgfqpoint{0.005525in}{0.020833in}}{\pgfqpoint{0.000000in}{0.020833in}}%
\pgfpathcurveto{\pgfqpoint{-0.005525in}{0.020833in}}{\pgfqpoint{-0.010825in}{0.018638in}}{\pgfqpoint{-0.014731in}{0.014731in}}%
\pgfpathcurveto{\pgfqpoint{-0.018638in}{0.010825in}}{\pgfqpoint{-0.020833in}{0.005525in}}{\pgfqpoint{-0.020833in}{0.000000in}}%
\pgfpathcurveto{\pgfqpoint{-0.020833in}{-0.005525in}}{\pgfqpoint{-0.018638in}{-0.010825in}}{\pgfqpoint{-0.014731in}{-0.014731in}}%
\pgfpathcurveto{\pgfqpoint{-0.010825in}{-0.018638in}}{\pgfqpoint{-0.005525in}{-0.020833in}}{\pgfqpoint{0.000000in}{-0.020833in}}%
\pgfpathclose%
\pgfusepath{stroke,fill}%
}%
\begin{pgfscope}%
\pgfsys@transformshift{1.966424in}{1.096045in}%
\pgfsys@useobject{currentmarker}{}%
\end{pgfscope}%
\end{pgfscope}%
\begin{pgfscope}%
\definecolor{textcolor}{rgb}{0.000000,0.000000,0.000000}%
\pgfsetstrokecolor{textcolor}%
\pgfsetfillcolor{textcolor}%
\pgftext[x=2.139924in,y=1.062308in,left,base]{\color{textcolor}\rmfamily\fontsize{6.940000}{8.328000}\selectfont \(\displaystyle tr(P_1)\omega^{(x)}\)}%
\end{pgfscope}%
\begin{pgfscope}%
\pgfsetrectcap%
\pgfsetroundjoin%
\pgfsetlinewidth{1.505625pt}%
\definecolor{currentstroke}{rgb}{0.000000,0.000000,1.000000}%
\pgfsetstrokecolor{currentstroke}%
\pgfsetdash{}{0pt}%
\pgfpathmoveto{\pgfqpoint{1.870035in}{0.922603in}}%
\pgfpathlineto{\pgfqpoint{2.062813in}{0.922603in}}%
\pgfusepath{stroke}%
\end{pgfscope}%
\begin{pgfscope}%
\pgfsetbuttcap%
\pgfsetroundjoin%
\definecolor{currentfill}{rgb}{0.000000,0.000000,1.000000}%
\pgfsetfillcolor{currentfill}%
\pgfsetlinewidth{1.003750pt}%
\definecolor{currentstroke}{rgb}{0.000000,0.000000,1.000000}%
\pgfsetstrokecolor{currentstroke}%
\pgfsetdash{}{0pt}%
\pgfsys@defobject{currentmarker}{\pgfqpoint{-0.020833in}{-0.020833in}}{\pgfqpoint{0.020833in}{0.020833in}}{%
\pgfpathmoveto{\pgfqpoint{0.000000in}{-0.020833in}}%
\pgfpathcurveto{\pgfqpoint{0.005525in}{-0.020833in}}{\pgfqpoint{0.010825in}{-0.018638in}}{\pgfqpoint{0.014731in}{-0.014731in}}%
\pgfpathcurveto{\pgfqpoint{0.018638in}{-0.010825in}}{\pgfqpoint{0.020833in}{-0.005525in}}{\pgfqpoint{0.020833in}{0.000000in}}%
\pgfpathcurveto{\pgfqpoint{0.020833in}{0.005525in}}{\pgfqpoint{0.018638in}{0.010825in}}{\pgfqpoint{0.014731in}{0.014731in}}%
\pgfpathcurveto{\pgfqpoint{0.010825in}{0.018638in}}{\pgfqpoint{0.005525in}{0.020833in}}{\pgfqpoint{0.000000in}{0.020833in}}%
\pgfpathcurveto{\pgfqpoint{-0.005525in}{0.020833in}}{\pgfqpoint{-0.010825in}{0.018638in}}{\pgfqpoint{-0.014731in}{0.014731in}}%
\pgfpathcurveto{\pgfqpoint{-0.018638in}{0.010825in}}{\pgfqpoint{-0.020833in}{0.005525in}}{\pgfqpoint{-0.020833in}{0.000000in}}%
\pgfpathcurveto{\pgfqpoint{-0.020833in}{-0.005525in}}{\pgfqpoint{-0.018638in}{-0.010825in}}{\pgfqpoint{-0.014731in}{-0.014731in}}%
\pgfpathcurveto{\pgfqpoint{-0.010825in}{-0.018638in}}{\pgfqpoint{-0.005525in}{-0.020833in}}{\pgfqpoint{0.000000in}{-0.020833in}}%
\pgfpathclose%
\pgfusepath{stroke,fill}%
}%
\begin{pgfscope}%
\pgfsys@transformshift{1.966424in}{0.922603in}%
\pgfsys@useobject{currentmarker}{}%
\end{pgfscope}%
\end{pgfscope}%
\begin{pgfscope}%
\definecolor{textcolor}{rgb}{0.000000,0.000000,0.000000}%
\pgfsetstrokecolor{textcolor}%
\pgfsetfillcolor{textcolor}%
\pgftext[x=2.139924in,y=0.888867in,left,base]{\color{textcolor}\rmfamily\fontsize{6.940000}{8.328000}\selectfont \(\displaystyle tr(P_2)(1-\omega^{(x)})\)}%
\end{pgfscope}%
\begin{pgfscope}%
\pgfsetbuttcap%
\pgfsetroundjoin%
\definecolor{currentfill}{rgb}{0.501961,0.501961,0.501961}%
\pgfsetfillcolor{currentfill}%
\pgfsetlinewidth{1.505625pt}%
\definecolor{currentstroke}{rgb}{0.501961,0.501961,0.501961}%
\pgfsetstrokecolor{currentstroke}%
\pgfsetdash{}{0pt}%
\pgfpathmoveto{\pgfqpoint{1.924758in}{0.732487in}}%
\pgfpathlineto{\pgfqpoint{2.008091in}{0.815820in}}%
\pgfpathmoveto{\pgfqpoint{1.924758in}{0.815820in}}%
\pgfpathlineto{\pgfqpoint{2.008091in}{0.732487in}}%
\pgfusepath{stroke,fill}%
\end{pgfscope}%
\begin{pgfscope}%
\definecolor{textcolor}{rgb}{0.000000,0.000000,0.000000}%
\pgfsetstrokecolor{textcolor}%
\pgfsetfillcolor{textcolor}%
\pgftext[x=2.139924in,y=0.748852in,left,base]{\color{textcolor}\rmfamily\fontsize{6.940000}{8.328000}\selectfont Solution limits}%
\end{pgfscope}%
\begin{pgfscope}%
\pgfsetbuttcap%
\pgfsetroundjoin%
\definecolor{currentfill}{rgb}{1.000000,0.000000,0.000000}%
\pgfsetfillcolor{currentfill}%
\pgfsetlinewidth{1.505625pt}%
\definecolor{currentstroke}{rgb}{1.000000,0.000000,0.000000}%
\pgfsetstrokecolor{currentstroke}%
\pgfsetdash{}{0pt}%
\pgfpathmoveto{\pgfqpoint{1.924758in}{0.597873in}}%
\pgfpathlineto{\pgfqpoint{2.008091in}{0.681207in}}%
\pgfpathmoveto{\pgfqpoint{1.924758in}{0.681207in}}%
\pgfpathlineto{\pgfqpoint{2.008091in}{0.597873in}}%
\pgfusepath{stroke,fill}%
\end{pgfscope}%
\begin{pgfscope}%
\definecolor{textcolor}{rgb}{0.000000,0.000000,0.000000}%
\pgfsetstrokecolor{textcolor}%
\pgfsetfillcolor{textcolor}%
\pgftext[x=2.139924in,y=0.614238in,left,base]{\color{textcolor}\rmfamily\fontsize{6.940000}{8.328000}\selectfont Approx. solution}%
\end{pgfscope}%
\end{pgfpicture}%
\makeatother%
\endgroup%

%DIFDELCMD <    %%%
\DIFdelendFL \DIFaddbeginFL \includegraphics{images/2_sensors.pdf}
   \DIFaddendFL \end{center}
   \vspace{-15pt}
   \caption{Approximation of $\omega_1$ with \DIFdelbeginFL \DIFdelFL{discretisation }\DIFdelendFL \DIFaddbeginFL \DIFaddFL{discretization }\DIFaddendFL step-size $s=0.1$. Only comparisons between line points are used.}
   \vspace{-\baselineskip}
   \label{fig:2_sensor_sol}
\end{figure}
Consecutive $\omega^{(x)}$ and $\omega^{(x+1)}$ for which list comparisons differ can be used to estimate the true intersection, and $\omega_1$, by
\begin{equation}
   \omega_1 \approx 0.5(\omega^{(x)} + \omega^{(x+1)})\DIFdelbegin \DIFdel{\,}\DIFdelend \DIFaddbegin \enspace\DIFaddend . \label{eqn:secfci_2sen_omega}
\end{equation}
In the case a comparison returns equality, the exact value of $\omega^{(x)}$ can be taken to be $\omega_1$.

The fusion center can then use its values for $\omega_1$ and $\omega_2 = 1-\omega_1$ and the received PHE encryptions of local information vectors and information matrices to compute \eqref{eqn:paillier_ci_cov} and \eqref{eqn:paillier_ci_estimate}.

%  .d8888b.                    8888888888 .d8888b. 8888888
% d88P  Y88b                   888       d88P  Y88b  888
% Y88b.                        888       888    888  888
%  "Y888b.    .d88b.   .d8888b 8888888   888         888
%     "Y88b. d8P  Y8b d88P"    888       888         888
%       "888 88888888 888      888       888    888  888
% Y88b  d88P Y8b.     Y88b.    888       Y88b  d88P  888
%  "Y8888P"   "Y8888   "Y8888P 888        "Y8888P" 8888888



\section{Multi-sensor Secure Fast Covariance Intersection} \label{sec:multi_secfci}
When computing the SecFCI fusion for $n$ sensors, we solve \eqref{eqn:ci_cov_estimate} homomorphically by computing
\begin{equation}
   \mathcal{E}(\mP^{-1}) = \mathcal{E}(\mP^{-1}_1)^{\omega_1}\cdots\mathcal{E}(\mP^{-1}_n)^{\omega_n} \label{eqn:n_sen_paillier_ci_cov}
\end{equation}
and
\begin{equation}
   \mathcal{E}(\mP^{-1}\mean{\vec{x}}) = \mathcal{E}(\mP^{-1}_1\mean{\vec{x}}_1)^{\omega_1}\cdots\mathcal{E}(\mP^{-1}_n\mean{\vec{x}}_n)^{\omega_n}\DIFdelbegin \DIFdel{\,}\DIFdelend \DIFaddbegin \enspace\DIFaddend . \label{eqn:n_sen_paillier_ci_estimate}
\end{equation}
As with the two sensor case, encoded results from \eqref{eqn:n_sen_paillier_ci_cov} and \eqref{eqn:n_sen_paillier_ci_estimate} contain exactly one multiplication factor to remove and can be decoded exactly with \eqref{eqn:qmn_mult_decode}. Again we are just left with \DIFaddbegin \DIFadd{the task of }\DIFaddend computing the plaintext weights $\omega_1,\dots,\omega_n$.

Our approach to the $n$ sensor case is to solve each $n-1$ conditions in \eqref{eqn:fci_eq} using the two sensor method, and combining partial solutions to compute the final result. When we consider a Euclidean dimension for each $\omega_i$, partial solutions can be considered geometrically as hyperplanes of $n-2$ dimension, over the $n-1$ dimensional solution space given by \eqref{eqn:ci_omega_sum_bound}. 

This can be visualized in the three sensor case, which requires solving partial solutions
\begin{equation}
   \omega_1 \tr(\mP_1) - \omega_2 \tr(\mP_2) = 0,\ \omega_1+\omega_2=1-\omega_3 \label{eqn:3_sensor_partial_sol_1}
\end{equation}
and
\begin{equation}
   \omega_2 \tr(\mP_2) - \omega_3 \tr(\mP_3) = 0,\ \omega_2+\omega_3=1-\omega_1\DIFdelbegin \DIFdel{\,}\DIFdelend \DIFaddbegin \enspace\DIFaddend . \label{eqn:3_sensor_partial_sol_2}
\end{equation}
We can use the two sensor method from section \ref{sec:secfci} to solve \eqref{eqn:3_sensor_partial_sol_1} exactly when $\omega_3=0$, and know that when $\omega_3=1$, then $\omega_1=\omega_2=0$. These two points are enough to define the two-dimensional partial solution \eqref{eqn:3_sensor_partial_sol_1} which can be seen plotted over the possible solution space in Fig. \ref{fig:3_sensor_partial_sol}. Fig. \ref{fig:3_sensor_partial_sols} shows both partial solutions \eqref{eqn:3_sensor_partial_sol_1} and \eqref{eqn:3_sensor_partial_sol_2} plotted over the solution space.
\begin{figure*}[tb]
   \begin{subfigure}[t]{0.3\textwidth}
      \vspace{-5pt}
      \begin{center}
         \DIFdelbeginFL %DIFDELCMD < \input{images/partial_sol1.pgf}
%DIFDELCMD <       %%%
\DIFdelendFL \DIFaddbeginFL \includegraphics{images/partial_sol1.pdf}
      \DIFaddendFL \end{center}
      \vspace{-10pt}
      \caption{Partial solution to \eqref{eqn:3_sensor_partial_sol_1}.}
      \label{fig:3_sensor_partial_sol}
   \end{subfigure}
   ~
   \begin{subfigure}[t]{0.3\textwidth}
      \vspace{-5pt}
      \begin{center}
         \DIFdelbeginFL %DIFDELCMD < %% Creator: Matplotlib, PGF backend
%%
%% To include the figure in your LaTeX document, write
%%   \input{<filename>.pgf}
%%
%% Make sure the required packages are loaded in your preamble
%%   \usepackage{pgf}
%%
%% Figures using additional raster images can only be included by \input if
%% they are in the same directory as the main LaTeX file. For loading figures
%% from other directories you can use the `import` package
%%   \usepackage{import}
%% and then include the figures with
%%   \import{<path to file>}{<filename>.pgf}
%%
%% Matplotlib used the following preamble
%%
\begingroup%
\makeatletter%
\begin{pgfpicture}%
\pgfpathrectangle{\pgfpointorigin}{\pgfqpoint{2.330000in}{1.800000in}}%
\pgfusepath{use as bounding box, clip}%
\begin{pgfscope}%
\pgfsetbuttcap%
\pgfsetmiterjoin%
\definecolor{currentfill}{rgb}{1.000000,1.000000,1.000000}%
\pgfsetfillcolor{currentfill}%
\pgfsetlinewidth{0.000000pt}%
\definecolor{currentstroke}{rgb}{1.000000,1.000000,1.000000}%
\pgfsetstrokecolor{currentstroke}%
\pgfsetdash{}{0pt}%
\pgfpathmoveto{\pgfqpoint{0.000000in}{0.000000in}}%
\pgfpathlineto{\pgfqpoint{2.330000in}{0.000000in}}%
\pgfpathlineto{\pgfqpoint{2.330000in}{1.800000in}}%
\pgfpathlineto{\pgfqpoint{0.000000in}{1.800000in}}%
\pgfpathclose%
\pgfusepath{fill}%
\end{pgfscope}%
\begin{pgfscope}%
\pgfsetbuttcap%
\pgfsetmiterjoin%
\definecolor{currentfill}{rgb}{1.000000,1.000000,1.000000}%
\pgfsetfillcolor{currentfill}%
\pgfsetlinewidth{0.000000pt}%
\definecolor{currentstroke}{rgb}{0.000000,0.000000,0.000000}%
\pgfsetstrokecolor{currentstroke}%
\pgfsetstrokeopacity{0.000000}%
\pgfsetdash{}{0pt}%
\pgfpathmoveto{\pgfqpoint{0.291250in}{0.198000in}}%
\pgfpathlineto{\pgfqpoint{2.097000in}{0.198000in}}%
\pgfpathlineto{\pgfqpoint{2.097000in}{1.584000in}}%
\pgfpathlineto{\pgfqpoint{0.291250in}{1.584000in}}%
\pgfpathclose%
\pgfusepath{fill}%
\end{pgfscope}%
\begin{pgfscope}%
\pgfsetbuttcap%
\pgfsetmiterjoin%
\definecolor{currentfill}{rgb}{0.950000,0.950000,0.950000}%
\pgfsetfillcolor{currentfill}%
\pgfsetfillopacity{0.500000}%
\pgfsetlinewidth{1.003750pt}%
\definecolor{currentstroke}{rgb}{0.950000,0.950000,0.950000}%
\pgfsetstrokecolor{currentstroke}%
\pgfsetstrokeopacity{0.500000}%
\pgfsetdash{}{0pt}%
\pgfpathmoveto{\pgfqpoint{0.833658in}{0.840538in}}%
\pgfpathlineto{\pgfqpoint{1.632216in}{0.828086in}}%
\pgfpathlineto{\pgfqpoint{1.652621in}{1.351825in}}%
\pgfpathlineto{\pgfqpoint{0.814720in}{1.363606in}}%
\pgfusepath{stroke,fill}%
\end{pgfscope}%
\begin{pgfscope}%
\pgfsetbuttcap%
\pgfsetmiterjoin%
\definecolor{currentfill}{rgb}{0.900000,0.900000,0.900000}%
\pgfsetfillcolor{currentfill}%
\pgfsetfillopacity{0.500000}%
\pgfsetlinewidth{1.003750pt}%
\definecolor{currentstroke}{rgb}{0.900000,0.900000,0.900000}%
\pgfsetstrokecolor{currentstroke}%
\pgfsetstrokeopacity{0.500000}%
\pgfsetdash{}{0pt}%
\pgfpathmoveto{\pgfqpoint{0.833658in}{0.840538in}}%
\pgfpathlineto{\pgfqpoint{0.776198in}{0.459248in}}%
\pgfpathlineto{\pgfqpoint{0.752789in}{1.001646in}}%
\pgfpathlineto{\pgfqpoint{0.814720in}{1.363606in}}%
\pgfusepath{stroke,fill}%
\end{pgfscope}%
\begin{pgfscope}%
\pgfsetbuttcap%
\pgfsetmiterjoin%
\definecolor{currentfill}{rgb}{0.925000,0.925000,0.925000}%
\pgfsetfillcolor{currentfill}%
\pgfsetfillopacity{0.500000}%
\pgfsetlinewidth{1.003750pt}%
\definecolor{currentstroke}{rgb}{0.925000,0.925000,0.925000}%
\pgfsetstrokecolor{currentstroke}%
\pgfsetstrokeopacity{0.500000}%
\pgfsetdash{}{0pt}%
\pgfpathmoveto{\pgfqpoint{0.833658in}{0.840538in}}%
\pgfpathlineto{\pgfqpoint{0.776198in}{0.459248in}}%
\pgfpathlineto{\pgfqpoint{1.632042in}{0.444942in}}%
\pgfpathlineto{\pgfqpoint{1.632216in}{0.828086in}}%
\pgfusepath{stroke,fill}%
\end{pgfscope}%
\begin{pgfscope}%
\pgfsetrectcap%
\pgfsetroundjoin%
\pgfsetlinewidth{0.803000pt}%
\definecolor{currentstroke}{rgb}{0.000000,0.000000,0.000000}%
\pgfsetstrokecolor{currentstroke}%
\pgfsetdash{}{0pt}%
\pgfpathmoveto{\pgfqpoint{1.632216in}{0.828086in}}%
\pgfpathlineto{\pgfqpoint{1.632042in}{0.444942in}}%
\pgfusepath{stroke}%
\end{pgfscope}%
\begin{pgfscope}%
\definecolor{textcolor}{rgb}{0.000000,0.000000,0.000000}%
\pgfsetstrokecolor{textcolor}%
\pgfsetfillcolor{textcolor}%
\pgftext[x=1.998003in,y=0.355322in,left,base,rotate=89.973991]{\color{textcolor}\rmfamily\fontsize{10.000000}{12.000000}\selectfont \(\displaystyle \omega_1\)}%
\end{pgfscope}%
\begin{pgfscope}%
\pgfsetbuttcap%
\pgfsetroundjoin%
\pgfsetlinewidth{0.803000pt}%
\definecolor{currentstroke}{rgb}{0.690196,0.690196,0.690196}%
\pgfsetstrokecolor{currentstroke}%
\pgfsetdash{}{0pt}%
\pgfpathmoveto{\pgfqpoint{1.632205in}{0.805242in}}%
\pgfpathlineto{\pgfqpoint{0.830231in}{0.817802in}}%
\pgfpathlineto{\pgfqpoint{0.811039in}{1.342094in}}%
\pgfusepath{stroke}%
\end{pgfscope}%
\begin{pgfscope}%
\pgfsetbuttcap%
\pgfsetroundjoin%
\pgfsetlinewidth{0.803000pt}%
\definecolor{currentstroke}{rgb}{0.690196,0.690196,0.690196}%
\pgfsetstrokecolor{currentstroke}%
\pgfsetdash{}{0pt}%
\pgfpathmoveto{\pgfqpoint{1.632132in}{0.643163in}}%
\pgfpathlineto{\pgfqpoint{0.805922in}{0.656494in}}%
\pgfpathlineto{\pgfqpoint{0.784881in}{1.189212in}}%
\pgfusepath{stroke}%
\end{pgfscope}%
\begin{pgfscope}%
\pgfsetbuttcap%
\pgfsetroundjoin%
\pgfsetlinewidth{0.803000pt}%
\definecolor{currentstroke}{rgb}{0.690196,0.690196,0.690196}%
\pgfsetstrokecolor{currentstroke}%
\pgfsetdash{}{0pt}%
\pgfpathmoveto{\pgfqpoint{1.632054in}{0.470958in}}%
\pgfpathlineto{\pgfqpoint{0.780099in}{0.485134in}}%
\pgfpathlineto{\pgfqpoint{0.757007in}{1.026301in}}%
\pgfusepath{stroke}%
\end{pgfscope}%
\begin{pgfscope}%
\pgfsetrectcap%
\pgfsetroundjoin%
\pgfsetlinewidth{0.803000pt}%
\definecolor{currentstroke}{rgb}{0.000000,0.000000,0.000000}%
\pgfsetstrokecolor{currentstroke}%
\pgfsetdash{}{0pt}%
\pgfpathmoveto{\pgfqpoint{1.625775in}{0.805343in}}%
\pgfpathlineto{\pgfqpoint{1.645068in}{0.805041in}}%
\pgfusepath{stroke}%
\end{pgfscope}%
\begin{pgfscope}%
\definecolor{textcolor}{rgb}{0.000000,0.000000,0.000000}%
\pgfsetstrokecolor{textcolor}%
\pgfsetfillcolor{textcolor}%
\pgftext[x=1.781636in,y=0.707549in,,top]{\color{textcolor}\rmfamily\fontsize{8.330000}{9.996000}\selectfont \(\displaystyle 0.0\)}%
\end{pgfscope}%
\begin{pgfscope}%
\pgfsetrectcap%
\pgfsetroundjoin%
\pgfsetlinewidth{0.803000pt}%
\definecolor{currentstroke}{rgb}{0.000000,0.000000,0.000000}%
\pgfsetstrokecolor{currentstroke}%
\pgfsetdash{}{0pt}%
\pgfpathmoveto{\pgfqpoint{1.625506in}{0.643270in}}%
\pgfpathlineto{\pgfqpoint{1.645384in}{0.642949in}}%
\pgfusepath{stroke}%
\end{pgfscope}%
\begin{pgfscope}%
\definecolor{textcolor}{rgb}{0.000000,0.000000,0.000000}%
\pgfsetstrokecolor{textcolor}%
\pgfsetfillcolor{textcolor}%
\pgftext[x=1.786050in,y=0.543964in,,top]{\color{textcolor}\rmfamily\fontsize{8.330000}{9.996000}\selectfont \(\displaystyle 0.5\)}%
\end{pgfscope}%
\begin{pgfscope}%
\pgfsetrectcap%
\pgfsetroundjoin%
\pgfsetlinewidth{0.803000pt}%
\definecolor{currentstroke}{rgb}{0.000000,0.000000,0.000000}%
\pgfsetstrokecolor{currentstroke}%
\pgfsetdash{}{0pt}%
\pgfpathmoveto{\pgfqpoint{1.625221in}{0.471072in}}%
\pgfpathlineto{\pgfqpoint{1.645720in}{0.470731in}}%
\pgfusepath{stroke}%
\end{pgfscope}%
\begin{pgfscope}%
\definecolor{textcolor}{rgb}{0.000000,0.000000,0.000000}%
\pgfsetstrokecolor{textcolor}%
\pgfsetfillcolor{textcolor}%
\pgftext[x=1.790737in,y=0.370250in,,top]{\color{textcolor}\rmfamily\fontsize{8.330000}{9.996000}\selectfont \(\displaystyle 1.0\)}%
\end{pgfscope}%
\begin{pgfscope}%
\pgfsetrectcap%
\pgfsetroundjoin%
\pgfsetlinewidth{0.803000pt}%
\definecolor{currentstroke}{rgb}{0.000000,0.000000,0.000000}%
\pgfsetstrokecolor{currentstroke}%
\pgfsetdash{}{0pt}%
\pgfpathmoveto{\pgfqpoint{0.776198in}{0.459248in}}%
\pgfpathlineto{\pgfqpoint{1.632042in}{0.444942in}}%
\pgfusepath{stroke}%
\end{pgfscope}%
\begin{pgfscope}%
\definecolor{textcolor}{rgb}{0.000000,0.000000,0.000000}%
\pgfsetstrokecolor{textcolor}%
\pgfsetfillcolor{textcolor}%
\pgftext[x=1.115899in,y=0.028675in,left,base,rotate=359.042353]{\color{textcolor}\rmfamily\fontsize{10.000000}{12.000000}\selectfont \(\displaystyle \omega_2\)}%
\end{pgfscope}%
\begin{pgfscope}%
\pgfsetbuttcap%
\pgfsetroundjoin%
\pgfsetlinewidth{0.803000pt}%
\definecolor{currentstroke}{rgb}{0.690196,0.690196,0.690196}%
\pgfsetstrokecolor{currentstroke}%
\pgfsetdash{}{0pt}%
\pgfpathmoveto{\pgfqpoint{0.867919in}{1.362858in}}%
\pgfpathlineto{\pgfqpoint{0.884364in}{0.839748in}}%
\pgfpathlineto{\pgfqpoint{0.830533in}{0.458339in}}%
\pgfusepath{stroke}%
\end{pgfscope}%
\begin{pgfscope}%
\pgfsetbuttcap%
\pgfsetroundjoin%
\pgfsetlinewidth{0.803000pt}%
\definecolor{currentstroke}{rgb}{0.690196,0.690196,0.690196}%
\pgfsetstrokecolor{currentstroke}%
\pgfsetdash{}{0pt}%
\pgfpathmoveto{\pgfqpoint{1.233156in}{1.357723in}}%
\pgfpathlineto{\pgfqpoint{1.232470in}{0.834319in}}%
\pgfpathlineto{\pgfqpoint{1.203583in}{0.452104in}}%
\pgfusepath{stroke}%
\end{pgfscope}%
\begin{pgfscope}%
\pgfsetbuttcap%
\pgfsetroundjoin%
\pgfsetlinewidth{0.803000pt}%
\definecolor{currentstroke}{rgb}{0.690196,0.690196,0.690196}%
\pgfsetstrokecolor{currentstroke}%
\pgfsetdash{}{0pt}%
\pgfpathmoveto{\pgfqpoint{1.599178in}{1.352576in}}%
\pgfpathlineto{\pgfqpoint{1.581287in}{0.828880in}}%
\pgfpathlineto{\pgfqpoint{1.577451in}{0.445854in}}%
\pgfusepath{stroke}%
\end{pgfscope}%
\begin{pgfscope}%
\pgfsetrectcap%
\pgfsetroundjoin%
\pgfsetlinewidth{0.803000pt}%
\definecolor{currentstroke}{rgb}{0.000000,0.000000,0.000000}%
\pgfsetstrokecolor{currentstroke}%
\pgfsetdash{}{0pt}%
\pgfpathmoveto{\pgfqpoint{0.830994in}{0.461608in}}%
\pgfpathlineto{\pgfqpoint{0.829608in}{0.451791in}}%
\pgfusepath{stroke}%
\end{pgfscope}%
\begin{pgfscope}%
\definecolor{textcolor}{rgb}{0.000000,0.000000,0.000000}%
\pgfsetstrokecolor{textcolor}%
\pgfsetfillcolor{textcolor}%
\pgftext[x=0.823164in,y=0.280056in,,top]{\color{textcolor}\rmfamily\fontsize{8.330000}{9.996000}\selectfont \(\displaystyle 0.0\)}%
\end{pgfscope}%
\begin{pgfscope}%
\pgfsetrectcap%
\pgfsetroundjoin%
\pgfsetlinewidth{0.803000pt}%
\definecolor{currentstroke}{rgb}{0.000000,0.000000,0.000000}%
\pgfsetstrokecolor{currentstroke}%
\pgfsetdash{}{0pt}%
\pgfpathmoveto{\pgfqpoint{1.203831in}{0.455379in}}%
\pgfpathlineto{\pgfqpoint{1.203087in}{0.445541in}}%
\pgfusepath{stroke}%
\end{pgfscope}%
\begin{pgfscope}%
\definecolor{textcolor}{rgb}{0.000000,0.000000,0.000000}%
\pgfsetstrokecolor{textcolor}%
\pgfsetfillcolor{textcolor}%
\pgftext[x=1.197754in,y=0.273598in,,top]{\color{textcolor}\rmfamily\fontsize{8.330000}{9.996000}\selectfont \(\displaystyle 0.5\)}%
\end{pgfscope}%
\begin{pgfscope}%
\pgfsetrectcap%
\pgfsetroundjoin%
\pgfsetlinewidth{0.803000pt}%
\definecolor{currentstroke}{rgb}{0.000000,0.000000,0.000000}%
\pgfsetstrokecolor{currentstroke}%
\pgfsetdash{}{0pt}%
\pgfpathmoveto{\pgfqpoint{1.577484in}{0.449137in}}%
\pgfpathlineto{\pgfqpoint{1.577385in}{0.439277in}}%
\pgfusepath{stroke}%
\end{pgfscope}%
\begin{pgfscope}%
\definecolor{textcolor}{rgb}{0.000000,0.000000,0.000000}%
\pgfsetstrokecolor{textcolor}%
\pgfsetfillcolor{textcolor}%
\pgftext[x=1.573168in,y=0.267126in,,top]{\color{textcolor}\rmfamily\fontsize{8.330000}{9.996000}\selectfont \(\displaystyle 1.0\)}%
\end{pgfscope}%
\begin{pgfscope}%
\pgfsetrectcap%
\pgfsetroundjoin%
\pgfsetlinewidth{0.803000pt}%
\definecolor{currentstroke}{rgb}{0.000000,0.000000,0.000000}%
\pgfsetstrokecolor{currentstroke}%
\pgfsetdash{}{0pt}%
\pgfpathmoveto{\pgfqpoint{0.776198in}{0.459248in}}%
\pgfpathlineto{\pgfqpoint{0.752789in}{1.001646in}}%
\pgfusepath{stroke}%
\end{pgfscope}%
\begin{pgfscope}%
\definecolor{textcolor}{rgb}{0.000000,0.000000,0.000000}%
\pgfsetstrokecolor{textcolor}%
\pgfsetfillcolor{textcolor}%
\pgftext[x=0.322348in,y=0.630186in,left,base,rotate=272.471244]{\color{textcolor}\rmfamily\fontsize{10.000000}{12.000000}\selectfont \(\displaystyle \omega_3\)}%
\end{pgfscope}%
\begin{pgfscope}%
\pgfsetbuttcap%
\pgfsetroundjoin%
\pgfsetlinewidth{0.803000pt}%
\definecolor{currentstroke}{rgb}{0.690196,0.690196,0.690196}%
\pgfsetstrokecolor{currentstroke}%
\pgfsetdash{}{0pt}%
\pgfpathmoveto{\pgfqpoint{0.774778in}{0.492134in}}%
\pgfpathlineto{\pgfqpoint{0.832506in}{0.872359in}}%
\pgfpathlineto{\pgfqpoint{1.633457in}{0.859943in}}%
\pgfusepath{stroke}%
\end{pgfscope}%
\begin{pgfscope}%
\pgfsetbuttcap%
\pgfsetroundjoin%
\pgfsetlinewidth{0.803000pt}%
\definecolor{currentstroke}{rgb}{0.690196,0.690196,0.690196}%
\pgfsetstrokecolor{currentstroke}%
\pgfsetdash{}{0pt}%
\pgfpathmoveto{\pgfqpoint{0.764795in}{0.723456in}}%
\pgfpathlineto{\pgfqpoint{0.824416in}{1.095792in}}%
\pgfpathlineto{\pgfqpoint{1.642173in}{1.083652in}}%
\pgfusepath{stroke}%
\end{pgfscope}%
\begin{pgfscope}%
\pgfsetbuttcap%
\pgfsetroundjoin%
\pgfsetlinewidth{0.803000pt}%
\definecolor{currentstroke}{rgb}{0.690196,0.690196,0.690196}%
\pgfsetstrokecolor{currentstroke}%
\pgfsetdash{}{0pt}%
\pgfpathmoveto{\pgfqpoint{0.754352in}{0.965425in}}%
\pgfpathlineto{\pgfqpoint{0.815981in}{1.328791in}}%
\pgfpathlineto{\pgfqpoint{1.651263in}{1.316962in}}%
\pgfusepath{stroke}%
\end{pgfscope}%
\begin{pgfscope}%
\pgfsetrectcap%
\pgfsetroundjoin%
\pgfsetlinewidth{0.803000pt}%
\definecolor{currentstroke}{rgb}{0.000000,0.000000,0.000000}%
\pgfsetstrokecolor{currentstroke}%
\pgfsetdash{}{0pt}%
\pgfpathmoveto{\pgfqpoint{0.775273in}{0.495393in}}%
\pgfpathlineto{\pgfqpoint{0.773787in}{0.485605in}}%
\pgfusepath{stroke}%
\end{pgfscope}%
\begin{pgfscope}%
\definecolor{textcolor}{rgb}{0.000000,0.000000,0.000000}%
\pgfsetstrokecolor{textcolor}%
\pgfsetfillcolor{textcolor}%
\pgftext[x=0.595925in,y=0.415524in,,top]{\color{textcolor}\rmfamily\fontsize{8.330000}{9.996000}\selectfont \(\displaystyle 0.0\)}%
\end{pgfscope}%
\begin{pgfscope}%
\pgfsetrectcap%
\pgfsetroundjoin%
\pgfsetlinewidth{0.803000pt}%
\definecolor{currentstroke}{rgb}{0.000000,0.000000,0.000000}%
\pgfsetstrokecolor{currentstroke}%
\pgfsetdash{}{0pt}%
\pgfpathmoveto{\pgfqpoint{0.765307in}{0.726652in}}%
\pgfpathlineto{\pgfqpoint{0.763769in}{0.717052in}}%
\pgfusepath{stroke}%
\end{pgfscope}%
\begin{pgfscope}%
\definecolor{textcolor}{rgb}{0.000000,0.000000,0.000000}%
\pgfsetstrokecolor{textcolor}%
\pgfsetfillcolor{textcolor}%
\pgftext[x=0.581724in,y=0.648298in,,top]{\color{textcolor}\rmfamily\fontsize{8.330000}{9.996000}\selectfont \(\displaystyle 0.5\)}%
\end{pgfscope}%
\begin{pgfscope}%
\pgfsetrectcap%
\pgfsetroundjoin%
\pgfsetlinewidth{0.803000pt}%
\definecolor{currentstroke}{rgb}{0.000000,0.000000,0.000000}%
\pgfsetstrokecolor{currentstroke}%
\pgfsetdash{}{0pt}%
\pgfpathmoveto{\pgfqpoint{0.754882in}{0.968549in}}%
\pgfpathlineto{\pgfqpoint{0.753290in}{0.959166in}}%
\pgfusepath{stroke}%
\end{pgfscope}%
\begin{pgfscope}%
\definecolor{textcolor}{rgb}{0.000000,0.000000,0.000000}%
\pgfsetstrokecolor{textcolor}%
\pgfsetfillcolor{textcolor}%
\pgftext[x=0.566860in,y=0.891940in,,top]{\color{textcolor}\rmfamily\fontsize{8.330000}{9.996000}\selectfont \(\displaystyle 1.0\)}%
\end{pgfscope}%
\begin{pgfscope}%
\pgfpathrectangle{\pgfqpoint{0.291250in}{0.198000in}}{\pgfqpoint{1.805750in}{1.386000in}}%
\pgfusepath{clip}%
\pgfsetbuttcap%
\pgfsetroundjoin%
\pgfsetlinewidth{1.505625pt}%
\definecolor{currentstroke}{rgb}{0.000000,0.500000,0.000000}%
\pgfsetstrokecolor{currentstroke}%
\pgfsetdash{{5.550000pt}{2.400000pt}}{0.000000pt}%
\pgfpathmoveto{\pgfqpoint{1.410230in}{0.764325in}}%
\pgfpathlineto{\pgfqpoint{0.865586in}{1.306433in}}%
\pgfusepath{stroke}%
\end{pgfscope}%
\begin{pgfscope}%
\pgfpathrectangle{\pgfqpoint{0.291250in}{0.198000in}}{\pgfqpoint{1.805750in}{1.386000in}}%
\pgfusepath{clip}%
\pgfsetbuttcap%
\pgfsetroundjoin%
\pgfsetlinewidth{1.505625pt}%
\definecolor{currentstroke}{rgb}{0.000000,0.000000,1.000000}%
\pgfsetstrokecolor{currentstroke}%
\pgfsetdash{{5.550000pt}{2.400000pt}}{0.000000pt}%
\pgfpathmoveto{\pgfqpoint{1.245356in}{1.058152in}}%
\pgfpathlineto{\pgfqpoint{0.832959in}{0.517056in}}%
\pgfusepath{stroke}%
\end{pgfscope}%
\begin{pgfscope}%
\pgfpathrectangle{\pgfqpoint{0.291250in}{0.198000in}}{\pgfqpoint{1.805750in}{1.386000in}}%
\pgfusepath{clip}%
\pgfsetbuttcap%
\pgfsetroundjoin%
\definecolor{currentfill}{rgb}{0.000000,0.500000,0.000000}%
\pgfsetfillcolor{currentfill}%
\pgfsetlinewidth{1.505625pt}%
\definecolor{currentstroke}{rgb}{0.000000,0.500000,0.000000}%
\pgfsetstrokecolor{currentstroke}%
\pgfsetdash{}{0pt}%
\pgfpathmoveto{\pgfqpoint{1.379173in}{0.733269in}}%
\pgfpathlineto{\pgfqpoint{1.441286in}{0.795382in}}%
\pgfpathmoveto{\pgfqpoint{1.379173in}{0.795382in}}%
\pgfpathlineto{\pgfqpoint{1.441286in}{0.733269in}}%
\pgfusepath{stroke,fill}%
\end{pgfscope}%
\begin{pgfscope}%
\pgfpathrectangle{\pgfqpoint{0.291250in}{0.198000in}}{\pgfqpoint{1.805750in}{1.386000in}}%
\pgfusepath{clip}%
\pgfsetbuttcap%
\pgfsetroundjoin%
\definecolor{currentfill}{rgb}{0.000000,0.500000,0.000000}%
\pgfsetfillcolor{currentfill}%
\pgfsetlinewidth{1.505625pt}%
\definecolor{currentstroke}{rgb}{0.000000,0.500000,0.000000}%
\pgfsetstrokecolor{currentstroke}%
\pgfsetdash{}{0pt}%
\pgfpathmoveto{\pgfqpoint{0.834530in}{1.275376in}}%
\pgfpathlineto{\pgfqpoint{0.896643in}{1.337489in}}%
\pgfpathmoveto{\pgfqpoint{0.834530in}{1.337489in}}%
\pgfpathlineto{\pgfqpoint{0.896643in}{1.275376in}}%
\pgfusepath{stroke,fill}%
\end{pgfscope}%
\begin{pgfscope}%
\pgfpathrectangle{\pgfqpoint{0.291250in}{0.198000in}}{\pgfqpoint{1.805750in}{1.386000in}}%
\pgfusepath{clip}%
\pgfsetbuttcap%
\pgfsetroundjoin%
\definecolor{currentfill}{rgb}{0.596451,0.170415,0.170415}%
\pgfsetfillcolor{currentfill}%
\pgfsetfillopacity{0.500000}%
\pgfsetlinewidth{0.000000pt}%
\definecolor{currentstroke}{rgb}{0.000000,0.000000,0.000000}%
\pgfsetstrokecolor{currentstroke}%
\pgfsetdash{}{0pt}%
\pgfpathmoveto{\pgfqpoint{0.832959in}{0.517056in}}%
\pgfpathlineto{\pgfqpoint{0.865586in}{1.306433in}}%
\pgfpathlineto{\pgfqpoint{1.582151in}{0.837966in}}%
\pgfpathclose%
\pgfusepath{fill}%
\end{pgfscope}%
\begin{pgfscope}%
\pgfpathrectangle{\pgfqpoint{0.291250in}{0.198000in}}{\pgfqpoint{1.805750in}{1.386000in}}%
\pgfusepath{clip}%
\pgfsetbuttcap%
\pgfsetroundjoin%
\definecolor{currentfill}{rgb}{0.000000,0.000000,1.000000}%
\pgfsetfillcolor{currentfill}%
\pgfsetlinewidth{1.505625pt}%
\definecolor{currentstroke}{rgb}{0.000000,0.000000,1.000000}%
\pgfsetstrokecolor{currentstroke}%
\pgfsetdash{}{0pt}%
\pgfpathmoveto{\pgfqpoint{1.214300in}{1.027095in}}%
\pgfpathlineto{\pgfqpoint{1.276413in}{1.089208in}}%
\pgfpathmoveto{\pgfqpoint{1.214300in}{1.089208in}}%
\pgfpathlineto{\pgfqpoint{1.276413in}{1.027095in}}%
\pgfusepath{stroke,fill}%
\end{pgfscope}%
\begin{pgfscope}%
\pgfpathrectangle{\pgfqpoint{0.291250in}{0.198000in}}{\pgfqpoint{1.805750in}{1.386000in}}%
\pgfusepath{clip}%
\pgfsetbuttcap%
\pgfsetroundjoin%
\definecolor{currentfill}{rgb}{0.000000,0.000000,1.000000}%
\pgfsetfillcolor{currentfill}%
\pgfsetlinewidth{1.505625pt}%
\definecolor{currentstroke}{rgb}{0.000000,0.000000,1.000000}%
\pgfsetstrokecolor{currentstroke}%
\pgfsetdash{}{0pt}%
\pgfpathmoveto{\pgfqpoint{0.801903in}{0.485999in}}%
\pgfpathlineto{\pgfqpoint{0.864016in}{0.548112in}}%
\pgfpathmoveto{\pgfqpoint{0.801903in}{0.548112in}}%
\pgfpathlineto{\pgfqpoint{0.864016in}{0.485999in}}%
\pgfusepath{stroke,fill}%
\end{pgfscope}%
\begin{pgfscope}%
\pgfsetbuttcap%
\pgfsetmiterjoin%
\definecolor{currentfill}{rgb}{1.000000,1.000000,1.000000}%
\pgfsetfillcolor{currentfill}%
\pgfsetfillopacity{0.600000}%
\pgfsetlinewidth{1.003750pt}%
\definecolor{currentstroke}{rgb}{0.800000,0.800000,0.800000}%
\pgfsetstrokecolor{currentstroke}%
\pgfsetstrokeopacity{0.600000}%
\pgfsetdash{}{0pt}%
\pgfpathmoveto{\pgfqpoint{0.623559in}{1.227787in}}%
\pgfpathlineto{\pgfqpoint{2.029528in}{1.227787in}}%
\pgfpathquadraticcurveto{\pgfqpoint{2.048806in}{1.227787in}}{\pgfqpoint{2.048806in}{1.247065in}}%
\pgfpathlineto{\pgfqpoint{2.048806in}{1.641268in}}%
\pgfpathquadraticcurveto{\pgfqpoint{2.048806in}{1.660546in}}{\pgfqpoint{2.029528in}{1.660546in}}%
\pgfpathlineto{\pgfqpoint{0.623559in}{1.660546in}}%
\pgfpathquadraticcurveto{\pgfqpoint{0.604281in}{1.660546in}}{\pgfqpoint{0.604281in}{1.641268in}}%
\pgfpathlineto{\pgfqpoint{0.604281in}{1.247065in}}%
\pgfpathquadraticcurveto{\pgfqpoint{0.604281in}{1.227787in}}{\pgfqpoint{0.623559in}{1.227787in}}%
\pgfpathclose%
\pgfusepath{stroke,fill}%
\end{pgfscope}%
\begin{pgfscope}%
\pgfsetbuttcap%
\pgfsetroundjoin%
\definecolor{currentfill}{rgb}{0.700000,0.200000,0.200000}%
\pgfsetfillcolor{currentfill}%
\pgfsetlinewidth{1.003750pt}%
\definecolor{currentstroke}{rgb}{0.700000,0.200000,0.200000}%
\pgfsetstrokecolor{currentstroke}%
\pgfsetdash{}{0pt}%
\pgfsys@defobject{currentmarker}{\pgfqpoint{-0.041667in}{-0.041667in}}{\pgfqpoint{0.041667in}{0.041667in}}{%
\pgfpathmoveto{\pgfqpoint{0.000000in}{-0.041667in}}%
\pgfpathcurveto{\pgfqpoint{0.011050in}{-0.041667in}}{\pgfqpoint{0.021649in}{-0.037276in}}{\pgfqpoint{0.029463in}{-0.029463in}}%
\pgfpathcurveto{\pgfqpoint{0.037276in}{-0.021649in}}{\pgfqpoint{0.041667in}{-0.011050in}}{\pgfqpoint{0.041667in}{0.000000in}}%
\pgfpathcurveto{\pgfqpoint{0.041667in}{0.011050in}}{\pgfqpoint{0.037276in}{0.021649in}}{\pgfqpoint{0.029463in}{0.029463in}}%
\pgfpathcurveto{\pgfqpoint{0.021649in}{0.037276in}}{\pgfqpoint{0.011050in}{0.041667in}}{\pgfqpoint{0.000000in}{0.041667in}}%
\pgfpathcurveto{\pgfqpoint{-0.011050in}{0.041667in}}{\pgfqpoint{-0.021649in}{0.037276in}}{\pgfqpoint{-0.029463in}{0.029463in}}%
\pgfpathcurveto{\pgfqpoint{-0.037276in}{0.021649in}}{\pgfqpoint{-0.041667in}{0.011050in}}{\pgfqpoint{-0.041667in}{0.000000in}}%
\pgfpathcurveto{\pgfqpoint{-0.041667in}{-0.011050in}}{\pgfqpoint{-0.037276in}{-0.021649in}}{\pgfqpoint{-0.029463in}{-0.029463in}}%
\pgfpathcurveto{\pgfqpoint{-0.021649in}{-0.037276in}}{\pgfqpoint{-0.011050in}{-0.041667in}}{\pgfqpoint{0.000000in}{-0.041667in}}%
\pgfpathclose%
\pgfusepath{stroke,fill}%
}%
\begin{pgfscope}%
\pgfsys@transformshift{0.739225in}{1.588211in}%
\pgfsys@useobject{currentmarker}{}%
\end{pgfscope}%
\end{pgfscope}%
\begin{pgfscope}%
\definecolor{textcolor}{rgb}{0.000000,0.000000,0.000000}%
\pgfsetstrokecolor{textcolor}%
\pgfsetfillcolor{textcolor}%
\pgftext[x=0.912725in,y=1.554475in,left,base]{\color{textcolor}\rmfamily\fontsize{6.940000}{8.328000}\selectfont \(\displaystyle \omega_i\) solution space}%
\end{pgfscope}%
\begin{pgfscope}%
\pgfsetbuttcap%
\pgfsetroundjoin%
\pgfsetlinewidth{1.505625pt}%
\definecolor{currentstroke}{rgb}{0.000000,0.500000,0.000000}%
\pgfsetstrokecolor{currentstroke}%
\pgfsetdash{{5.550000pt}{2.400000pt}}{0.000000pt}%
\pgfpathmoveto{\pgfqpoint{0.642837in}{1.453597in}}%
\pgfpathlineto{\pgfqpoint{0.835614in}{1.453597in}}%
\pgfusepath{stroke}%
\end{pgfscope}%
\begin{pgfscope}%
\definecolor{textcolor}{rgb}{0.000000,0.000000,0.000000}%
\pgfsetstrokecolor{textcolor}%
\pgfsetfillcolor{textcolor}%
\pgftext[x=0.912725in,y=1.419861in,left,base]{\color{textcolor}\rmfamily\fontsize{6.940000}{8.328000}\selectfont \(\displaystyle \omega_1\), \(\displaystyle \omega_2\) partial solution}%
\end{pgfscope}%
\begin{pgfscope}%
\pgfsetbuttcap%
\pgfsetroundjoin%
\pgfsetlinewidth{1.505625pt}%
\definecolor{currentstroke}{rgb}{0.000000,0.000000,1.000000}%
\pgfsetstrokecolor{currentstroke}%
\pgfsetdash{{5.550000pt}{2.400000pt}}{0.000000pt}%
\pgfpathmoveto{\pgfqpoint{0.642837in}{1.318983in}}%
\pgfpathlineto{\pgfqpoint{0.835614in}{1.318983in}}%
\pgfusepath{stroke}%
\end{pgfscope}%
\begin{pgfscope}%
\definecolor{textcolor}{rgb}{0.000000,0.000000,0.000000}%
\pgfsetstrokecolor{textcolor}%
\pgfsetfillcolor{textcolor}%
\pgftext[x=0.912725in,y=1.285247in,left,base]{\color{textcolor}\rmfamily\fontsize{6.940000}{8.328000}\selectfont \(\displaystyle \omega_2\), \(\displaystyle \omega_3\) partial solution}%
\end{pgfscope}%
\end{pgfpicture}%
\makeatother%
\endgroup%

%DIFDELCMD <       %%%
\DIFdelendFL \DIFaddbeginFL \includegraphics{images/partial_sols.pdf}
      \DIFaddendFL \end{center}
      \vspace{-10pt}
      \caption{Partial solutions to \eqref{eqn:3_sensor_partial_sol_1} and \eqref{eqn:3_sensor_partial_sol_2}.}
      \label{fig:3_sensor_partial_sols}
   \end{subfigure}
   ~
   \begin{subfigure}[t]{0.3\textwidth}
      \vspace{-5pt}
      \begin{center}
         \DIFdelbeginFL %DIFDELCMD < \input{images/partial_sol_planes.pgf}
%DIFDELCMD <       %%%
\DIFdelendFL \DIFaddbeginFL \includegraphics{images/partial_sol_planes.pdf}
      \DIFaddendFL \end{center}
      \vspace{-10pt}
      \caption{Partial solutions as planes.}
      \label{fig:3sen_planes}
   \end{subfigure}
   \caption{Partial solutions over $\omega_1$, $\omega_2$, and $\omega_3$ solution space.}
   \vspace{-\baselineskip}
   \label{fig:partial_sols_and_planes}
\end{figure*}
The final solution from all partial solutions is computed by finding their intersection. This can be seen in Fig. \ref{fig:3_sensor_partial_sols} as the intersection of the $(\omega_1,\omega_2)$ and $(\omega_2,\omega_3)$ partial solution lines.

To simplify computing the partial solution intersection, we define equivalent planes for each of the partial solutions, perpendicular to the solution space, in the form
\begin{equation}
   a_1\omega_1 + a_2\omega_2 +a_3\omega_3 + a_4 = 0\DIFdelbegin \DIFdel{\,}\DIFdelend \DIFaddbegin \enspace\DIFaddend , \label{eqn:3sen_plane_eq}
\end{equation}
and solve the resulting linear system for finding the intersection of all planes and the solution space. This is given by
\begin{equation}
   \begin{bmatrix}
      a_1^{(1)} & a_2^{(1)} & a_3^{(1)} \\
      a_1^{(2)} & a_2^{(2)} & a_3^{(2)} \\
      1 & 1 & 1
   \end{bmatrix}
   \begin{bmatrix}
      \omega_1 \\
      \omega_2 \\
      \omega_3
   \end{bmatrix}
   =
   \begin{bmatrix}
      a_3^{(1)} \\
      a_4^{(2)} \\
      1
   \end{bmatrix}\DIFdelbegin \DIFdel{\,}\DIFdelend \DIFaddbegin \enspace\DIFaddend , \label{eqn:3sen_plane_sol_eq}
\end{equation}
where $a_i^{(j)}$ denotes parameter $i$ of partial solution $j$, and has been shown visually in Fig. \ref{fig:3sen_planes}.

In the $n$ sensor case, we can similarly solve partial solutions by first using the method from section \ref{sec:secfci} to solve equations with two parameters $\omega_k$ and $\omega_{k+1}$ when letting all $\omega_i=0,\ i\neq k,k+1$. For each equation we can then compute remaining partial solution points at $\omega_i=1,\ i\neq k,k+1$ with $\omega_j=0,\ j\neq i$. Perpendicular hyperplanes can then be similarly defined in the form 
\begin{equation}
   a_1\omega_1 + \dots +a_n\omega_n + a_{n+1} = 0\DIFdelbegin \DIFdel{\,}\DIFdelend \DIFaddbegin \enspace\DIFaddend . \label{eqn:nsen_plane_eq}
\end{equation}
Due to their inherent orthogonality, and that all meaningful covariance traces are strictly positive, the $n-1$ partial solution hyperplanes are guaranteed to intersect at exactly one point. The hyperplane intersection results in the linear system 
\begin{equation}
   \begingroup
   \setlength\arraycolsep{2pt}
   \begin{bmatrix}
      a_1^{(1)} & a_2^{(1)} & \cdots & a_{n}^{(1)} \\
      \vdots & \vdots & \ddots & \vdots \\
      a_1^{(n-1)} & a_2^{(n-1)} & \cdots & a_{n}^{(n-1)} \\
      1 & 1 & \cdots & 1
   \end{bmatrix}
   \begin{bmatrix}
      \omega_1 \\
      \vdots \\
      \omega_{n-1} \\
      \omega_{n}
   \end{bmatrix}
   =
   \begin{bmatrix}
      a_{n+1}^{(1)} \\
      \vdots \\
      a_{n+1}^{(n-1)} \\
      1
   \end{bmatrix}\DIFdelbegin \DIFdel{\,}\DIFdelend \DIFaddbegin \enspace\DIFaddend , \label{eqn:hyperplane_sol_eq}
   \endgroup
\end{equation}
and gives the solution to the SecFCI $\omega_i$ weights.

As all $O(n\log{p})$ ORE comparisons are done between sequential sensors $i$ and $i+1$, $L$ and $R$ ORE encryptions can be used to the same effect as for the two sensor case. The ORE ordered list sent from each sensor $i$ is given by
\begin{equation}
   \DIFdelbegin %DIFDELCMD < \begin{aligned} \label{eqn:sensor_lists}
%DIFDELCMD <       &[\mathcal{E}^L_{ORE}(\omega^{(1)}\tr(\mP_i)),\dots,\mathcal{E}^L_{ORE}(\omega^{(p)}\tr(\mP_i))],\,i\text{ odd} \\
%DIFDELCMD <       &[\mathcal{E}^R_{ORE}(\omega^{(1)}\tr(\mP_i)),\dots,\mathcal{E}^R_{ORE}(\omega^{(p)}\tr(\mP_i))],\,i\text{ even}.
%DIFDELCMD <    \end{aligned}%%%
\DIFdelend \DIFaddbegin \begin{aligned} \label{eqn:sensor_lists}
      &[\mathcal{E}^L_{ORE}(\omega^{(1)}\tr(\mP_i)),\dots,\mathcal{E}^L_{ORE}(\omega^{(p)}\tr(\mP_i))],\,i\text{ odd} \\
      &[\mathcal{E}^R_{ORE}(\omega^{(1)}\tr(\mP_i)),\dots,\mathcal{E}^R_{ORE}(\omega^{(p)}\tr(\mP_i))],\,i\text{ even}\enspace.
   \end{aligned}\DIFaddend 
\end{equation}
When combining \eqref{eqn:sensor_lists} with PHE encryptions of local information vectors and information matrices, SecFCI can be computed entirely homomorphically by \eqref{eqn:n_sen_paillier_ci_cov} and \eqref{eqn:n_sen_paillier_ci_estimate}.

Briefly considering the security of our scheme, we note that any leaked information from ORE lists \eqref{eqn:sensor_lists}, as described in \cite{chenettePracticalOrderRevealingEncryption2016}, can be considered a subset of knowing the estimated fusion weights $\omega_1,\dots,\omega_n$, which specify relative sizes of sensor covariance traces, and we already consider public. Thus only IND-CPA and IND-OCPA (after accounting for leakage through public weights) encryptions are made available to the fusion center.

\subsection{Computational Complexity} \label{subsec:complexity}
Given the state estimates and estimate errors at each sensor, we wish to show the computational complexity of the SecFCI algorithm for the $n$ sensor case. We will assume that both Lewi ORE and Paillier PHE schemes use the same length security parameter (and equivalently key size), such that $\lambda_{Lewi} = \lambda_{Paillier} = \log{N}$, where $\lambda_{s}$ represents encryption scheme $s$'s security parameter, and $N$ the Paillier modulus and encryptable integer limit. We also note the distinction between floating-point or small integer operations, which are typically treated as having $O(1)$ runtime, and large integer operations whose complexities are dependent on bit length. While architectures exist for speeding up encryption operations \cite{gueronIntelAdvancedEncryption2010}, we consider software implementations and treat large integer operations in terms of bit operations explicitly.

From \cite{paillierPublicKeyCryptosystemsBased1999,lewiOrderRevealingEncryptionNew2016}, and the assumptions made above, we have summarized the operation complixites of the two schemes in Table \ref{tab:complex_ops}.
\begin{table}[tb]
   %\vspace{-5pt}
   \centering
   \caption{Computation complexity of encryption operations.}
   \label{tab:complex_ops}
   \begin{tabular}{ |c|c| }
      \hline
      \textbf{Operation} & \textbf{Complexity} \\ 
      \hline
      Paillier enc. & \DIFdelbeginFL \DIFdelFL{$O(\log{N}\log^2{N^2})$ }\DIFdelendFL \DIFaddbeginFL \DIFaddFL{$O(\log^3{N})$ }\DIFaddendFL \\ 
      Paillier dec. & \DIFdelbeginFL \DIFdelFL{$O(\log{N}\log^2{N^2})$ }\DIFdelendFL \DIFaddbeginFL \DIFaddFL{$O(\log^3{N})$ }\DIFaddendFL \\ 
      Paillier add. & \DIFdelbeginFL \DIFdelFL{$O(\log^2{N^2})$ }\DIFdelendFL \DIFaddbeginFL \DIFaddFL{$O(\log^2{N})$ }\DIFaddendFL \\ 
      Paillier scalar mult. & \DIFdelbeginFL \DIFdelFL{$O(\log{N}\log^2{N^2})$ }\DIFdelendFL \DIFaddbeginFL \DIFaddFL{$O(\log^3{N})$ }\DIFaddendFL \\ 
      Lewi $L$ enc. & \DIFdelbeginFL \DIFdelFL{$O(\log^2{N^2})$ }\DIFdelendFL \DIFaddbeginFL \DIFaddFL{$O(\log^2{N})$ }\DIFaddendFL \\ 
      Lewi $R$ enc. & \DIFdelbeginFL \DIFdelFL{$O(\log^2{N^2})$ }\DIFdelendFL \DIFaddbeginFL \DIFaddFL{$O(\log^2{N})$ }\DIFaddendFL \\ 
      Lewi comp. & \DIFdelbeginFL \DIFdelFL{$O(\log^2{N^2})$ }\DIFdelendFL \DIFaddbeginFL \DIFaddFL{$O(\log^2{N})$ }\DIFaddendFL \\ 
      \hline
   \end{tabular}
   \vspace{-5pt}
   %\vspace{-\baselineskip}
\end{table}
In contrast to some current FHE schemes, these operations are of a much lower complexity than \cite{vandijkFullyHomomorphicEncryption2010a}, which has complexity $O(\lambda^{10})$ for integer operations, and \cite{stehleFasterFullyHomomorphic2010}, which computes single bit operations in $O(\lambda^{3.5})$ adding significant overhead for integer arithmetic.

Finally, applying the operations from Table \ref{tab:complex_ops} to the SecFCI algorithm, we summarize the total complexity of SecFCI at the sensors and the fusion center in Table \ref{tab:complex}, with the unencrypted complexities of FCI shown for reference. 
\begin{table}[tb]
   %\vspace{-5pt}
   \centering
   \caption{Computation complexity at sensors and fusion center.}
   \label{tab:complex}
   \begin{tabular}{ |c|c|c| }
      \hline
       & \textbf{FCI} & \textbf{SecFCI} \\ 
      \hline
      Sensors & $O(1)$ & \DIFdelbeginFL \DIFdelFL{$O\left((p + \log{N})\log^2{N^2}\right)$ }\DIFdelendFL \DIFaddbeginFL \DIFaddFL{$O\left(p\log^2{N} + \log^3{N}\right)$ }\DIFaddendFL \\ 
      Fusion & $O(n^3)$ & \DIFdelbeginFL \DIFdelFL{$O\left((\log{p} + \log{N})n\log^2{N^2} + n^3\right)$ }\DIFdelendFL \DIFaddbeginFL \DIFaddFL{$O\left(n\log{p}\log^2{N} + n\log^3{N} + n^3\right)$ }\DIFaddendFL \\ 
      \hline
   \end{tabular}
   %\vspace{-5pt}
   %\vspace{-\baselineskip}
\end{table}

% 8888888b.                            888 888
% 888   Y88b                           888 888
% 888    888                           888 888
% 888   d88P .d88b.  .d8888b  888  888 888 888888 .d8888b
% 8888888P" d8P  Y8b 88K      888  888 888 888    88K
% 888 T88b  88888888 "Y8888b. 888  888 888 888    "Y8888b.
% 888  T88b Y8b.          X88 Y88b 888 888 Y88b.       X88
% 888   T88b "Y8888   88888P'  "Y88888 888  "Y888  88888P'



\section{Simulation Results} \label{sec:results}
We have implemented a simulation to demonstrate the accuracy of SecFCI approximating FCI. Three sensors independently measure a constant-speed linear process and simultaneously run a Kalman filter on their measurements. Estimates are sent both encrypted and unencrypted to a fusion center which computes the SecFCI and FCI fusions on the received data respectively. Encrypted estimates are comprised of PHE encryptions of the information vector and information matrix, $\mathcal{E}(\mP^{-1}_i\mean{\vec{x}}_i)$ and $\mathcal{E}(\mP^{-1}_i)$, in addition to the ORE list given by \eqref{eqn:sensor_lists} with discretization step $s=0.1$. Unencrypted estimates consist of the state estimate $\mean{\vec{x}}_i$ and covariance $\mP_i$. The trajectory and fused estimates are shown in Fig. \ref{fig:fci_secfci_traj}.
\begin{figure}[tb]
   %\vspace{-5pt}
   \vspace{-10pt}
   \begin{center}
      \DIFdelbeginFL %DIFDELCMD < %% Creator: Matplotlib, PGF backend
%%
%% To include the figure in your LaTeX document, write
%%   \input{<filename>.pgf}
%%
%% Make sure the required packages are loaded in your preamble
%%   \usepackage{pgf}
%%
%% Figures using additional raster images can only be included by \input if
%% they are in the same directory as the main LaTeX file. For loading figures
%% from other directories you can use the `import` package
%%   \usepackage{import}
%% and then include the figures with
%%   \import{<path to file>}{<filename>.pgf}
%%
%% Matplotlib used the following preamble
%%
\begingroup%
\makeatletter%
\begin{pgfpicture}%
\pgfpathrectangle{\pgfpointorigin}{\pgfqpoint{3.200000in}{1.800000in}}%
\pgfusepath{use as bounding box, clip}%
\begin{pgfscope}%
\pgfsetbuttcap%
\pgfsetmiterjoin%
\definecolor{currentfill}{rgb}{1.000000,1.000000,1.000000}%
\pgfsetfillcolor{currentfill}%
\pgfsetlinewidth{0.000000pt}%
\definecolor{currentstroke}{rgb}{1.000000,1.000000,1.000000}%
\pgfsetstrokecolor{currentstroke}%
\pgfsetdash{}{0pt}%
\pgfpathmoveto{\pgfqpoint{0.000000in}{0.000000in}}%
\pgfpathlineto{\pgfqpoint{3.200000in}{0.000000in}}%
\pgfpathlineto{\pgfqpoint{3.200000in}{1.800000in}}%
\pgfpathlineto{\pgfqpoint{0.000000in}{1.800000in}}%
\pgfpathclose%
\pgfusepath{fill}%
\end{pgfscope}%
\begin{pgfscope}%
\pgfsetbuttcap%
\pgfsetmiterjoin%
\definecolor{currentfill}{rgb}{1.000000,1.000000,1.000000}%
\pgfsetfillcolor{currentfill}%
\pgfsetlinewidth{0.000000pt}%
\definecolor{currentstroke}{rgb}{0.000000,0.000000,0.000000}%
\pgfsetstrokecolor{currentstroke}%
\pgfsetstrokeopacity{0.000000}%
\pgfsetdash{}{0pt}%
\pgfpathmoveto{\pgfqpoint{0.644217in}{0.500309in}}%
\pgfpathlineto{\pgfqpoint{3.050000in}{0.500309in}}%
\pgfpathlineto{\pgfqpoint{3.050000in}{1.650000in}}%
\pgfpathlineto{\pgfqpoint{0.644217in}{1.650000in}}%
\pgfpathclose%
\pgfusepath{fill}%
\end{pgfscope}%
\begin{pgfscope}%
\pgfpathrectangle{\pgfqpoint{0.644217in}{0.500309in}}{\pgfqpoint{2.405783in}{1.149691in}}%
\pgfusepath{clip}%
\pgfsetbuttcap%
\pgfsetroundjoin%
\definecolor{currentfill}{rgb}{1.000000,0.000000,0.000000}%
\pgfsetfillcolor{currentfill}%
\pgfsetlinewidth{1.003750pt}%
\definecolor{currentstroke}{rgb}{1.000000,0.000000,0.000000}%
\pgfsetstrokecolor{currentstroke}%
\pgfsetdash{}{0pt}%
\pgfpathmoveto{\pgfqpoint{1.003806in}{0.823773in}}%
\pgfpathcurveto{\pgfqpoint{1.009331in}{0.823773in}}{\pgfqpoint{1.014630in}{0.825968in}}{\pgfqpoint{1.018537in}{0.829875in}}%
\pgfpathcurveto{\pgfqpoint{1.022444in}{0.833782in}}{\pgfqpoint{1.024639in}{0.839081in}}{\pgfqpoint{1.024639in}{0.844606in}}%
\pgfpathcurveto{\pgfqpoint{1.024639in}{0.850131in}}{\pgfqpoint{1.022444in}{0.855431in}}{\pgfqpoint{1.018537in}{0.859338in}}%
\pgfpathcurveto{\pgfqpoint{1.014630in}{0.863245in}}{\pgfqpoint{1.009331in}{0.865440in}}{\pgfqpoint{1.003806in}{0.865440in}}%
\pgfpathcurveto{\pgfqpoint{0.998281in}{0.865440in}}{\pgfqpoint{0.992981in}{0.863245in}}{\pgfqpoint{0.989074in}{0.859338in}}%
\pgfpathcurveto{\pgfqpoint{0.985167in}{0.855431in}}{\pgfqpoint{0.982972in}{0.850131in}}{\pgfqpoint{0.982972in}{0.844606in}}%
\pgfpathcurveto{\pgfqpoint{0.982972in}{0.839081in}}{\pgfqpoint{0.985167in}{0.833782in}}{\pgfqpoint{0.989074in}{0.829875in}}%
\pgfpathcurveto{\pgfqpoint{0.992981in}{0.825968in}}{\pgfqpoint{0.998281in}{0.823773in}}{\pgfqpoint{1.003806in}{0.823773in}}%
\pgfpathclose%
\pgfusepath{stroke,fill}%
\end{pgfscope}%
\begin{pgfscope}%
\pgfpathrectangle{\pgfqpoint{0.644217in}{0.500309in}}{\pgfqpoint{2.405783in}{1.149691in}}%
\pgfusepath{clip}%
\pgfsetbuttcap%
\pgfsetroundjoin%
\definecolor{currentfill}{rgb}{1.000000,0.000000,0.000000}%
\pgfsetfillcolor{currentfill}%
\pgfsetlinewidth{1.003750pt}%
\definecolor{currentstroke}{rgb}{1.000000,0.000000,0.000000}%
\pgfsetstrokecolor{currentstroke}%
\pgfsetdash{}{0pt}%
\pgfpathmoveto{\pgfqpoint{1.054149in}{0.863509in}}%
\pgfpathcurveto{\pgfqpoint{1.059674in}{0.863509in}}{\pgfqpoint{1.064974in}{0.865704in}}{\pgfqpoint{1.068880in}{0.869611in}}%
\pgfpathcurveto{\pgfqpoint{1.072787in}{0.873518in}}{\pgfqpoint{1.074982in}{0.878817in}}{\pgfqpoint{1.074982in}{0.884342in}}%
\pgfpathcurveto{\pgfqpoint{1.074982in}{0.889867in}}{\pgfqpoint{1.072787in}{0.895167in}}{\pgfqpoint{1.068880in}{0.899074in}}%
\pgfpathcurveto{\pgfqpoint{1.064974in}{0.902980in}}{\pgfqpoint{1.059674in}{0.905175in}}{\pgfqpoint{1.054149in}{0.905175in}}%
\pgfpathcurveto{\pgfqpoint{1.048624in}{0.905175in}}{\pgfqpoint{1.043325in}{0.902980in}}{\pgfqpoint{1.039418in}{0.899074in}}%
\pgfpathcurveto{\pgfqpoint{1.035511in}{0.895167in}}{\pgfqpoint{1.033316in}{0.889867in}}{\pgfqpoint{1.033316in}{0.884342in}}%
\pgfpathcurveto{\pgfqpoint{1.033316in}{0.878817in}}{\pgfqpoint{1.035511in}{0.873518in}}{\pgfqpoint{1.039418in}{0.869611in}}%
\pgfpathcurveto{\pgfqpoint{1.043325in}{0.865704in}}{\pgfqpoint{1.048624in}{0.863509in}}{\pgfqpoint{1.054149in}{0.863509in}}%
\pgfpathclose%
\pgfusepath{stroke,fill}%
\end{pgfscope}%
\begin{pgfscope}%
\pgfpathrectangle{\pgfqpoint{0.644217in}{0.500309in}}{\pgfqpoint{2.405783in}{1.149691in}}%
\pgfusepath{clip}%
\pgfsetbuttcap%
\pgfsetroundjoin%
\definecolor{currentfill}{rgb}{1.000000,0.000000,0.000000}%
\pgfsetfillcolor{currentfill}%
\pgfsetlinewidth{1.003750pt}%
\definecolor{currentstroke}{rgb}{1.000000,0.000000,0.000000}%
\pgfsetstrokecolor{currentstroke}%
\pgfsetdash{}{0pt}%
\pgfpathmoveto{\pgfqpoint{1.118734in}{0.904424in}}%
\pgfpathcurveto{\pgfqpoint{1.124259in}{0.904424in}}{\pgfqpoint{1.129559in}{0.906619in}}{\pgfqpoint{1.133466in}{0.910525in}}%
\pgfpathcurveto{\pgfqpoint{1.137372in}{0.914432in}}{\pgfqpoint{1.139568in}{0.919732in}}{\pgfqpoint{1.139568in}{0.925257in}}%
\pgfpathcurveto{\pgfqpoint{1.139568in}{0.930782in}}{\pgfqpoint{1.137372in}{0.936081in}}{\pgfqpoint{1.133466in}{0.939988in}}%
\pgfpathcurveto{\pgfqpoint{1.129559in}{0.943895in}}{\pgfqpoint{1.124259in}{0.946090in}}{\pgfqpoint{1.118734in}{0.946090in}}%
\pgfpathcurveto{\pgfqpoint{1.113209in}{0.946090in}}{\pgfqpoint{1.107910in}{0.943895in}}{\pgfqpoint{1.104003in}{0.939988in}}%
\pgfpathcurveto{\pgfqpoint{1.100096in}{0.936081in}}{\pgfqpoint{1.097901in}{0.930782in}}{\pgfqpoint{1.097901in}{0.925257in}}%
\pgfpathcurveto{\pgfqpoint{1.097901in}{0.919732in}}{\pgfqpoint{1.100096in}{0.914432in}}{\pgfqpoint{1.104003in}{0.910525in}}%
\pgfpathcurveto{\pgfqpoint{1.107910in}{0.906619in}}{\pgfqpoint{1.113209in}{0.904424in}}{\pgfqpoint{1.118734in}{0.904424in}}%
\pgfpathclose%
\pgfusepath{stroke,fill}%
\end{pgfscope}%
\begin{pgfscope}%
\pgfpathrectangle{\pgfqpoint{0.644217in}{0.500309in}}{\pgfqpoint{2.405783in}{1.149691in}}%
\pgfusepath{clip}%
\pgfsetbuttcap%
\pgfsetroundjoin%
\definecolor{currentfill}{rgb}{1.000000,0.000000,0.000000}%
\pgfsetfillcolor{currentfill}%
\pgfsetlinewidth{1.003750pt}%
\definecolor{currentstroke}{rgb}{1.000000,0.000000,0.000000}%
\pgfsetstrokecolor{currentstroke}%
\pgfsetdash{}{0pt}%
\pgfpathmoveto{\pgfqpoint{1.179704in}{0.978402in}}%
\pgfpathcurveto{\pgfqpoint{1.185229in}{0.978402in}}{\pgfqpoint{1.190529in}{0.980597in}}{\pgfqpoint{1.194436in}{0.984504in}}%
\pgfpathcurveto{\pgfqpoint{1.198343in}{0.988411in}}{\pgfqpoint{1.200538in}{0.993710in}}{\pgfqpoint{1.200538in}{0.999235in}}%
\pgfpathcurveto{\pgfqpoint{1.200538in}{1.004760in}}{\pgfqpoint{1.198343in}{1.010060in}}{\pgfqpoint{1.194436in}{1.013967in}}%
\pgfpathcurveto{\pgfqpoint{1.190529in}{1.017873in}}{\pgfqpoint{1.185229in}{1.020068in}}{\pgfqpoint{1.179704in}{1.020068in}}%
\pgfpathcurveto{\pgfqpoint{1.174179in}{1.020068in}}{\pgfqpoint{1.168880in}{1.017873in}}{\pgfqpoint{1.164973in}{1.013967in}}%
\pgfpathcurveto{\pgfqpoint{1.161066in}{1.010060in}}{\pgfqpoint{1.158871in}{1.004760in}}{\pgfqpoint{1.158871in}{0.999235in}}%
\pgfpathcurveto{\pgfqpoint{1.158871in}{0.993710in}}{\pgfqpoint{1.161066in}{0.988411in}}{\pgfqpoint{1.164973in}{0.984504in}}%
\pgfpathcurveto{\pgfqpoint{1.168880in}{0.980597in}}{\pgfqpoint{1.174179in}{0.978402in}}{\pgfqpoint{1.179704in}{0.978402in}}%
\pgfpathclose%
\pgfusepath{stroke,fill}%
\end{pgfscope}%
\begin{pgfscope}%
\pgfpathrectangle{\pgfqpoint{0.644217in}{0.500309in}}{\pgfqpoint{2.405783in}{1.149691in}}%
\pgfusepath{clip}%
\pgfsetbuttcap%
\pgfsetroundjoin%
\definecolor{currentfill}{rgb}{1.000000,0.000000,0.000000}%
\pgfsetfillcolor{currentfill}%
\pgfsetlinewidth{1.003750pt}%
\definecolor{currentstroke}{rgb}{1.000000,0.000000,0.000000}%
\pgfsetstrokecolor{currentstroke}%
\pgfsetdash{}{0pt}%
\pgfpathmoveto{\pgfqpoint{1.263854in}{1.018091in}}%
\pgfpathcurveto{\pgfqpoint{1.269379in}{1.018091in}}{\pgfqpoint{1.274678in}{1.020286in}}{\pgfqpoint{1.278585in}{1.024193in}}%
\pgfpathcurveto{\pgfqpoint{1.282492in}{1.028100in}}{\pgfqpoint{1.284687in}{1.033399in}}{\pgfqpoint{1.284687in}{1.038925in}}%
\pgfpathcurveto{\pgfqpoint{1.284687in}{1.044450in}}{\pgfqpoint{1.282492in}{1.049749in}}{\pgfqpoint{1.278585in}{1.053656in}}%
\pgfpathcurveto{\pgfqpoint{1.274678in}{1.057563in}}{\pgfqpoint{1.269379in}{1.059758in}}{\pgfqpoint{1.263854in}{1.059758in}}%
\pgfpathcurveto{\pgfqpoint{1.258329in}{1.059758in}}{\pgfqpoint{1.253029in}{1.057563in}}{\pgfqpoint{1.249122in}{1.053656in}}%
\pgfpathcurveto{\pgfqpoint{1.245216in}{1.049749in}}{\pgfqpoint{1.243020in}{1.044450in}}{\pgfqpoint{1.243020in}{1.038925in}}%
\pgfpathcurveto{\pgfqpoint{1.243020in}{1.033399in}}{\pgfqpoint{1.245216in}{1.028100in}}{\pgfqpoint{1.249122in}{1.024193in}}%
\pgfpathcurveto{\pgfqpoint{1.253029in}{1.020286in}}{\pgfqpoint{1.258329in}{1.018091in}}{\pgfqpoint{1.263854in}{1.018091in}}%
\pgfpathclose%
\pgfusepath{stroke,fill}%
\end{pgfscope}%
\begin{pgfscope}%
\pgfpathrectangle{\pgfqpoint{0.644217in}{0.500309in}}{\pgfqpoint{2.405783in}{1.149691in}}%
\pgfusepath{clip}%
\pgfsetbuttcap%
\pgfsetroundjoin%
\definecolor{currentfill}{rgb}{1.000000,0.000000,0.000000}%
\pgfsetfillcolor{currentfill}%
\pgfsetlinewidth{1.003750pt}%
\definecolor{currentstroke}{rgb}{1.000000,0.000000,0.000000}%
\pgfsetstrokecolor{currentstroke}%
\pgfsetdash{}{0pt}%
\pgfpathmoveto{\pgfqpoint{1.386939in}{1.025015in}}%
\pgfpathcurveto{\pgfqpoint{1.392464in}{1.025015in}}{\pgfqpoint{1.397764in}{1.027210in}}{\pgfqpoint{1.401671in}{1.031117in}}%
\pgfpathcurveto{\pgfqpoint{1.405577in}{1.035024in}}{\pgfqpoint{1.407772in}{1.040323in}}{\pgfqpoint{1.407772in}{1.045848in}}%
\pgfpathcurveto{\pgfqpoint{1.407772in}{1.051373in}}{\pgfqpoint{1.405577in}{1.056673in}}{\pgfqpoint{1.401671in}{1.060580in}}%
\pgfpathcurveto{\pgfqpoint{1.397764in}{1.064486in}}{\pgfqpoint{1.392464in}{1.066682in}}{\pgfqpoint{1.386939in}{1.066682in}}%
\pgfpathcurveto{\pgfqpoint{1.381414in}{1.066682in}}{\pgfqpoint{1.376115in}{1.064486in}}{\pgfqpoint{1.372208in}{1.060580in}}%
\pgfpathcurveto{\pgfqpoint{1.368301in}{1.056673in}}{\pgfqpoint{1.366106in}{1.051373in}}{\pgfqpoint{1.366106in}{1.045848in}}%
\pgfpathcurveto{\pgfqpoint{1.366106in}{1.040323in}}{\pgfqpoint{1.368301in}{1.035024in}}{\pgfqpoint{1.372208in}{1.031117in}}%
\pgfpathcurveto{\pgfqpoint{1.376115in}{1.027210in}}{\pgfqpoint{1.381414in}{1.025015in}}{\pgfqpoint{1.386939in}{1.025015in}}%
\pgfpathclose%
\pgfusepath{stroke,fill}%
\end{pgfscope}%
\begin{pgfscope}%
\pgfpathrectangle{\pgfqpoint{0.644217in}{0.500309in}}{\pgfqpoint{2.405783in}{1.149691in}}%
\pgfusepath{clip}%
\pgfsetbuttcap%
\pgfsetroundjoin%
\definecolor{currentfill}{rgb}{1.000000,0.000000,0.000000}%
\pgfsetfillcolor{currentfill}%
\pgfsetlinewidth{1.003750pt}%
\definecolor{currentstroke}{rgb}{1.000000,0.000000,0.000000}%
\pgfsetstrokecolor{currentstroke}%
\pgfsetdash{}{0pt}%
\pgfpathmoveto{\pgfqpoint{1.392619in}{1.043462in}}%
\pgfpathcurveto{\pgfqpoint{1.398144in}{1.043462in}}{\pgfqpoint{1.403444in}{1.045657in}}{\pgfqpoint{1.407351in}{1.049564in}}%
\pgfpathcurveto{\pgfqpoint{1.411257in}{1.053471in}}{\pgfqpoint{1.413453in}{1.058771in}}{\pgfqpoint{1.413453in}{1.064296in}}%
\pgfpathcurveto{\pgfqpoint{1.413453in}{1.069821in}}{\pgfqpoint{1.411257in}{1.075120in}}{\pgfqpoint{1.407351in}{1.079027in}}%
\pgfpathcurveto{\pgfqpoint{1.403444in}{1.082934in}}{\pgfqpoint{1.398144in}{1.085129in}}{\pgfqpoint{1.392619in}{1.085129in}}%
\pgfpathcurveto{\pgfqpoint{1.387094in}{1.085129in}}{\pgfqpoint{1.381795in}{1.082934in}}{\pgfqpoint{1.377888in}{1.079027in}}%
\pgfpathcurveto{\pgfqpoint{1.373981in}{1.075120in}}{\pgfqpoint{1.371786in}{1.069821in}}{\pgfqpoint{1.371786in}{1.064296in}}%
\pgfpathcurveto{\pgfqpoint{1.371786in}{1.058771in}}{\pgfqpoint{1.373981in}{1.053471in}}{\pgfqpoint{1.377888in}{1.049564in}}%
\pgfpathcurveto{\pgfqpoint{1.381795in}{1.045657in}}{\pgfqpoint{1.387094in}{1.043462in}}{\pgfqpoint{1.392619in}{1.043462in}}%
\pgfpathclose%
\pgfusepath{stroke,fill}%
\end{pgfscope}%
\begin{pgfscope}%
\pgfpathrectangle{\pgfqpoint{0.644217in}{0.500309in}}{\pgfqpoint{2.405783in}{1.149691in}}%
\pgfusepath{clip}%
\pgfsetbuttcap%
\pgfsetroundjoin%
\definecolor{currentfill}{rgb}{1.000000,0.000000,0.000000}%
\pgfsetfillcolor{currentfill}%
\pgfsetlinewidth{1.003750pt}%
\definecolor{currentstroke}{rgb}{1.000000,0.000000,0.000000}%
\pgfsetstrokecolor{currentstroke}%
\pgfsetdash{}{0pt}%
\pgfpathmoveto{\pgfqpoint{1.562943in}{1.151246in}}%
\pgfpathcurveto{\pgfqpoint{1.568468in}{1.151246in}}{\pgfqpoint{1.573767in}{1.153441in}}{\pgfqpoint{1.577674in}{1.157348in}}%
\pgfpathcurveto{\pgfqpoint{1.581581in}{1.161255in}}{\pgfqpoint{1.583776in}{1.166554in}}{\pgfqpoint{1.583776in}{1.172079in}}%
\pgfpathcurveto{\pgfqpoint{1.583776in}{1.177604in}}{\pgfqpoint{1.581581in}{1.182904in}}{\pgfqpoint{1.577674in}{1.186811in}}%
\pgfpathcurveto{\pgfqpoint{1.573767in}{1.190717in}}{\pgfqpoint{1.568468in}{1.192913in}}{\pgfqpoint{1.562943in}{1.192913in}}%
\pgfpathcurveto{\pgfqpoint{1.557418in}{1.192913in}}{\pgfqpoint{1.552118in}{1.190717in}}{\pgfqpoint{1.548211in}{1.186811in}}%
\pgfpathcurveto{\pgfqpoint{1.544305in}{1.182904in}}{\pgfqpoint{1.542109in}{1.177604in}}{\pgfqpoint{1.542109in}{1.172079in}}%
\pgfpathcurveto{\pgfqpoint{1.542109in}{1.166554in}}{\pgfqpoint{1.544305in}{1.161255in}}{\pgfqpoint{1.548211in}{1.157348in}}%
\pgfpathcurveto{\pgfqpoint{1.552118in}{1.153441in}}{\pgfqpoint{1.557418in}{1.151246in}}{\pgfqpoint{1.562943in}{1.151246in}}%
\pgfpathclose%
\pgfusepath{stroke,fill}%
\end{pgfscope}%
\begin{pgfscope}%
\pgfpathrectangle{\pgfqpoint{0.644217in}{0.500309in}}{\pgfqpoint{2.405783in}{1.149691in}}%
\pgfusepath{clip}%
\pgfsetbuttcap%
\pgfsetroundjoin%
\definecolor{currentfill}{rgb}{1.000000,0.000000,0.000000}%
\pgfsetfillcolor{currentfill}%
\pgfsetlinewidth{1.003750pt}%
\definecolor{currentstroke}{rgb}{1.000000,0.000000,0.000000}%
\pgfsetstrokecolor{currentstroke}%
\pgfsetdash{}{0pt}%
\pgfpathmoveto{\pgfqpoint{1.594124in}{1.217971in}}%
\pgfpathcurveto{\pgfqpoint{1.599650in}{1.217971in}}{\pgfqpoint{1.604949in}{1.220166in}}{\pgfqpoint{1.608856in}{1.224073in}}%
\pgfpathcurveto{\pgfqpoint{1.612763in}{1.227980in}}{\pgfqpoint{1.614958in}{1.233279in}}{\pgfqpoint{1.614958in}{1.238804in}}%
\pgfpathcurveto{\pgfqpoint{1.614958in}{1.244329in}}{\pgfqpoint{1.612763in}{1.249629in}}{\pgfqpoint{1.608856in}{1.253536in}}%
\pgfpathcurveto{\pgfqpoint{1.604949in}{1.257442in}}{\pgfqpoint{1.599650in}{1.259638in}}{\pgfqpoint{1.594124in}{1.259638in}}%
\pgfpathcurveto{\pgfqpoint{1.588599in}{1.259638in}}{\pgfqpoint{1.583300in}{1.257442in}}{\pgfqpoint{1.579393in}{1.253536in}}%
\pgfpathcurveto{\pgfqpoint{1.575486in}{1.249629in}}{\pgfqpoint{1.573291in}{1.244329in}}{\pgfqpoint{1.573291in}{1.238804in}}%
\pgfpathcurveto{\pgfqpoint{1.573291in}{1.233279in}}{\pgfqpoint{1.575486in}{1.227980in}}{\pgfqpoint{1.579393in}{1.224073in}}%
\pgfpathcurveto{\pgfqpoint{1.583300in}{1.220166in}}{\pgfqpoint{1.588599in}{1.217971in}}{\pgfqpoint{1.594124in}{1.217971in}}%
\pgfpathclose%
\pgfusepath{stroke,fill}%
\end{pgfscope}%
\begin{pgfscope}%
\pgfpathrectangle{\pgfqpoint{0.644217in}{0.500309in}}{\pgfqpoint{2.405783in}{1.149691in}}%
\pgfusepath{clip}%
\pgfsetbuttcap%
\pgfsetroundjoin%
\definecolor{currentfill}{rgb}{1.000000,0.000000,0.000000}%
\pgfsetfillcolor{currentfill}%
\pgfsetlinewidth{1.003750pt}%
\definecolor{currentstroke}{rgb}{1.000000,0.000000,0.000000}%
\pgfsetstrokecolor{currentstroke}%
\pgfsetdash{}{0pt}%
\pgfpathmoveto{\pgfqpoint{1.620731in}{1.333998in}}%
\pgfpathcurveto{\pgfqpoint{1.626256in}{1.333998in}}{\pgfqpoint{1.631556in}{1.336193in}}{\pgfqpoint{1.635462in}{1.340100in}}%
\pgfpathcurveto{\pgfqpoint{1.639369in}{1.344007in}}{\pgfqpoint{1.641564in}{1.349306in}}{\pgfqpoint{1.641564in}{1.354831in}}%
\pgfpathcurveto{\pgfqpoint{1.641564in}{1.360356in}}{\pgfqpoint{1.639369in}{1.365656in}}{\pgfqpoint{1.635462in}{1.369563in}}%
\pgfpathcurveto{\pgfqpoint{1.631556in}{1.373469in}}{\pgfqpoint{1.626256in}{1.375665in}}{\pgfqpoint{1.620731in}{1.375665in}}%
\pgfpathcurveto{\pgfqpoint{1.615206in}{1.375665in}}{\pgfqpoint{1.609906in}{1.373469in}}{\pgfqpoint{1.606000in}{1.369563in}}%
\pgfpathcurveto{\pgfqpoint{1.602093in}{1.365656in}}{\pgfqpoint{1.599898in}{1.360356in}}{\pgfqpoint{1.599898in}{1.354831in}}%
\pgfpathcurveto{\pgfqpoint{1.599898in}{1.349306in}}{\pgfqpoint{1.602093in}{1.344007in}}{\pgfqpoint{1.606000in}{1.340100in}}%
\pgfpathcurveto{\pgfqpoint{1.609906in}{1.336193in}}{\pgfqpoint{1.615206in}{1.333998in}}{\pgfqpoint{1.620731in}{1.333998in}}%
\pgfpathclose%
\pgfusepath{stroke,fill}%
\end{pgfscope}%
\begin{pgfscope}%
\pgfpathrectangle{\pgfqpoint{0.644217in}{0.500309in}}{\pgfqpoint{2.405783in}{1.149691in}}%
\pgfusepath{clip}%
\pgfsetbuttcap%
\pgfsetroundjoin%
\definecolor{currentfill}{rgb}{1.000000,0.000000,0.000000}%
\pgfsetfillcolor{currentfill}%
\pgfsetlinewidth{1.003750pt}%
\definecolor{currentstroke}{rgb}{1.000000,0.000000,0.000000}%
\pgfsetstrokecolor{currentstroke}%
\pgfsetdash{}{0pt}%
\pgfpathmoveto{\pgfqpoint{1.892159in}{1.366090in}}%
\pgfpathcurveto{\pgfqpoint{1.897684in}{1.366090in}}{\pgfqpoint{1.902984in}{1.368286in}}{\pgfqpoint{1.906891in}{1.372192in}}%
\pgfpathcurveto{\pgfqpoint{1.910797in}{1.376099in}}{\pgfqpoint{1.912993in}{1.381399in}}{\pgfqpoint{1.912993in}{1.386924in}}%
\pgfpathcurveto{\pgfqpoint{1.912993in}{1.392449in}}{\pgfqpoint{1.910797in}{1.397748in}}{\pgfqpoint{1.906891in}{1.401655in}}%
\pgfpathcurveto{\pgfqpoint{1.902984in}{1.405562in}}{\pgfqpoint{1.897684in}{1.407757in}}{\pgfqpoint{1.892159in}{1.407757in}}%
\pgfpathcurveto{\pgfqpoint{1.886634in}{1.407757in}}{\pgfqpoint{1.881335in}{1.405562in}}{\pgfqpoint{1.877428in}{1.401655in}}%
\pgfpathcurveto{\pgfqpoint{1.873521in}{1.397748in}}{\pgfqpoint{1.871326in}{1.392449in}}{\pgfqpoint{1.871326in}{1.386924in}}%
\pgfpathcurveto{\pgfqpoint{1.871326in}{1.381399in}}{\pgfqpoint{1.873521in}{1.376099in}}{\pgfqpoint{1.877428in}{1.372192in}}%
\pgfpathcurveto{\pgfqpoint{1.881335in}{1.368286in}}{\pgfqpoint{1.886634in}{1.366090in}}{\pgfqpoint{1.892159in}{1.366090in}}%
\pgfpathclose%
\pgfusepath{stroke,fill}%
\end{pgfscope}%
\begin{pgfscope}%
\pgfpathrectangle{\pgfqpoint{0.644217in}{0.500309in}}{\pgfqpoint{2.405783in}{1.149691in}}%
\pgfusepath{clip}%
\pgfsetbuttcap%
\pgfsetroundjoin%
\definecolor{currentfill}{rgb}{1.000000,0.000000,0.000000}%
\pgfsetfillcolor{currentfill}%
\pgfsetlinewidth{1.003750pt}%
\definecolor{currentstroke}{rgb}{1.000000,0.000000,0.000000}%
\pgfsetstrokecolor{currentstroke}%
\pgfsetdash{}{0pt}%
\pgfpathmoveto{\pgfqpoint{2.049721in}{1.372892in}}%
\pgfpathcurveto{\pgfqpoint{2.055246in}{1.372892in}}{\pgfqpoint{2.060545in}{1.375087in}}{\pgfqpoint{2.064452in}{1.378994in}}%
\pgfpathcurveto{\pgfqpoint{2.068359in}{1.382901in}}{\pgfqpoint{2.070554in}{1.388200in}}{\pgfqpoint{2.070554in}{1.393725in}}%
\pgfpathcurveto{\pgfqpoint{2.070554in}{1.399250in}}{\pgfqpoint{2.068359in}{1.404550in}}{\pgfqpoint{2.064452in}{1.408457in}}%
\pgfpathcurveto{\pgfqpoint{2.060545in}{1.412364in}}{\pgfqpoint{2.055246in}{1.414559in}}{\pgfqpoint{2.049721in}{1.414559in}}%
\pgfpathcurveto{\pgfqpoint{2.044196in}{1.414559in}}{\pgfqpoint{2.038896in}{1.412364in}}{\pgfqpoint{2.034989in}{1.408457in}}%
\pgfpathcurveto{\pgfqpoint{2.031082in}{1.404550in}}{\pgfqpoint{2.028887in}{1.399250in}}{\pgfqpoint{2.028887in}{1.393725in}}%
\pgfpathcurveto{\pgfqpoint{2.028887in}{1.388200in}}{\pgfqpoint{2.031082in}{1.382901in}}{\pgfqpoint{2.034989in}{1.378994in}}%
\pgfpathcurveto{\pgfqpoint{2.038896in}{1.375087in}}{\pgfqpoint{2.044196in}{1.372892in}}{\pgfqpoint{2.049721in}{1.372892in}}%
\pgfpathclose%
\pgfusepath{stroke,fill}%
\end{pgfscope}%
\begin{pgfscope}%
\pgfpathrectangle{\pgfqpoint{0.644217in}{0.500309in}}{\pgfqpoint{2.405783in}{1.149691in}}%
\pgfusepath{clip}%
\pgfsetbuttcap%
\pgfsetroundjoin%
\definecolor{currentfill}{rgb}{1.000000,0.000000,0.000000}%
\pgfsetfillcolor{currentfill}%
\pgfsetlinewidth{1.003750pt}%
\definecolor{currentstroke}{rgb}{1.000000,0.000000,0.000000}%
\pgfsetstrokecolor{currentstroke}%
\pgfsetdash{}{0pt}%
\pgfpathmoveto{\pgfqpoint{2.216111in}{1.386895in}}%
\pgfpathcurveto{\pgfqpoint{2.221636in}{1.386895in}}{\pgfqpoint{2.226935in}{1.389090in}}{\pgfqpoint{2.230842in}{1.392997in}}%
\pgfpathcurveto{\pgfqpoint{2.234749in}{1.396904in}}{\pgfqpoint{2.236944in}{1.402203in}}{\pgfqpoint{2.236944in}{1.407728in}}%
\pgfpathcurveto{\pgfqpoint{2.236944in}{1.413253in}}{\pgfqpoint{2.234749in}{1.418553in}}{\pgfqpoint{2.230842in}{1.422460in}}%
\pgfpathcurveto{\pgfqpoint{2.226935in}{1.426367in}}{\pgfqpoint{2.221636in}{1.428562in}}{\pgfqpoint{2.216111in}{1.428562in}}%
\pgfpathcurveto{\pgfqpoint{2.210586in}{1.428562in}}{\pgfqpoint{2.205286in}{1.426367in}}{\pgfqpoint{2.201379in}{1.422460in}}%
\pgfpathcurveto{\pgfqpoint{2.197473in}{1.418553in}}{\pgfqpoint{2.195277in}{1.413253in}}{\pgfqpoint{2.195277in}{1.407728in}}%
\pgfpathcurveto{\pgfqpoint{2.195277in}{1.402203in}}{\pgfqpoint{2.197473in}{1.396904in}}{\pgfqpoint{2.201379in}{1.392997in}}%
\pgfpathcurveto{\pgfqpoint{2.205286in}{1.389090in}}{\pgfqpoint{2.210586in}{1.386895in}}{\pgfqpoint{2.216111in}{1.386895in}}%
\pgfpathclose%
\pgfusepath{stroke,fill}%
\end{pgfscope}%
\begin{pgfscope}%
\pgfpathrectangle{\pgfqpoint{0.644217in}{0.500309in}}{\pgfqpoint{2.405783in}{1.149691in}}%
\pgfusepath{clip}%
\pgfsetbuttcap%
\pgfsetroundjoin%
\definecolor{currentfill}{rgb}{1.000000,0.000000,0.000000}%
\pgfsetfillcolor{currentfill}%
\pgfsetlinewidth{1.003750pt}%
\definecolor{currentstroke}{rgb}{1.000000,0.000000,0.000000}%
\pgfsetstrokecolor{currentstroke}%
\pgfsetdash{}{0pt}%
\pgfpathmoveto{\pgfqpoint{2.421917in}{1.429370in}}%
\pgfpathcurveto{\pgfqpoint{2.427442in}{1.429370in}}{\pgfqpoint{2.432742in}{1.431566in}}{\pgfqpoint{2.436649in}{1.435472in}}%
\pgfpathcurveto{\pgfqpoint{2.440555in}{1.439379in}}{\pgfqpoint{2.442751in}{1.444679in}}{\pgfqpoint{2.442751in}{1.450204in}}%
\pgfpathcurveto{\pgfqpoint{2.442751in}{1.455729in}}{\pgfqpoint{2.440555in}{1.461028in}}{\pgfqpoint{2.436649in}{1.464935in}}%
\pgfpathcurveto{\pgfqpoint{2.432742in}{1.468842in}}{\pgfqpoint{2.427442in}{1.471037in}}{\pgfqpoint{2.421917in}{1.471037in}}%
\pgfpathcurveto{\pgfqpoint{2.416392in}{1.471037in}}{\pgfqpoint{2.411093in}{1.468842in}}{\pgfqpoint{2.407186in}{1.464935in}}%
\pgfpathcurveto{\pgfqpoint{2.403279in}{1.461028in}}{\pgfqpoint{2.401084in}{1.455729in}}{\pgfqpoint{2.401084in}{1.450204in}}%
\pgfpathcurveto{\pgfqpoint{2.401084in}{1.444679in}}{\pgfqpoint{2.403279in}{1.439379in}}{\pgfqpoint{2.407186in}{1.435472in}}%
\pgfpathcurveto{\pgfqpoint{2.411093in}{1.431566in}}{\pgfqpoint{2.416392in}{1.429370in}}{\pgfqpoint{2.421917in}{1.429370in}}%
\pgfpathclose%
\pgfusepath{stroke,fill}%
\end{pgfscope}%
\begin{pgfscope}%
\pgfpathrectangle{\pgfqpoint{0.644217in}{0.500309in}}{\pgfqpoint{2.405783in}{1.149691in}}%
\pgfusepath{clip}%
\pgfsetbuttcap%
\pgfsetroundjoin%
\definecolor{currentfill}{rgb}{1.000000,0.000000,0.000000}%
\pgfsetfillcolor{currentfill}%
\pgfsetlinewidth{1.003750pt}%
\definecolor{currentstroke}{rgb}{1.000000,0.000000,0.000000}%
\pgfsetstrokecolor{currentstroke}%
\pgfsetdash{}{0pt}%
\pgfpathmoveto{\pgfqpoint{2.602884in}{1.432821in}}%
\pgfpathcurveto{\pgfqpoint{2.608409in}{1.432821in}}{\pgfqpoint{2.613708in}{1.435016in}}{\pgfqpoint{2.617615in}{1.438923in}}%
\pgfpathcurveto{\pgfqpoint{2.621522in}{1.442830in}}{\pgfqpoint{2.623717in}{1.448130in}}{\pgfqpoint{2.623717in}{1.453655in}}%
\pgfpathcurveto{\pgfqpoint{2.623717in}{1.459180in}}{\pgfqpoint{2.621522in}{1.464479in}}{\pgfqpoint{2.617615in}{1.468386in}}%
\pgfpathcurveto{\pgfqpoint{2.613708in}{1.472293in}}{\pgfqpoint{2.608409in}{1.474488in}}{\pgfqpoint{2.602884in}{1.474488in}}%
\pgfpathcurveto{\pgfqpoint{2.597359in}{1.474488in}}{\pgfqpoint{2.592059in}{1.472293in}}{\pgfqpoint{2.588152in}{1.468386in}}%
\pgfpathcurveto{\pgfqpoint{2.584246in}{1.464479in}}{\pgfqpoint{2.582051in}{1.459180in}}{\pgfqpoint{2.582051in}{1.453655in}}%
\pgfpathcurveto{\pgfqpoint{2.582051in}{1.448130in}}{\pgfqpoint{2.584246in}{1.442830in}}{\pgfqpoint{2.588152in}{1.438923in}}%
\pgfpathcurveto{\pgfqpoint{2.592059in}{1.435016in}}{\pgfqpoint{2.597359in}{1.432821in}}{\pgfqpoint{2.602884in}{1.432821in}}%
\pgfpathclose%
\pgfusepath{stroke,fill}%
\end{pgfscope}%
\begin{pgfscope}%
\pgfpathrectangle{\pgfqpoint{0.644217in}{0.500309in}}{\pgfqpoint{2.405783in}{1.149691in}}%
\pgfusepath{clip}%
\pgfsetbuttcap%
\pgfsetroundjoin%
\definecolor{currentfill}{rgb}{1.000000,0.000000,0.000000}%
\pgfsetfillcolor{currentfill}%
\pgfsetlinewidth{1.003750pt}%
\definecolor{currentstroke}{rgb}{1.000000,0.000000,0.000000}%
\pgfsetstrokecolor{currentstroke}%
\pgfsetdash{}{0pt}%
\pgfpathmoveto{\pgfqpoint{2.729507in}{1.415176in}}%
\pgfpathcurveto{\pgfqpoint{2.735032in}{1.415176in}}{\pgfqpoint{2.740331in}{1.417371in}}{\pgfqpoint{2.744238in}{1.421278in}}%
\pgfpathcurveto{\pgfqpoint{2.748145in}{1.425184in}}{\pgfqpoint{2.750340in}{1.430484in}}{\pgfqpoint{2.750340in}{1.436009in}}%
\pgfpathcurveto{\pgfqpoint{2.750340in}{1.441534in}}{\pgfqpoint{2.748145in}{1.446834in}}{\pgfqpoint{2.744238in}{1.450740in}}%
\pgfpathcurveto{\pgfqpoint{2.740331in}{1.454647in}}{\pgfqpoint{2.735032in}{1.456842in}}{\pgfqpoint{2.729507in}{1.456842in}}%
\pgfpathcurveto{\pgfqpoint{2.723982in}{1.456842in}}{\pgfqpoint{2.718682in}{1.454647in}}{\pgfqpoint{2.714775in}{1.450740in}}%
\pgfpathcurveto{\pgfqpoint{2.710869in}{1.446834in}}{\pgfqpoint{2.708674in}{1.441534in}}{\pgfqpoint{2.708674in}{1.436009in}}%
\pgfpathcurveto{\pgfqpoint{2.708674in}{1.430484in}}{\pgfqpoint{2.710869in}{1.425184in}}{\pgfqpoint{2.714775in}{1.421278in}}%
\pgfpathcurveto{\pgfqpoint{2.718682in}{1.417371in}}{\pgfqpoint{2.723982in}{1.415176in}}{\pgfqpoint{2.729507in}{1.415176in}}%
\pgfpathclose%
\pgfusepath{stroke,fill}%
\end{pgfscope}%
\begin{pgfscope}%
\pgfpathrectangle{\pgfqpoint{0.644217in}{0.500309in}}{\pgfqpoint{2.405783in}{1.149691in}}%
\pgfusepath{clip}%
\pgfsetbuttcap%
\pgfsetroundjoin%
\definecolor{currentfill}{rgb}{1.000000,0.000000,0.000000}%
\pgfsetfillcolor{currentfill}%
\pgfsetlinewidth{1.003750pt}%
\definecolor{currentstroke}{rgb}{1.000000,0.000000,0.000000}%
\pgfsetstrokecolor{currentstroke}%
\pgfsetdash{}{0pt}%
\pgfpathmoveto{\pgfqpoint{2.870227in}{1.476295in}}%
\pgfpathcurveto{\pgfqpoint{2.875752in}{1.476295in}}{\pgfqpoint{2.881052in}{1.478490in}}{\pgfqpoint{2.884958in}{1.482397in}}%
\pgfpathcurveto{\pgfqpoint{2.888865in}{1.486304in}}{\pgfqpoint{2.891060in}{1.491603in}}{\pgfqpoint{2.891060in}{1.497128in}}%
\pgfpathcurveto{\pgfqpoint{2.891060in}{1.502653in}}{\pgfqpoint{2.888865in}{1.507953in}}{\pgfqpoint{2.884958in}{1.511860in}}%
\pgfpathcurveto{\pgfqpoint{2.881052in}{1.515767in}}{\pgfqpoint{2.875752in}{1.517962in}}{\pgfqpoint{2.870227in}{1.517962in}}%
\pgfpathcurveto{\pgfqpoint{2.864702in}{1.517962in}}{\pgfqpoint{2.859403in}{1.515767in}}{\pgfqpoint{2.855496in}{1.511860in}}%
\pgfpathcurveto{\pgfqpoint{2.851589in}{1.507953in}}{\pgfqpoint{2.849394in}{1.502653in}}{\pgfqpoint{2.849394in}{1.497128in}}%
\pgfpathcurveto{\pgfqpoint{2.849394in}{1.491603in}}{\pgfqpoint{2.851589in}{1.486304in}}{\pgfqpoint{2.855496in}{1.482397in}}%
\pgfpathcurveto{\pgfqpoint{2.859403in}{1.478490in}}{\pgfqpoint{2.864702in}{1.476295in}}{\pgfqpoint{2.870227in}{1.476295in}}%
\pgfpathclose%
\pgfusepath{stroke,fill}%
\end{pgfscope}%
\begin{pgfscope}%
\pgfpathrectangle{\pgfqpoint{0.644217in}{0.500309in}}{\pgfqpoint{2.405783in}{1.149691in}}%
\pgfusepath{clip}%
\pgfsetbuttcap%
\pgfsetroundjoin%
\definecolor{currentfill}{rgb}{0.000000,0.000000,1.000000}%
\pgfsetfillcolor{currentfill}%
\pgfsetlinewidth{1.003750pt}%
\definecolor{currentstroke}{rgb}{0.000000,0.000000,1.000000}%
\pgfsetstrokecolor{currentstroke}%
\pgfsetdash{}{0pt}%
\pgfpathmoveto{\pgfqpoint{1.002226in}{0.825360in}}%
\pgfpathcurveto{\pgfqpoint{1.007751in}{0.825360in}}{\pgfqpoint{1.013050in}{0.827555in}}{\pgfqpoint{1.016957in}{0.831462in}}%
\pgfpathcurveto{\pgfqpoint{1.020864in}{0.835369in}}{\pgfqpoint{1.023059in}{0.840669in}}{\pgfqpoint{1.023059in}{0.846194in}}%
\pgfpathcurveto{\pgfqpoint{1.023059in}{0.851719in}}{\pgfqpoint{1.020864in}{0.857018in}}{\pgfqpoint{1.016957in}{0.860925in}}%
\pgfpathcurveto{\pgfqpoint{1.013050in}{0.864832in}}{\pgfqpoint{1.007751in}{0.867027in}}{\pgfqpoint{1.002226in}{0.867027in}}%
\pgfpathcurveto{\pgfqpoint{0.996701in}{0.867027in}}{\pgfqpoint{0.991401in}{0.864832in}}{\pgfqpoint{0.987494in}{0.860925in}}%
\pgfpathcurveto{\pgfqpoint{0.983587in}{0.857018in}}{\pgfqpoint{0.981392in}{0.851719in}}{\pgfqpoint{0.981392in}{0.846194in}}%
\pgfpathcurveto{\pgfqpoint{0.981392in}{0.840669in}}{\pgfqpoint{0.983587in}{0.835369in}}{\pgfqpoint{0.987494in}{0.831462in}}%
\pgfpathcurveto{\pgfqpoint{0.991401in}{0.827555in}}{\pgfqpoint{0.996701in}{0.825360in}}{\pgfqpoint{1.002226in}{0.825360in}}%
\pgfpathclose%
\pgfusepath{stroke,fill}%
\end{pgfscope}%
\begin{pgfscope}%
\pgfpathrectangle{\pgfqpoint{0.644217in}{0.500309in}}{\pgfqpoint{2.405783in}{1.149691in}}%
\pgfusepath{clip}%
\pgfsetbuttcap%
\pgfsetroundjoin%
\definecolor{currentfill}{rgb}{0.000000,0.000000,1.000000}%
\pgfsetfillcolor{currentfill}%
\pgfsetlinewidth{1.003750pt}%
\definecolor{currentstroke}{rgb}{0.000000,0.000000,1.000000}%
\pgfsetstrokecolor{currentstroke}%
\pgfsetdash{}{0pt}%
\pgfpathmoveto{\pgfqpoint{1.048408in}{0.854146in}}%
\pgfpathcurveto{\pgfqpoint{1.053933in}{0.854146in}}{\pgfqpoint{1.059233in}{0.856342in}}{\pgfqpoint{1.063140in}{0.860248in}}%
\pgfpathcurveto{\pgfqpoint{1.067046in}{0.864155in}}{\pgfqpoint{1.069242in}{0.869455in}}{\pgfqpoint{1.069242in}{0.874980in}}%
\pgfpathcurveto{\pgfqpoint{1.069242in}{0.880505in}}{\pgfqpoint{1.067046in}{0.885804in}}{\pgfqpoint{1.063140in}{0.889711in}}%
\pgfpathcurveto{\pgfqpoint{1.059233in}{0.893618in}}{\pgfqpoint{1.053933in}{0.895813in}}{\pgfqpoint{1.048408in}{0.895813in}}%
\pgfpathcurveto{\pgfqpoint{1.042883in}{0.895813in}}{\pgfqpoint{1.037584in}{0.893618in}}{\pgfqpoint{1.033677in}{0.889711in}}%
\pgfpathcurveto{\pgfqpoint{1.029770in}{0.885804in}}{\pgfqpoint{1.027575in}{0.880505in}}{\pgfqpoint{1.027575in}{0.874980in}}%
\pgfpathcurveto{\pgfqpoint{1.027575in}{0.869455in}}{\pgfqpoint{1.029770in}{0.864155in}}{\pgfqpoint{1.033677in}{0.860248in}}%
\pgfpathcurveto{\pgfqpoint{1.037584in}{0.856342in}}{\pgfqpoint{1.042883in}{0.854146in}}{\pgfqpoint{1.048408in}{0.854146in}}%
\pgfpathclose%
\pgfusepath{stroke,fill}%
\end{pgfscope}%
\begin{pgfscope}%
\pgfpathrectangle{\pgfqpoint{0.644217in}{0.500309in}}{\pgfqpoint{2.405783in}{1.149691in}}%
\pgfusepath{clip}%
\pgfsetbuttcap%
\pgfsetroundjoin%
\definecolor{currentfill}{rgb}{0.000000,0.000000,1.000000}%
\pgfsetfillcolor{currentfill}%
\pgfsetlinewidth{1.003750pt}%
\definecolor{currentstroke}{rgb}{0.000000,0.000000,1.000000}%
\pgfsetstrokecolor{currentstroke}%
\pgfsetdash{}{0pt}%
\pgfpathmoveto{\pgfqpoint{1.111749in}{0.895024in}}%
\pgfpathcurveto{\pgfqpoint{1.117274in}{0.895024in}}{\pgfqpoint{1.122573in}{0.897219in}}{\pgfqpoint{1.126480in}{0.901126in}}%
\pgfpathcurveto{\pgfqpoint{1.130387in}{0.905033in}}{\pgfqpoint{1.132582in}{0.910333in}}{\pgfqpoint{1.132582in}{0.915858in}}%
\pgfpathcurveto{\pgfqpoint{1.132582in}{0.921383in}}{\pgfqpoint{1.130387in}{0.926682in}}{\pgfqpoint{1.126480in}{0.930589in}}%
\pgfpathcurveto{\pgfqpoint{1.122573in}{0.934496in}}{\pgfqpoint{1.117274in}{0.936691in}}{\pgfqpoint{1.111749in}{0.936691in}}%
\pgfpathcurveto{\pgfqpoint{1.106224in}{0.936691in}}{\pgfqpoint{1.100924in}{0.934496in}}{\pgfqpoint{1.097017in}{0.930589in}}%
\pgfpathcurveto{\pgfqpoint{1.093111in}{0.926682in}}{\pgfqpoint{1.090915in}{0.921383in}}{\pgfqpoint{1.090915in}{0.915858in}}%
\pgfpathcurveto{\pgfqpoint{1.090915in}{0.910333in}}{\pgfqpoint{1.093111in}{0.905033in}}{\pgfqpoint{1.097017in}{0.901126in}}%
\pgfpathcurveto{\pgfqpoint{1.100924in}{0.897219in}}{\pgfqpoint{1.106224in}{0.895024in}}{\pgfqpoint{1.111749in}{0.895024in}}%
\pgfpathclose%
\pgfusepath{stroke,fill}%
\end{pgfscope}%
\begin{pgfscope}%
\pgfpathrectangle{\pgfqpoint{0.644217in}{0.500309in}}{\pgfqpoint{2.405783in}{1.149691in}}%
\pgfusepath{clip}%
\pgfsetbuttcap%
\pgfsetroundjoin%
\definecolor{currentfill}{rgb}{0.000000,0.000000,1.000000}%
\pgfsetfillcolor{currentfill}%
\pgfsetlinewidth{1.003750pt}%
\definecolor{currentstroke}{rgb}{0.000000,0.000000,1.000000}%
\pgfsetstrokecolor{currentstroke}%
\pgfsetdash{}{0pt}%
\pgfpathmoveto{\pgfqpoint{1.184294in}{0.972798in}}%
\pgfpathcurveto{\pgfqpoint{1.189819in}{0.972798in}}{\pgfqpoint{1.195118in}{0.974993in}}{\pgfqpoint{1.199025in}{0.978900in}}%
\pgfpathcurveto{\pgfqpoint{1.202932in}{0.982807in}}{\pgfqpoint{1.205127in}{0.988106in}}{\pgfqpoint{1.205127in}{0.993631in}}%
\pgfpathcurveto{\pgfqpoint{1.205127in}{0.999156in}}{\pgfqpoint{1.202932in}{1.004456in}}{\pgfqpoint{1.199025in}{1.008363in}}%
\pgfpathcurveto{\pgfqpoint{1.195118in}{1.012270in}}{\pgfqpoint{1.189819in}{1.014465in}}{\pgfqpoint{1.184294in}{1.014465in}}%
\pgfpathcurveto{\pgfqpoint{1.178769in}{1.014465in}}{\pgfqpoint{1.173469in}{1.012270in}}{\pgfqpoint{1.169562in}{1.008363in}}%
\pgfpathcurveto{\pgfqpoint{1.165656in}{1.004456in}}{\pgfqpoint{1.163460in}{0.999156in}}{\pgfqpoint{1.163460in}{0.993631in}}%
\pgfpathcurveto{\pgfqpoint{1.163460in}{0.988106in}}{\pgfqpoint{1.165656in}{0.982807in}}{\pgfqpoint{1.169562in}{0.978900in}}%
\pgfpathcurveto{\pgfqpoint{1.173469in}{0.974993in}}{\pgfqpoint{1.178769in}{0.972798in}}{\pgfqpoint{1.184294in}{0.972798in}}%
\pgfpathclose%
\pgfusepath{stroke,fill}%
\end{pgfscope}%
\begin{pgfscope}%
\pgfpathrectangle{\pgfqpoint{0.644217in}{0.500309in}}{\pgfqpoint{2.405783in}{1.149691in}}%
\pgfusepath{clip}%
\pgfsetbuttcap%
\pgfsetroundjoin%
\definecolor{currentfill}{rgb}{0.000000,0.000000,1.000000}%
\pgfsetfillcolor{currentfill}%
\pgfsetlinewidth{1.003750pt}%
\definecolor{currentstroke}{rgb}{0.000000,0.000000,1.000000}%
\pgfsetstrokecolor{currentstroke}%
\pgfsetdash{}{0pt}%
\pgfpathmoveto{\pgfqpoint{1.265743in}{1.017980in}}%
\pgfpathcurveto{\pgfqpoint{1.271268in}{1.017980in}}{\pgfqpoint{1.276567in}{1.020175in}}{\pgfqpoint{1.280474in}{1.024082in}}%
\pgfpathcurveto{\pgfqpoint{1.284381in}{1.027989in}}{\pgfqpoint{1.286576in}{1.033288in}}{\pgfqpoint{1.286576in}{1.038814in}}%
\pgfpathcurveto{\pgfqpoint{1.286576in}{1.044339in}}{\pgfqpoint{1.284381in}{1.049638in}}{\pgfqpoint{1.280474in}{1.053545in}}%
\pgfpathcurveto{\pgfqpoint{1.276567in}{1.057452in}}{\pgfqpoint{1.271268in}{1.059647in}}{\pgfqpoint{1.265743in}{1.059647in}}%
\pgfpathcurveto{\pgfqpoint{1.260218in}{1.059647in}}{\pgfqpoint{1.254918in}{1.057452in}}{\pgfqpoint{1.251011in}{1.053545in}}%
\pgfpathcurveto{\pgfqpoint{1.247104in}{1.049638in}}{\pgfqpoint{1.244909in}{1.044339in}}{\pgfqpoint{1.244909in}{1.038814in}}%
\pgfpathcurveto{\pgfqpoint{1.244909in}{1.033288in}}{\pgfqpoint{1.247104in}{1.027989in}}{\pgfqpoint{1.251011in}{1.024082in}}%
\pgfpathcurveto{\pgfqpoint{1.254918in}{1.020175in}}{\pgfqpoint{1.260218in}{1.017980in}}{\pgfqpoint{1.265743in}{1.017980in}}%
\pgfpathclose%
\pgfusepath{stroke,fill}%
\end{pgfscope}%
\begin{pgfscope}%
\pgfpathrectangle{\pgfqpoint{0.644217in}{0.500309in}}{\pgfqpoint{2.405783in}{1.149691in}}%
\pgfusepath{clip}%
\pgfsetbuttcap%
\pgfsetroundjoin%
\definecolor{currentfill}{rgb}{0.000000,0.000000,1.000000}%
\pgfsetfillcolor{currentfill}%
\pgfsetlinewidth{1.003750pt}%
\definecolor{currentstroke}{rgb}{0.000000,0.000000,1.000000}%
\pgfsetstrokecolor{currentstroke}%
\pgfsetdash{}{0pt}%
\pgfpathmoveto{\pgfqpoint{1.396983in}{1.025742in}}%
\pgfpathcurveto{\pgfqpoint{1.402508in}{1.025742in}}{\pgfqpoint{1.407807in}{1.027937in}}{\pgfqpoint{1.411714in}{1.031843in}}%
\pgfpathcurveto{\pgfqpoint{1.415621in}{1.035750in}}{\pgfqpoint{1.417816in}{1.041050in}}{\pgfqpoint{1.417816in}{1.046575in}}%
\pgfpathcurveto{\pgfqpoint{1.417816in}{1.052100in}}{\pgfqpoint{1.415621in}{1.057399in}}{\pgfqpoint{1.411714in}{1.061306in}}%
\pgfpathcurveto{\pgfqpoint{1.407807in}{1.065213in}}{\pgfqpoint{1.402508in}{1.067408in}}{\pgfqpoint{1.396983in}{1.067408in}}%
\pgfpathcurveto{\pgfqpoint{1.391458in}{1.067408in}}{\pgfqpoint{1.386158in}{1.065213in}}{\pgfqpoint{1.382251in}{1.061306in}}%
\pgfpathcurveto{\pgfqpoint{1.378344in}{1.057399in}}{\pgfqpoint{1.376149in}{1.052100in}}{\pgfqpoint{1.376149in}{1.046575in}}%
\pgfpathcurveto{\pgfqpoint{1.376149in}{1.041050in}}{\pgfqpoint{1.378344in}{1.035750in}}{\pgfqpoint{1.382251in}{1.031843in}}%
\pgfpathcurveto{\pgfqpoint{1.386158in}{1.027937in}}{\pgfqpoint{1.391458in}{1.025742in}}{\pgfqpoint{1.396983in}{1.025742in}}%
\pgfpathclose%
\pgfusepath{stroke,fill}%
\end{pgfscope}%
\begin{pgfscope}%
\pgfpathrectangle{\pgfqpoint{0.644217in}{0.500309in}}{\pgfqpoint{2.405783in}{1.149691in}}%
\pgfusepath{clip}%
\pgfsetbuttcap%
\pgfsetroundjoin%
\definecolor{currentfill}{rgb}{0.000000,0.000000,1.000000}%
\pgfsetfillcolor{currentfill}%
\pgfsetlinewidth{1.003750pt}%
\definecolor{currentstroke}{rgb}{0.000000,0.000000,1.000000}%
\pgfsetstrokecolor{currentstroke}%
\pgfsetdash{}{0pt}%
\pgfpathmoveto{\pgfqpoint{1.387694in}{1.041886in}}%
\pgfpathcurveto{\pgfqpoint{1.393219in}{1.041886in}}{\pgfqpoint{1.398519in}{1.044081in}}{\pgfqpoint{1.402426in}{1.047988in}}%
\pgfpathcurveto{\pgfqpoint{1.406333in}{1.051895in}}{\pgfqpoint{1.408528in}{1.057194in}}{\pgfqpoint{1.408528in}{1.062719in}}%
\pgfpathcurveto{\pgfqpoint{1.408528in}{1.068244in}}{\pgfqpoint{1.406333in}{1.073544in}}{\pgfqpoint{1.402426in}{1.077451in}}%
\pgfpathcurveto{\pgfqpoint{1.398519in}{1.081357in}}{\pgfqpoint{1.393219in}{1.083553in}}{\pgfqpoint{1.387694in}{1.083553in}}%
\pgfpathcurveto{\pgfqpoint{1.382169in}{1.083553in}}{\pgfqpoint{1.376870in}{1.081357in}}{\pgfqpoint{1.372963in}{1.077451in}}%
\pgfpathcurveto{\pgfqpoint{1.369056in}{1.073544in}}{\pgfqpoint{1.366861in}{1.068244in}}{\pgfqpoint{1.366861in}{1.062719in}}%
\pgfpathcurveto{\pgfqpoint{1.366861in}{1.057194in}}{\pgfqpoint{1.369056in}{1.051895in}}{\pgfqpoint{1.372963in}{1.047988in}}%
\pgfpathcurveto{\pgfqpoint{1.376870in}{1.044081in}}{\pgfqpoint{1.382169in}{1.041886in}}{\pgfqpoint{1.387694in}{1.041886in}}%
\pgfpathclose%
\pgfusepath{stroke,fill}%
\end{pgfscope}%
\begin{pgfscope}%
\pgfpathrectangle{\pgfqpoint{0.644217in}{0.500309in}}{\pgfqpoint{2.405783in}{1.149691in}}%
\pgfusepath{clip}%
\pgfsetbuttcap%
\pgfsetroundjoin%
\definecolor{currentfill}{rgb}{0.000000,0.000000,1.000000}%
\pgfsetfillcolor{currentfill}%
\pgfsetlinewidth{1.003750pt}%
\definecolor{currentstroke}{rgb}{0.000000,0.000000,1.000000}%
\pgfsetstrokecolor{currentstroke}%
\pgfsetdash{}{0pt}%
\pgfpathmoveto{\pgfqpoint{1.560803in}{1.152352in}}%
\pgfpathcurveto{\pgfqpoint{1.566328in}{1.152352in}}{\pgfqpoint{1.571628in}{1.154547in}}{\pgfqpoint{1.575535in}{1.158454in}}%
\pgfpathcurveto{\pgfqpoint{1.579441in}{1.162360in}}{\pgfqpoint{1.581637in}{1.167660in}}{\pgfqpoint{1.581637in}{1.173185in}}%
\pgfpathcurveto{\pgfqpoint{1.581637in}{1.178710in}}{\pgfqpoint{1.579441in}{1.184010in}}{\pgfqpoint{1.575535in}{1.187916in}}%
\pgfpathcurveto{\pgfqpoint{1.571628in}{1.191823in}}{\pgfqpoint{1.566328in}{1.194018in}}{\pgfqpoint{1.560803in}{1.194018in}}%
\pgfpathcurveto{\pgfqpoint{1.555278in}{1.194018in}}{\pgfqpoint{1.549979in}{1.191823in}}{\pgfqpoint{1.546072in}{1.187916in}}%
\pgfpathcurveto{\pgfqpoint{1.542165in}{1.184010in}}{\pgfqpoint{1.539970in}{1.178710in}}{\pgfqpoint{1.539970in}{1.173185in}}%
\pgfpathcurveto{\pgfqpoint{1.539970in}{1.167660in}}{\pgfqpoint{1.542165in}{1.162360in}}{\pgfqpoint{1.546072in}{1.158454in}}%
\pgfpathcurveto{\pgfqpoint{1.549979in}{1.154547in}}{\pgfqpoint{1.555278in}{1.152352in}}{\pgfqpoint{1.560803in}{1.152352in}}%
\pgfpathclose%
\pgfusepath{stroke,fill}%
\end{pgfscope}%
\begin{pgfscope}%
\pgfpathrectangle{\pgfqpoint{0.644217in}{0.500309in}}{\pgfqpoint{2.405783in}{1.149691in}}%
\pgfusepath{clip}%
\pgfsetbuttcap%
\pgfsetroundjoin%
\definecolor{currentfill}{rgb}{0.000000,0.000000,1.000000}%
\pgfsetfillcolor{currentfill}%
\pgfsetlinewidth{1.003750pt}%
\definecolor{currentstroke}{rgb}{0.000000,0.000000,1.000000}%
\pgfsetstrokecolor{currentstroke}%
\pgfsetdash{}{0pt}%
\pgfpathmoveto{\pgfqpoint{1.600318in}{1.217069in}}%
\pgfpathcurveto{\pgfqpoint{1.605844in}{1.217069in}}{\pgfqpoint{1.611143in}{1.219264in}}{\pgfqpoint{1.615050in}{1.223171in}}%
\pgfpathcurveto{\pgfqpoint{1.618957in}{1.227077in}}{\pgfqpoint{1.621152in}{1.232377in}}{\pgfqpoint{1.621152in}{1.237902in}}%
\pgfpathcurveto{\pgfqpoint{1.621152in}{1.243427in}}{\pgfqpoint{1.618957in}{1.248727in}}{\pgfqpoint{1.615050in}{1.252633in}}%
\pgfpathcurveto{\pgfqpoint{1.611143in}{1.256540in}}{\pgfqpoint{1.605844in}{1.258735in}}{\pgfqpoint{1.600318in}{1.258735in}}%
\pgfpathcurveto{\pgfqpoint{1.594793in}{1.258735in}}{\pgfqpoint{1.589494in}{1.256540in}}{\pgfqpoint{1.585587in}{1.252633in}}%
\pgfpathcurveto{\pgfqpoint{1.581680in}{1.248727in}}{\pgfqpoint{1.579485in}{1.243427in}}{\pgfqpoint{1.579485in}{1.237902in}}%
\pgfpathcurveto{\pgfqpoint{1.579485in}{1.232377in}}{\pgfqpoint{1.581680in}{1.227077in}}{\pgfqpoint{1.585587in}{1.223171in}}%
\pgfpathcurveto{\pgfqpoint{1.589494in}{1.219264in}}{\pgfqpoint{1.594793in}{1.217069in}}{\pgfqpoint{1.600318in}{1.217069in}}%
\pgfpathclose%
\pgfusepath{stroke,fill}%
\end{pgfscope}%
\begin{pgfscope}%
\pgfpathrectangle{\pgfqpoint{0.644217in}{0.500309in}}{\pgfqpoint{2.405783in}{1.149691in}}%
\pgfusepath{clip}%
\pgfsetbuttcap%
\pgfsetroundjoin%
\definecolor{currentfill}{rgb}{0.000000,0.000000,1.000000}%
\pgfsetfillcolor{currentfill}%
\pgfsetlinewidth{1.003750pt}%
\definecolor{currentstroke}{rgb}{0.000000,0.000000,1.000000}%
\pgfsetstrokecolor{currentstroke}%
\pgfsetdash{}{0pt}%
\pgfpathmoveto{\pgfqpoint{1.619343in}{1.333187in}}%
\pgfpathcurveto{\pgfqpoint{1.624868in}{1.333187in}}{\pgfqpoint{1.630168in}{1.335382in}}{\pgfqpoint{1.634074in}{1.339289in}}%
\pgfpathcurveto{\pgfqpoint{1.637981in}{1.343195in}}{\pgfqpoint{1.640176in}{1.348495in}}{\pgfqpoint{1.640176in}{1.354020in}}%
\pgfpathcurveto{\pgfqpoint{1.640176in}{1.359545in}}{\pgfqpoint{1.637981in}{1.364845in}}{\pgfqpoint{1.634074in}{1.368751in}}%
\pgfpathcurveto{\pgfqpoint{1.630168in}{1.372658in}}{\pgfqpoint{1.624868in}{1.374853in}}{\pgfqpoint{1.619343in}{1.374853in}}%
\pgfpathcurveto{\pgfqpoint{1.613818in}{1.374853in}}{\pgfqpoint{1.608519in}{1.372658in}}{\pgfqpoint{1.604612in}{1.368751in}}%
\pgfpathcurveto{\pgfqpoint{1.600705in}{1.364845in}}{\pgfqpoint{1.598510in}{1.359545in}}{\pgfqpoint{1.598510in}{1.354020in}}%
\pgfpathcurveto{\pgfqpoint{1.598510in}{1.348495in}}{\pgfqpoint{1.600705in}{1.343195in}}{\pgfqpoint{1.604612in}{1.339289in}}%
\pgfpathcurveto{\pgfqpoint{1.608519in}{1.335382in}}{\pgfqpoint{1.613818in}{1.333187in}}{\pgfqpoint{1.619343in}{1.333187in}}%
\pgfpathclose%
\pgfusepath{stroke,fill}%
\end{pgfscope}%
\begin{pgfscope}%
\pgfpathrectangle{\pgfqpoint{0.644217in}{0.500309in}}{\pgfqpoint{2.405783in}{1.149691in}}%
\pgfusepath{clip}%
\pgfsetbuttcap%
\pgfsetroundjoin%
\definecolor{currentfill}{rgb}{0.000000,0.000000,1.000000}%
\pgfsetfillcolor{currentfill}%
\pgfsetlinewidth{1.003750pt}%
\definecolor{currentstroke}{rgb}{0.000000,0.000000,1.000000}%
\pgfsetstrokecolor{currentstroke}%
\pgfsetdash{}{0pt}%
\pgfpathmoveto{\pgfqpoint{1.899674in}{1.366927in}}%
\pgfpathcurveto{\pgfqpoint{1.905199in}{1.366927in}}{\pgfqpoint{1.910499in}{1.369122in}}{\pgfqpoint{1.914406in}{1.373029in}}%
\pgfpathcurveto{\pgfqpoint{1.918312in}{1.376935in}}{\pgfqpoint{1.920508in}{1.382235in}}{\pgfqpoint{1.920508in}{1.387760in}}%
\pgfpathcurveto{\pgfqpoint{1.920508in}{1.393285in}}{\pgfqpoint{1.918312in}{1.398584in}}{\pgfqpoint{1.914406in}{1.402491in}}%
\pgfpathcurveto{\pgfqpoint{1.910499in}{1.406398in}}{\pgfqpoint{1.905199in}{1.408593in}}{\pgfqpoint{1.899674in}{1.408593in}}%
\pgfpathcurveto{\pgfqpoint{1.894149in}{1.408593in}}{\pgfqpoint{1.888850in}{1.406398in}}{\pgfqpoint{1.884943in}{1.402491in}}%
\pgfpathcurveto{\pgfqpoint{1.881036in}{1.398584in}}{\pgfqpoint{1.878841in}{1.393285in}}{\pgfqpoint{1.878841in}{1.387760in}}%
\pgfpathcurveto{\pgfqpoint{1.878841in}{1.382235in}}{\pgfqpoint{1.881036in}{1.376935in}}{\pgfqpoint{1.884943in}{1.373029in}}%
\pgfpathcurveto{\pgfqpoint{1.888850in}{1.369122in}}{\pgfqpoint{1.894149in}{1.366927in}}{\pgfqpoint{1.899674in}{1.366927in}}%
\pgfpathclose%
\pgfusepath{stroke,fill}%
\end{pgfscope}%
\begin{pgfscope}%
\pgfpathrectangle{\pgfqpoint{0.644217in}{0.500309in}}{\pgfqpoint{2.405783in}{1.149691in}}%
\pgfusepath{clip}%
\pgfsetbuttcap%
\pgfsetroundjoin%
\definecolor{currentfill}{rgb}{0.000000,0.000000,1.000000}%
\pgfsetfillcolor{currentfill}%
\pgfsetlinewidth{1.003750pt}%
\definecolor{currentstroke}{rgb}{0.000000,0.000000,1.000000}%
\pgfsetstrokecolor{currentstroke}%
\pgfsetdash{}{0pt}%
\pgfpathmoveto{\pgfqpoint{2.059039in}{1.375282in}}%
\pgfpathcurveto{\pgfqpoint{2.064564in}{1.375282in}}{\pgfqpoint{2.069864in}{1.377477in}}{\pgfqpoint{2.073771in}{1.381384in}}%
\pgfpathcurveto{\pgfqpoint{2.077678in}{1.385291in}}{\pgfqpoint{2.079873in}{1.390590in}}{\pgfqpoint{2.079873in}{1.396115in}}%
\pgfpathcurveto{\pgfqpoint{2.079873in}{1.401640in}}{\pgfqpoint{2.077678in}{1.406940in}}{\pgfqpoint{2.073771in}{1.410847in}}%
\pgfpathcurveto{\pgfqpoint{2.069864in}{1.414754in}}{\pgfqpoint{2.064564in}{1.416949in}}{\pgfqpoint{2.059039in}{1.416949in}}%
\pgfpathcurveto{\pgfqpoint{2.053514in}{1.416949in}}{\pgfqpoint{2.048215in}{1.414754in}}{\pgfqpoint{2.044308in}{1.410847in}}%
\pgfpathcurveto{\pgfqpoint{2.040401in}{1.406940in}}{\pgfqpoint{2.038206in}{1.401640in}}{\pgfqpoint{2.038206in}{1.396115in}}%
\pgfpathcurveto{\pgfqpoint{2.038206in}{1.390590in}}{\pgfqpoint{2.040401in}{1.385291in}}{\pgfqpoint{2.044308in}{1.381384in}}%
\pgfpathcurveto{\pgfqpoint{2.048215in}{1.377477in}}{\pgfqpoint{2.053514in}{1.375282in}}{\pgfqpoint{2.059039in}{1.375282in}}%
\pgfpathclose%
\pgfusepath{stroke,fill}%
\end{pgfscope}%
\begin{pgfscope}%
\pgfpathrectangle{\pgfqpoint{0.644217in}{0.500309in}}{\pgfqpoint{2.405783in}{1.149691in}}%
\pgfusepath{clip}%
\pgfsetbuttcap%
\pgfsetroundjoin%
\definecolor{currentfill}{rgb}{0.000000,0.000000,1.000000}%
\pgfsetfillcolor{currentfill}%
\pgfsetlinewidth{1.003750pt}%
\definecolor{currentstroke}{rgb}{0.000000,0.000000,1.000000}%
\pgfsetstrokecolor{currentstroke}%
\pgfsetdash{}{0pt}%
\pgfpathmoveto{\pgfqpoint{2.209712in}{1.381392in}}%
\pgfpathcurveto{\pgfqpoint{2.215237in}{1.381392in}}{\pgfqpoint{2.220536in}{1.383588in}}{\pgfqpoint{2.224443in}{1.387494in}}%
\pgfpathcurveto{\pgfqpoint{2.228350in}{1.391401in}}{\pgfqpoint{2.230545in}{1.396701in}}{\pgfqpoint{2.230545in}{1.402226in}}%
\pgfpathcurveto{\pgfqpoint{2.230545in}{1.407751in}}{\pgfqpoint{2.228350in}{1.413050in}}{\pgfqpoint{2.224443in}{1.416957in}}%
\pgfpathcurveto{\pgfqpoint{2.220536in}{1.420864in}}{\pgfqpoint{2.215237in}{1.423059in}}{\pgfqpoint{2.209712in}{1.423059in}}%
\pgfpathcurveto{\pgfqpoint{2.204187in}{1.423059in}}{\pgfqpoint{2.198887in}{1.420864in}}{\pgfqpoint{2.194980in}{1.416957in}}%
\pgfpathcurveto{\pgfqpoint{2.191074in}{1.413050in}}{\pgfqpoint{2.188878in}{1.407751in}}{\pgfqpoint{2.188878in}{1.402226in}}%
\pgfpathcurveto{\pgfqpoint{2.188878in}{1.396701in}}{\pgfqpoint{2.191074in}{1.391401in}}{\pgfqpoint{2.194980in}{1.387494in}}%
\pgfpathcurveto{\pgfqpoint{2.198887in}{1.383588in}}{\pgfqpoint{2.204187in}{1.381392in}}{\pgfqpoint{2.209712in}{1.381392in}}%
\pgfpathclose%
\pgfusepath{stroke,fill}%
\end{pgfscope}%
\begin{pgfscope}%
\pgfpathrectangle{\pgfqpoint{0.644217in}{0.500309in}}{\pgfqpoint{2.405783in}{1.149691in}}%
\pgfusepath{clip}%
\pgfsetbuttcap%
\pgfsetroundjoin%
\definecolor{currentfill}{rgb}{0.000000,0.000000,1.000000}%
\pgfsetfillcolor{currentfill}%
\pgfsetlinewidth{1.003750pt}%
\definecolor{currentstroke}{rgb}{0.000000,0.000000,1.000000}%
\pgfsetstrokecolor{currentstroke}%
\pgfsetdash{}{0pt}%
\pgfpathmoveto{\pgfqpoint{2.417269in}{1.421691in}}%
\pgfpathcurveto{\pgfqpoint{2.422794in}{1.421691in}}{\pgfqpoint{2.428094in}{1.423886in}}{\pgfqpoint{2.432001in}{1.427793in}}%
\pgfpathcurveto{\pgfqpoint{2.435907in}{1.431699in}}{\pgfqpoint{2.438103in}{1.436999in}}{\pgfqpoint{2.438103in}{1.442524in}}%
\pgfpathcurveto{\pgfqpoint{2.438103in}{1.448049in}}{\pgfqpoint{2.435907in}{1.453349in}}{\pgfqpoint{2.432001in}{1.457255in}}%
\pgfpathcurveto{\pgfqpoint{2.428094in}{1.461162in}}{\pgfqpoint{2.422794in}{1.463357in}}{\pgfqpoint{2.417269in}{1.463357in}}%
\pgfpathcurveto{\pgfqpoint{2.411744in}{1.463357in}}{\pgfqpoint{2.406445in}{1.461162in}}{\pgfqpoint{2.402538in}{1.457255in}}%
\pgfpathcurveto{\pgfqpoint{2.398631in}{1.453349in}}{\pgfqpoint{2.396436in}{1.448049in}}{\pgfqpoint{2.396436in}{1.442524in}}%
\pgfpathcurveto{\pgfqpoint{2.396436in}{1.436999in}}{\pgfqpoint{2.398631in}{1.431699in}}{\pgfqpoint{2.402538in}{1.427793in}}%
\pgfpathcurveto{\pgfqpoint{2.406445in}{1.423886in}}{\pgfqpoint{2.411744in}{1.421691in}}{\pgfqpoint{2.417269in}{1.421691in}}%
\pgfpathclose%
\pgfusepath{stroke,fill}%
\end{pgfscope}%
\begin{pgfscope}%
\pgfpathrectangle{\pgfqpoint{0.644217in}{0.500309in}}{\pgfqpoint{2.405783in}{1.149691in}}%
\pgfusepath{clip}%
\pgfsetbuttcap%
\pgfsetroundjoin%
\definecolor{currentfill}{rgb}{0.000000,0.000000,1.000000}%
\pgfsetfillcolor{currentfill}%
\pgfsetlinewidth{1.003750pt}%
\definecolor{currentstroke}{rgb}{0.000000,0.000000,1.000000}%
\pgfsetstrokecolor{currentstroke}%
\pgfsetdash{}{0pt}%
\pgfpathmoveto{\pgfqpoint{2.602536in}{1.428717in}}%
\pgfpathcurveto{\pgfqpoint{2.608061in}{1.428717in}}{\pgfqpoint{2.613360in}{1.430912in}}{\pgfqpoint{2.617267in}{1.434819in}}%
\pgfpathcurveto{\pgfqpoint{2.621174in}{1.438726in}}{\pgfqpoint{2.623369in}{1.444025in}}{\pgfqpoint{2.623369in}{1.449550in}}%
\pgfpathcurveto{\pgfqpoint{2.623369in}{1.455075in}}{\pgfqpoint{2.621174in}{1.460375in}}{\pgfqpoint{2.617267in}{1.464282in}}%
\pgfpathcurveto{\pgfqpoint{2.613360in}{1.468189in}}{\pgfqpoint{2.608061in}{1.470384in}}{\pgfqpoint{2.602536in}{1.470384in}}%
\pgfpathcurveto{\pgfqpoint{2.597010in}{1.470384in}}{\pgfqpoint{2.591711in}{1.468189in}}{\pgfqpoint{2.587804in}{1.464282in}}%
\pgfpathcurveto{\pgfqpoint{2.583897in}{1.460375in}}{\pgfqpoint{2.581702in}{1.455075in}}{\pgfqpoint{2.581702in}{1.449550in}}%
\pgfpathcurveto{\pgfqpoint{2.581702in}{1.444025in}}{\pgfqpoint{2.583897in}{1.438726in}}{\pgfqpoint{2.587804in}{1.434819in}}%
\pgfpathcurveto{\pgfqpoint{2.591711in}{1.430912in}}{\pgfqpoint{2.597010in}{1.428717in}}{\pgfqpoint{2.602536in}{1.428717in}}%
\pgfpathclose%
\pgfusepath{stroke,fill}%
\end{pgfscope}%
\begin{pgfscope}%
\pgfpathrectangle{\pgfqpoint{0.644217in}{0.500309in}}{\pgfqpoint{2.405783in}{1.149691in}}%
\pgfusepath{clip}%
\pgfsetbuttcap%
\pgfsetroundjoin%
\definecolor{currentfill}{rgb}{0.000000,0.000000,1.000000}%
\pgfsetfillcolor{currentfill}%
\pgfsetlinewidth{1.003750pt}%
\definecolor{currentstroke}{rgb}{0.000000,0.000000,1.000000}%
\pgfsetstrokecolor{currentstroke}%
\pgfsetdash{}{0pt}%
\pgfpathmoveto{\pgfqpoint{2.725529in}{1.413010in}}%
\pgfpathcurveto{\pgfqpoint{2.731054in}{1.413010in}}{\pgfqpoint{2.736353in}{1.415205in}}{\pgfqpoint{2.740260in}{1.419112in}}%
\pgfpathcurveto{\pgfqpoint{2.744167in}{1.423019in}}{\pgfqpoint{2.746362in}{1.428318in}}{\pgfqpoint{2.746362in}{1.433843in}}%
\pgfpathcurveto{\pgfqpoint{2.746362in}{1.439368in}}{\pgfqpoint{2.744167in}{1.444668in}}{\pgfqpoint{2.740260in}{1.448574in}}%
\pgfpathcurveto{\pgfqpoint{2.736353in}{1.452481in}}{\pgfqpoint{2.731054in}{1.454676in}}{\pgfqpoint{2.725529in}{1.454676in}}%
\pgfpathcurveto{\pgfqpoint{2.720004in}{1.454676in}}{\pgfqpoint{2.714704in}{1.452481in}}{\pgfqpoint{2.710797in}{1.448574in}}%
\pgfpathcurveto{\pgfqpoint{2.706891in}{1.444668in}}{\pgfqpoint{2.704695in}{1.439368in}}{\pgfqpoint{2.704695in}{1.433843in}}%
\pgfpathcurveto{\pgfqpoint{2.704695in}{1.428318in}}{\pgfqpoint{2.706891in}{1.423019in}}{\pgfqpoint{2.710797in}{1.419112in}}%
\pgfpathcurveto{\pgfqpoint{2.714704in}{1.415205in}}{\pgfqpoint{2.720004in}{1.413010in}}{\pgfqpoint{2.725529in}{1.413010in}}%
\pgfpathclose%
\pgfusepath{stroke,fill}%
\end{pgfscope}%
\begin{pgfscope}%
\pgfpathrectangle{\pgfqpoint{0.644217in}{0.500309in}}{\pgfqpoint{2.405783in}{1.149691in}}%
\pgfusepath{clip}%
\pgfsetbuttcap%
\pgfsetroundjoin%
\definecolor{currentfill}{rgb}{0.000000,0.000000,1.000000}%
\pgfsetfillcolor{currentfill}%
\pgfsetlinewidth{1.003750pt}%
\definecolor{currentstroke}{rgb}{0.000000,0.000000,1.000000}%
\pgfsetstrokecolor{currentstroke}%
\pgfsetdash{}{0pt}%
\pgfpathmoveto{\pgfqpoint{2.876477in}{1.479367in}}%
\pgfpathcurveto{\pgfqpoint{2.882002in}{1.479367in}}{\pgfqpoint{2.887301in}{1.481562in}}{\pgfqpoint{2.891208in}{1.485469in}}%
\pgfpathcurveto{\pgfqpoint{2.895115in}{1.489375in}}{\pgfqpoint{2.897310in}{1.494675in}}{\pgfqpoint{2.897310in}{1.500200in}}%
\pgfpathcurveto{\pgfqpoint{2.897310in}{1.505725in}}{\pgfqpoint{2.895115in}{1.511025in}}{\pgfqpoint{2.891208in}{1.514931in}}%
\pgfpathcurveto{\pgfqpoint{2.887301in}{1.518838in}}{\pgfqpoint{2.882002in}{1.521033in}}{\pgfqpoint{2.876477in}{1.521033in}}%
\pgfpathcurveto{\pgfqpoint{2.870952in}{1.521033in}}{\pgfqpoint{2.865652in}{1.518838in}}{\pgfqpoint{2.861745in}{1.514931in}}%
\pgfpathcurveto{\pgfqpoint{2.857838in}{1.511025in}}{\pgfqpoint{2.855643in}{1.505725in}}{\pgfqpoint{2.855643in}{1.500200in}}%
\pgfpathcurveto{\pgfqpoint{2.855643in}{1.494675in}}{\pgfqpoint{2.857838in}{1.489375in}}{\pgfqpoint{2.861745in}{1.485469in}}%
\pgfpathcurveto{\pgfqpoint{2.865652in}{1.481562in}}{\pgfqpoint{2.870952in}{1.479367in}}{\pgfqpoint{2.876477in}{1.479367in}}%
\pgfpathclose%
\pgfusepath{stroke,fill}%
\end{pgfscope}%
\begin{pgfscope}%
\pgfpathrectangle{\pgfqpoint{0.644217in}{0.500309in}}{\pgfqpoint{2.405783in}{1.149691in}}%
\pgfusepath{clip}%
\pgfsetrectcap%
\pgfsetroundjoin%
\pgfsetlinewidth{1.505625pt}%
\definecolor{currentstroke}{rgb}{0.827451,0.827451,0.827451}%
\pgfsetstrokecolor{currentstroke}%
\pgfsetdash{}{0pt}%
\pgfpathmoveto{\pgfqpoint{0.890286in}{0.742031in}}%
\pgfpathlineto{\pgfqpoint{0.915457in}{0.766126in}}%
\pgfpathlineto{\pgfqpoint{0.938601in}{0.797030in}}%
\pgfpathlineto{\pgfqpoint{0.970421in}{0.830136in}}%
\pgfpathlineto{\pgfqpoint{0.999210in}{0.855406in}}%
\pgfpathlineto{\pgfqpoint{1.029587in}{0.873402in}}%
\pgfpathlineto{\pgfqpoint{1.058608in}{0.892924in}}%
\pgfpathlineto{\pgfqpoint{1.081956in}{0.914549in}}%
\pgfpathlineto{\pgfqpoint{1.101990in}{0.936163in}}%
\pgfpathlineto{\pgfqpoint{1.122525in}{0.957494in}}%
\pgfpathlineto{\pgfqpoint{1.147001in}{0.971118in}}%
\pgfpathlineto{\pgfqpoint{1.176622in}{0.987565in}}%
\pgfpathlineto{\pgfqpoint{1.220332in}{1.009728in}}%
\pgfpathlineto{\pgfqpoint{1.264965in}{1.030980in}}%
\pgfpathlineto{\pgfqpoint{1.304903in}{1.050381in}}%
\pgfpathlineto{\pgfqpoint{1.344093in}{1.067359in}}%
\pgfpathlineto{\pgfqpoint{1.384337in}{1.082507in}}%
\pgfpathlineto{\pgfqpoint{1.416214in}{1.094209in}}%
\pgfpathlineto{\pgfqpoint{1.439096in}{1.110657in}}%
\pgfpathlineto{\pgfqpoint{1.457354in}{1.126638in}}%
\pgfpathlineto{\pgfqpoint{1.475082in}{1.143259in}}%
\pgfpathlineto{\pgfqpoint{1.498532in}{1.157791in}}%
\pgfpathlineto{\pgfqpoint{1.524249in}{1.177749in}}%
\pgfpathlineto{\pgfqpoint{1.546251in}{1.205435in}}%
\pgfpathlineto{\pgfqpoint{1.574448in}{1.233248in}}%
\pgfpathlineto{\pgfqpoint{1.604655in}{1.265164in}}%
\pgfpathlineto{\pgfqpoint{1.639535in}{1.299087in}}%
\pgfpathlineto{\pgfqpoint{1.687910in}{1.335776in}}%
\pgfpathlineto{\pgfqpoint{1.734451in}{1.363974in}}%
\pgfpathlineto{\pgfqpoint{1.782298in}{1.383786in}}%
\pgfpathlineto{\pgfqpoint{1.832497in}{1.396627in}}%
\pgfpathlineto{\pgfqpoint{1.887997in}{1.400891in}}%
\pgfpathlineto{\pgfqpoint{1.948652in}{1.397628in}}%
\pgfpathlineto{\pgfqpoint{2.011849in}{1.394304in}}%
\pgfpathlineto{\pgfqpoint{2.074395in}{1.394480in}}%
\pgfpathlineto{\pgfqpoint{2.128128in}{1.400515in}}%
\pgfpathlineto{\pgfqpoint{2.175642in}{1.404142in}}%
\pgfpathlineto{\pgfqpoint{2.225319in}{1.410196in}}%
\pgfpathlineto{\pgfqpoint{2.275660in}{1.415858in}}%
\pgfpathlineto{\pgfqpoint{2.330917in}{1.417184in}}%
\pgfpathlineto{\pgfqpoint{2.384491in}{1.418446in}}%
\pgfpathlineto{\pgfqpoint{2.433634in}{1.419552in}}%
\pgfpathlineto{\pgfqpoint{2.487173in}{1.421035in}}%
\pgfpathlineto{\pgfqpoint{2.547092in}{1.423829in}}%
\pgfpathlineto{\pgfqpoint{2.608559in}{1.430538in}}%
\pgfpathlineto{\pgfqpoint{2.669357in}{1.438916in}}%
\pgfpathlineto{\pgfqpoint{2.728416in}{1.450315in}}%
\pgfpathlineto{\pgfqpoint{2.791050in}{1.459181in}}%
\pgfpathlineto{\pgfqpoint{2.851356in}{1.471237in}}%
\pgfpathlineto{\pgfqpoint{2.914599in}{1.487270in}}%
\pgfusepath{stroke}%
\end{pgfscope}%
\begin{pgfscope}%
\pgfpathrectangle{\pgfqpoint{0.644217in}{0.500309in}}{\pgfqpoint{2.405783in}{1.149691in}}%
\pgfusepath{clip}%
\pgfsetbuttcap%
\pgfsetroundjoin%
\definecolor{currentfill}{rgb}{0.827451,0.827451,0.827451}%
\pgfsetfillcolor{currentfill}%
\pgfsetlinewidth{1.003750pt}%
\definecolor{currentstroke}{rgb}{0.827451,0.827451,0.827451}%
\pgfsetstrokecolor{currentstroke}%
\pgfsetdash{}{0pt}%
\pgfsys@defobject{currentmarker}{\pgfqpoint{-0.020833in}{-0.020833in}}{\pgfqpoint{0.020833in}{0.020833in}}{%
\pgfpathmoveto{\pgfqpoint{0.000000in}{-0.020833in}}%
\pgfpathcurveto{\pgfqpoint{0.005525in}{-0.020833in}}{\pgfqpoint{0.010825in}{-0.018638in}}{\pgfqpoint{0.014731in}{-0.014731in}}%
\pgfpathcurveto{\pgfqpoint{0.018638in}{-0.010825in}}{\pgfqpoint{0.020833in}{-0.005525in}}{\pgfqpoint{0.020833in}{0.000000in}}%
\pgfpathcurveto{\pgfqpoint{0.020833in}{0.005525in}}{\pgfqpoint{0.018638in}{0.010825in}}{\pgfqpoint{0.014731in}{0.014731in}}%
\pgfpathcurveto{\pgfqpoint{0.010825in}{0.018638in}}{\pgfqpoint{0.005525in}{0.020833in}}{\pgfqpoint{0.000000in}{0.020833in}}%
\pgfpathcurveto{\pgfqpoint{-0.005525in}{0.020833in}}{\pgfqpoint{-0.010825in}{0.018638in}}{\pgfqpoint{-0.014731in}{0.014731in}}%
\pgfpathcurveto{\pgfqpoint{-0.018638in}{0.010825in}}{\pgfqpoint{-0.020833in}{0.005525in}}{\pgfqpoint{-0.020833in}{0.000000in}}%
\pgfpathcurveto{\pgfqpoint{-0.020833in}{-0.005525in}}{\pgfqpoint{-0.018638in}{-0.010825in}}{\pgfqpoint{-0.014731in}{-0.014731in}}%
\pgfpathcurveto{\pgfqpoint{-0.010825in}{-0.018638in}}{\pgfqpoint{-0.005525in}{-0.020833in}}{\pgfqpoint{0.000000in}{-0.020833in}}%
\pgfpathclose%
\pgfusepath{stroke,fill}%
}%
\begin{pgfscope}%
\pgfsys@transformshift{0.890286in}{0.742031in}%
\pgfsys@useobject{currentmarker}{}%
\end{pgfscope}%
\begin{pgfscope}%
\pgfsys@transformshift{0.915457in}{0.766126in}%
\pgfsys@useobject{currentmarker}{}%
\end{pgfscope}%
\begin{pgfscope}%
\pgfsys@transformshift{0.938601in}{0.797030in}%
\pgfsys@useobject{currentmarker}{}%
\end{pgfscope}%
\begin{pgfscope}%
\pgfsys@transformshift{0.970421in}{0.830136in}%
\pgfsys@useobject{currentmarker}{}%
\end{pgfscope}%
\begin{pgfscope}%
\pgfsys@transformshift{0.999210in}{0.855406in}%
\pgfsys@useobject{currentmarker}{}%
\end{pgfscope}%
\begin{pgfscope}%
\pgfsys@transformshift{1.029587in}{0.873402in}%
\pgfsys@useobject{currentmarker}{}%
\end{pgfscope}%
\begin{pgfscope}%
\pgfsys@transformshift{1.058608in}{0.892924in}%
\pgfsys@useobject{currentmarker}{}%
\end{pgfscope}%
\begin{pgfscope}%
\pgfsys@transformshift{1.081956in}{0.914549in}%
\pgfsys@useobject{currentmarker}{}%
\end{pgfscope}%
\begin{pgfscope}%
\pgfsys@transformshift{1.101990in}{0.936163in}%
\pgfsys@useobject{currentmarker}{}%
\end{pgfscope}%
\begin{pgfscope}%
\pgfsys@transformshift{1.122525in}{0.957494in}%
\pgfsys@useobject{currentmarker}{}%
\end{pgfscope}%
\begin{pgfscope}%
\pgfsys@transformshift{1.147001in}{0.971118in}%
\pgfsys@useobject{currentmarker}{}%
\end{pgfscope}%
\begin{pgfscope}%
\pgfsys@transformshift{1.176622in}{0.987565in}%
\pgfsys@useobject{currentmarker}{}%
\end{pgfscope}%
\begin{pgfscope}%
\pgfsys@transformshift{1.220332in}{1.009728in}%
\pgfsys@useobject{currentmarker}{}%
\end{pgfscope}%
\begin{pgfscope}%
\pgfsys@transformshift{1.264965in}{1.030980in}%
\pgfsys@useobject{currentmarker}{}%
\end{pgfscope}%
\begin{pgfscope}%
\pgfsys@transformshift{1.304903in}{1.050381in}%
\pgfsys@useobject{currentmarker}{}%
\end{pgfscope}%
\begin{pgfscope}%
\pgfsys@transformshift{1.344093in}{1.067359in}%
\pgfsys@useobject{currentmarker}{}%
\end{pgfscope}%
\begin{pgfscope}%
\pgfsys@transformshift{1.384337in}{1.082507in}%
\pgfsys@useobject{currentmarker}{}%
\end{pgfscope}%
\begin{pgfscope}%
\pgfsys@transformshift{1.416214in}{1.094209in}%
\pgfsys@useobject{currentmarker}{}%
\end{pgfscope}%
\begin{pgfscope}%
\pgfsys@transformshift{1.439096in}{1.110657in}%
\pgfsys@useobject{currentmarker}{}%
\end{pgfscope}%
\begin{pgfscope}%
\pgfsys@transformshift{1.457354in}{1.126638in}%
\pgfsys@useobject{currentmarker}{}%
\end{pgfscope}%
\begin{pgfscope}%
\pgfsys@transformshift{1.475082in}{1.143259in}%
\pgfsys@useobject{currentmarker}{}%
\end{pgfscope}%
\begin{pgfscope}%
\pgfsys@transformshift{1.498532in}{1.157791in}%
\pgfsys@useobject{currentmarker}{}%
\end{pgfscope}%
\begin{pgfscope}%
\pgfsys@transformshift{1.524249in}{1.177749in}%
\pgfsys@useobject{currentmarker}{}%
\end{pgfscope}%
\begin{pgfscope}%
\pgfsys@transformshift{1.546251in}{1.205435in}%
\pgfsys@useobject{currentmarker}{}%
\end{pgfscope}%
\begin{pgfscope}%
\pgfsys@transformshift{1.574448in}{1.233248in}%
\pgfsys@useobject{currentmarker}{}%
\end{pgfscope}%
\begin{pgfscope}%
\pgfsys@transformshift{1.604655in}{1.265164in}%
\pgfsys@useobject{currentmarker}{}%
\end{pgfscope}%
\begin{pgfscope}%
\pgfsys@transformshift{1.639535in}{1.299087in}%
\pgfsys@useobject{currentmarker}{}%
\end{pgfscope}%
\begin{pgfscope}%
\pgfsys@transformshift{1.687910in}{1.335776in}%
\pgfsys@useobject{currentmarker}{}%
\end{pgfscope}%
\begin{pgfscope}%
\pgfsys@transformshift{1.734451in}{1.363974in}%
\pgfsys@useobject{currentmarker}{}%
\end{pgfscope}%
\begin{pgfscope}%
\pgfsys@transformshift{1.782298in}{1.383786in}%
\pgfsys@useobject{currentmarker}{}%
\end{pgfscope}%
\begin{pgfscope}%
\pgfsys@transformshift{1.832497in}{1.396627in}%
\pgfsys@useobject{currentmarker}{}%
\end{pgfscope}%
\begin{pgfscope}%
\pgfsys@transformshift{1.887997in}{1.400891in}%
\pgfsys@useobject{currentmarker}{}%
\end{pgfscope}%
\begin{pgfscope}%
\pgfsys@transformshift{1.948652in}{1.397628in}%
\pgfsys@useobject{currentmarker}{}%
\end{pgfscope}%
\begin{pgfscope}%
\pgfsys@transformshift{2.011849in}{1.394304in}%
\pgfsys@useobject{currentmarker}{}%
\end{pgfscope}%
\begin{pgfscope}%
\pgfsys@transformshift{2.074395in}{1.394480in}%
\pgfsys@useobject{currentmarker}{}%
\end{pgfscope}%
\begin{pgfscope}%
\pgfsys@transformshift{2.128128in}{1.400515in}%
\pgfsys@useobject{currentmarker}{}%
\end{pgfscope}%
\begin{pgfscope}%
\pgfsys@transformshift{2.175642in}{1.404142in}%
\pgfsys@useobject{currentmarker}{}%
\end{pgfscope}%
\begin{pgfscope}%
\pgfsys@transformshift{2.225319in}{1.410196in}%
\pgfsys@useobject{currentmarker}{}%
\end{pgfscope}%
\begin{pgfscope}%
\pgfsys@transformshift{2.275660in}{1.415858in}%
\pgfsys@useobject{currentmarker}{}%
\end{pgfscope}%
\begin{pgfscope}%
\pgfsys@transformshift{2.330917in}{1.417184in}%
\pgfsys@useobject{currentmarker}{}%
\end{pgfscope}%
\begin{pgfscope}%
\pgfsys@transformshift{2.384491in}{1.418446in}%
\pgfsys@useobject{currentmarker}{}%
\end{pgfscope}%
\begin{pgfscope}%
\pgfsys@transformshift{2.433634in}{1.419552in}%
\pgfsys@useobject{currentmarker}{}%
\end{pgfscope}%
\begin{pgfscope}%
\pgfsys@transformshift{2.487173in}{1.421035in}%
\pgfsys@useobject{currentmarker}{}%
\end{pgfscope}%
\begin{pgfscope}%
\pgfsys@transformshift{2.547092in}{1.423829in}%
\pgfsys@useobject{currentmarker}{}%
\end{pgfscope}%
\begin{pgfscope}%
\pgfsys@transformshift{2.608559in}{1.430538in}%
\pgfsys@useobject{currentmarker}{}%
\end{pgfscope}%
\begin{pgfscope}%
\pgfsys@transformshift{2.669357in}{1.438916in}%
\pgfsys@useobject{currentmarker}{}%
\end{pgfscope}%
\begin{pgfscope}%
\pgfsys@transformshift{2.728416in}{1.450315in}%
\pgfsys@useobject{currentmarker}{}%
\end{pgfscope}%
\begin{pgfscope}%
\pgfsys@transformshift{2.791050in}{1.459181in}%
\pgfsys@useobject{currentmarker}{}%
\end{pgfscope}%
\begin{pgfscope}%
\pgfsys@transformshift{2.851356in}{1.471237in}%
\pgfsys@useobject{currentmarker}{}%
\end{pgfscope}%
\begin{pgfscope}%
\pgfsys@transformshift{2.914599in}{1.487270in}%
\pgfsys@useobject{currentmarker}{}%
\end{pgfscope}%
\end{pgfscope}%
\begin{pgfscope}%
\pgfsetbuttcap%
\pgfsetroundjoin%
\definecolor{currentfill}{rgb}{0.000000,0.000000,0.000000}%
\pgfsetfillcolor{currentfill}%
\pgfsetlinewidth{0.803000pt}%
\definecolor{currentstroke}{rgb}{0.000000,0.000000,0.000000}%
\pgfsetstrokecolor{currentstroke}%
\pgfsetdash{}{0pt}%
\pgfsys@defobject{currentmarker}{\pgfqpoint{0.000000in}{-0.048611in}}{\pgfqpoint{0.000000in}{0.000000in}}{%
\pgfpathmoveto{\pgfqpoint{0.000000in}{0.000000in}}%
\pgfpathlineto{\pgfqpoint{0.000000in}{-0.048611in}}%
\pgfusepath{stroke,fill}%
}%
\begin{pgfscope}%
\pgfsys@transformshift{0.890286in}{0.500309in}%
\pgfsys@useobject{currentmarker}{}%
\end{pgfscope}%
\end{pgfscope}%
\begin{pgfscope}%
\definecolor{textcolor}{rgb}{0.000000,0.000000,0.000000}%
\pgfsetstrokecolor{textcolor}%
\pgfsetfillcolor{textcolor}%
\pgftext[x=0.890286in,y=0.403087in,,top]{\color{textcolor}\rmfamily\fontsize{8.330000}{9.996000}\selectfont \(\displaystyle 0\)}%
\end{pgfscope}%
\begin{pgfscope}%
\pgfsetbuttcap%
\pgfsetroundjoin%
\definecolor{currentfill}{rgb}{0.000000,0.000000,0.000000}%
\pgfsetfillcolor{currentfill}%
\pgfsetlinewidth{0.803000pt}%
\definecolor{currentstroke}{rgb}{0.000000,0.000000,0.000000}%
\pgfsetstrokecolor{currentstroke}%
\pgfsetdash{}{0pt}%
\pgfsys@defobject{currentmarker}{\pgfqpoint{0.000000in}{-0.048611in}}{\pgfqpoint{0.000000in}{0.000000in}}{%
\pgfpathmoveto{\pgfqpoint{0.000000in}{0.000000in}}%
\pgfpathlineto{\pgfqpoint{0.000000in}{-0.048611in}}%
\pgfusepath{stroke,fill}%
}%
\begin{pgfscope}%
\pgfsys@transformshift{1.458855in}{0.500309in}%
\pgfsys@useobject{currentmarker}{}%
\end{pgfscope}%
\end{pgfscope}%
\begin{pgfscope}%
\definecolor{textcolor}{rgb}{0.000000,0.000000,0.000000}%
\pgfsetstrokecolor{textcolor}%
\pgfsetfillcolor{textcolor}%
\pgftext[x=1.458855in,y=0.403087in,,top]{\color{textcolor}\rmfamily\fontsize{8.330000}{9.996000}\selectfont \(\displaystyle 5\)}%
\end{pgfscope}%
\begin{pgfscope}%
\pgfsetbuttcap%
\pgfsetroundjoin%
\definecolor{currentfill}{rgb}{0.000000,0.000000,0.000000}%
\pgfsetfillcolor{currentfill}%
\pgfsetlinewidth{0.803000pt}%
\definecolor{currentstroke}{rgb}{0.000000,0.000000,0.000000}%
\pgfsetstrokecolor{currentstroke}%
\pgfsetdash{}{0pt}%
\pgfsys@defobject{currentmarker}{\pgfqpoint{0.000000in}{-0.048611in}}{\pgfqpoint{0.000000in}{0.000000in}}{%
\pgfpathmoveto{\pgfqpoint{0.000000in}{0.000000in}}%
\pgfpathlineto{\pgfqpoint{0.000000in}{-0.048611in}}%
\pgfusepath{stroke,fill}%
}%
\begin{pgfscope}%
\pgfsys@transformshift{2.027425in}{0.500309in}%
\pgfsys@useobject{currentmarker}{}%
\end{pgfscope}%
\end{pgfscope}%
\begin{pgfscope}%
\definecolor{textcolor}{rgb}{0.000000,0.000000,0.000000}%
\pgfsetstrokecolor{textcolor}%
\pgfsetfillcolor{textcolor}%
\pgftext[x=2.027425in,y=0.403087in,,top]{\color{textcolor}\rmfamily\fontsize{8.330000}{9.996000}\selectfont \(\displaystyle 10\)}%
\end{pgfscope}%
\begin{pgfscope}%
\pgfsetbuttcap%
\pgfsetroundjoin%
\definecolor{currentfill}{rgb}{0.000000,0.000000,0.000000}%
\pgfsetfillcolor{currentfill}%
\pgfsetlinewidth{0.803000pt}%
\definecolor{currentstroke}{rgb}{0.000000,0.000000,0.000000}%
\pgfsetstrokecolor{currentstroke}%
\pgfsetdash{}{0pt}%
\pgfsys@defobject{currentmarker}{\pgfqpoint{0.000000in}{-0.048611in}}{\pgfqpoint{0.000000in}{0.000000in}}{%
\pgfpathmoveto{\pgfqpoint{0.000000in}{0.000000in}}%
\pgfpathlineto{\pgfqpoint{0.000000in}{-0.048611in}}%
\pgfusepath{stroke,fill}%
}%
\begin{pgfscope}%
\pgfsys@transformshift{2.595995in}{0.500309in}%
\pgfsys@useobject{currentmarker}{}%
\end{pgfscope}%
\end{pgfscope}%
\begin{pgfscope}%
\definecolor{textcolor}{rgb}{0.000000,0.000000,0.000000}%
\pgfsetstrokecolor{textcolor}%
\pgfsetfillcolor{textcolor}%
\pgftext[x=2.595995in,y=0.403087in,,top]{\color{textcolor}\rmfamily\fontsize{8.330000}{9.996000}\selectfont \(\displaystyle 15\)}%
\end{pgfscope}%
\begin{pgfscope}%
\definecolor{textcolor}{rgb}{0.000000,0.000000,0.000000}%
\pgfsetstrokecolor{textcolor}%
\pgfsetfillcolor{textcolor}%
\pgftext[x=1.847108in,y=0.248766in,,top]{\color{textcolor}\rmfamily\fontsize{8.330000}{9.996000}\selectfont Location \(\displaystyle x\)}%
\end{pgfscope}%
\begin{pgfscope}%
\pgfsetbuttcap%
\pgfsetroundjoin%
\definecolor{currentfill}{rgb}{0.000000,0.000000,0.000000}%
\pgfsetfillcolor{currentfill}%
\pgfsetlinewidth{0.803000pt}%
\definecolor{currentstroke}{rgb}{0.000000,0.000000,0.000000}%
\pgfsetstrokecolor{currentstroke}%
\pgfsetdash{}{0pt}%
\pgfsys@defobject{currentmarker}{\pgfqpoint{-0.048611in}{0.000000in}}{\pgfqpoint{0.000000in}{0.000000in}}{%
\pgfpathmoveto{\pgfqpoint{0.000000in}{0.000000in}}%
\pgfpathlineto{\pgfqpoint{-0.048611in}{0.000000in}}%
\pgfusepath{stroke,fill}%
}%
\begin{pgfscope}%
\pgfsys@transformshift{0.644217in}{0.742031in}%
\pgfsys@useobject{currentmarker}{}%
\end{pgfscope}%
\end{pgfscope}%
\begin{pgfscope}%
\definecolor{textcolor}{rgb}{0.000000,0.000000,0.000000}%
\pgfsetstrokecolor{textcolor}%
\pgfsetfillcolor{textcolor}%
\pgftext[x=0.487966in,y=0.703451in,left,base]{\color{textcolor}\rmfamily\fontsize{8.330000}{9.996000}\selectfont \(\displaystyle 0\)}%
\end{pgfscope}%
\begin{pgfscope}%
\pgfsetbuttcap%
\pgfsetroundjoin%
\definecolor{currentfill}{rgb}{0.000000,0.000000,0.000000}%
\pgfsetfillcolor{currentfill}%
\pgfsetlinewidth{0.803000pt}%
\definecolor{currentstroke}{rgb}{0.000000,0.000000,0.000000}%
\pgfsetstrokecolor{currentstroke}%
\pgfsetdash{}{0pt}%
\pgfsys@defobject{currentmarker}{\pgfqpoint{-0.048611in}{0.000000in}}{\pgfqpoint{0.000000in}{0.000000in}}{%
\pgfpathmoveto{\pgfqpoint{0.000000in}{0.000000in}}%
\pgfpathlineto{\pgfqpoint{-0.048611in}{0.000000in}}%
\pgfusepath{stroke,fill}%
}%
\begin{pgfscope}%
\pgfsys@transformshift{0.644217in}{1.214228in}%
\pgfsys@useobject{currentmarker}{}%
\end{pgfscope}%
\end{pgfscope}%
\begin{pgfscope}%
\definecolor{textcolor}{rgb}{0.000000,0.000000,0.000000}%
\pgfsetstrokecolor{textcolor}%
\pgfsetfillcolor{textcolor}%
\pgftext[x=0.487966in,y=1.175647in,left,base]{\color{textcolor}\rmfamily\fontsize{8.330000}{9.996000}\selectfont \(\displaystyle 5\)}%
\end{pgfscope}%
\begin{pgfscope}%
\definecolor{textcolor}{rgb}{0.000000,0.000000,0.000000}%
\pgfsetstrokecolor{textcolor}%
\pgfsetfillcolor{textcolor}%
\pgftext[x=0.432410in,y=1.075154in,,bottom,rotate=90.000000]{\color{textcolor}\rmfamily\fontsize{8.330000}{9.996000}\selectfont Location \(\displaystyle y\)}%
\end{pgfscope}%
\begin{pgfscope}%
\pgfpathrectangle{\pgfqpoint{0.644217in}{0.500309in}}{\pgfqpoint{2.405783in}{1.149691in}}%
\pgfusepath{clip}%
\pgfsetbuttcap%
\pgfsetroundjoin%
\definecolor{currentfill}{rgb}{0.196078,0.803922,0.196078}%
\pgfsetfillcolor{currentfill}%
\pgfsetlinewidth{1.505625pt}%
\definecolor{currentstroke}{rgb}{0.196078,0.803922,0.196078}%
\pgfsetstrokecolor{currentstroke}%
\pgfsetdash{}{0pt}%
\pgfpathmoveto{\pgfqpoint{0.787458in}{0.528645in}}%
\pgfpathlineto{\pgfqpoint{0.870791in}{0.611978in}}%
\pgfpathmoveto{\pgfqpoint{0.787458in}{0.611978in}}%
\pgfpathlineto{\pgfqpoint{0.870791in}{0.528645in}}%
\pgfusepath{stroke,fill}%
\end{pgfscope}%
\begin{pgfscope}%
\pgfpathrectangle{\pgfqpoint{0.644217in}{0.500309in}}{\pgfqpoint{2.405783in}{1.149691in}}%
\pgfusepath{clip}%
\pgfsetbuttcap%
\pgfsetroundjoin%
\definecolor{currentfill}{rgb}{0.196078,0.803922,0.196078}%
\pgfsetfillcolor{currentfill}%
\pgfsetlinewidth{1.505625pt}%
\definecolor{currentstroke}{rgb}{0.196078,0.803922,0.196078}%
\pgfsetstrokecolor{currentstroke}%
\pgfsetdash{}{0pt}%
\pgfpathmoveto{\pgfqpoint{1.083996in}{0.916622in}}%
\pgfpathlineto{\pgfqpoint{1.167329in}{0.999955in}}%
\pgfpathmoveto{\pgfqpoint{1.083996in}{0.999955in}}%
\pgfpathlineto{\pgfqpoint{1.167329in}{0.916622in}}%
\pgfusepath{stroke,fill}%
\end{pgfscope}%
\begin{pgfscope}%
\pgfpathrectangle{\pgfqpoint{0.644217in}{0.500309in}}{\pgfqpoint{2.405783in}{1.149691in}}%
\pgfusepath{clip}%
\pgfsetbuttcap%
\pgfsetroundjoin%
\definecolor{currentfill}{rgb}{0.196078,0.803922,0.196078}%
\pgfsetfillcolor{currentfill}%
\pgfsetlinewidth{1.505625pt}%
\definecolor{currentstroke}{rgb}{0.196078,0.803922,0.196078}%
\pgfsetstrokecolor{currentstroke}%
\pgfsetdash{}{0pt}%
\pgfpathmoveto{\pgfqpoint{0.998885in}{0.918456in}}%
\pgfpathlineto{\pgfqpoint{1.082219in}{1.001789in}}%
\pgfpathmoveto{\pgfqpoint{0.998885in}{1.001789in}}%
\pgfpathlineto{\pgfqpoint{1.082219in}{0.918456in}}%
\pgfusepath{stroke,fill}%
\end{pgfscope}%
\begin{pgfscope}%
\pgfpathrectangle{\pgfqpoint{0.644217in}{0.500309in}}{\pgfqpoint{2.405783in}{1.149691in}}%
\pgfusepath{clip}%
\pgfsetbuttcap%
\pgfsetroundjoin%
\definecolor{currentfill}{rgb}{0.196078,0.803922,0.196078}%
\pgfsetfillcolor{currentfill}%
\pgfsetlinewidth{1.505625pt}%
\definecolor{currentstroke}{rgb}{0.196078,0.803922,0.196078}%
\pgfsetstrokecolor{currentstroke}%
\pgfsetdash{}{0pt}%
\pgfpathmoveto{\pgfqpoint{1.082991in}{0.845652in}}%
\pgfpathlineto{\pgfqpoint{1.166324in}{0.928986in}}%
\pgfpathmoveto{\pgfqpoint{1.082991in}{0.928986in}}%
\pgfpathlineto{\pgfqpoint{1.166324in}{0.845652in}}%
\pgfusepath{stroke,fill}%
\end{pgfscope}%
\begin{pgfscope}%
\pgfpathrectangle{\pgfqpoint{0.644217in}{0.500309in}}{\pgfqpoint{2.405783in}{1.149691in}}%
\pgfusepath{clip}%
\pgfsetbuttcap%
\pgfsetroundjoin%
\definecolor{currentfill}{rgb}{0.196078,0.803922,0.196078}%
\pgfsetfillcolor{currentfill}%
\pgfsetlinewidth{1.505625pt}%
\definecolor{currentstroke}{rgb}{0.196078,0.803922,0.196078}%
\pgfsetstrokecolor{currentstroke}%
\pgfsetdash{}{0pt}%
\pgfpathmoveto{\pgfqpoint{1.261488in}{0.910499in}}%
\pgfpathlineto{\pgfqpoint{1.344821in}{0.993832in}}%
\pgfpathmoveto{\pgfqpoint{1.261488in}{0.993832in}}%
\pgfpathlineto{\pgfqpoint{1.344821in}{0.910499in}}%
\pgfusepath{stroke,fill}%
\end{pgfscope}%
\begin{pgfscope}%
\pgfpathrectangle{\pgfqpoint{0.644217in}{0.500309in}}{\pgfqpoint{2.405783in}{1.149691in}}%
\pgfusepath{clip}%
\pgfsetbuttcap%
\pgfsetroundjoin%
\definecolor{currentfill}{rgb}{0.196078,0.803922,0.196078}%
\pgfsetfillcolor{currentfill}%
\pgfsetlinewidth{1.505625pt}%
\definecolor{currentstroke}{rgb}{0.196078,0.803922,0.196078}%
\pgfsetstrokecolor{currentstroke}%
\pgfsetdash{}{0pt}%
\pgfpathmoveto{\pgfqpoint{1.106372in}{1.036227in}}%
\pgfpathlineto{\pgfqpoint{1.189705in}{1.119560in}}%
\pgfpathmoveto{\pgfqpoint{1.106372in}{1.119560in}}%
\pgfpathlineto{\pgfqpoint{1.189705in}{1.036227in}}%
\pgfusepath{stroke,fill}%
\end{pgfscope}%
\begin{pgfscope}%
\pgfpathrectangle{\pgfqpoint{0.644217in}{0.500309in}}{\pgfqpoint{2.405783in}{1.149691in}}%
\pgfusepath{clip}%
\pgfsetbuttcap%
\pgfsetroundjoin%
\definecolor{currentfill}{rgb}{0.196078,0.803922,0.196078}%
\pgfsetfillcolor{currentfill}%
\pgfsetlinewidth{1.505625pt}%
\definecolor{currentstroke}{rgb}{0.196078,0.803922,0.196078}%
\pgfsetstrokecolor{currentstroke}%
\pgfsetdash{}{0pt}%
\pgfpathmoveto{\pgfqpoint{1.496885in}{0.992287in}}%
\pgfpathlineto{\pgfqpoint{1.580219in}{1.075621in}}%
\pgfpathmoveto{\pgfqpoint{1.496885in}{1.075621in}}%
\pgfpathlineto{\pgfqpoint{1.580219in}{0.992287in}}%
\pgfusepath{stroke,fill}%
\end{pgfscope}%
\begin{pgfscope}%
\pgfpathrectangle{\pgfqpoint{0.644217in}{0.500309in}}{\pgfqpoint{2.405783in}{1.149691in}}%
\pgfusepath{clip}%
\pgfsetbuttcap%
\pgfsetroundjoin%
\definecolor{currentfill}{rgb}{0.196078,0.803922,0.196078}%
\pgfsetfillcolor{currentfill}%
\pgfsetlinewidth{1.505625pt}%
\definecolor{currentstroke}{rgb}{0.196078,0.803922,0.196078}%
\pgfsetstrokecolor{currentstroke}%
\pgfsetdash{}{0pt}%
\pgfpathmoveto{\pgfqpoint{1.579739in}{1.069556in}}%
\pgfpathlineto{\pgfqpoint{1.663072in}{1.152890in}}%
\pgfpathmoveto{\pgfqpoint{1.579739in}{1.152890in}}%
\pgfpathlineto{\pgfqpoint{1.663072in}{1.069556in}}%
\pgfusepath{stroke,fill}%
\end{pgfscope}%
\begin{pgfscope}%
\pgfpathrectangle{\pgfqpoint{0.644217in}{0.500309in}}{\pgfqpoint{2.405783in}{1.149691in}}%
\pgfusepath{clip}%
\pgfsetbuttcap%
\pgfsetroundjoin%
\definecolor{currentfill}{rgb}{0.196078,0.803922,0.196078}%
\pgfsetfillcolor{currentfill}%
\pgfsetlinewidth{1.505625pt}%
\definecolor{currentstroke}{rgb}{0.196078,0.803922,0.196078}%
\pgfsetstrokecolor{currentstroke}%
\pgfsetdash{}{0pt}%
\pgfpathmoveto{\pgfqpoint{1.437232in}{1.152821in}}%
\pgfpathlineto{\pgfqpoint{1.520565in}{1.236154in}}%
\pgfpathmoveto{\pgfqpoint{1.437232in}{1.236154in}}%
\pgfpathlineto{\pgfqpoint{1.520565in}{1.152821in}}%
\pgfusepath{stroke,fill}%
\end{pgfscope}%
\begin{pgfscope}%
\pgfpathrectangle{\pgfqpoint{0.644217in}{0.500309in}}{\pgfqpoint{2.405783in}{1.149691in}}%
\pgfusepath{clip}%
\pgfsetbuttcap%
\pgfsetroundjoin%
\definecolor{currentfill}{rgb}{0.196078,0.803922,0.196078}%
\pgfsetfillcolor{currentfill}%
\pgfsetlinewidth{1.505625pt}%
\definecolor{currentstroke}{rgb}{0.196078,0.803922,0.196078}%
\pgfsetstrokecolor{currentstroke}%
\pgfsetdash{}{0pt}%
\pgfpathmoveto{\pgfqpoint{1.606638in}{1.384007in}}%
\pgfpathlineto{\pgfqpoint{1.689972in}{1.467340in}}%
\pgfpathmoveto{\pgfqpoint{1.606638in}{1.467340in}}%
\pgfpathlineto{\pgfqpoint{1.689972in}{1.384007in}}%
\pgfusepath{stroke,fill}%
\end{pgfscope}%
\begin{pgfscope}%
\pgfpathrectangle{\pgfqpoint{0.644217in}{0.500309in}}{\pgfqpoint{2.405783in}{1.149691in}}%
\pgfusepath{clip}%
\pgfsetbuttcap%
\pgfsetroundjoin%
\definecolor{currentfill}{rgb}{0.196078,0.803922,0.196078}%
\pgfsetfillcolor{currentfill}%
\pgfsetlinewidth{1.505625pt}%
\definecolor{currentstroke}{rgb}{0.196078,0.803922,0.196078}%
\pgfsetstrokecolor{currentstroke}%
\pgfsetdash{}{0pt}%
\pgfpathmoveto{\pgfqpoint{1.745685in}{1.463999in}}%
\pgfpathlineto{\pgfqpoint{1.829018in}{1.547333in}}%
\pgfpathmoveto{\pgfqpoint{1.745685in}{1.547333in}}%
\pgfpathlineto{\pgfqpoint{1.829018in}{1.463999in}}%
\pgfusepath{stroke,fill}%
\end{pgfscope}%
\begin{pgfscope}%
\pgfpathrectangle{\pgfqpoint{0.644217in}{0.500309in}}{\pgfqpoint{2.405783in}{1.149691in}}%
\pgfusepath{clip}%
\pgfsetbuttcap%
\pgfsetroundjoin%
\definecolor{currentfill}{rgb}{0.196078,0.803922,0.196078}%
\pgfsetfillcolor{currentfill}%
\pgfsetlinewidth{1.505625pt}%
\definecolor{currentstroke}{rgb}{0.196078,0.803922,0.196078}%
\pgfsetstrokecolor{currentstroke}%
\pgfsetdash{}{0pt}%
\pgfpathmoveto{\pgfqpoint{1.962687in}{1.383293in}}%
\pgfpathlineto{\pgfqpoint{2.046021in}{1.466626in}}%
\pgfpathmoveto{\pgfqpoint{1.962687in}{1.466626in}}%
\pgfpathlineto{\pgfqpoint{2.046021in}{1.383293in}}%
\pgfusepath{stroke,fill}%
\end{pgfscope}%
\begin{pgfscope}%
\pgfpathrectangle{\pgfqpoint{0.644217in}{0.500309in}}{\pgfqpoint{2.405783in}{1.149691in}}%
\pgfusepath{clip}%
\pgfsetbuttcap%
\pgfsetroundjoin%
\definecolor{currentfill}{rgb}{0.196078,0.803922,0.196078}%
\pgfsetfillcolor{currentfill}%
\pgfsetlinewidth{1.505625pt}%
\definecolor{currentstroke}{rgb}{0.196078,0.803922,0.196078}%
\pgfsetstrokecolor{currentstroke}%
\pgfsetdash{}{0pt}%
\pgfpathmoveto{\pgfqpoint{2.152368in}{1.345860in}}%
\pgfpathlineto{\pgfqpoint{2.235701in}{1.429193in}}%
\pgfpathmoveto{\pgfqpoint{2.152368in}{1.429193in}}%
\pgfpathlineto{\pgfqpoint{2.235701in}{1.345860in}}%
\pgfusepath{stroke,fill}%
\end{pgfscope}%
\begin{pgfscope}%
\pgfpathrectangle{\pgfqpoint{0.644217in}{0.500309in}}{\pgfqpoint{2.405783in}{1.149691in}}%
\pgfusepath{clip}%
\pgfsetbuttcap%
\pgfsetroundjoin%
\definecolor{currentfill}{rgb}{0.196078,0.803922,0.196078}%
\pgfsetfillcolor{currentfill}%
\pgfsetlinewidth{1.505625pt}%
\definecolor{currentstroke}{rgb}{0.196078,0.803922,0.196078}%
\pgfsetstrokecolor{currentstroke}%
\pgfsetdash{}{0pt}%
\pgfpathmoveto{\pgfqpoint{2.409144in}{1.515090in}}%
\pgfpathlineto{\pgfqpoint{2.492477in}{1.598423in}}%
\pgfpathmoveto{\pgfqpoint{2.409144in}{1.598423in}}%
\pgfpathlineto{\pgfqpoint{2.492477in}{1.515090in}}%
\pgfusepath{stroke,fill}%
\end{pgfscope}%
\begin{pgfscope}%
\pgfpathrectangle{\pgfqpoint{0.644217in}{0.500309in}}{\pgfqpoint{2.405783in}{1.149691in}}%
\pgfusepath{clip}%
\pgfsetbuttcap%
\pgfsetroundjoin%
\definecolor{currentfill}{rgb}{0.196078,0.803922,0.196078}%
\pgfsetfillcolor{currentfill}%
\pgfsetlinewidth{1.505625pt}%
\definecolor{currentstroke}{rgb}{0.196078,0.803922,0.196078}%
\pgfsetstrokecolor{currentstroke}%
\pgfsetdash{}{0pt}%
\pgfpathmoveto{\pgfqpoint{2.486013in}{1.538331in}}%
\pgfpathlineto{\pgfqpoint{2.569346in}{1.621664in}}%
\pgfpathmoveto{\pgfqpoint{2.486013in}{1.621664in}}%
\pgfpathlineto{\pgfqpoint{2.569346in}{1.538331in}}%
\pgfusepath{stroke,fill}%
\end{pgfscope}%
\begin{pgfscope}%
\pgfpathrectangle{\pgfqpoint{0.644217in}{0.500309in}}{\pgfqpoint{2.405783in}{1.149691in}}%
\pgfusepath{clip}%
\pgfsetbuttcap%
\pgfsetroundjoin%
\definecolor{currentfill}{rgb}{0.196078,0.803922,0.196078}%
\pgfsetfillcolor{currentfill}%
\pgfsetlinewidth{1.505625pt}%
\definecolor{currentstroke}{rgb}{0.196078,0.803922,0.196078}%
\pgfsetstrokecolor{currentstroke}%
\pgfsetdash{}{0pt}%
\pgfpathmoveto{\pgfqpoint{2.585366in}{1.379862in}}%
\pgfpathlineto{\pgfqpoint{2.668699in}{1.463196in}}%
\pgfpathmoveto{\pgfqpoint{2.585366in}{1.463196in}}%
\pgfpathlineto{\pgfqpoint{2.668699in}{1.379862in}}%
\pgfusepath{stroke,fill}%
\end{pgfscope}%
\begin{pgfscope}%
\pgfpathrectangle{\pgfqpoint{0.644217in}{0.500309in}}{\pgfqpoint{2.405783in}{1.149691in}}%
\pgfusepath{clip}%
\pgfsetbuttcap%
\pgfsetroundjoin%
\definecolor{currentfill}{rgb}{0.196078,0.803922,0.196078}%
\pgfsetfillcolor{currentfill}%
\pgfsetlinewidth{1.505625pt}%
\definecolor{currentstroke}{rgb}{0.196078,0.803922,0.196078}%
\pgfsetstrokecolor{currentstroke}%
\pgfsetdash{}{0pt}%
\pgfpathmoveto{\pgfqpoint{2.616408in}{1.222854in}}%
\pgfpathlineto{\pgfqpoint{2.699741in}{1.306188in}}%
\pgfpathmoveto{\pgfqpoint{2.616408in}{1.306188in}}%
\pgfpathlineto{\pgfqpoint{2.699741in}{1.222854in}}%
\pgfusepath{stroke,fill}%
\end{pgfscope}%
\begin{pgfscope}%
\pgfpathrectangle{\pgfqpoint{0.644217in}{0.500309in}}{\pgfqpoint{2.405783in}{1.149691in}}%
\pgfusepath{clip}%
\pgfsetbuttcap%
\pgfsetroundjoin%
\definecolor{currentfill}{rgb}{0.392157,0.584314,0.929412}%
\pgfsetfillcolor{currentfill}%
\pgfsetlinewidth{1.505625pt}%
\definecolor{currentstroke}{rgb}{0.392157,0.584314,0.929412}%
\pgfsetstrokecolor{currentstroke}%
\pgfsetdash{}{0pt}%
\pgfpathmoveto{\pgfqpoint{0.740012in}{0.633551in}}%
\pgfpathlineto{\pgfqpoint{0.823346in}{0.716884in}}%
\pgfpathmoveto{\pgfqpoint{0.740012in}{0.716884in}}%
\pgfpathlineto{\pgfqpoint{0.823346in}{0.633551in}}%
\pgfusepath{stroke,fill}%
\end{pgfscope}%
\begin{pgfscope}%
\pgfpathrectangle{\pgfqpoint{0.644217in}{0.500309in}}{\pgfqpoint{2.405783in}{1.149691in}}%
\pgfusepath{clip}%
\pgfsetbuttcap%
\pgfsetroundjoin%
\definecolor{currentfill}{rgb}{0.392157,0.584314,0.929412}%
\pgfsetfillcolor{currentfill}%
\pgfsetlinewidth{1.505625pt}%
\definecolor{currentstroke}{rgb}{0.392157,0.584314,0.929412}%
\pgfsetstrokecolor{currentstroke}%
\pgfsetdash{}{0pt}%
\pgfpathmoveto{\pgfqpoint{1.053262in}{0.850348in}}%
\pgfpathlineto{\pgfqpoint{1.136595in}{0.933681in}}%
\pgfpathmoveto{\pgfqpoint{1.053262in}{0.933681in}}%
\pgfpathlineto{\pgfqpoint{1.136595in}{0.850348in}}%
\pgfusepath{stroke,fill}%
\end{pgfscope}%
\begin{pgfscope}%
\pgfpathrectangle{\pgfqpoint{0.644217in}{0.500309in}}{\pgfqpoint{2.405783in}{1.149691in}}%
\pgfusepath{clip}%
\pgfsetbuttcap%
\pgfsetroundjoin%
\definecolor{currentfill}{rgb}{0.392157,0.584314,0.929412}%
\pgfsetfillcolor{currentfill}%
\pgfsetlinewidth{1.505625pt}%
\definecolor{currentstroke}{rgb}{0.392157,0.584314,0.929412}%
\pgfsetstrokecolor{currentstroke}%
\pgfsetdash{}{0pt}%
\pgfpathmoveto{\pgfqpoint{1.208935in}{0.849606in}}%
\pgfpathlineto{\pgfqpoint{1.292268in}{0.932939in}}%
\pgfpathmoveto{\pgfqpoint{1.208935in}{0.932939in}}%
\pgfpathlineto{\pgfqpoint{1.292268in}{0.849606in}}%
\pgfusepath{stroke,fill}%
\end{pgfscope}%
\begin{pgfscope}%
\pgfpathrectangle{\pgfqpoint{0.644217in}{0.500309in}}{\pgfqpoint{2.405783in}{1.149691in}}%
\pgfusepath{clip}%
\pgfsetbuttcap%
\pgfsetroundjoin%
\definecolor{currentfill}{rgb}{0.392157,0.584314,0.929412}%
\pgfsetfillcolor{currentfill}%
\pgfsetlinewidth{1.505625pt}%
\definecolor{currentstroke}{rgb}{0.392157,0.584314,0.929412}%
\pgfsetstrokecolor{currentstroke}%
\pgfsetdash{}{0pt}%
\pgfpathmoveto{\pgfqpoint{1.031693in}{0.934859in}}%
\pgfpathlineto{\pgfqpoint{1.115026in}{1.018192in}}%
\pgfpathmoveto{\pgfqpoint{1.031693in}{1.018192in}}%
\pgfpathlineto{\pgfqpoint{1.115026in}{0.934859in}}%
\pgfusepath{stroke,fill}%
\end{pgfscope}%
\begin{pgfscope}%
\pgfpathrectangle{\pgfqpoint{0.644217in}{0.500309in}}{\pgfqpoint{2.405783in}{1.149691in}}%
\pgfusepath{clip}%
\pgfsetbuttcap%
\pgfsetroundjoin%
\definecolor{currentfill}{rgb}{0.392157,0.584314,0.929412}%
\pgfsetfillcolor{currentfill}%
\pgfsetlinewidth{1.505625pt}%
\definecolor{currentstroke}{rgb}{0.392157,0.584314,0.929412}%
\pgfsetstrokecolor{currentstroke}%
\pgfsetdash{}{0pt}%
\pgfpathmoveto{\pgfqpoint{1.226711in}{0.950760in}}%
\pgfpathlineto{\pgfqpoint{1.310044in}{1.034094in}}%
\pgfpathmoveto{\pgfqpoint{1.226711in}{1.034094in}}%
\pgfpathlineto{\pgfqpoint{1.310044in}{0.950760in}}%
\pgfusepath{stroke,fill}%
\end{pgfscope}%
\begin{pgfscope}%
\pgfpathrectangle{\pgfqpoint{0.644217in}{0.500309in}}{\pgfqpoint{2.405783in}{1.149691in}}%
\pgfusepath{clip}%
\pgfsetbuttcap%
\pgfsetroundjoin%
\definecolor{currentfill}{rgb}{0.392157,0.584314,0.929412}%
\pgfsetfillcolor{currentfill}%
\pgfsetlinewidth{1.505625pt}%
\definecolor{currentstroke}{rgb}{0.392157,0.584314,0.929412}%
\pgfsetstrokecolor{currentstroke}%
\pgfsetdash{}{0pt}%
\pgfpathmoveto{\pgfqpoint{1.445866in}{1.121310in}}%
\pgfpathlineto{\pgfqpoint{1.529199in}{1.204644in}}%
\pgfpathmoveto{\pgfqpoint{1.445866in}{1.204644in}}%
\pgfpathlineto{\pgfqpoint{1.529199in}{1.121310in}}%
\pgfusepath{stroke,fill}%
\end{pgfscope}%
\begin{pgfscope}%
\pgfpathrectangle{\pgfqpoint{0.644217in}{0.500309in}}{\pgfqpoint{2.405783in}{1.149691in}}%
\pgfusepath{clip}%
\pgfsetbuttcap%
\pgfsetroundjoin%
\definecolor{currentfill}{rgb}{0.392157,0.584314,0.929412}%
\pgfsetfillcolor{currentfill}%
\pgfsetlinewidth{1.505625pt}%
\definecolor{currentstroke}{rgb}{0.392157,0.584314,0.929412}%
\pgfsetstrokecolor{currentstroke}%
\pgfsetdash{}{0pt}%
\pgfpathmoveto{\pgfqpoint{1.333699in}{1.052891in}}%
\pgfpathlineto{\pgfqpoint{1.417032in}{1.136224in}}%
\pgfpathmoveto{\pgfqpoint{1.333699in}{1.136224in}}%
\pgfpathlineto{\pgfqpoint{1.417032in}{1.052891in}}%
\pgfusepath{stroke,fill}%
\end{pgfscope}%
\begin{pgfscope}%
\pgfpathrectangle{\pgfqpoint{0.644217in}{0.500309in}}{\pgfqpoint{2.405783in}{1.149691in}}%
\pgfusepath{clip}%
\pgfsetbuttcap%
\pgfsetroundjoin%
\definecolor{currentfill}{rgb}{0.392157,0.584314,0.929412}%
\pgfsetfillcolor{currentfill}%
\pgfsetlinewidth{1.505625pt}%
\definecolor{currentstroke}{rgb}{0.392157,0.584314,0.929412}%
\pgfsetstrokecolor{currentstroke}%
\pgfsetdash{}{0pt}%
\pgfpathmoveto{\pgfqpoint{1.629712in}{1.232787in}}%
\pgfpathlineto{\pgfqpoint{1.713045in}{1.316120in}}%
\pgfpathmoveto{\pgfqpoint{1.629712in}{1.316120in}}%
\pgfpathlineto{\pgfqpoint{1.713045in}{1.232787in}}%
\pgfusepath{stroke,fill}%
\end{pgfscope}%
\begin{pgfscope}%
\pgfpathrectangle{\pgfqpoint{0.644217in}{0.500309in}}{\pgfqpoint{2.405783in}{1.149691in}}%
\pgfusepath{clip}%
\pgfsetbuttcap%
\pgfsetroundjoin%
\definecolor{currentfill}{rgb}{0.392157,0.584314,0.929412}%
\pgfsetfillcolor{currentfill}%
\pgfsetlinewidth{1.505625pt}%
\definecolor{currentstroke}{rgb}{0.392157,0.584314,0.929412}%
\pgfsetstrokecolor{currentstroke}%
\pgfsetdash{}{0pt}%
\pgfpathmoveto{\pgfqpoint{1.789627in}{1.348644in}}%
\pgfpathlineto{\pgfqpoint{1.872960in}{1.431978in}}%
\pgfpathmoveto{\pgfqpoint{1.789627in}{1.431978in}}%
\pgfpathlineto{\pgfqpoint{1.872960in}{1.348644in}}%
\pgfusepath{stroke,fill}%
\end{pgfscope}%
\begin{pgfscope}%
\pgfpathrectangle{\pgfqpoint{0.644217in}{0.500309in}}{\pgfqpoint{2.405783in}{1.149691in}}%
\pgfusepath{clip}%
\pgfsetbuttcap%
\pgfsetroundjoin%
\definecolor{currentfill}{rgb}{0.392157,0.584314,0.929412}%
\pgfsetfillcolor{currentfill}%
\pgfsetlinewidth{1.505625pt}%
\definecolor{currentstroke}{rgb}{0.392157,0.584314,0.929412}%
\pgfsetstrokecolor{currentstroke}%
\pgfsetdash{}{0pt}%
\pgfpathmoveto{\pgfqpoint{1.648770in}{1.335473in}}%
\pgfpathlineto{\pgfqpoint{1.732103in}{1.418806in}}%
\pgfpathmoveto{\pgfqpoint{1.648770in}{1.418806in}}%
\pgfpathlineto{\pgfqpoint{1.732103in}{1.335473in}}%
\pgfusepath{stroke,fill}%
\end{pgfscope}%
\begin{pgfscope}%
\pgfpathrectangle{\pgfqpoint{0.644217in}{0.500309in}}{\pgfqpoint{2.405783in}{1.149691in}}%
\pgfusepath{clip}%
\pgfsetbuttcap%
\pgfsetroundjoin%
\definecolor{currentfill}{rgb}{0.392157,0.584314,0.929412}%
\pgfsetfillcolor{currentfill}%
\pgfsetlinewidth{1.505625pt}%
\definecolor{currentstroke}{rgb}{0.392157,0.584314,0.929412}%
\pgfsetstrokecolor{currentstroke}%
\pgfsetdash{}{0pt}%
\pgfpathmoveto{\pgfqpoint{1.846055in}{1.364810in}}%
\pgfpathlineto{\pgfqpoint{1.929389in}{1.448143in}}%
\pgfpathmoveto{\pgfqpoint{1.846055in}{1.448143in}}%
\pgfpathlineto{\pgfqpoint{1.929389in}{1.364810in}}%
\pgfusepath{stroke,fill}%
\end{pgfscope}%
\begin{pgfscope}%
\pgfpathrectangle{\pgfqpoint{0.644217in}{0.500309in}}{\pgfqpoint{2.405783in}{1.149691in}}%
\pgfusepath{clip}%
\pgfsetbuttcap%
\pgfsetroundjoin%
\definecolor{currentfill}{rgb}{0.392157,0.584314,0.929412}%
\pgfsetfillcolor{currentfill}%
\pgfsetlinewidth{1.505625pt}%
\definecolor{currentstroke}{rgb}{0.392157,0.584314,0.929412}%
\pgfsetstrokecolor{currentstroke}%
\pgfsetdash{}{0pt}%
\pgfpathmoveto{\pgfqpoint{1.953642in}{1.377768in}}%
\pgfpathlineto{\pgfqpoint{2.036975in}{1.461101in}}%
\pgfpathmoveto{\pgfqpoint{1.953642in}{1.461101in}}%
\pgfpathlineto{\pgfqpoint{2.036975in}{1.377768in}}%
\pgfusepath{stroke,fill}%
\end{pgfscope}%
\begin{pgfscope}%
\pgfpathrectangle{\pgfqpoint{0.644217in}{0.500309in}}{\pgfqpoint{2.405783in}{1.149691in}}%
\pgfusepath{clip}%
\pgfsetbuttcap%
\pgfsetroundjoin%
\definecolor{currentfill}{rgb}{0.392157,0.584314,0.929412}%
\pgfsetfillcolor{currentfill}%
\pgfsetlinewidth{1.505625pt}%
\definecolor{currentstroke}{rgb}{0.392157,0.584314,0.929412}%
\pgfsetstrokecolor{currentstroke}%
\pgfsetdash{}{0pt}%
\pgfpathmoveto{\pgfqpoint{2.142225in}{1.418576in}}%
\pgfpathlineto{\pgfqpoint{2.225559in}{1.501909in}}%
\pgfpathmoveto{\pgfqpoint{2.142225in}{1.501909in}}%
\pgfpathlineto{\pgfqpoint{2.225559in}{1.418576in}}%
\pgfusepath{stroke,fill}%
\end{pgfscope}%
\begin{pgfscope}%
\pgfpathrectangle{\pgfqpoint{0.644217in}{0.500309in}}{\pgfqpoint{2.405783in}{1.149691in}}%
\pgfusepath{clip}%
\pgfsetbuttcap%
\pgfsetroundjoin%
\definecolor{currentfill}{rgb}{0.392157,0.584314,0.929412}%
\pgfsetfillcolor{currentfill}%
\pgfsetlinewidth{1.505625pt}%
\definecolor{currentstroke}{rgb}{0.392157,0.584314,0.929412}%
\pgfsetstrokecolor{currentstroke}%
\pgfsetdash{}{0pt}%
\pgfpathmoveto{\pgfqpoint{2.274770in}{1.433236in}}%
\pgfpathlineto{\pgfqpoint{2.358103in}{1.516570in}}%
\pgfpathmoveto{\pgfqpoint{2.274770in}{1.516570in}}%
\pgfpathlineto{\pgfqpoint{2.358103in}{1.433236in}}%
\pgfusepath{stroke,fill}%
\end{pgfscope}%
\begin{pgfscope}%
\pgfpathrectangle{\pgfqpoint{0.644217in}{0.500309in}}{\pgfqpoint{2.405783in}{1.149691in}}%
\pgfusepath{clip}%
\pgfsetbuttcap%
\pgfsetroundjoin%
\definecolor{currentfill}{rgb}{0.392157,0.584314,0.929412}%
\pgfsetfillcolor{currentfill}%
\pgfsetlinewidth{1.505625pt}%
\definecolor{currentstroke}{rgb}{0.392157,0.584314,0.929412}%
\pgfsetstrokecolor{currentstroke}%
\pgfsetdash{}{0pt}%
\pgfpathmoveto{\pgfqpoint{2.525022in}{1.448769in}}%
\pgfpathlineto{\pgfqpoint{2.608356in}{1.532102in}}%
\pgfpathmoveto{\pgfqpoint{2.525022in}{1.532102in}}%
\pgfpathlineto{\pgfqpoint{2.608356in}{1.448769in}}%
\pgfusepath{stroke,fill}%
\end{pgfscope}%
\begin{pgfscope}%
\pgfpathrectangle{\pgfqpoint{0.644217in}{0.500309in}}{\pgfqpoint{2.405783in}{1.149691in}}%
\pgfusepath{clip}%
\pgfsetbuttcap%
\pgfsetroundjoin%
\definecolor{currentfill}{rgb}{0.392157,0.584314,0.929412}%
\pgfsetfillcolor{currentfill}%
\pgfsetlinewidth{1.505625pt}%
\definecolor{currentstroke}{rgb}{0.392157,0.584314,0.929412}%
\pgfsetstrokecolor{currentstroke}%
\pgfsetdash{}{0pt}%
\pgfpathmoveto{\pgfqpoint{2.626857in}{1.346415in}}%
\pgfpathlineto{\pgfqpoint{2.710190in}{1.429748in}}%
\pgfpathmoveto{\pgfqpoint{2.626857in}{1.429748in}}%
\pgfpathlineto{\pgfqpoint{2.710190in}{1.346415in}}%
\pgfusepath{stroke,fill}%
\end{pgfscope}%
\begin{pgfscope}%
\pgfpathrectangle{\pgfqpoint{0.644217in}{0.500309in}}{\pgfqpoint{2.405783in}{1.149691in}}%
\pgfusepath{clip}%
\pgfsetbuttcap%
\pgfsetroundjoin%
\definecolor{currentfill}{rgb}{0.392157,0.584314,0.929412}%
\pgfsetfillcolor{currentfill}%
\pgfsetlinewidth{1.505625pt}%
\definecolor{currentstroke}{rgb}{0.392157,0.584314,0.929412}%
\pgfsetstrokecolor{currentstroke}%
\pgfsetdash{}{0pt}%
\pgfpathmoveto{\pgfqpoint{2.858843in}{1.469075in}}%
\pgfpathlineto{\pgfqpoint{2.942176in}{1.552408in}}%
\pgfpathmoveto{\pgfqpoint{2.858843in}{1.552408in}}%
\pgfpathlineto{\pgfqpoint{2.942176in}{1.469075in}}%
\pgfusepath{stroke,fill}%
\end{pgfscope}%
\begin{pgfscope}%
\pgfpathrectangle{\pgfqpoint{0.644217in}{0.500309in}}{\pgfqpoint{2.405783in}{1.149691in}}%
\pgfusepath{clip}%
\pgfsetbuttcap%
\pgfsetroundjoin%
\definecolor{currentfill}{rgb}{1.000000,0.647059,0.000000}%
\pgfsetfillcolor{currentfill}%
\pgfsetlinewidth{1.505625pt}%
\definecolor{currentstroke}{rgb}{1.000000,0.647059,0.000000}%
\pgfsetstrokecolor{currentstroke}%
\pgfsetdash{}{0pt}%
\pgfpathmoveto{\pgfqpoint{1.042104in}{0.733971in}}%
\pgfpathlineto{\pgfqpoint{1.125438in}{0.817304in}}%
\pgfpathmoveto{\pgfqpoint{1.042104in}{0.817304in}}%
\pgfpathlineto{\pgfqpoint{1.125438in}{0.733971in}}%
\pgfusepath{stroke,fill}%
\end{pgfscope}%
\begin{pgfscope}%
\pgfpathrectangle{\pgfqpoint{0.644217in}{0.500309in}}{\pgfqpoint{2.405783in}{1.149691in}}%
\pgfusepath{clip}%
\pgfsetbuttcap%
\pgfsetroundjoin%
\definecolor{currentfill}{rgb}{1.000000,0.647059,0.000000}%
\pgfsetfillcolor{currentfill}%
\pgfsetlinewidth{1.505625pt}%
\definecolor{currentstroke}{rgb}{1.000000,0.647059,0.000000}%
\pgfsetstrokecolor{currentstroke}%
\pgfsetdash{}{0pt}%
\pgfpathmoveto{\pgfqpoint{0.959256in}{0.804470in}}%
\pgfpathlineto{\pgfqpoint{1.042589in}{0.887803in}}%
\pgfpathmoveto{\pgfqpoint{0.959256in}{0.887803in}}%
\pgfpathlineto{\pgfqpoint{1.042589in}{0.804470in}}%
\pgfusepath{stroke,fill}%
\end{pgfscope}%
\begin{pgfscope}%
\pgfpathrectangle{\pgfqpoint{0.644217in}{0.500309in}}{\pgfqpoint{2.405783in}{1.149691in}}%
\pgfusepath{clip}%
\pgfsetbuttcap%
\pgfsetroundjoin%
\definecolor{currentfill}{rgb}{1.000000,0.647059,0.000000}%
\pgfsetfillcolor{currentfill}%
\pgfsetlinewidth{1.505625pt}%
\definecolor{currentstroke}{rgb}{1.000000,0.647059,0.000000}%
\pgfsetstrokecolor{currentstroke}%
\pgfsetdash{}{0pt}%
\pgfpathmoveto{\pgfqpoint{0.963370in}{0.772347in}}%
\pgfpathlineto{\pgfqpoint{1.046704in}{0.855680in}}%
\pgfpathmoveto{\pgfqpoint{0.963370in}{0.855680in}}%
\pgfpathlineto{\pgfqpoint{1.046704in}{0.772347in}}%
\pgfusepath{stroke,fill}%
\end{pgfscope}%
\begin{pgfscope}%
\pgfpathrectangle{\pgfqpoint{0.644217in}{0.500309in}}{\pgfqpoint{2.405783in}{1.149691in}}%
\pgfusepath{clip}%
\pgfsetbuttcap%
\pgfsetroundjoin%
\definecolor{currentfill}{rgb}{1.000000,0.647059,0.000000}%
\pgfsetfillcolor{currentfill}%
\pgfsetlinewidth{1.505625pt}%
\definecolor{currentstroke}{rgb}{1.000000,0.647059,0.000000}%
\pgfsetstrokecolor{currentstroke}%
\pgfsetdash{}{0pt}%
\pgfpathmoveto{\pgfqpoint{1.219777in}{0.972998in}}%
\pgfpathlineto{\pgfqpoint{1.303111in}{1.056332in}}%
\pgfpathmoveto{\pgfqpoint{1.219777in}{1.056332in}}%
\pgfpathlineto{\pgfqpoint{1.303111in}{0.972998in}}%
\pgfusepath{stroke,fill}%
\end{pgfscope}%
\begin{pgfscope}%
\pgfpathrectangle{\pgfqpoint{0.644217in}{0.500309in}}{\pgfqpoint{2.405783in}{1.149691in}}%
\pgfusepath{clip}%
\pgfsetbuttcap%
\pgfsetroundjoin%
\definecolor{currentfill}{rgb}{1.000000,0.647059,0.000000}%
\pgfsetfillcolor{currentfill}%
\pgfsetlinewidth{1.505625pt}%
\definecolor{currentstroke}{rgb}{1.000000,0.647059,0.000000}%
\pgfsetstrokecolor{currentstroke}%
\pgfsetdash{}{0pt}%
\pgfpathmoveto{\pgfqpoint{1.084382in}{0.960739in}}%
\pgfpathlineto{\pgfqpoint{1.167715in}{1.044073in}}%
\pgfpathmoveto{\pgfqpoint{1.084382in}{1.044073in}}%
\pgfpathlineto{\pgfqpoint{1.167715in}{0.960739in}}%
\pgfusepath{stroke,fill}%
\end{pgfscope}%
\begin{pgfscope}%
\pgfpathrectangle{\pgfqpoint{0.644217in}{0.500309in}}{\pgfqpoint{2.405783in}{1.149691in}}%
\pgfusepath{clip}%
\pgfsetbuttcap%
\pgfsetroundjoin%
\definecolor{currentfill}{rgb}{1.000000,0.647059,0.000000}%
\pgfsetfillcolor{currentfill}%
\pgfsetlinewidth{1.505625pt}%
\definecolor{currentstroke}{rgb}{1.000000,0.647059,0.000000}%
\pgfsetstrokecolor{currentstroke}%
\pgfsetdash{}{0pt}%
\pgfpathmoveto{\pgfqpoint{1.410365in}{1.038059in}}%
\pgfpathlineto{\pgfqpoint{1.493698in}{1.121392in}}%
\pgfpathmoveto{\pgfqpoint{1.410365in}{1.121392in}}%
\pgfpathlineto{\pgfqpoint{1.493698in}{1.038059in}}%
\pgfusepath{stroke,fill}%
\end{pgfscope}%
\begin{pgfscope}%
\pgfpathrectangle{\pgfqpoint{0.644217in}{0.500309in}}{\pgfqpoint{2.405783in}{1.149691in}}%
\pgfusepath{clip}%
\pgfsetbuttcap%
\pgfsetroundjoin%
\definecolor{currentfill}{rgb}{1.000000,0.647059,0.000000}%
\pgfsetfillcolor{currentfill}%
\pgfsetlinewidth{1.505625pt}%
\definecolor{currentstroke}{rgb}{1.000000,0.647059,0.000000}%
\pgfsetstrokecolor{currentstroke}%
\pgfsetdash{}{0pt}%
\pgfpathmoveto{\pgfqpoint{1.194744in}{0.983130in}}%
\pgfpathlineto{\pgfqpoint{1.278077in}{1.066464in}}%
\pgfpathmoveto{\pgfqpoint{1.194744in}{1.066464in}}%
\pgfpathlineto{\pgfqpoint{1.278077in}{0.983130in}}%
\pgfusepath{stroke,fill}%
\end{pgfscope}%
\begin{pgfscope}%
\pgfpathrectangle{\pgfqpoint{0.644217in}{0.500309in}}{\pgfqpoint{2.405783in}{1.149691in}}%
\pgfusepath{clip}%
\pgfsetbuttcap%
\pgfsetroundjoin%
\definecolor{currentfill}{rgb}{1.000000,0.647059,0.000000}%
\pgfsetfillcolor{currentfill}%
\pgfsetlinewidth{1.505625pt}%
\definecolor{currentstroke}{rgb}{1.000000,0.647059,0.000000}%
\pgfsetstrokecolor{currentstroke}%
\pgfsetdash{}{0pt}%
\pgfpathmoveto{\pgfqpoint{1.528939in}{1.087440in}}%
\pgfpathlineto{\pgfqpoint{1.612273in}{1.170774in}}%
\pgfpathmoveto{\pgfqpoint{1.528939in}{1.170774in}}%
\pgfpathlineto{\pgfqpoint{1.612273in}{1.087440in}}%
\pgfusepath{stroke,fill}%
\end{pgfscope}%
\begin{pgfscope}%
\pgfpathrectangle{\pgfqpoint{0.644217in}{0.500309in}}{\pgfqpoint{2.405783in}{1.149691in}}%
\pgfusepath{clip}%
\pgfsetbuttcap%
\pgfsetroundjoin%
\definecolor{currentfill}{rgb}{1.000000,0.647059,0.000000}%
\pgfsetfillcolor{currentfill}%
\pgfsetlinewidth{1.505625pt}%
\definecolor{currentstroke}{rgb}{1.000000,0.647059,0.000000}%
\pgfsetstrokecolor{currentstroke}%
\pgfsetdash{}{0pt}%
\pgfpathmoveto{\pgfqpoint{1.513655in}{1.162701in}}%
\pgfpathlineto{\pgfqpoint{1.596989in}{1.246034in}}%
\pgfpathmoveto{\pgfqpoint{1.513655in}{1.246034in}}%
\pgfpathlineto{\pgfqpoint{1.596989in}{1.162701in}}%
\pgfusepath{stroke,fill}%
\end{pgfscope}%
\begin{pgfscope}%
\pgfpathrectangle{\pgfqpoint{0.644217in}{0.500309in}}{\pgfqpoint{2.405783in}{1.149691in}}%
\pgfusepath{clip}%
\pgfsetbuttcap%
\pgfsetroundjoin%
\definecolor{currentfill}{rgb}{1.000000,0.647059,0.000000}%
\pgfsetfillcolor{currentfill}%
\pgfsetlinewidth{1.505625pt}%
\definecolor{currentstroke}{rgb}{1.000000,0.647059,0.000000}%
\pgfsetstrokecolor{currentstroke}%
\pgfsetdash{}{0pt}%
\pgfpathmoveto{\pgfqpoint{1.678287in}{1.320386in}}%
\pgfpathlineto{\pgfqpoint{1.761621in}{1.403720in}}%
\pgfpathmoveto{\pgfqpoint{1.678287in}{1.403720in}}%
\pgfpathlineto{\pgfqpoint{1.761621in}{1.320386in}}%
\pgfusepath{stroke,fill}%
\end{pgfscope}%
\begin{pgfscope}%
\pgfpathrectangle{\pgfqpoint{0.644217in}{0.500309in}}{\pgfqpoint{2.405783in}{1.149691in}}%
\pgfusepath{clip}%
\pgfsetbuttcap%
\pgfsetroundjoin%
\definecolor{currentfill}{rgb}{1.000000,0.647059,0.000000}%
\pgfsetfillcolor{currentfill}%
\pgfsetlinewidth{1.505625pt}%
\definecolor{currentstroke}{rgb}{1.000000,0.647059,0.000000}%
\pgfsetstrokecolor{currentstroke}%
\pgfsetdash{}{0pt}%
\pgfpathmoveto{\pgfqpoint{1.827267in}{1.348106in}}%
\pgfpathlineto{\pgfqpoint{1.910600in}{1.431440in}}%
\pgfpathmoveto{\pgfqpoint{1.827267in}{1.431440in}}%
\pgfpathlineto{\pgfqpoint{1.910600in}{1.348106in}}%
\pgfusepath{stroke,fill}%
\end{pgfscope}%
\begin{pgfscope}%
\pgfpathrectangle{\pgfqpoint{0.644217in}{0.500309in}}{\pgfqpoint{2.405783in}{1.149691in}}%
\pgfusepath{clip}%
\pgfsetbuttcap%
\pgfsetroundjoin%
\definecolor{currentfill}{rgb}{1.000000,0.647059,0.000000}%
\pgfsetfillcolor{currentfill}%
\pgfsetlinewidth{1.505625pt}%
\definecolor{currentstroke}{rgb}{1.000000,0.647059,0.000000}%
\pgfsetstrokecolor{currentstroke}%
\pgfsetdash{}{0pt}%
\pgfpathmoveto{\pgfqpoint{2.094729in}{1.421860in}}%
\pgfpathlineto{\pgfqpoint{2.178062in}{1.505193in}}%
\pgfpathmoveto{\pgfqpoint{2.094729in}{1.505193in}}%
\pgfpathlineto{\pgfqpoint{2.178062in}{1.421860in}}%
\pgfusepath{stroke,fill}%
\end{pgfscope}%
\begin{pgfscope}%
\pgfpathrectangle{\pgfqpoint{0.644217in}{0.500309in}}{\pgfqpoint{2.405783in}{1.149691in}}%
\pgfusepath{clip}%
\pgfsetbuttcap%
\pgfsetroundjoin%
\definecolor{currentfill}{rgb}{1.000000,0.647059,0.000000}%
\pgfsetfillcolor{currentfill}%
\pgfsetlinewidth{1.505625pt}%
\definecolor{currentstroke}{rgb}{1.000000,0.647059,0.000000}%
\pgfsetstrokecolor{currentstroke}%
\pgfsetdash{}{0pt}%
\pgfpathmoveto{\pgfqpoint{1.853371in}{1.304270in}}%
\pgfpathlineto{\pgfqpoint{1.936704in}{1.387604in}}%
\pgfpathmoveto{\pgfqpoint{1.853371in}{1.387604in}}%
\pgfpathlineto{\pgfqpoint{1.936704in}{1.304270in}}%
\pgfusepath{stroke,fill}%
\end{pgfscope}%
\begin{pgfscope}%
\pgfpathrectangle{\pgfqpoint{0.644217in}{0.500309in}}{\pgfqpoint{2.405783in}{1.149691in}}%
\pgfusepath{clip}%
\pgfsetbuttcap%
\pgfsetroundjoin%
\definecolor{currentfill}{rgb}{1.000000,0.647059,0.000000}%
\pgfsetfillcolor{currentfill}%
\pgfsetlinewidth{1.505625pt}%
\definecolor{currentstroke}{rgb}{1.000000,0.647059,0.000000}%
\pgfsetstrokecolor{currentstroke}%
\pgfsetdash{}{0pt}%
\pgfpathmoveto{\pgfqpoint{2.264187in}{1.358252in}}%
\pgfpathlineto{\pgfqpoint{2.347521in}{1.441585in}}%
\pgfpathmoveto{\pgfqpoint{2.264187in}{1.441585in}}%
\pgfpathlineto{\pgfqpoint{2.347521in}{1.358252in}}%
\pgfusepath{stroke,fill}%
\end{pgfscope}%
\begin{pgfscope}%
\pgfpathrectangle{\pgfqpoint{0.644217in}{0.500309in}}{\pgfqpoint{2.405783in}{1.149691in}}%
\pgfusepath{clip}%
\pgfsetbuttcap%
\pgfsetroundjoin%
\definecolor{currentfill}{rgb}{1.000000,0.647059,0.000000}%
\pgfsetfillcolor{currentfill}%
\pgfsetlinewidth{1.505625pt}%
\definecolor{currentstroke}{rgb}{1.000000,0.647059,0.000000}%
\pgfsetstrokecolor{currentstroke}%
\pgfsetdash{}{0pt}%
\pgfpathmoveto{\pgfqpoint{2.333758in}{1.337612in}}%
\pgfpathlineto{\pgfqpoint{2.417091in}{1.420945in}}%
\pgfpathmoveto{\pgfqpoint{2.333758in}{1.420945in}}%
\pgfpathlineto{\pgfqpoint{2.417091in}{1.337612in}}%
\pgfusepath{stroke,fill}%
\end{pgfscope}%
\begin{pgfscope}%
\pgfpathrectangle{\pgfqpoint{0.644217in}{0.500309in}}{\pgfqpoint{2.405783in}{1.149691in}}%
\pgfusepath{clip}%
\pgfsetbuttcap%
\pgfsetroundjoin%
\definecolor{currentfill}{rgb}{1.000000,0.647059,0.000000}%
\pgfsetfillcolor{currentfill}%
\pgfsetlinewidth{1.505625pt}%
\definecolor{currentstroke}{rgb}{1.000000,0.647059,0.000000}%
\pgfsetstrokecolor{currentstroke}%
\pgfsetdash{}{0pt}%
\pgfpathmoveto{\pgfqpoint{2.593338in}{1.383503in}}%
\pgfpathlineto{\pgfqpoint{2.676671in}{1.466836in}}%
\pgfpathmoveto{\pgfqpoint{2.593338in}{1.466836in}}%
\pgfpathlineto{\pgfqpoint{2.676671in}{1.383503in}}%
\pgfusepath{stroke,fill}%
\end{pgfscope}%
\begin{pgfscope}%
\pgfpathrectangle{\pgfqpoint{0.644217in}{0.500309in}}{\pgfqpoint{2.405783in}{1.149691in}}%
\pgfusepath{clip}%
\pgfsetbuttcap%
\pgfsetroundjoin%
\definecolor{currentfill}{rgb}{1.000000,0.647059,0.000000}%
\pgfsetfillcolor{currentfill}%
\pgfsetlinewidth{1.505625pt}%
\definecolor{currentstroke}{rgb}{1.000000,0.647059,0.000000}%
\pgfsetstrokecolor{currentstroke}%
\pgfsetdash{}{0pt}%
\pgfpathmoveto{\pgfqpoint{2.870037in}{1.451554in}}%
\pgfpathlineto{\pgfqpoint{2.953371in}{1.534888in}}%
\pgfpathmoveto{\pgfqpoint{2.870037in}{1.534888in}}%
\pgfpathlineto{\pgfqpoint{2.953371in}{1.451554in}}%
\pgfusepath{stroke,fill}%
\end{pgfscope}%
\begin{pgfscope}%
\pgfsetrectcap%
\pgfsetmiterjoin%
\pgfsetlinewidth{0.803000pt}%
\definecolor{currentstroke}{rgb}{0.000000,0.000000,0.000000}%
\pgfsetstrokecolor{currentstroke}%
\pgfsetdash{}{0pt}%
\pgfpathmoveto{\pgfqpoint{0.644217in}{0.500309in}}%
\pgfpathlineto{\pgfqpoint{0.644217in}{1.650000in}}%
\pgfusepath{stroke}%
\end{pgfscope}%
\begin{pgfscope}%
\pgfsetrectcap%
\pgfsetmiterjoin%
\pgfsetlinewidth{0.803000pt}%
\definecolor{currentstroke}{rgb}{0.000000,0.000000,0.000000}%
\pgfsetstrokecolor{currentstroke}%
\pgfsetdash{}{0pt}%
\pgfpathmoveto{\pgfqpoint{3.050000in}{0.500309in}}%
\pgfpathlineto{\pgfqpoint{3.050000in}{1.650000in}}%
\pgfusepath{stroke}%
\end{pgfscope}%
\begin{pgfscope}%
\pgfsetrectcap%
\pgfsetmiterjoin%
\pgfsetlinewidth{0.803000pt}%
\definecolor{currentstroke}{rgb}{0.000000,0.000000,0.000000}%
\pgfsetstrokecolor{currentstroke}%
\pgfsetdash{}{0pt}%
\pgfpathmoveto{\pgfqpoint{0.644217in}{0.500309in}}%
\pgfpathlineto{\pgfqpoint{3.050000in}{0.500309in}}%
\pgfusepath{stroke}%
\end{pgfscope}%
\begin{pgfscope}%
\pgfsetrectcap%
\pgfsetmiterjoin%
\pgfsetlinewidth{0.803000pt}%
\definecolor{currentstroke}{rgb}{0.000000,0.000000,0.000000}%
\pgfsetstrokecolor{currentstroke}%
\pgfsetdash{}{0pt}%
\pgfpathmoveto{\pgfqpoint{0.644217in}{1.650000in}}%
\pgfpathlineto{\pgfqpoint{3.050000in}{1.650000in}}%
\pgfusepath{stroke}%
\end{pgfscope}%
\begin{pgfscope}%
\pgfpathrectangle{\pgfqpoint{0.644217in}{0.500309in}}{\pgfqpoint{2.405783in}{1.149691in}}%
\pgfusepath{clip}%
\pgfsetbuttcap%
\pgfsetroundjoin%
\pgfsetlinewidth{1.505625pt}%
\definecolor{currentstroke}{rgb}{0.827451,0.827451,0.827451}%
\pgfsetstrokecolor{currentstroke}%
\pgfsetdash{{5.550000pt}{2.400000pt}}{0.000000pt}%
\pgfpathmoveto{\pgfqpoint{0.890286in}{0.742031in}}%
\pgfpathlineto{\pgfqpoint{0.829124in}{0.570311in}}%
\pgfusepath{stroke}%
\end{pgfscope}%
\begin{pgfscope}%
\pgfpathrectangle{\pgfqpoint{0.644217in}{0.500309in}}{\pgfqpoint{2.405783in}{1.149691in}}%
\pgfusepath{clip}%
\pgfsetbuttcap%
\pgfsetroundjoin%
\pgfsetlinewidth{1.505625pt}%
\definecolor{currentstroke}{rgb}{0.827451,0.827451,0.827451}%
\pgfsetstrokecolor{currentstroke}%
\pgfsetdash{{5.550000pt}{2.400000pt}}{0.000000pt}%
\pgfpathmoveto{\pgfqpoint{0.970421in}{0.830136in}}%
\pgfpathlineto{\pgfqpoint{1.125662in}{0.958289in}}%
\pgfusepath{stroke}%
\end{pgfscope}%
\begin{pgfscope}%
\pgfpathrectangle{\pgfqpoint{0.644217in}{0.500309in}}{\pgfqpoint{2.405783in}{1.149691in}}%
\pgfusepath{clip}%
\pgfsetbuttcap%
\pgfsetroundjoin%
\pgfsetlinewidth{1.505625pt}%
\definecolor{currentstroke}{rgb}{0.827451,0.827451,0.827451}%
\pgfsetstrokecolor{currentstroke}%
\pgfsetdash{{5.550000pt}{2.400000pt}}{0.000000pt}%
\pgfpathmoveto{\pgfqpoint{1.058608in}{0.892924in}}%
\pgfpathlineto{\pgfqpoint{1.040552in}{0.960123in}}%
\pgfusepath{stroke}%
\end{pgfscope}%
\begin{pgfscope}%
\pgfpathrectangle{\pgfqpoint{0.644217in}{0.500309in}}{\pgfqpoint{2.405783in}{1.149691in}}%
\pgfusepath{clip}%
\pgfsetbuttcap%
\pgfsetroundjoin%
\pgfsetlinewidth{1.505625pt}%
\definecolor{currentstroke}{rgb}{0.827451,0.827451,0.827451}%
\pgfsetstrokecolor{currentstroke}%
\pgfsetdash{{5.550000pt}{2.400000pt}}{0.000000pt}%
\pgfpathmoveto{\pgfqpoint{1.122525in}{0.957494in}}%
\pgfpathlineto{\pgfqpoint{1.124657in}{0.887319in}}%
\pgfusepath{stroke}%
\end{pgfscope}%
\begin{pgfscope}%
\pgfpathrectangle{\pgfqpoint{0.644217in}{0.500309in}}{\pgfqpoint{2.405783in}{1.149691in}}%
\pgfusepath{clip}%
\pgfsetbuttcap%
\pgfsetroundjoin%
\pgfsetlinewidth{1.505625pt}%
\definecolor{currentstroke}{rgb}{0.827451,0.827451,0.827451}%
\pgfsetstrokecolor{currentstroke}%
\pgfsetdash{{5.550000pt}{2.400000pt}}{0.000000pt}%
\pgfpathmoveto{\pgfqpoint{1.220332in}{1.009728in}}%
\pgfpathlineto{\pgfqpoint{1.303155in}{0.952166in}}%
\pgfusepath{stroke}%
\end{pgfscope}%
\begin{pgfscope}%
\pgfpathrectangle{\pgfqpoint{0.644217in}{0.500309in}}{\pgfqpoint{2.405783in}{1.149691in}}%
\pgfusepath{clip}%
\pgfsetbuttcap%
\pgfsetroundjoin%
\pgfsetlinewidth{1.505625pt}%
\definecolor{currentstroke}{rgb}{0.827451,0.827451,0.827451}%
\pgfsetstrokecolor{currentstroke}%
\pgfsetdash{{5.550000pt}{2.400000pt}}{0.000000pt}%
\pgfpathmoveto{\pgfqpoint{1.344093in}{1.067359in}}%
\pgfpathlineto{\pgfqpoint{1.148039in}{1.077894in}}%
\pgfusepath{stroke}%
\end{pgfscope}%
\begin{pgfscope}%
\pgfpathrectangle{\pgfqpoint{0.644217in}{0.500309in}}{\pgfqpoint{2.405783in}{1.149691in}}%
\pgfusepath{clip}%
\pgfsetbuttcap%
\pgfsetroundjoin%
\pgfsetlinewidth{1.505625pt}%
\definecolor{currentstroke}{rgb}{0.827451,0.827451,0.827451}%
\pgfsetstrokecolor{currentstroke}%
\pgfsetdash{{5.550000pt}{2.400000pt}}{0.000000pt}%
\pgfpathmoveto{\pgfqpoint{1.439096in}{1.110657in}}%
\pgfpathlineto{\pgfqpoint{1.538552in}{1.033954in}}%
\pgfusepath{stroke}%
\end{pgfscope}%
\begin{pgfscope}%
\pgfpathrectangle{\pgfqpoint{0.644217in}{0.500309in}}{\pgfqpoint{2.405783in}{1.149691in}}%
\pgfusepath{clip}%
\pgfsetbuttcap%
\pgfsetroundjoin%
\pgfsetlinewidth{1.505625pt}%
\definecolor{currentstroke}{rgb}{0.827451,0.827451,0.827451}%
\pgfsetstrokecolor{currentstroke}%
\pgfsetdash{{5.550000pt}{2.400000pt}}{0.000000pt}%
\pgfpathmoveto{\pgfqpoint{1.498532in}{1.157791in}}%
\pgfpathlineto{\pgfqpoint{1.621406in}{1.111223in}}%
\pgfusepath{stroke}%
\end{pgfscope}%
\begin{pgfscope}%
\pgfpathrectangle{\pgfqpoint{0.644217in}{0.500309in}}{\pgfqpoint{2.405783in}{1.149691in}}%
\pgfusepath{clip}%
\pgfsetbuttcap%
\pgfsetroundjoin%
\pgfsetlinewidth{1.505625pt}%
\definecolor{currentstroke}{rgb}{0.827451,0.827451,0.827451}%
\pgfsetstrokecolor{currentstroke}%
\pgfsetdash{{5.550000pt}{2.400000pt}}{0.000000pt}%
\pgfpathmoveto{\pgfqpoint{1.574448in}{1.233248in}}%
\pgfpathlineto{\pgfqpoint{1.478898in}{1.194488in}}%
\pgfusepath{stroke}%
\end{pgfscope}%
\begin{pgfscope}%
\pgfpathrectangle{\pgfqpoint{0.644217in}{0.500309in}}{\pgfqpoint{2.405783in}{1.149691in}}%
\pgfusepath{clip}%
\pgfsetbuttcap%
\pgfsetroundjoin%
\pgfsetlinewidth{1.505625pt}%
\definecolor{currentstroke}{rgb}{0.827451,0.827451,0.827451}%
\pgfsetstrokecolor{currentstroke}%
\pgfsetdash{{5.550000pt}{2.400000pt}}{0.000000pt}%
\pgfpathmoveto{\pgfqpoint{1.687910in}{1.335776in}}%
\pgfpathlineto{\pgfqpoint{1.648305in}{1.425674in}}%
\pgfusepath{stroke}%
\end{pgfscope}%
\begin{pgfscope}%
\pgfpathrectangle{\pgfqpoint{0.644217in}{0.500309in}}{\pgfqpoint{2.405783in}{1.149691in}}%
\pgfusepath{clip}%
\pgfsetbuttcap%
\pgfsetroundjoin%
\pgfsetlinewidth{1.505625pt}%
\definecolor{currentstroke}{rgb}{0.827451,0.827451,0.827451}%
\pgfsetstrokecolor{currentstroke}%
\pgfsetdash{{5.550000pt}{2.400000pt}}{0.000000pt}%
\pgfpathmoveto{\pgfqpoint{1.832497in}{1.396627in}}%
\pgfpathlineto{\pgfqpoint{1.787351in}{1.505666in}}%
\pgfusepath{stroke}%
\end{pgfscope}%
\begin{pgfscope}%
\pgfpathrectangle{\pgfqpoint{0.644217in}{0.500309in}}{\pgfqpoint{2.405783in}{1.149691in}}%
\pgfusepath{clip}%
\pgfsetbuttcap%
\pgfsetroundjoin%
\pgfsetlinewidth{1.505625pt}%
\definecolor{currentstroke}{rgb}{0.827451,0.827451,0.827451}%
\pgfsetstrokecolor{currentstroke}%
\pgfsetdash{{5.550000pt}{2.400000pt}}{0.000000pt}%
\pgfpathmoveto{\pgfqpoint{2.011849in}{1.394304in}}%
\pgfpathlineto{\pgfqpoint{2.004354in}{1.424959in}}%
\pgfusepath{stroke}%
\end{pgfscope}%
\begin{pgfscope}%
\pgfpathrectangle{\pgfqpoint{0.644217in}{0.500309in}}{\pgfqpoint{2.405783in}{1.149691in}}%
\pgfusepath{clip}%
\pgfsetbuttcap%
\pgfsetroundjoin%
\pgfsetlinewidth{1.505625pt}%
\definecolor{currentstroke}{rgb}{0.827451,0.827451,0.827451}%
\pgfsetstrokecolor{currentstroke}%
\pgfsetdash{{5.550000pt}{2.400000pt}}{0.000000pt}%
\pgfpathmoveto{\pgfqpoint{2.175642in}{1.404142in}}%
\pgfpathlineto{\pgfqpoint{2.194035in}{1.387527in}}%
\pgfusepath{stroke}%
\end{pgfscope}%
\begin{pgfscope}%
\pgfpathrectangle{\pgfqpoint{0.644217in}{0.500309in}}{\pgfqpoint{2.405783in}{1.149691in}}%
\pgfusepath{clip}%
\pgfsetbuttcap%
\pgfsetroundjoin%
\pgfsetlinewidth{1.505625pt}%
\definecolor{currentstroke}{rgb}{0.827451,0.827451,0.827451}%
\pgfsetstrokecolor{currentstroke}%
\pgfsetdash{{5.550000pt}{2.400000pt}}{0.000000pt}%
\pgfpathmoveto{\pgfqpoint{2.330917in}{1.417184in}}%
\pgfpathlineto{\pgfqpoint{2.450810in}{1.556757in}}%
\pgfusepath{stroke}%
\end{pgfscope}%
\begin{pgfscope}%
\pgfpathrectangle{\pgfqpoint{0.644217in}{0.500309in}}{\pgfqpoint{2.405783in}{1.149691in}}%
\pgfusepath{clip}%
\pgfsetbuttcap%
\pgfsetroundjoin%
\pgfsetlinewidth{1.505625pt}%
\definecolor{currentstroke}{rgb}{0.827451,0.827451,0.827451}%
\pgfsetstrokecolor{currentstroke}%
\pgfsetdash{{5.550000pt}{2.400000pt}}{0.000000pt}%
\pgfpathmoveto{\pgfqpoint{2.487173in}{1.421035in}}%
\pgfpathlineto{\pgfqpoint{2.527680in}{1.579998in}}%
\pgfusepath{stroke}%
\end{pgfscope}%
\begin{pgfscope}%
\pgfpathrectangle{\pgfqpoint{0.644217in}{0.500309in}}{\pgfqpoint{2.405783in}{1.149691in}}%
\pgfusepath{clip}%
\pgfsetbuttcap%
\pgfsetroundjoin%
\pgfsetlinewidth{1.505625pt}%
\definecolor{currentstroke}{rgb}{0.827451,0.827451,0.827451}%
\pgfsetstrokecolor{currentstroke}%
\pgfsetdash{{5.550000pt}{2.400000pt}}{0.000000pt}%
\pgfpathmoveto{\pgfqpoint{2.669357in}{1.438916in}}%
\pgfpathlineto{\pgfqpoint{2.627033in}{1.421529in}}%
\pgfusepath{stroke}%
\end{pgfscope}%
\begin{pgfscope}%
\pgfpathrectangle{\pgfqpoint{0.644217in}{0.500309in}}{\pgfqpoint{2.405783in}{1.149691in}}%
\pgfusepath{clip}%
\pgfsetbuttcap%
\pgfsetroundjoin%
\pgfsetlinewidth{1.505625pt}%
\definecolor{currentstroke}{rgb}{0.827451,0.827451,0.827451}%
\pgfsetstrokecolor{currentstroke}%
\pgfsetdash{{5.550000pt}{2.400000pt}}{0.000000pt}%
\pgfpathmoveto{\pgfqpoint{2.851356in}{1.471237in}}%
\pgfpathlineto{\pgfqpoint{2.658074in}{1.264521in}}%
\pgfusepath{stroke}%
\end{pgfscope}%
\begin{pgfscope}%
\pgfpathrectangle{\pgfqpoint{0.644217in}{0.500309in}}{\pgfqpoint{2.405783in}{1.149691in}}%
\pgfusepath{clip}%
\pgfsetbuttcap%
\pgfsetroundjoin%
\pgfsetlinewidth{1.505625pt}%
\definecolor{currentstroke}{rgb}{0.827451,0.827451,0.827451}%
\pgfsetstrokecolor{currentstroke}%
\pgfsetdash{{5.550000pt}{2.400000pt}}{0.000000pt}%
\pgfpathmoveto{\pgfqpoint{0.890286in}{0.742031in}}%
\pgfpathlineto{\pgfqpoint{0.781679in}{0.675218in}}%
\pgfusepath{stroke}%
\end{pgfscope}%
\begin{pgfscope}%
\pgfpathrectangle{\pgfqpoint{0.644217in}{0.500309in}}{\pgfqpoint{2.405783in}{1.149691in}}%
\pgfusepath{clip}%
\pgfsetbuttcap%
\pgfsetroundjoin%
\pgfsetlinewidth{1.505625pt}%
\definecolor{currentstroke}{rgb}{0.827451,0.827451,0.827451}%
\pgfsetstrokecolor{currentstroke}%
\pgfsetdash{{5.550000pt}{2.400000pt}}{0.000000pt}%
\pgfpathmoveto{\pgfqpoint{0.970421in}{0.830136in}}%
\pgfpathlineto{\pgfqpoint{1.094929in}{0.892015in}}%
\pgfusepath{stroke}%
\end{pgfscope}%
\begin{pgfscope}%
\pgfpathrectangle{\pgfqpoint{0.644217in}{0.500309in}}{\pgfqpoint{2.405783in}{1.149691in}}%
\pgfusepath{clip}%
\pgfsetbuttcap%
\pgfsetroundjoin%
\pgfsetlinewidth{1.505625pt}%
\definecolor{currentstroke}{rgb}{0.827451,0.827451,0.827451}%
\pgfsetstrokecolor{currentstroke}%
\pgfsetdash{{5.550000pt}{2.400000pt}}{0.000000pt}%
\pgfpathmoveto{\pgfqpoint{1.058608in}{0.892924in}}%
\pgfpathlineto{\pgfqpoint{1.250602in}{0.891272in}}%
\pgfusepath{stroke}%
\end{pgfscope}%
\begin{pgfscope}%
\pgfpathrectangle{\pgfqpoint{0.644217in}{0.500309in}}{\pgfqpoint{2.405783in}{1.149691in}}%
\pgfusepath{clip}%
\pgfsetbuttcap%
\pgfsetroundjoin%
\pgfsetlinewidth{1.505625pt}%
\definecolor{currentstroke}{rgb}{0.827451,0.827451,0.827451}%
\pgfsetstrokecolor{currentstroke}%
\pgfsetdash{{5.550000pt}{2.400000pt}}{0.000000pt}%
\pgfpathmoveto{\pgfqpoint{1.122525in}{0.957494in}}%
\pgfpathlineto{\pgfqpoint{1.073359in}{0.976526in}}%
\pgfusepath{stroke}%
\end{pgfscope}%
\begin{pgfscope}%
\pgfpathrectangle{\pgfqpoint{0.644217in}{0.500309in}}{\pgfqpoint{2.405783in}{1.149691in}}%
\pgfusepath{clip}%
\pgfsetbuttcap%
\pgfsetroundjoin%
\pgfsetlinewidth{1.505625pt}%
\definecolor{currentstroke}{rgb}{0.827451,0.827451,0.827451}%
\pgfsetstrokecolor{currentstroke}%
\pgfsetdash{{5.550000pt}{2.400000pt}}{0.000000pt}%
\pgfpathmoveto{\pgfqpoint{1.220332in}{1.009728in}}%
\pgfpathlineto{\pgfqpoint{1.268378in}{0.992427in}}%
\pgfusepath{stroke}%
\end{pgfscope}%
\begin{pgfscope}%
\pgfpathrectangle{\pgfqpoint{0.644217in}{0.500309in}}{\pgfqpoint{2.405783in}{1.149691in}}%
\pgfusepath{clip}%
\pgfsetbuttcap%
\pgfsetroundjoin%
\pgfsetlinewidth{1.505625pt}%
\definecolor{currentstroke}{rgb}{0.827451,0.827451,0.827451}%
\pgfsetstrokecolor{currentstroke}%
\pgfsetdash{{5.550000pt}{2.400000pt}}{0.000000pt}%
\pgfpathmoveto{\pgfqpoint{1.344093in}{1.067359in}}%
\pgfpathlineto{\pgfqpoint{1.487532in}{1.162977in}}%
\pgfusepath{stroke}%
\end{pgfscope}%
\begin{pgfscope}%
\pgfpathrectangle{\pgfqpoint{0.644217in}{0.500309in}}{\pgfqpoint{2.405783in}{1.149691in}}%
\pgfusepath{clip}%
\pgfsetbuttcap%
\pgfsetroundjoin%
\pgfsetlinewidth{1.505625pt}%
\definecolor{currentstroke}{rgb}{0.827451,0.827451,0.827451}%
\pgfsetstrokecolor{currentstroke}%
\pgfsetdash{{5.550000pt}{2.400000pt}}{0.000000pt}%
\pgfpathmoveto{\pgfqpoint{1.439096in}{1.110657in}}%
\pgfpathlineto{\pgfqpoint{1.375365in}{1.094557in}}%
\pgfusepath{stroke}%
\end{pgfscope}%
\begin{pgfscope}%
\pgfpathrectangle{\pgfqpoint{0.644217in}{0.500309in}}{\pgfqpoint{2.405783in}{1.149691in}}%
\pgfusepath{clip}%
\pgfsetbuttcap%
\pgfsetroundjoin%
\pgfsetlinewidth{1.505625pt}%
\definecolor{currentstroke}{rgb}{0.827451,0.827451,0.827451}%
\pgfsetstrokecolor{currentstroke}%
\pgfsetdash{{5.550000pt}{2.400000pt}}{0.000000pt}%
\pgfpathmoveto{\pgfqpoint{1.498532in}{1.157791in}}%
\pgfpathlineto{\pgfqpoint{1.671378in}{1.274453in}}%
\pgfusepath{stroke}%
\end{pgfscope}%
\begin{pgfscope}%
\pgfpathrectangle{\pgfqpoint{0.644217in}{0.500309in}}{\pgfqpoint{2.405783in}{1.149691in}}%
\pgfusepath{clip}%
\pgfsetbuttcap%
\pgfsetroundjoin%
\pgfsetlinewidth{1.505625pt}%
\definecolor{currentstroke}{rgb}{0.827451,0.827451,0.827451}%
\pgfsetstrokecolor{currentstroke}%
\pgfsetdash{{5.550000pt}{2.400000pt}}{0.000000pt}%
\pgfpathmoveto{\pgfqpoint{1.574448in}{1.233248in}}%
\pgfpathlineto{\pgfqpoint{1.831294in}{1.390311in}}%
\pgfusepath{stroke}%
\end{pgfscope}%
\begin{pgfscope}%
\pgfpathrectangle{\pgfqpoint{0.644217in}{0.500309in}}{\pgfqpoint{2.405783in}{1.149691in}}%
\pgfusepath{clip}%
\pgfsetbuttcap%
\pgfsetroundjoin%
\pgfsetlinewidth{1.505625pt}%
\definecolor{currentstroke}{rgb}{0.827451,0.827451,0.827451}%
\pgfsetstrokecolor{currentstroke}%
\pgfsetdash{{5.550000pt}{2.400000pt}}{0.000000pt}%
\pgfpathmoveto{\pgfqpoint{1.687910in}{1.335776in}}%
\pgfpathlineto{\pgfqpoint{1.690436in}{1.377139in}}%
\pgfusepath{stroke}%
\end{pgfscope}%
\begin{pgfscope}%
\pgfpathrectangle{\pgfqpoint{0.644217in}{0.500309in}}{\pgfqpoint{2.405783in}{1.149691in}}%
\pgfusepath{clip}%
\pgfsetbuttcap%
\pgfsetroundjoin%
\pgfsetlinewidth{1.505625pt}%
\definecolor{currentstroke}{rgb}{0.827451,0.827451,0.827451}%
\pgfsetstrokecolor{currentstroke}%
\pgfsetdash{{5.550000pt}{2.400000pt}}{0.000000pt}%
\pgfpathmoveto{\pgfqpoint{1.832497in}{1.396627in}}%
\pgfpathlineto{\pgfqpoint{1.887722in}{1.406476in}}%
\pgfusepath{stroke}%
\end{pgfscope}%
\begin{pgfscope}%
\pgfpathrectangle{\pgfqpoint{0.644217in}{0.500309in}}{\pgfqpoint{2.405783in}{1.149691in}}%
\pgfusepath{clip}%
\pgfsetbuttcap%
\pgfsetroundjoin%
\pgfsetlinewidth{1.505625pt}%
\definecolor{currentstroke}{rgb}{0.827451,0.827451,0.827451}%
\pgfsetstrokecolor{currentstroke}%
\pgfsetdash{{5.550000pt}{2.400000pt}}{0.000000pt}%
\pgfpathmoveto{\pgfqpoint{2.011849in}{1.394304in}}%
\pgfpathlineto{\pgfqpoint{1.995308in}{1.419435in}}%
\pgfusepath{stroke}%
\end{pgfscope}%
\begin{pgfscope}%
\pgfpathrectangle{\pgfqpoint{0.644217in}{0.500309in}}{\pgfqpoint{2.405783in}{1.149691in}}%
\pgfusepath{clip}%
\pgfsetbuttcap%
\pgfsetroundjoin%
\pgfsetlinewidth{1.505625pt}%
\definecolor{currentstroke}{rgb}{0.827451,0.827451,0.827451}%
\pgfsetstrokecolor{currentstroke}%
\pgfsetdash{{5.550000pt}{2.400000pt}}{0.000000pt}%
\pgfpathmoveto{\pgfqpoint{2.175642in}{1.404142in}}%
\pgfpathlineto{\pgfqpoint{2.183892in}{1.460243in}}%
\pgfusepath{stroke}%
\end{pgfscope}%
\begin{pgfscope}%
\pgfpathrectangle{\pgfqpoint{0.644217in}{0.500309in}}{\pgfqpoint{2.405783in}{1.149691in}}%
\pgfusepath{clip}%
\pgfsetbuttcap%
\pgfsetroundjoin%
\pgfsetlinewidth{1.505625pt}%
\definecolor{currentstroke}{rgb}{0.827451,0.827451,0.827451}%
\pgfsetstrokecolor{currentstroke}%
\pgfsetdash{{5.550000pt}{2.400000pt}}{0.000000pt}%
\pgfpathmoveto{\pgfqpoint{2.330917in}{1.417184in}}%
\pgfpathlineto{\pgfqpoint{2.316436in}{1.474903in}}%
\pgfusepath{stroke}%
\end{pgfscope}%
\begin{pgfscope}%
\pgfpathrectangle{\pgfqpoint{0.644217in}{0.500309in}}{\pgfqpoint{2.405783in}{1.149691in}}%
\pgfusepath{clip}%
\pgfsetbuttcap%
\pgfsetroundjoin%
\pgfsetlinewidth{1.505625pt}%
\definecolor{currentstroke}{rgb}{0.827451,0.827451,0.827451}%
\pgfsetstrokecolor{currentstroke}%
\pgfsetdash{{5.550000pt}{2.400000pt}}{0.000000pt}%
\pgfpathmoveto{\pgfqpoint{2.487173in}{1.421035in}}%
\pgfpathlineto{\pgfqpoint{2.566689in}{1.490435in}}%
\pgfusepath{stroke}%
\end{pgfscope}%
\begin{pgfscope}%
\pgfpathrectangle{\pgfqpoint{0.644217in}{0.500309in}}{\pgfqpoint{2.405783in}{1.149691in}}%
\pgfusepath{clip}%
\pgfsetbuttcap%
\pgfsetroundjoin%
\pgfsetlinewidth{1.505625pt}%
\definecolor{currentstroke}{rgb}{0.827451,0.827451,0.827451}%
\pgfsetstrokecolor{currentstroke}%
\pgfsetdash{{5.550000pt}{2.400000pt}}{0.000000pt}%
\pgfpathmoveto{\pgfqpoint{2.669357in}{1.438916in}}%
\pgfpathlineto{\pgfqpoint{2.668523in}{1.388082in}}%
\pgfusepath{stroke}%
\end{pgfscope}%
\begin{pgfscope}%
\pgfpathrectangle{\pgfqpoint{0.644217in}{0.500309in}}{\pgfqpoint{2.405783in}{1.149691in}}%
\pgfusepath{clip}%
\pgfsetbuttcap%
\pgfsetroundjoin%
\pgfsetlinewidth{1.505625pt}%
\definecolor{currentstroke}{rgb}{0.827451,0.827451,0.827451}%
\pgfsetstrokecolor{currentstroke}%
\pgfsetdash{{5.550000pt}{2.400000pt}}{0.000000pt}%
\pgfpathmoveto{\pgfqpoint{2.851356in}{1.471237in}}%
\pgfpathlineto{\pgfqpoint{2.900510in}{1.510742in}}%
\pgfusepath{stroke}%
\end{pgfscope}%
\begin{pgfscope}%
\pgfpathrectangle{\pgfqpoint{0.644217in}{0.500309in}}{\pgfqpoint{2.405783in}{1.149691in}}%
\pgfusepath{clip}%
\pgfsetbuttcap%
\pgfsetroundjoin%
\pgfsetlinewidth{1.505625pt}%
\definecolor{currentstroke}{rgb}{0.827451,0.827451,0.827451}%
\pgfsetstrokecolor{currentstroke}%
\pgfsetdash{{5.550000pt}{2.400000pt}}{0.000000pt}%
\pgfpathmoveto{\pgfqpoint{0.890286in}{0.742031in}}%
\pgfpathlineto{\pgfqpoint{1.083771in}{0.775637in}}%
\pgfusepath{stroke}%
\end{pgfscope}%
\begin{pgfscope}%
\pgfpathrectangle{\pgfqpoint{0.644217in}{0.500309in}}{\pgfqpoint{2.405783in}{1.149691in}}%
\pgfusepath{clip}%
\pgfsetbuttcap%
\pgfsetroundjoin%
\pgfsetlinewidth{1.505625pt}%
\definecolor{currentstroke}{rgb}{0.827451,0.827451,0.827451}%
\pgfsetstrokecolor{currentstroke}%
\pgfsetdash{{5.550000pt}{2.400000pt}}{0.000000pt}%
\pgfpathmoveto{\pgfqpoint{0.970421in}{0.830136in}}%
\pgfpathlineto{\pgfqpoint{1.000922in}{0.846137in}}%
\pgfusepath{stroke}%
\end{pgfscope}%
\begin{pgfscope}%
\pgfpathrectangle{\pgfqpoint{0.644217in}{0.500309in}}{\pgfqpoint{2.405783in}{1.149691in}}%
\pgfusepath{clip}%
\pgfsetbuttcap%
\pgfsetroundjoin%
\pgfsetlinewidth{1.505625pt}%
\definecolor{currentstroke}{rgb}{0.827451,0.827451,0.827451}%
\pgfsetstrokecolor{currentstroke}%
\pgfsetdash{{5.550000pt}{2.400000pt}}{0.000000pt}%
\pgfpathmoveto{\pgfqpoint{1.058608in}{0.892924in}}%
\pgfpathlineto{\pgfqpoint{1.005037in}{0.814013in}}%
\pgfusepath{stroke}%
\end{pgfscope}%
\begin{pgfscope}%
\pgfpathrectangle{\pgfqpoint{0.644217in}{0.500309in}}{\pgfqpoint{2.405783in}{1.149691in}}%
\pgfusepath{clip}%
\pgfsetbuttcap%
\pgfsetroundjoin%
\pgfsetlinewidth{1.505625pt}%
\definecolor{currentstroke}{rgb}{0.827451,0.827451,0.827451}%
\pgfsetstrokecolor{currentstroke}%
\pgfsetdash{{5.550000pt}{2.400000pt}}{0.000000pt}%
\pgfpathmoveto{\pgfqpoint{1.122525in}{0.957494in}}%
\pgfpathlineto{\pgfqpoint{1.261444in}{1.014665in}}%
\pgfusepath{stroke}%
\end{pgfscope}%
\begin{pgfscope}%
\pgfpathrectangle{\pgfqpoint{0.644217in}{0.500309in}}{\pgfqpoint{2.405783in}{1.149691in}}%
\pgfusepath{clip}%
\pgfsetbuttcap%
\pgfsetroundjoin%
\pgfsetlinewidth{1.505625pt}%
\definecolor{currentstroke}{rgb}{0.827451,0.827451,0.827451}%
\pgfsetstrokecolor{currentstroke}%
\pgfsetdash{{5.550000pt}{2.400000pt}}{0.000000pt}%
\pgfpathmoveto{\pgfqpoint{1.220332in}{1.009728in}}%
\pgfpathlineto{\pgfqpoint{1.126048in}{1.002406in}}%
\pgfusepath{stroke}%
\end{pgfscope}%
\begin{pgfscope}%
\pgfpathrectangle{\pgfqpoint{0.644217in}{0.500309in}}{\pgfqpoint{2.405783in}{1.149691in}}%
\pgfusepath{clip}%
\pgfsetbuttcap%
\pgfsetroundjoin%
\pgfsetlinewidth{1.505625pt}%
\definecolor{currentstroke}{rgb}{0.827451,0.827451,0.827451}%
\pgfsetstrokecolor{currentstroke}%
\pgfsetdash{{5.550000pt}{2.400000pt}}{0.000000pt}%
\pgfpathmoveto{\pgfqpoint{1.344093in}{1.067359in}}%
\pgfpathlineto{\pgfqpoint{1.452032in}{1.079726in}}%
\pgfusepath{stroke}%
\end{pgfscope}%
\begin{pgfscope}%
\pgfpathrectangle{\pgfqpoint{0.644217in}{0.500309in}}{\pgfqpoint{2.405783in}{1.149691in}}%
\pgfusepath{clip}%
\pgfsetbuttcap%
\pgfsetroundjoin%
\pgfsetlinewidth{1.505625pt}%
\definecolor{currentstroke}{rgb}{0.827451,0.827451,0.827451}%
\pgfsetstrokecolor{currentstroke}%
\pgfsetdash{{5.550000pt}{2.400000pt}}{0.000000pt}%
\pgfpathmoveto{\pgfqpoint{1.439096in}{1.110657in}}%
\pgfpathlineto{\pgfqpoint{1.236411in}{1.024797in}}%
\pgfusepath{stroke}%
\end{pgfscope}%
\begin{pgfscope}%
\pgfpathrectangle{\pgfqpoint{0.644217in}{0.500309in}}{\pgfqpoint{2.405783in}{1.149691in}}%
\pgfusepath{clip}%
\pgfsetbuttcap%
\pgfsetroundjoin%
\pgfsetlinewidth{1.505625pt}%
\definecolor{currentstroke}{rgb}{0.827451,0.827451,0.827451}%
\pgfsetstrokecolor{currentstroke}%
\pgfsetdash{{5.550000pt}{2.400000pt}}{0.000000pt}%
\pgfpathmoveto{\pgfqpoint{1.498532in}{1.157791in}}%
\pgfpathlineto{\pgfqpoint{1.570606in}{1.129107in}}%
\pgfusepath{stroke}%
\end{pgfscope}%
\begin{pgfscope}%
\pgfpathrectangle{\pgfqpoint{0.644217in}{0.500309in}}{\pgfqpoint{2.405783in}{1.149691in}}%
\pgfusepath{clip}%
\pgfsetbuttcap%
\pgfsetroundjoin%
\pgfsetlinewidth{1.505625pt}%
\definecolor{currentstroke}{rgb}{0.827451,0.827451,0.827451}%
\pgfsetstrokecolor{currentstroke}%
\pgfsetdash{{5.550000pt}{2.400000pt}}{0.000000pt}%
\pgfpathmoveto{\pgfqpoint{1.574448in}{1.233248in}}%
\pgfpathlineto{\pgfqpoint{1.555322in}{1.204368in}}%
\pgfusepath{stroke}%
\end{pgfscope}%
\begin{pgfscope}%
\pgfpathrectangle{\pgfqpoint{0.644217in}{0.500309in}}{\pgfqpoint{2.405783in}{1.149691in}}%
\pgfusepath{clip}%
\pgfsetbuttcap%
\pgfsetroundjoin%
\pgfsetlinewidth{1.505625pt}%
\definecolor{currentstroke}{rgb}{0.827451,0.827451,0.827451}%
\pgfsetstrokecolor{currentstroke}%
\pgfsetdash{{5.550000pt}{2.400000pt}}{0.000000pt}%
\pgfpathmoveto{\pgfqpoint{1.687910in}{1.335776in}}%
\pgfpathlineto{\pgfqpoint{1.719954in}{1.362053in}}%
\pgfusepath{stroke}%
\end{pgfscope}%
\begin{pgfscope}%
\pgfpathrectangle{\pgfqpoint{0.644217in}{0.500309in}}{\pgfqpoint{2.405783in}{1.149691in}}%
\pgfusepath{clip}%
\pgfsetbuttcap%
\pgfsetroundjoin%
\pgfsetlinewidth{1.505625pt}%
\definecolor{currentstroke}{rgb}{0.827451,0.827451,0.827451}%
\pgfsetstrokecolor{currentstroke}%
\pgfsetdash{{5.550000pt}{2.400000pt}}{0.000000pt}%
\pgfpathmoveto{\pgfqpoint{1.832497in}{1.396627in}}%
\pgfpathlineto{\pgfqpoint{1.868933in}{1.389773in}}%
\pgfusepath{stroke}%
\end{pgfscope}%
\begin{pgfscope}%
\pgfpathrectangle{\pgfqpoint{0.644217in}{0.500309in}}{\pgfqpoint{2.405783in}{1.149691in}}%
\pgfusepath{clip}%
\pgfsetbuttcap%
\pgfsetroundjoin%
\pgfsetlinewidth{1.505625pt}%
\definecolor{currentstroke}{rgb}{0.827451,0.827451,0.827451}%
\pgfsetstrokecolor{currentstroke}%
\pgfsetdash{{5.550000pt}{2.400000pt}}{0.000000pt}%
\pgfpathmoveto{\pgfqpoint{2.011849in}{1.394304in}}%
\pgfpathlineto{\pgfqpoint{2.136396in}{1.463526in}}%
\pgfusepath{stroke}%
\end{pgfscope}%
\begin{pgfscope}%
\pgfpathrectangle{\pgfqpoint{0.644217in}{0.500309in}}{\pgfqpoint{2.405783in}{1.149691in}}%
\pgfusepath{clip}%
\pgfsetbuttcap%
\pgfsetroundjoin%
\pgfsetlinewidth{1.505625pt}%
\definecolor{currentstroke}{rgb}{0.827451,0.827451,0.827451}%
\pgfsetstrokecolor{currentstroke}%
\pgfsetdash{{5.550000pt}{2.400000pt}}{0.000000pt}%
\pgfpathmoveto{\pgfqpoint{2.175642in}{1.404142in}}%
\pgfpathlineto{\pgfqpoint{1.895037in}{1.345937in}}%
\pgfusepath{stroke}%
\end{pgfscope}%
\begin{pgfscope}%
\pgfpathrectangle{\pgfqpoint{0.644217in}{0.500309in}}{\pgfqpoint{2.405783in}{1.149691in}}%
\pgfusepath{clip}%
\pgfsetbuttcap%
\pgfsetroundjoin%
\pgfsetlinewidth{1.505625pt}%
\definecolor{currentstroke}{rgb}{0.827451,0.827451,0.827451}%
\pgfsetstrokecolor{currentstroke}%
\pgfsetdash{{5.550000pt}{2.400000pt}}{0.000000pt}%
\pgfpathmoveto{\pgfqpoint{2.330917in}{1.417184in}}%
\pgfpathlineto{\pgfqpoint{2.305854in}{1.399918in}}%
\pgfusepath{stroke}%
\end{pgfscope}%
\begin{pgfscope}%
\pgfpathrectangle{\pgfqpoint{0.644217in}{0.500309in}}{\pgfqpoint{2.405783in}{1.149691in}}%
\pgfusepath{clip}%
\pgfsetbuttcap%
\pgfsetroundjoin%
\pgfsetlinewidth{1.505625pt}%
\definecolor{currentstroke}{rgb}{0.827451,0.827451,0.827451}%
\pgfsetstrokecolor{currentstroke}%
\pgfsetdash{{5.550000pt}{2.400000pt}}{0.000000pt}%
\pgfpathmoveto{\pgfqpoint{2.487173in}{1.421035in}}%
\pgfpathlineto{\pgfqpoint{2.375425in}{1.379279in}}%
\pgfusepath{stroke}%
\end{pgfscope}%
\begin{pgfscope}%
\pgfpathrectangle{\pgfqpoint{0.644217in}{0.500309in}}{\pgfqpoint{2.405783in}{1.149691in}}%
\pgfusepath{clip}%
\pgfsetbuttcap%
\pgfsetroundjoin%
\pgfsetlinewidth{1.505625pt}%
\definecolor{currentstroke}{rgb}{0.827451,0.827451,0.827451}%
\pgfsetstrokecolor{currentstroke}%
\pgfsetdash{{5.550000pt}{2.400000pt}}{0.000000pt}%
\pgfpathmoveto{\pgfqpoint{2.669357in}{1.438916in}}%
\pgfpathlineto{\pgfqpoint{2.635004in}{1.425169in}}%
\pgfusepath{stroke}%
\end{pgfscope}%
\begin{pgfscope}%
\pgfpathrectangle{\pgfqpoint{0.644217in}{0.500309in}}{\pgfqpoint{2.405783in}{1.149691in}}%
\pgfusepath{clip}%
\pgfsetbuttcap%
\pgfsetroundjoin%
\pgfsetlinewidth{1.505625pt}%
\definecolor{currentstroke}{rgb}{0.827451,0.827451,0.827451}%
\pgfsetstrokecolor{currentstroke}%
\pgfsetdash{{5.550000pt}{2.400000pt}}{0.000000pt}%
\pgfpathmoveto{\pgfqpoint{2.851356in}{1.471237in}}%
\pgfpathlineto{\pgfqpoint{2.911704in}{1.493221in}}%
\pgfusepath{stroke}%
\end{pgfscope}%
\begin{pgfscope}%
\pgfpathrectangle{\pgfqpoint{0.644217in}{0.500309in}}{\pgfqpoint{2.405783in}{1.149691in}}%
\pgfusepath{clip}%
\pgfsetbuttcap%
\pgfsetmiterjoin%
\pgfsetlinewidth{1.003750pt}%
\definecolor{currentstroke}{rgb}{1.000000,0.000000,0.000000}%
\pgfsetstrokecolor{currentstroke}%
\pgfsetdash{}{0pt}%
\pgfpathmoveto{\pgfqpoint{0.949189in}{0.908243in}}%
\pgfpathcurveto{\pgfqpoint{0.912177in}{0.886333in}}{\pgfqpoint{0.882430in}{0.858612in}}{\pgfqpoint{0.866500in}{0.831186in}}%
\pgfpathcurveto{\pgfqpoint{0.850570in}{0.803759in}}{\pgfqpoint{0.849758in}{0.778867in}}{\pgfqpoint{0.864242in}{0.761990in}}%
\pgfpathcurveto{\pgfqpoint{0.878726in}{0.745113in}}{\pgfqpoint{0.907325in}{0.737631in}}{\pgfqpoint{0.943739in}{0.741190in}}%
\pgfpathcurveto{\pgfqpoint{0.980153in}{0.744749in}}{\pgfqpoint{1.021409in}{0.759060in}}{\pgfqpoint{1.058422in}{0.780970in}}%
\pgfpathcurveto{\pgfqpoint{1.095435in}{0.802880in}}{\pgfqpoint{1.125182in}{0.830601in}}{\pgfqpoint{1.141112in}{0.858027in}}%
\pgfpathcurveto{\pgfqpoint{1.157042in}{0.885453in}}{\pgfqpoint{1.157854in}{0.910346in}}{\pgfqpoint{1.143369in}{0.927223in}}%
\pgfpathcurveto{\pgfqpoint{1.128885in}{0.944099in}}{\pgfqpoint{1.100286in}{0.951582in}}{\pgfqpoint{1.063872in}{0.948023in}}%
\pgfpathcurveto{\pgfqpoint{1.027458in}{0.944464in}}{\pgfqpoint{0.986202in}{0.930153in}}{\pgfqpoint{0.949189in}{0.908243in}}%
\pgfpathclose%
\pgfusepath{stroke}%
\end{pgfscope}%
\begin{pgfscope}%
\pgfpathrectangle{\pgfqpoint{0.644217in}{0.500309in}}{\pgfqpoint{2.405783in}{1.149691in}}%
\pgfusepath{clip}%
\pgfsetbuttcap%
\pgfsetmiterjoin%
\pgfsetlinewidth{1.003750pt}%
\definecolor{currentstroke}{rgb}{1.000000,0.000000,0.000000}%
\pgfsetstrokecolor{currentstroke}%
\pgfsetdash{}{0pt}%
\pgfpathmoveto{\pgfqpoint{1.001475in}{0.945777in}}%
\pgfpathcurveto{\pgfqpoint{0.965290in}{0.924379in}}{\pgfqpoint{0.936133in}{0.897381in}}{\pgfqpoint{0.920424in}{0.870729in}}%
\pgfpathcurveto{\pgfqpoint{0.904715in}{0.844077in}}{\pgfqpoint{0.903738in}{0.819948in}}{\pgfqpoint{0.917707in}{0.803655in}}%
\pgfpathcurveto{\pgfqpoint{0.931677in}{0.787362in}}{\pgfqpoint{0.959452in}{0.780236in}}{\pgfqpoint{0.994916in}{0.783846in}}%
\pgfpathcurveto{\pgfqpoint{1.030381in}{0.787457in}}{\pgfqpoint{1.070638in}{0.801508in}}{\pgfqpoint{1.106823in}{0.822907in}}%
\pgfpathcurveto{\pgfqpoint{1.143008in}{0.844305in}}{\pgfqpoint{1.172165in}{0.871304in}}{\pgfqpoint{1.187874in}{0.897955in}}%
\pgfpathcurveto{\pgfqpoint{1.203583in}{0.924607in}}{\pgfqpoint{1.204560in}{0.948737in}}{\pgfqpoint{1.190591in}{0.965029in}}%
\pgfpathcurveto{\pgfqpoint{1.176622in}{0.981322in}}{\pgfqpoint{1.148846in}{0.988448in}}{\pgfqpoint{1.113382in}{0.984838in}}%
\pgfpathcurveto{\pgfqpoint{1.077917in}{0.981228in}}{\pgfqpoint{1.037660in}{0.967176in}}{\pgfqpoint{1.001475in}{0.945777in}}%
\pgfpathclose%
\pgfusepath{stroke}%
\end{pgfscope}%
\begin{pgfscope}%
\pgfpathrectangle{\pgfqpoint{0.644217in}{0.500309in}}{\pgfqpoint{2.405783in}{1.149691in}}%
\pgfusepath{clip}%
\pgfsetbuttcap%
\pgfsetmiterjoin%
\pgfsetlinewidth{1.003750pt}%
\definecolor{currentstroke}{rgb}{1.000000,0.000000,0.000000}%
\pgfsetstrokecolor{currentstroke}%
\pgfsetdash{}{0pt}%
\pgfpathmoveto{\pgfqpoint{1.064850in}{0.986145in}}%
\pgfpathcurveto{\pgfqpoint{1.029958in}{0.964848in}}{\pgfqpoint{1.002168in}{0.938004in}}{\pgfqpoint{0.987600in}{0.911526in}}%
\pgfpathcurveto{\pgfqpoint{0.973033in}{0.885048in}}{\pgfqpoint{0.972877in}{0.861098in}}{\pgfqpoint{0.987167in}{0.844950in}}%
\pgfpathcurveto{\pgfqpoint{1.001457in}{0.828802in}}{\pgfqpoint{1.029027in}{0.821775in}}{\pgfqpoint{1.063804in}{0.825417in}}%
\pgfpathcurveto{\pgfqpoint{1.098581in}{0.829058in}}{\pgfqpoint{1.137726in}{0.843071in}}{\pgfqpoint{1.172618in}{0.864369in}}%
\pgfpathcurveto{\pgfqpoint{1.207510in}{0.885666in}}{\pgfqpoint{1.235300in}{0.912510in}}{\pgfqpoint{1.249868in}{0.938988in}}%
\pgfpathcurveto{\pgfqpoint{1.264436in}{0.965466in}}{\pgfqpoint{1.264592in}{0.989416in}}{\pgfqpoint{1.250302in}{1.005564in}}%
\pgfpathcurveto{\pgfqpoint{1.236012in}{1.021711in}}{\pgfqpoint{1.208442in}{1.028738in}}{\pgfqpoint{1.173665in}{1.025097in}}%
\pgfpathcurveto{\pgfqpoint{1.138888in}{1.021455in}}{\pgfqpoint{1.099742in}{1.007443in}}{\pgfqpoint{1.064850in}{0.986145in}}%
\pgfpathclose%
\pgfusepath{stroke}%
\end{pgfscope}%
\begin{pgfscope}%
\pgfpathrectangle{\pgfqpoint{0.644217in}{0.500309in}}{\pgfqpoint{2.405783in}{1.149691in}}%
\pgfusepath{clip}%
\pgfsetbuttcap%
\pgfsetmiterjoin%
\pgfsetlinewidth{1.003750pt}%
\definecolor{currentstroke}{rgb}{1.000000,0.000000,0.000000}%
\pgfsetstrokecolor{currentstroke}%
\pgfsetdash{}{0pt}%
\pgfpathmoveto{\pgfqpoint{1.127121in}{1.059421in}}%
\pgfpathcurveto{\pgfqpoint{1.093061in}{1.038896in}}{\pgfqpoint{1.065932in}{1.012867in}}{\pgfqpoint{1.051708in}{0.987067in}}%
\pgfpathcurveto{\pgfqpoint{1.037485in}{0.961267in}}{\pgfqpoint{1.037329in}{0.937803in}}{\pgfqpoint{1.051274in}{0.921842in}}%
\pgfpathcurveto{\pgfqpoint{1.065219in}{0.905880in}}{\pgfqpoint{1.092128in}{0.898725in}}{\pgfqpoint{1.126073in}{0.901952in}}%
\pgfpathcurveto{\pgfqpoint{1.160018in}{0.905179in}}{\pgfqpoint{1.198228in}{0.918525in}}{\pgfqpoint{1.232288in}{0.939050in}}%
\pgfpathcurveto{\pgfqpoint{1.266348in}{0.959575in}}{\pgfqpoint{1.293477in}{0.985603in}}{\pgfqpoint{1.307701in}{1.011403in}}%
\pgfpathcurveto{\pgfqpoint{1.321924in}{1.037203in}}{\pgfqpoint{1.322080in}{1.060667in}}{\pgfqpoint{1.308135in}{1.076629in}}%
\pgfpathcurveto{\pgfqpoint{1.294189in}{1.092590in}}{\pgfqpoint{1.267281in}{1.099745in}}{\pgfqpoint{1.233336in}{1.096518in}}%
\pgfpathcurveto{\pgfqpoint{1.199391in}{1.093291in}}{\pgfqpoint{1.161181in}{1.079946in}}{\pgfqpoint{1.127121in}{1.059421in}}%
\pgfpathclose%
\pgfusepath{stroke}%
\end{pgfscope}%
\begin{pgfscope}%
\pgfpathrectangle{\pgfqpoint{0.644217in}{0.500309in}}{\pgfqpoint{2.405783in}{1.149691in}}%
\pgfusepath{clip}%
\pgfsetbuttcap%
\pgfsetmiterjoin%
\pgfsetlinewidth{1.003750pt}%
\definecolor{currentstroke}{rgb}{1.000000,0.000000,0.000000}%
\pgfsetstrokecolor{currentstroke}%
\pgfsetdash{}{0pt}%
\pgfpathmoveto{\pgfqpoint{1.212177in}{1.098670in}}%
\pgfpathcurveto{\pgfqpoint{1.178559in}{1.078614in}}{\pgfqpoint{1.151759in}{1.053082in}}{\pgfqpoint{1.137679in}{1.027696in}}%
\pgfpathcurveto{\pgfqpoint{1.123599in}{1.002311in}}{\pgfqpoint{1.123388in}{0.979145in}}{\pgfqpoint{1.137093in}{0.963300in}}%
\pgfpathcurveto{\pgfqpoint{1.150798in}{0.947456in}}{\pgfqpoint{1.177299in}{0.940226in}}{\pgfqpoint{1.210761in}{0.943204in}}%
\pgfpathcurveto{\pgfqpoint{1.244223in}{0.946182in}}{\pgfqpoint{1.281913in}{0.959124in}}{\pgfqpoint{1.315531in}{0.979180in}}%
\pgfpathcurveto{\pgfqpoint{1.349148in}{0.999235in}}{\pgfqpoint{1.375948in}{1.024768in}}{\pgfqpoint{1.390029in}{1.050153in}}%
\pgfpathcurveto{\pgfqpoint{1.404109in}{1.075538in}}{\pgfqpoint{1.404320in}{1.098704in}}{\pgfqpoint{1.390615in}{1.114549in}}%
\pgfpathcurveto{\pgfqpoint{1.376910in}{1.130393in}}{\pgfqpoint{1.350408in}{1.137623in}}{\pgfqpoint{1.316946in}{1.134645in}}%
\pgfpathcurveto{\pgfqpoint{1.283484in}{1.131667in}}{\pgfqpoint{1.245794in}{1.118725in}}{\pgfqpoint{1.212177in}{1.098670in}}%
\pgfpathclose%
\pgfusepath{stroke}%
\end{pgfscope}%
\begin{pgfscope}%
\pgfpathrectangle{\pgfqpoint{0.644217in}{0.500309in}}{\pgfqpoint{2.405783in}{1.149691in}}%
\pgfusepath{clip}%
\pgfsetbuttcap%
\pgfsetmiterjoin%
\pgfsetlinewidth{1.003750pt}%
\definecolor{currentstroke}{rgb}{1.000000,0.000000,0.000000}%
\pgfsetstrokecolor{currentstroke}%
\pgfsetdash{}{0pt}%
\pgfpathmoveto{\pgfqpoint{1.335751in}{1.105350in}}%
\pgfpathcurveto{\pgfqpoint{1.302361in}{1.085538in}}{\pgfqpoint{1.275728in}{1.060265in}}{\pgfqpoint{1.261717in}{1.035097in}}%
\pgfpathcurveto{\pgfqpoint{1.247706in}{1.009929in}}{\pgfqpoint{1.247461in}{0.986922in}}{\pgfqpoint{1.261036in}{0.971142in}}%
\pgfpathcurveto{\pgfqpoint{1.274612in}{0.955362in}}{\pgfqpoint{1.300899in}{0.948098in}}{\pgfqpoint{1.334108in}{0.950949in}}%
\pgfpathcurveto{\pgfqpoint{1.367317in}{0.953800in}}{\pgfqpoint{1.404738in}{0.966534in}}{\pgfqpoint{1.438128in}{0.986347in}}%
\pgfpathcurveto{\pgfqpoint{1.471517in}{1.006159in}}{\pgfqpoint{1.498151in}{1.031432in}}{\pgfqpoint{1.512162in}{1.056600in}}%
\pgfpathcurveto{\pgfqpoint{1.526173in}{1.081767in}}{\pgfqpoint{1.526417in}{1.104775in}}{\pgfqpoint{1.512842in}{1.120555in}}%
\pgfpathcurveto{\pgfqpoint{1.499267in}{1.136335in}}{\pgfqpoint{1.472980in}{1.143599in}}{\pgfqpoint{1.439770in}{1.140748in}}%
\pgfpathcurveto{\pgfqpoint{1.406561in}{1.137896in}}{\pgfqpoint{1.369141in}{1.125162in}}{\pgfqpoint{1.335751in}{1.105350in}}%
\pgfpathclose%
\pgfusepath{stroke}%
\end{pgfscope}%
\begin{pgfscope}%
\pgfpathrectangle{\pgfqpoint{0.644217in}{0.500309in}}{\pgfqpoint{2.405783in}{1.149691in}}%
\pgfusepath{clip}%
\pgfsetbuttcap%
\pgfsetmiterjoin%
\pgfsetlinewidth{1.003750pt}%
\definecolor{currentstroke}{rgb}{1.000000,0.000000,0.000000}%
\pgfsetstrokecolor{currentstroke}%
\pgfsetdash{}{0pt}%
\pgfpathmoveto{\pgfqpoint{1.341692in}{1.123661in}}%
\pgfpathcurveto{\pgfqpoint{1.308421in}{1.103975in}}{\pgfqpoint{1.281875in}{1.078837in}}{\pgfqpoint{1.267899in}{1.053784in}}%
\pgfpathcurveto{\pgfqpoint{1.253923in}{1.028732in}}{\pgfqpoint{1.253659in}{1.005809in}}{\pgfqpoint{1.267165in}{0.990065in}}%
\pgfpathcurveto{\pgfqpoint{1.280672in}{0.974322in}}{\pgfqpoint{1.306845in}{0.967042in}}{\pgfqpoint{1.339921in}{0.969829in}}%
\pgfpathcurveto{\pgfqpoint{1.372997in}{0.972617in}}{\pgfqpoint{1.410276in}{0.985244in}}{\pgfqpoint{1.443546in}{1.004931in}}%
\pgfpathcurveto{\pgfqpoint{1.476817in}{1.024617in}}{\pgfqpoint{1.503364in}{1.049754in}}{\pgfqpoint{1.517339in}{1.074807in}}%
\pgfpathcurveto{\pgfqpoint{1.531315in}{1.099860in}}{\pgfqpoint{1.531579in}{1.122782in}}{\pgfqpoint{1.518073in}{1.138526in}}%
\pgfpathcurveto{\pgfqpoint{1.504567in}{1.154270in}}{\pgfqpoint{1.478394in}{1.161549in}}{\pgfqpoint{1.445317in}{1.158762in}}%
\pgfpathcurveto{\pgfqpoint{1.412241in}{1.155974in}}{\pgfqpoint{1.374963in}{1.143347in}}{\pgfqpoint{1.341692in}{1.123661in}}%
\pgfpathclose%
\pgfusepath{stroke}%
\end{pgfscope}%
\begin{pgfscope}%
\pgfpathrectangle{\pgfqpoint{0.644217in}{0.500309in}}{\pgfqpoint{2.405783in}{1.149691in}}%
\pgfusepath{clip}%
\pgfsetbuttcap%
\pgfsetmiterjoin%
\pgfsetlinewidth{1.003750pt}%
\definecolor{currentstroke}{rgb}{1.000000,0.000000,0.000000}%
\pgfsetstrokecolor{currentstroke}%
\pgfsetdash{}{0pt}%
\pgfpathmoveto{\pgfqpoint{1.512158in}{1.231366in}}%
\pgfpathcurveto{\pgfqpoint{1.478951in}{1.211746in}}{\pgfqpoint{1.452450in}{1.186681in}}{\pgfqpoint{1.438493in}{1.161690in}}%
\pgfpathcurveto{\pgfqpoint{1.424535in}{1.136699in}}{\pgfqpoint{1.424260in}{1.113823in}}{\pgfqpoint{1.437729in}{1.098100in}}%
\pgfpathcurveto{\pgfqpoint{1.451197in}{1.082377in}}{\pgfqpoint{1.477309in}{1.075091in}}{\pgfqpoint{1.510313in}{1.077846in}}%
\pgfpathcurveto{\pgfqpoint{1.543318in}{1.080602in}}{\pgfqpoint{1.580520in}{1.093173in}}{\pgfqpoint{1.613728in}{1.112793in}}%
\pgfpathcurveto{\pgfqpoint{1.646935in}{1.132412in}}{\pgfqpoint{1.673435in}{1.157478in}}{\pgfqpoint{1.687393in}{1.182469in}}%
\pgfpathcurveto{\pgfqpoint{1.701350in}{1.207459in}}{\pgfqpoint{1.701625in}{1.230335in}}{\pgfqpoint{1.688157in}{1.246058in}}%
\pgfpathcurveto{\pgfqpoint{1.674689in}{1.261781in}}{\pgfqpoint{1.648577in}{1.269068in}}{\pgfqpoint{1.615572in}{1.266312in}}%
\pgfpathcurveto{\pgfqpoint{1.582568in}{1.263557in}}{\pgfqpoint{1.545365in}{1.250985in}}{\pgfqpoint{1.512158in}{1.231366in}}%
\pgfpathclose%
\pgfusepath{stroke}%
\end{pgfscope}%
\begin{pgfscope}%
\pgfpathrectangle{\pgfqpoint{0.644217in}{0.500309in}}{\pgfqpoint{2.405783in}{1.149691in}}%
\pgfusepath{clip}%
\pgfsetbuttcap%
\pgfsetmiterjoin%
\pgfsetlinewidth{1.003750pt}%
\definecolor{currentstroke}{rgb}{1.000000,0.000000,0.000000}%
\pgfsetstrokecolor{currentstroke}%
\pgfsetdash{}{0pt}%
\pgfpathmoveto{\pgfqpoint{1.543419in}{1.298045in}}%
\pgfpathcurveto{\pgfqpoint{1.510246in}{1.278461in}}{\pgfqpoint{1.483770in}{1.253434in}}{\pgfqpoint{1.469822in}{1.228477in}}%
\pgfpathcurveto{\pgfqpoint{1.455874in}{1.203520in}}{\pgfqpoint{1.455593in}{1.180670in}}{\pgfqpoint{1.469040in}{1.164959in}}%
\pgfpathcurveto{\pgfqpoint{1.482487in}{1.149249in}}{\pgfqpoint{1.508565in}{1.141960in}}{\pgfqpoint{1.541531in}{1.144699in}}%
\pgfpathcurveto{\pgfqpoint{1.574496in}{1.147437in}}{\pgfqpoint{1.611657in}{1.159980in}}{\pgfqpoint{1.644830in}{1.179564in}}%
\pgfpathcurveto{\pgfqpoint{1.678003in}{1.199148in}}{\pgfqpoint{1.704479in}{1.224174in}}{\pgfqpoint{1.718427in}{1.249131in}}%
\pgfpathcurveto{\pgfqpoint{1.732375in}{1.274088in}}{\pgfqpoint{1.732656in}{1.296938in}}{\pgfqpoint{1.719209in}{1.312649in}}%
\pgfpathcurveto{\pgfqpoint{1.705762in}{1.328360in}}{\pgfqpoint{1.679684in}{1.335648in}}{\pgfqpoint{1.646718in}{1.332910in}}%
\pgfpathcurveto{\pgfqpoint{1.613753in}{1.330171in}}{\pgfqpoint{1.576592in}{1.317628in}}{\pgfqpoint{1.543419in}{1.298045in}}%
\pgfpathclose%
\pgfusepath{stroke}%
\end{pgfscope}%
\begin{pgfscope}%
\pgfpathrectangle{\pgfqpoint{0.644217in}{0.500309in}}{\pgfqpoint{2.405783in}{1.149691in}}%
\pgfusepath{clip}%
\pgfsetbuttcap%
\pgfsetmiterjoin%
\pgfsetlinewidth{1.003750pt}%
\definecolor{currentstroke}{rgb}{1.000000,0.000000,0.000000}%
\pgfsetstrokecolor{currentstroke}%
\pgfsetdash{}{0pt}%
\pgfpathmoveto{\pgfqpoint{1.570070in}{1.414044in}}%
\pgfpathcurveto{\pgfqpoint{1.536916in}{1.394480in}}{\pgfqpoint{1.510453in}{1.369475in}}{\pgfqpoint{1.496510in}{1.344537in}}%
\pgfpathcurveto{\pgfqpoint{1.482567in}{1.319598in}}{\pgfqpoint{1.482282in}{1.296763in}}{\pgfqpoint{1.495717in}{1.281059in}}%
\pgfpathcurveto{\pgfqpoint{1.509153in}{1.265356in}}{\pgfqpoint{1.535212in}{1.258066in}}{\pgfqpoint{1.568156in}{1.260797in}}%
\pgfpathcurveto{\pgfqpoint{1.601100in}{1.263527in}}{\pgfqpoint{1.638238in}{1.276053in}}{\pgfqpoint{1.671392in}{1.295618in}}%
\pgfpathcurveto{\pgfqpoint{1.704546in}{1.315183in}}{\pgfqpoint{1.731009in}{1.340188in}}{\pgfqpoint{1.744952in}{1.365126in}}%
\pgfpathcurveto{\pgfqpoint{1.758895in}{1.390064in}}{\pgfqpoint{1.759180in}{1.412900in}}{\pgfqpoint{1.745745in}{1.428603in}}%
\pgfpathcurveto{\pgfqpoint{1.732309in}{1.444307in}}{\pgfqpoint{1.706250in}{1.451596in}}{\pgfqpoint{1.673306in}{1.448866in}}%
\pgfpathcurveto{\pgfqpoint{1.640363in}{1.446136in}}{\pgfqpoint{1.603224in}{1.433609in}}{\pgfqpoint{1.570070in}{1.414044in}}%
\pgfpathclose%
\pgfusepath{stroke}%
\end{pgfscope}%
\begin{pgfscope}%
\pgfpathrectangle{\pgfqpoint{0.644217in}{0.500309in}}{\pgfqpoint{2.405783in}{1.149691in}}%
\pgfusepath{clip}%
\pgfsetbuttcap%
\pgfsetmiterjoin%
\pgfsetlinewidth{1.003750pt}%
\definecolor{currentstroke}{rgb}{1.000000,0.000000,0.000000}%
\pgfsetstrokecolor{currentstroke}%
\pgfsetdash{}{0pt}%
\pgfpathmoveto{\pgfqpoint{1.841523in}{1.446121in}}%
\pgfpathcurveto{\pgfqpoint{1.808380in}{1.426567in}}{\pgfqpoint{1.781924in}{1.401573in}}{\pgfqpoint{1.767984in}{1.376646in}}%
\pgfpathcurveto{\pgfqpoint{1.754043in}{1.351718in}}{\pgfqpoint{1.753756in}{1.328891in}}{\pgfqpoint{1.767184in}{1.313191in}}%
\pgfpathcurveto{\pgfqpoint{1.780613in}{1.297492in}}{\pgfqpoint{1.806662in}{1.290203in}}{\pgfqpoint{1.839594in}{1.292928in}}%
\pgfpathcurveto{\pgfqpoint{1.872525in}{1.295654in}}{\pgfqpoint{1.909651in}{1.308173in}}{\pgfqpoint{1.942795in}{1.327727in}}%
\pgfpathcurveto{\pgfqpoint{1.975939in}{1.347281in}}{\pgfqpoint{2.002394in}{1.372274in}}{\pgfqpoint{2.016335in}{1.397202in}}%
\pgfpathcurveto{\pgfqpoint{2.030275in}{1.422130in}}{\pgfqpoint{2.030563in}{1.444957in}}{\pgfqpoint{2.017134in}{1.460656in}}%
\pgfpathcurveto{\pgfqpoint{2.003705in}{1.476355in}}{\pgfqpoint{1.977657in}{1.483645in}}{\pgfqpoint{1.944725in}{1.480919in}}%
\pgfpathcurveto{\pgfqpoint{1.911793in}{1.478193in}}{\pgfqpoint{1.874667in}{1.465675in}}{\pgfqpoint{1.841523in}{1.446121in}}%
\pgfpathclose%
\pgfusepath{stroke}%
\end{pgfscope}%
\begin{pgfscope}%
\pgfpathrectangle{\pgfqpoint{0.644217in}{0.500309in}}{\pgfqpoint{2.405783in}{1.149691in}}%
\pgfusepath{clip}%
\pgfsetbuttcap%
\pgfsetmiterjoin%
\pgfsetlinewidth{1.003750pt}%
\definecolor{currentstroke}{rgb}{1.000000,0.000000,0.000000}%
\pgfsetstrokecolor{currentstroke}%
\pgfsetdash{}{0pt}%
\pgfpathmoveto{\pgfqpoint{1.999099in}{1.452913in}}%
\pgfpathcurveto{\pgfqpoint{1.965961in}{1.433364in}}{\pgfqpoint{1.939510in}{1.408378in}}{\pgfqpoint{1.925571in}{1.383456in}}%
\pgfpathcurveto{\pgfqpoint{1.911631in}{1.358534in}}{\pgfqpoint{1.911342in}{1.335711in}}{\pgfqpoint{1.924767in}{1.320015in}}%
\pgfpathcurveto{\pgfqpoint{1.938192in}{1.304318in}}{\pgfqpoint{1.964235in}{1.297029in}}{\pgfqpoint{1.997160in}{1.299752in}}%
\pgfpathcurveto{\pgfqpoint{2.030085in}{1.302476in}}{\pgfqpoint{2.067204in}{1.314990in}}{\pgfqpoint{2.100342in}{1.334538in}}%
\pgfpathcurveto{\pgfqpoint{2.133480in}{1.354086in}}{\pgfqpoint{2.159931in}{1.379073in}}{\pgfqpoint{2.173871in}{1.403995in}}%
\pgfpathcurveto{\pgfqpoint{2.187810in}{1.428917in}}{\pgfqpoint{2.188099in}{1.451740in}}{\pgfqpoint{2.174674in}{1.467436in}}%
\pgfpathcurveto{\pgfqpoint{2.161249in}{1.483133in}}{\pgfqpoint{2.135206in}{1.490422in}}{\pgfqpoint{2.102281in}{1.487699in}}%
\pgfpathcurveto{\pgfqpoint{2.069356in}{1.484975in}}{\pgfqpoint{2.032237in}{1.472461in}}{\pgfqpoint{1.999099in}{1.452913in}}%
\pgfpathclose%
\pgfusepath{stroke}%
\end{pgfscope}%
\begin{pgfscope}%
\pgfpathrectangle{\pgfqpoint{0.644217in}{0.500309in}}{\pgfqpoint{2.405783in}{1.149691in}}%
\pgfusepath{clip}%
\pgfsetbuttcap%
\pgfsetmiterjoin%
\pgfsetlinewidth{1.003750pt}%
\definecolor{currentstroke}{rgb}{1.000000,0.000000,0.000000}%
\pgfsetstrokecolor{currentstroke}%
\pgfsetdash{}{0pt}%
\pgfpathmoveto{\pgfqpoint{2.165498in}{1.466910in}}%
\pgfpathcurveto{\pgfqpoint{2.132363in}{1.447365in}}{\pgfqpoint{2.105913in}{1.422382in}}{\pgfqpoint{2.091975in}{1.397463in}}%
\pgfpathcurveto{\pgfqpoint{2.078036in}{1.372544in}}{\pgfqpoint{2.077746in}{1.349725in}}{\pgfqpoint{2.091169in}{1.334029in}}%
\pgfpathcurveto{\pgfqpoint{2.104592in}{1.318334in}}{\pgfqpoint{2.130631in}{1.311045in}}{\pgfqpoint{2.163553in}{1.313768in}}%
\pgfpathcurveto{\pgfqpoint{2.196474in}{1.316490in}}{\pgfqpoint{2.233589in}{1.329002in}}{\pgfqpoint{2.266724in}{1.348547in}}%
\pgfpathcurveto{\pgfqpoint{2.299859in}{1.368092in}}{\pgfqpoint{2.326308in}{1.393075in}}{\pgfqpoint{2.340247in}{1.417994in}}%
\pgfpathcurveto{\pgfqpoint{2.354185in}{1.442912in}}{\pgfqpoint{2.354475in}{1.465732in}}{\pgfqpoint{2.341052in}{1.481427in}}%
\pgfpathcurveto{\pgfqpoint{2.327630in}{1.497123in}}{\pgfqpoint{2.301590in}{1.504412in}}{\pgfqpoint{2.268669in}{1.501689in}}%
\pgfpathcurveto{\pgfqpoint{2.235748in}{1.498967in}}{\pgfqpoint{2.198633in}{1.486455in}}{\pgfqpoint{2.165498in}{1.466910in}}%
\pgfpathclose%
\pgfusepath{stroke}%
\end{pgfscope}%
\begin{pgfscope}%
\pgfpathrectangle{\pgfqpoint{0.644217in}{0.500309in}}{\pgfqpoint{2.405783in}{1.149691in}}%
\pgfusepath{clip}%
\pgfsetbuttcap%
\pgfsetmiterjoin%
\pgfsetlinewidth{1.003750pt}%
\definecolor{currentstroke}{rgb}{1.000000,0.000000,0.000000}%
\pgfsetstrokecolor{currentstroke}%
\pgfsetdash{}{0pt}%
\pgfpathmoveto{\pgfqpoint{2.371309in}{1.509382in}}%
\pgfpathcurveto{\pgfqpoint{2.338176in}{1.489838in}}{\pgfqpoint{2.311728in}{1.464857in}}{\pgfqpoint{2.297789in}{1.439941in}}%
\pgfpathcurveto{\pgfqpoint{2.283851in}{1.415024in}}{\pgfqpoint{2.283561in}{1.392206in}}{\pgfqpoint{2.296982in}{1.376511in}}%
\pgfpathcurveto{\pgfqpoint{2.310404in}{1.360817in}}{\pgfqpoint{2.336441in}{1.353528in}}{\pgfqpoint{2.369360in}{1.356250in}}%
\pgfpathcurveto{\pgfqpoint{2.402279in}{1.358972in}}{\pgfqpoint{2.439392in}{1.371482in}}{\pgfqpoint{2.472526in}{1.391026in}}%
\pgfpathcurveto{\pgfqpoint{2.505659in}{1.410569in}}{\pgfqpoint{2.532107in}{1.435550in}}{\pgfqpoint{2.546045in}{1.460467in}}%
\pgfpathcurveto{\pgfqpoint{2.559983in}{1.485384in}}{\pgfqpoint{2.560274in}{1.508202in}}{\pgfqpoint{2.546852in}{1.523896in}}%
\pgfpathcurveto{\pgfqpoint{2.533431in}{1.539591in}}{\pgfqpoint{2.507393in}{1.546879in}}{\pgfqpoint{2.474474in}{1.544158in}}%
\pgfpathcurveto{\pgfqpoint{2.441555in}{1.541436in}}{\pgfqpoint{2.404442in}{1.528925in}}{\pgfqpoint{2.371309in}{1.509382in}}%
\pgfpathclose%
\pgfusepath{stroke}%
\end{pgfscope}%
\begin{pgfscope}%
\pgfpathrectangle{\pgfqpoint{0.644217in}{0.500309in}}{\pgfqpoint{2.405783in}{1.149691in}}%
\pgfusepath{clip}%
\pgfsetbuttcap%
\pgfsetmiterjoin%
\pgfsetlinewidth{1.003750pt}%
\definecolor{currentstroke}{rgb}{1.000000,0.000000,0.000000}%
\pgfsetstrokecolor{currentstroke}%
\pgfsetdash{}{0pt}%
\pgfpathmoveto{\pgfqpoint{2.552278in}{1.512831in}}%
\pgfpathcurveto{\pgfqpoint{2.519146in}{1.493288in}}{\pgfqpoint{2.492698in}{1.468308in}}{\pgfqpoint{2.478760in}{1.443392in}}%
\pgfpathcurveto{\pgfqpoint{2.464822in}{1.418477in}}{\pgfqpoint{2.464532in}{1.395659in}}{\pgfqpoint{2.477952in}{1.379966in}}%
\pgfpathcurveto{\pgfqpoint{2.491373in}{1.364272in}}{\pgfqpoint{2.517410in}{1.356983in}}{\pgfqpoint{2.550328in}{1.359705in}}%
\pgfpathcurveto{\pgfqpoint{2.583245in}{1.362426in}}{\pgfqpoint{2.620357in}{1.374936in}}{\pgfqpoint{2.653490in}{1.394478in}}%
\pgfpathcurveto{\pgfqpoint{2.686622in}{1.414021in}}{\pgfqpoint{2.713069in}{1.439001in}}{\pgfqpoint{2.727007in}{1.463917in}}%
\pgfpathcurveto{\pgfqpoint{2.740945in}{1.488833in}}{\pgfqpoint{2.741236in}{1.511650in}}{\pgfqpoint{2.727815in}{1.527344in}}%
\pgfpathcurveto{\pgfqpoint{2.714395in}{1.543037in}}{\pgfqpoint{2.688358in}{1.550326in}}{\pgfqpoint{2.655440in}{1.547604in}}%
\pgfpathcurveto{\pgfqpoint{2.622522in}{1.544883in}}{\pgfqpoint{2.585410in}{1.532373in}}{\pgfqpoint{2.552278in}{1.512831in}}%
\pgfpathclose%
\pgfusepath{stroke}%
\end{pgfscope}%
\begin{pgfscope}%
\pgfpathrectangle{\pgfqpoint{0.644217in}{0.500309in}}{\pgfqpoint{2.405783in}{1.149691in}}%
\pgfusepath{clip}%
\pgfsetbuttcap%
\pgfsetmiterjoin%
\pgfsetlinewidth{1.003750pt}%
\definecolor{currentstroke}{rgb}{1.000000,0.000000,0.000000}%
\pgfsetstrokecolor{currentstroke}%
\pgfsetdash{}{0pt}%
\pgfpathmoveto{\pgfqpoint{2.678903in}{1.495184in}}%
\pgfpathcurveto{\pgfqpoint{2.645771in}{1.475642in}}{\pgfqpoint{2.619324in}{1.450663in}}{\pgfqpoint{2.605386in}{1.425747in}}%
\pgfpathcurveto{\pgfqpoint{2.591448in}{1.400832in}}{\pgfqpoint{2.591157in}{1.378015in}}{\pgfqpoint{2.604577in}{1.362322in}}%
\pgfpathcurveto{\pgfqpoint{2.617998in}{1.346629in}}{\pgfqpoint{2.644034in}{1.339340in}}{\pgfqpoint{2.676951in}{1.342061in}}%
\pgfpathcurveto{\pgfqpoint{2.709868in}{1.344783in}}{\pgfqpoint{2.746979in}{1.357292in}}{\pgfqpoint{2.780111in}{1.376834in}}%
\pgfpathcurveto{\pgfqpoint{2.813243in}{1.396376in}}{\pgfqpoint{2.839690in}{1.421355in}}{\pgfqpoint{2.853628in}{1.446271in}}%
\pgfpathcurveto{\pgfqpoint{2.867566in}{1.471186in}}{\pgfqpoint{2.867857in}{1.494003in}}{\pgfqpoint{2.854436in}{1.509696in}}%
\pgfpathcurveto{\pgfqpoint{2.841016in}{1.525389in}}{\pgfqpoint{2.814980in}{1.532678in}}{\pgfqpoint{2.782063in}{1.529957in}}%
\pgfpathcurveto{\pgfqpoint{2.749146in}{1.527235in}}{\pgfqpoint{2.712034in}{1.514726in}}{\pgfqpoint{2.678903in}{1.495184in}}%
\pgfpathclose%
\pgfusepath{stroke}%
\end{pgfscope}%
\begin{pgfscope}%
\pgfpathrectangle{\pgfqpoint{0.644217in}{0.500309in}}{\pgfqpoint{2.405783in}{1.149691in}}%
\pgfusepath{clip}%
\pgfsetbuttcap%
\pgfsetmiterjoin%
\pgfsetlinewidth{1.003750pt}%
\definecolor{currentstroke}{rgb}{1.000000,0.000000,0.000000}%
\pgfsetstrokecolor{currentstroke}%
\pgfsetdash{}{0pt}%
\pgfpathmoveto{\pgfqpoint{2.819624in}{1.556303in}}%
\pgfpathcurveto{\pgfqpoint{2.786493in}{1.536761in}}{\pgfqpoint{2.760046in}{1.511782in}}{\pgfqpoint{2.746108in}{1.486867in}}%
\pgfpathcurveto{\pgfqpoint{2.732170in}{1.461952in}}{\pgfqpoint{2.731879in}{1.439136in}}{\pgfqpoint{2.745299in}{1.423442in}}%
\pgfpathcurveto{\pgfqpoint{2.758719in}{1.407749in}}{\pgfqpoint{2.784754in}{1.400461in}}{\pgfqpoint{2.817671in}{1.403182in}}%
\pgfpathcurveto{\pgfqpoint{2.850588in}{1.405903in}}{\pgfqpoint{2.887699in}{1.418412in}}{\pgfqpoint{2.920830in}{1.437954in}}%
\pgfpathcurveto{\pgfqpoint{2.953962in}{1.457496in}}{\pgfqpoint{2.980409in}{1.482475in}}{\pgfqpoint{2.994347in}{1.507390in}}%
\pgfpathcurveto{\pgfqpoint{3.008285in}{1.532305in}}{\pgfqpoint{3.008576in}{1.555121in}}{\pgfqpoint{2.995155in}{1.570814in}}%
\pgfpathcurveto{\pgfqpoint{2.981735in}{1.586507in}}{\pgfqpoint{2.955700in}{1.593796in}}{\pgfqpoint{2.922783in}{1.591075in}}%
\pgfpathcurveto{\pgfqpoint{2.889866in}{1.588353in}}{\pgfqpoint{2.852755in}{1.575844in}}{\pgfqpoint{2.819624in}{1.556303in}}%
\pgfpathclose%
\pgfusepath{stroke}%
\end{pgfscope}%
\begin{pgfscope}%
\pgfpathrectangle{\pgfqpoint{0.644217in}{0.500309in}}{\pgfqpoint{2.405783in}{1.149691in}}%
\pgfusepath{clip}%
\pgfsetbuttcap%
\pgfsetmiterjoin%
\pgfsetlinewidth{1.003750pt}%
\definecolor{currentstroke}{rgb}{0.000000,0.000000,1.000000}%
\pgfsetstrokecolor{currentstroke}%
\pgfsetdash{}{0pt}%
\pgfpathmoveto{\pgfqpoint{0.949518in}{0.909597in}}%
\pgfpathcurveto{\pgfqpoint{0.911852in}{0.888001in}}{\pgfqpoint{0.881277in}{0.860605in}}{\pgfqpoint{0.864527in}{0.833444in}}%
\pgfpathcurveto{\pgfqpoint{0.847777in}{0.806283in}}{\pgfqpoint{0.846219in}{0.781574in}}{\pgfqpoint{0.860197in}{0.764759in}}%
\pgfpathcurveto{\pgfqpoint{0.874175in}{0.747944in}}{\pgfqpoint{0.902548in}{0.740396in}}{\pgfqpoint{0.939066in}{0.743778in}}%
\pgfpathcurveto{\pgfqpoint{0.975584in}{0.747159in}}{\pgfqpoint{1.017267in}{0.761193in}}{\pgfqpoint{1.054933in}{0.782790in}}%
\pgfpathcurveto{\pgfqpoint{1.092599in}{0.804387in}}{\pgfqpoint{1.123174in}{0.831782in}}{\pgfqpoint{1.139925in}{0.858943in}}%
\pgfpathcurveto{\pgfqpoint{1.156675in}{0.886104in}}{\pgfqpoint{1.158232in}{0.910813in}}{\pgfqpoint{1.144254in}{0.927628in}}%
\pgfpathcurveto{\pgfqpoint{1.130276in}{0.944443in}}{\pgfqpoint{1.101903in}{0.951991in}}{\pgfqpoint{1.065385in}{0.948610in}}%
\pgfpathcurveto{\pgfqpoint{1.028867in}{0.945229in}}{\pgfqpoint{0.987185in}{0.931194in}}{\pgfqpoint{0.949518in}{0.909597in}}%
\pgfpathclose%
\pgfusepath{stroke}%
\end{pgfscope}%
\begin{pgfscope}%
\pgfpathrectangle{\pgfqpoint{0.644217in}{0.500309in}}{\pgfqpoint{2.405783in}{1.149691in}}%
\pgfusepath{clip}%
\pgfsetbuttcap%
\pgfsetmiterjoin%
\pgfsetlinewidth{1.003750pt}%
\definecolor{currentstroke}{rgb}{0.000000,0.000000,1.000000}%
\pgfsetstrokecolor{currentstroke}%
\pgfsetdash{}{0pt}%
\pgfpathmoveto{\pgfqpoint{0.997148in}{0.936252in}}%
\pgfpathcurveto{\pgfqpoint{0.960423in}{0.915061in}}{\pgfqpoint{0.930599in}{0.888280in}}{\pgfqpoint{0.914244in}{0.861805in}}%
\pgfpathcurveto{\pgfqpoint{0.897889in}{0.835330in}}{\pgfqpoint{0.896337in}{0.811325in}}{\pgfqpoint{0.909932in}{0.795075in}}%
\pgfpathcurveto{\pgfqpoint{0.923526in}{0.778825in}}{\pgfqpoint{0.951157in}{0.771658in}}{\pgfqpoint{0.986737in}{0.775152in}}%
\pgfpathcurveto{\pgfqpoint{1.022318in}{0.778646in}}{\pgfqpoint{1.062945in}{0.792516in}}{\pgfqpoint{1.099669in}{0.813707in}}%
\pgfpathcurveto{\pgfqpoint{1.136393in}{0.834898in}}{\pgfqpoint{1.166217in}{0.861680in}}{\pgfqpoint{1.182573in}{0.888155in}}%
\pgfpathcurveto{\pgfqpoint{1.198928in}{0.914629in}}{\pgfqpoint{1.200479in}{0.938635in}}{\pgfqpoint{1.186885in}{0.954884in}}%
\pgfpathcurveto{\pgfqpoint{1.173290in}{0.971134in}}{\pgfqpoint{1.145660in}{0.978301in}}{\pgfqpoint{1.110079in}{0.974807in}}%
\pgfpathcurveto{\pgfqpoint{1.074498in}{0.971313in}}{\pgfqpoint{1.033872in}{0.957443in}}{\pgfqpoint{0.997148in}{0.936252in}}%
\pgfpathclose%
\pgfusepath{stroke}%
\end{pgfscope}%
\begin{pgfscope}%
\pgfpathrectangle{\pgfqpoint{0.644217in}{0.500309in}}{\pgfqpoint{2.405783in}{1.149691in}}%
\pgfusepath{clip}%
\pgfsetbuttcap%
\pgfsetmiterjoin%
\pgfsetlinewidth{1.003750pt}%
\definecolor{currentstroke}{rgb}{0.000000,0.000000,1.000000}%
\pgfsetstrokecolor{currentstroke}%
\pgfsetdash{}{0pt}%
\pgfpathmoveto{\pgfqpoint{1.059468in}{0.976651in}}%
\pgfpathcurveto{\pgfqpoint{1.024061in}{0.955649in}}{\pgfqpoint{0.995607in}{0.929099in}}{\pgfqpoint{0.980375in}{0.902849in}}%
\pgfpathcurveto{\pgfqpoint{0.965142in}{0.876598in}}{\pgfqpoint{0.964373in}{0.852789in}}{\pgfqpoint{0.978238in}{0.836667in}}%
\pgfpathcurveto{\pgfqpoint{0.992103in}{0.820544in}}{\pgfqpoint{1.019470in}{0.813424in}}{\pgfqpoint{1.054310in}{0.816874in}}%
\pgfpathcurveto{\pgfqpoint{1.089151in}{0.820324in}}{\pgfqpoint{1.128622in}{0.834062in}}{\pgfqpoint{1.164029in}{0.855064in}}%
\pgfpathcurveto{\pgfqpoint{1.199437in}{0.876066in}}{\pgfqpoint{1.227890in}{0.902616in}}{\pgfqpoint{1.243123in}{0.928867in}}%
\pgfpathcurveto{\pgfqpoint{1.258356in}{0.955118in}}{\pgfqpoint{1.259124in}{0.978926in}}{\pgfqpoint{1.245259in}{0.995049in}}%
\pgfpathcurveto{\pgfqpoint{1.231394in}{1.011171in}}{\pgfqpoint{1.204028in}{1.018292in}}{\pgfqpoint{1.169187in}{1.014842in}}%
\pgfpathcurveto{\pgfqpoint{1.134346in}{1.011392in}}{\pgfqpoint{1.094876in}{0.997653in}}{\pgfqpoint{1.059468in}{0.976651in}}%
\pgfpathclose%
\pgfusepath{stroke}%
\end{pgfscope}%
\begin{pgfscope}%
\pgfpathrectangle{\pgfqpoint{0.644217in}{0.500309in}}{\pgfqpoint{2.405783in}{1.149691in}}%
\pgfusepath{clip}%
\pgfsetbuttcap%
\pgfsetmiterjoin%
\pgfsetlinewidth{1.003750pt}%
\definecolor{currentstroke}{rgb}{0.000000,0.000000,1.000000}%
\pgfsetstrokecolor{currentstroke}%
\pgfsetdash{}{0pt}%
\pgfpathmoveto{\pgfqpoint{1.133148in}{1.053749in}}%
\pgfpathcurveto{\pgfqpoint{1.098609in}{1.033482in}}{\pgfqpoint{1.070870in}{1.007708in}}{\pgfqpoint{1.056039in}{0.982103in}}%
\pgfpathcurveto{\pgfqpoint{1.041208in}{0.956498in}}{\pgfqpoint{1.040496in}{0.933154in}}{\pgfqpoint{1.054060in}{0.917210in}}%
\pgfpathcurveto{\pgfqpoint{1.067624in}{0.901267in}}{\pgfqpoint{1.094356in}{0.894027in}}{\pgfqpoint{1.128370in}{0.897084in}}%
\pgfpathcurveto{\pgfqpoint{1.162384in}{0.900141in}}{\pgfqpoint{1.200901in}{0.913247in}}{\pgfqpoint{1.235440in}{0.933514in}}%
\pgfpathcurveto{\pgfqpoint{1.269978in}{0.953781in}}{\pgfqpoint{1.297718in}{0.979555in}}{\pgfqpoint{1.312549in}{1.005160in}}%
\pgfpathcurveto{\pgfqpoint{1.327380in}{1.030764in}}{\pgfqpoint{1.328092in}{1.054109in}}{\pgfqpoint{1.314528in}{1.070052in}}%
\pgfpathcurveto{\pgfqpoint{1.300964in}{1.085996in}}{\pgfqpoint{1.274231in}{1.093236in}}{\pgfqpoint{1.240217in}{1.090179in}}%
\pgfpathcurveto{\pgfqpoint{1.206204in}{1.087121in}}{\pgfqpoint{1.167686in}{1.074016in}}{\pgfqpoint{1.133148in}{1.053749in}}%
\pgfpathclose%
\pgfusepath{stroke}%
\end{pgfscope}%
\begin{pgfscope}%
\pgfpathrectangle{\pgfqpoint{0.644217in}{0.500309in}}{\pgfqpoint{2.405783in}{1.149691in}}%
\pgfusepath{clip}%
\pgfsetbuttcap%
\pgfsetmiterjoin%
\pgfsetlinewidth{1.003750pt}%
\definecolor{currentstroke}{rgb}{0.000000,0.000000,1.000000}%
\pgfsetstrokecolor{currentstroke}%
\pgfsetdash{}{0pt}%
\pgfpathmoveto{\pgfqpoint{1.215376in}{1.098506in}}%
\pgfpathcurveto{\pgfqpoint{1.181302in}{1.078675in}}{\pgfqpoint{1.153926in}{1.053365in}}{\pgfqpoint{1.139277in}{1.028149in}}%
\pgfpathcurveto{\pgfqpoint{1.124628in}{1.002934in}}{\pgfqpoint{1.123903in}{0.979871in}}{\pgfqpoint{1.137260in}{0.964040in}}%
\pgfpathcurveto{\pgfqpoint{1.150617in}{0.948210in}}{\pgfqpoint{1.176967in}{0.940904in}}{\pgfqpoint{1.210506in}{0.943732in}}%
\pgfpathcurveto{\pgfqpoint{1.244045in}{0.946560in}}{\pgfqpoint{1.282035in}{0.959291in}}{\pgfqpoint{1.316109in}{0.979121in}}%
\pgfpathcurveto{\pgfqpoint{1.350183in}{0.998952in}}{\pgfqpoint{1.377559in}{1.024262in}}{\pgfqpoint{1.392208in}{1.049478in}}%
\pgfpathcurveto{\pgfqpoint{1.406857in}{1.074694in}}{\pgfqpoint{1.407583in}{1.097756in}}{\pgfqpoint{1.394225in}{1.113587in}}%
\pgfpathcurveto{\pgfqpoint{1.380868in}{1.129417in}}{\pgfqpoint{1.354518in}{1.136723in}}{\pgfqpoint{1.320979in}{1.133895in}}%
\pgfpathcurveto{\pgfqpoint{1.287440in}{1.131067in}}{\pgfqpoint{1.249450in}{1.118336in}}{\pgfqpoint{1.215376in}{1.098506in}}%
\pgfpathclose%
\pgfusepath{stroke}%
\end{pgfscope}%
\begin{pgfscope}%
\pgfpathrectangle{\pgfqpoint{0.644217in}{0.500309in}}{\pgfqpoint{2.405783in}{1.149691in}}%
\pgfusepath{clip}%
\pgfsetbuttcap%
\pgfsetmiterjoin%
\pgfsetlinewidth{1.003750pt}%
\definecolor{currentstroke}{rgb}{0.000000,0.000000,1.000000}%
\pgfsetstrokecolor{currentstroke}%
\pgfsetdash{}{0pt}%
\pgfpathmoveto{\pgfqpoint{1.347027in}{1.106033in}}%
\pgfpathcurveto{\pgfqpoint{1.313193in}{1.086427in}}{\pgfqpoint{1.286004in}{1.061356in}}{\pgfqpoint{1.271448in}{1.036342in}}%
\pgfpathcurveto{\pgfqpoint{1.256891in}{1.011328in}}{\pgfqpoint{1.256156in}{0.988413in}}{\pgfqpoint{1.269405in}{0.972645in}}%
\pgfpathcurveto{\pgfqpoint{1.282653in}{0.956876in}}{\pgfqpoint{1.308803in}{0.949541in}}{\pgfqpoint{1.342095in}{0.952255in}}%
\pgfpathcurveto{\pgfqpoint{1.375387in}{0.954969in}}{\pgfqpoint{1.413104in}{0.967510in}}{\pgfqpoint{1.446938in}{0.987116in}}%
\pgfpathcurveto{\pgfqpoint{1.480772in}{1.006723in}}{\pgfqpoint{1.507961in}{1.031794in}}{\pgfqpoint{1.522517in}{1.056808in}}%
\pgfpathcurveto{\pgfqpoint{1.537074in}{1.081822in}}{\pgfqpoint{1.537809in}{1.104736in}}{\pgfqpoint{1.524561in}{1.120505in}}%
\pgfpathcurveto{\pgfqpoint{1.511312in}{1.136273in}}{\pgfqpoint{1.485162in}{1.143609in}}{\pgfqpoint{1.451870in}{1.140895in}}%
\pgfpathcurveto{\pgfqpoint{1.418578in}{1.138181in}}{\pgfqpoint{1.380861in}{1.125640in}}{\pgfqpoint{1.347027in}{1.106033in}}%
\pgfpathclose%
\pgfusepath{stroke}%
\end{pgfscope}%
\begin{pgfscope}%
\pgfpathrectangle{\pgfqpoint{0.644217in}{0.500309in}}{\pgfqpoint{2.405783in}{1.149691in}}%
\pgfusepath{clip}%
\pgfsetbuttcap%
\pgfsetmiterjoin%
\pgfsetlinewidth{1.003750pt}%
\definecolor{currentstroke}{rgb}{0.000000,0.000000,1.000000}%
\pgfsetstrokecolor{currentstroke}%
\pgfsetdash{}{0pt}%
\pgfpathmoveto{\pgfqpoint{1.337953in}{1.122048in}}%
\pgfpathcurveto{\pgfqpoint{1.304245in}{1.102556in}}{\pgfqpoint{1.277154in}{1.077608in}}{\pgfqpoint{1.262647in}{1.052699in}}%
\pgfpathcurveto{\pgfqpoint{1.248140in}{1.027790in}}{\pgfqpoint{1.247401in}{1.004953in}}{\pgfqpoint{1.260592in}{0.989219in}}%
\pgfpathcurveto{\pgfqpoint{1.273784in}{0.973485in}}{\pgfqpoint{1.299829in}{0.966138in}}{\pgfqpoint{1.332992in}{0.968795in}}%
\pgfpathcurveto{\pgfqpoint{1.366155in}{0.971453in}}{\pgfqpoint{1.403728in}{0.983898in}}{\pgfqpoint{1.437436in}{1.003391in}}%
\pgfpathcurveto{\pgfqpoint{1.471144in}{1.022883in}}{\pgfqpoint{1.498235in}{1.047831in}}{\pgfqpoint{1.512742in}{1.072740in}}%
\pgfpathcurveto{\pgfqpoint{1.527249in}{1.097649in}}{\pgfqpoint{1.527988in}{1.120485in}}{\pgfqpoint{1.514796in}{1.136219in}}%
\pgfpathcurveto{\pgfqpoint{1.501605in}{1.151953in}}{\pgfqpoint{1.475559in}{1.159301in}}{\pgfqpoint{1.442396in}{1.156643in}}%
\pgfpathcurveto{\pgfqpoint{1.409233in}{1.153986in}}{\pgfqpoint{1.371660in}{1.141540in}}{\pgfqpoint{1.337953in}{1.122048in}}%
\pgfpathclose%
\pgfusepath{stroke}%
\end{pgfscope}%
\begin{pgfscope}%
\pgfpathrectangle{\pgfqpoint{0.644217in}{0.500309in}}{\pgfqpoint{2.405783in}{1.149691in}}%
\pgfusepath{clip}%
\pgfsetbuttcap%
\pgfsetmiterjoin%
\pgfsetlinewidth{1.003750pt}%
\definecolor{currentstroke}{rgb}{0.000000,0.000000,1.000000}%
\pgfsetstrokecolor{currentstroke}%
\pgfsetdash{}{0pt}%
\pgfpathmoveto{\pgfqpoint{1.511174in}{1.232440in}}%
\pgfpathcurveto{\pgfqpoint{1.477534in}{1.213006in}}{\pgfqpoint{1.450496in}{1.188123in}}{\pgfqpoint{1.436016in}{1.163269in}}%
\pgfpathcurveto{\pgfqpoint{1.421536in}{1.138416in}}{\pgfqpoint{1.420795in}{1.115622in}}{\pgfqpoint{1.433957in}{1.099907in}}%
\pgfpathcurveto{\pgfqpoint{1.447119in}{1.084193in}}{\pgfqpoint{1.473108in}{1.076841in}}{\pgfqpoint{1.506202in}{1.079471in}}%
\pgfpathcurveto{\pgfqpoint{1.539296in}{1.082100in}}{\pgfqpoint{1.576792in}{1.094497in}}{\pgfqpoint{1.610432in}{1.113930in}}%
\pgfpathcurveto{\pgfqpoint{1.644072in}{1.133364in}}{\pgfqpoint{1.671110in}{1.158247in}}{\pgfqpoint{1.685590in}{1.183101in}}%
\pgfpathcurveto{\pgfqpoint{1.700071in}{1.207954in}}{\pgfqpoint{1.700812in}{1.230748in}}{\pgfqpoint{1.687650in}{1.246463in}}%
\pgfpathcurveto{\pgfqpoint{1.674488in}{1.262177in}}{\pgfqpoint{1.648498in}{1.269529in}}{\pgfqpoint{1.615404in}{1.266899in}}%
\pgfpathcurveto{\pgfqpoint{1.582310in}{1.264270in}}{\pgfqpoint{1.544814in}{1.251873in}}{\pgfqpoint{1.511174in}{1.232440in}}%
\pgfpathclose%
\pgfusepath{stroke}%
\end{pgfscope}%
\begin{pgfscope}%
\pgfpathrectangle{\pgfqpoint{0.644217in}{0.500309in}}{\pgfqpoint{2.405783in}{1.149691in}}%
\pgfusepath{clip}%
\pgfsetbuttcap%
\pgfsetmiterjoin%
\pgfsetlinewidth{1.003750pt}%
\definecolor{currentstroke}{rgb}{0.000000,0.000000,1.000000}%
\pgfsetstrokecolor{currentstroke}%
\pgfsetdash{}{0pt}%
\pgfpathmoveto{\pgfqpoint{1.550750in}{1.297114in}}%
\pgfpathcurveto{\pgfqpoint{1.517146in}{1.277711in}}{\pgfqpoint{1.490138in}{1.252862in}}{\pgfqpoint{1.475672in}{1.228038in}}%
\pgfpathcurveto{\pgfqpoint{1.461207in}{1.203215in}}{\pgfqpoint{1.460465in}{1.180444in}}{\pgfqpoint{1.473611in}{1.164741in}}%
\pgfpathcurveto{\pgfqpoint{1.486757in}{1.149038in}}{\pgfqpoint{1.512717in}{1.141684in}}{\pgfqpoint{1.545773in}{1.144300in}}%
\pgfpathcurveto{\pgfqpoint{1.578830in}{1.146916in}}{\pgfqpoint{1.616284in}{1.159288in}}{\pgfqpoint{1.649887in}{1.178690in}}%
\pgfpathcurveto{\pgfqpoint{1.683491in}{1.198093in}}{\pgfqpoint{1.710499in}{1.222942in}}{\pgfqpoint{1.724965in}{1.247766in}}%
\pgfpathcurveto{\pgfqpoint{1.739430in}{1.272589in}}{\pgfqpoint{1.740172in}{1.295360in}}{\pgfqpoint{1.727026in}{1.311063in}}%
\pgfpathcurveto{\pgfqpoint{1.713880in}{1.326766in}}{\pgfqpoint{1.687920in}{1.334120in}}{\pgfqpoint{1.654864in}{1.331504in}}%
\pgfpathcurveto{\pgfqpoint{1.621807in}{1.328888in}}{\pgfqpoint{1.584353in}{1.316516in}}{\pgfqpoint{1.550750in}{1.297114in}}%
\pgfpathclose%
\pgfusepath{stroke}%
\end{pgfscope}%
\begin{pgfscope}%
\pgfpathrectangle{\pgfqpoint{0.644217in}{0.500309in}}{\pgfqpoint{2.405783in}{1.149691in}}%
\pgfusepath{clip}%
\pgfsetbuttcap%
\pgfsetmiterjoin%
\pgfsetlinewidth{1.003750pt}%
\definecolor{currentstroke}{rgb}{0.000000,0.000000,1.000000}%
\pgfsetstrokecolor{currentstroke}%
\pgfsetdash{}{0pt}%
\pgfpathmoveto{\pgfqpoint{1.569807in}{1.413207in}}%
\pgfpathcurveto{\pgfqpoint{1.536224in}{1.393820in}}{\pgfqpoint{1.509231in}{1.368989in}}{\pgfqpoint{1.494774in}{1.344182in}}%
\pgfpathcurveto{\pgfqpoint{1.480317in}{1.319374in}}{\pgfqpoint{1.479575in}{1.296616in}}{\pgfqpoint{1.492712in}{1.280920in}}%
\pgfpathcurveto{\pgfqpoint{1.505849in}{1.265224in}}{\pgfqpoint{1.531793in}{1.257870in}}{\pgfqpoint{1.564829in}{1.260479in}}%
\pgfpathcurveto{\pgfqpoint{1.597865in}{1.263088in}}{\pgfqpoint{1.635296in}{1.275447in}}{\pgfqpoint{1.668879in}{1.294834in}}%
\pgfpathcurveto{\pgfqpoint{1.702462in}{1.314220in}}{\pgfqpoint{1.729455in}{1.339051in}}{\pgfqpoint{1.743912in}{1.363858in}}%
\pgfpathcurveto{\pgfqpoint{1.758369in}{1.388666in}}{\pgfqpoint{1.759111in}{1.411424in}}{\pgfqpoint{1.745974in}{1.427120in}}%
\pgfpathcurveto{\pgfqpoint{1.732837in}{1.442817in}}{\pgfqpoint{1.706893in}{1.450170in}}{\pgfqpoint{1.673857in}{1.447561in}}%
\pgfpathcurveto{\pgfqpoint{1.640821in}{1.444952in}}{\pgfqpoint{1.603390in}{1.432593in}}{\pgfqpoint{1.569807in}{1.413207in}}%
\pgfpathclose%
\pgfusepath{stroke}%
\end{pgfscope}%
\begin{pgfscope}%
\pgfpathrectangle{\pgfqpoint{0.644217in}{0.500309in}}{\pgfqpoint{2.405783in}{1.149691in}}%
\pgfusepath{clip}%
\pgfsetbuttcap%
\pgfsetmiterjoin%
\pgfsetlinewidth{1.003750pt}%
\definecolor{currentstroke}{rgb}{0.000000,0.000000,1.000000}%
\pgfsetstrokecolor{currentstroke}%
\pgfsetdash{}{0pt}%
\pgfpathmoveto{\pgfqpoint{1.850156in}{1.446932in}}%
\pgfpathcurveto{\pgfqpoint{1.816584in}{1.427554in}}{\pgfqpoint{1.789601in}{1.402732in}}{\pgfqpoint{1.775148in}{1.377934in}}%
\pgfpathcurveto{\pgfqpoint{1.760695in}{1.353135in}}{\pgfqpoint{1.759953in}{1.330385in}}{\pgfqpoint{1.773085in}{1.314692in}}%
\pgfpathcurveto{\pgfqpoint{1.786218in}{1.299000in}}{\pgfqpoint{1.812152in}{1.291647in}}{\pgfqpoint{1.845177in}{1.294253in}}%
\pgfpathcurveto{\pgfqpoint{1.878202in}{1.296858in}}{\pgfqpoint{1.915621in}{1.309211in}}{\pgfqpoint{1.949193in}{1.328588in}}%
\pgfpathcurveto{\pgfqpoint{1.982764in}{1.347966in}}{\pgfqpoint{2.009748in}{1.372788in}}{\pgfqpoint{2.024201in}{1.397586in}}%
\pgfpathcurveto{\pgfqpoint{2.038653in}{1.422384in}}{\pgfqpoint{2.039395in}{1.445135in}}{\pgfqpoint{2.026263in}{1.460828in}}%
\pgfpathcurveto{\pgfqpoint{2.013131in}{1.476520in}}{\pgfqpoint{1.987196in}{1.483873in}}{\pgfqpoint{1.954171in}{1.481267in}}%
\pgfpathcurveto{\pgfqpoint{1.921146in}{1.478661in}}{\pgfqpoint{1.883728in}{1.466309in}}{\pgfqpoint{1.850156in}{1.446932in}}%
\pgfpathclose%
\pgfusepath{stroke}%
\end{pgfscope}%
\begin{pgfscope}%
\pgfpathrectangle{\pgfqpoint{0.644217in}{0.500309in}}{\pgfqpoint{2.405783in}{1.149691in}}%
\pgfusepath{clip}%
\pgfsetbuttcap%
\pgfsetmiterjoin%
\pgfsetlinewidth{1.003750pt}%
\definecolor{currentstroke}{rgb}{0.000000,0.000000,1.000000}%
\pgfsetstrokecolor{currentstroke}%
\pgfsetdash{}{0pt}%
\pgfpathmoveto{\pgfqpoint{2.009531in}{1.455278in}}%
\pgfpathcurveto{\pgfqpoint{1.975965in}{1.435905in}}{\pgfqpoint{1.948987in}{1.411089in}}{\pgfqpoint{1.934536in}{1.386295in}}%
\pgfpathcurveto{\pgfqpoint{1.920086in}{1.361502in}}{\pgfqpoint{1.919344in}{1.338755in}}{\pgfqpoint{1.932474in}{1.323065in}}%
\pgfpathcurveto{\pgfqpoint{1.945604in}{1.307375in}}{\pgfqpoint{1.971533in}{1.300022in}}{\pgfqpoint{2.004552in}{1.302626in}}%
\pgfpathcurveto{\pgfqpoint{2.037571in}{1.305231in}}{\pgfqpoint{2.074982in}{1.317579in}}{\pgfqpoint{2.108548in}{1.336953in}}%
\pgfpathcurveto{\pgfqpoint{2.142113in}{1.356326in}}{\pgfqpoint{2.169092in}{1.381142in}}{\pgfqpoint{2.183542in}{1.405936in}}%
\pgfpathcurveto{\pgfqpoint{2.197993in}{1.430729in}}{\pgfqpoint{2.198735in}{1.453476in}}{\pgfqpoint{2.185605in}{1.469166in}}%
\pgfpathcurveto{\pgfqpoint{2.172475in}{1.484856in}}{\pgfqpoint{2.146545in}{1.492209in}}{\pgfqpoint{2.113527in}{1.489604in}}%
\pgfpathcurveto{\pgfqpoint{2.080508in}{1.487000in}}{\pgfqpoint{2.043096in}{1.474651in}}{\pgfqpoint{2.009531in}{1.455278in}}%
\pgfpathclose%
\pgfusepath{stroke}%
\end{pgfscope}%
\begin{pgfscope}%
\pgfpathrectangle{\pgfqpoint{0.644217in}{0.500309in}}{\pgfqpoint{2.405783in}{1.149691in}}%
\pgfusepath{clip}%
\pgfsetbuttcap%
\pgfsetmiterjoin%
\pgfsetlinewidth{1.003750pt}%
\definecolor{currentstroke}{rgb}{0.000000,0.000000,1.000000}%
\pgfsetstrokecolor{currentstroke}%
\pgfsetdash{}{0pt}%
\pgfpathmoveto{\pgfqpoint{2.160209in}{1.461383in}}%
\pgfpathcurveto{\pgfqpoint{2.126647in}{1.442013in}}{\pgfqpoint{2.099671in}{1.417199in}}{\pgfqpoint{2.085222in}{1.392408in}}%
\pgfpathcurveto{\pgfqpoint{2.070773in}{1.367617in}}{\pgfqpoint{2.070031in}{1.344873in}}{\pgfqpoint{2.083159in}{1.329184in}}%
\pgfpathcurveto{\pgfqpoint{2.096288in}{1.313495in}}{\pgfqpoint{2.122215in}{1.306143in}}{\pgfqpoint{2.155230in}{1.308747in}}%
\pgfpathcurveto{\pgfqpoint{2.188245in}{1.311350in}}{\pgfqpoint{2.225653in}{1.323697in}}{\pgfqpoint{2.259215in}{1.343068in}}%
\pgfpathcurveto{\pgfqpoint{2.292777in}{1.362439in}}{\pgfqpoint{2.319753in}{1.387252in}}{\pgfqpoint{2.334202in}{1.412043in}}%
\pgfpathcurveto{\pgfqpoint{2.348651in}{1.436834in}}{\pgfqpoint{2.349392in}{1.459579in}}{\pgfqpoint{2.336264in}{1.475267in}}%
\pgfpathcurveto{\pgfqpoint{2.323136in}{1.490956in}}{\pgfqpoint{2.297209in}{1.498308in}}{\pgfqpoint{2.264194in}{1.495705in}}%
\pgfpathcurveto{\pgfqpoint{2.231179in}{1.493101in}}{\pgfqpoint{2.193771in}{1.480754in}}{\pgfqpoint{2.160209in}{1.461383in}}%
\pgfpathclose%
\pgfusepath{stroke}%
\end{pgfscope}%
\begin{pgfscope}%
\pgfpathrectangle{\pgfqpoint{0.644217in}{0.500309in}}{\pgfqpoint{2.405783in}{1.149691in}}%
\pgfusepath{clip}%
\pgfsetbuttcap%
\pgfsetmiterjoin%
\pgfsetlinewidth{1.003750pt}%
\definecolor{currentstroke}{rgb}{0.000000,0.000000,1.000000}%
\pgfsetstrokecolor{currentstroke}%
\pgfsetdash{}{0pt}%
\pgfpathmoveto{\pgfqpoint{2.367769in}{1.501679in}}%
\pgfpathcurveto{\pgfqpoint{2.334209in}{1.482309in}}{\pgfqpoint{2.307234in}{1.457497in}}{\pgfqpoint{2.292786in}{1.432708in}}%
\pgfpathcurveto{\pgfqpoint{2.278338in}{1.407918in}}{\pgfqpoint{2.277597in}{1.385175in}}{\pgfqpoint{2.290724in}{1.369487in}}%
\pgfpathcurveto{\pgfqpoint{2.303852in}{1.353799in}}{\pgfqpoint{2.329777in}{1.346447in}}{\pgfqpoint{2.362790in}{1.349051in}}%
\pgfpathcurveto{\pgfqpoint{2.395804in}{1.351654in}}{\pgfqpoint{2.433209in}{1.364000in}}{\pgfqpoint{2.466769in}{1.383369in}}%
\pgfpathcurveto{\pgfqpoint{2.500330in}{1.402739in}}{\pgfqpoint{2.527304in}{1.427551in}}{\pgfqpoint{2.541752in}{1.452340in}}%
\pgfpathcurveto{\pgfqpoint{2.556200in}{1.477130in}}{\pgfqpoint{2.556942in}{1.499873in}}{\pgfqpoint{2.543814in}{1.515561in}}%
\pgfpathcurveto{\pgfqpoint{2.530687in}{1.531249in}}{\pgfqpoint{2.504761in}{1.538601in}}{\pgfqpoint{2.471748in}{1.535997in}}%
\pgfpathcurveto{\pgfqpoint{2.438735in}{1.533394in}}{\pgfqpoint{2.401329in}{1.521048in}}{\pgfqpoint{2.367769in}{1.501679in}}%
\pgfpathclose%
\pgfusepath{stroke}%
\end{pgfscope}%
\begin{pgfscope}%
\pgfpathrectangle{\pgfqpoint{0.644217in}{0.500309in}}{\pgfqpoint{2.405783in}{1.149691in}}%
\pgfusepath{clip}%
\pgfsetbuttcap%
\pgfsetmiterjoin%
\pgfsetlinewidth{1.003750pt}%
\definecolor{currentstroke}{rgb}{0.000000,0.000000,1.000000}%
\pgfsetstrokecolor{currentstroke}%
\pgfsetdash{}{0pt}%
\pgfpathmoveto{\pgfqpoint{2.553037in}{1.508703in}}%
\pgfpathcurveto{\pgfqpoint{2.519478in}{1.489334in}}{\pgfqpoint{2.492504in}{1.464523in}}{\pgfqpoint{2.478057in}{1.439735in}}%
\pgfpathcurveto{\pgfqpoint{2.463609in}{1.414946in}}{\pgfqpoint{2.462867in}{1.392204in}}{\pgfqpoint{2.475994in}{1.376516in}}%
\pgfpathcurveto{\pgfqpoint{2.489122in}{1.360829in}}{\pgfqpoint{2.515046in}{1.353477in}}{\pgfqpoint{2.548058in}{1.356080in}}%
\pgfpathcurveto{\pgfqpoint{2.581070in}{1.358683in}}{\pgfqpoint{2.618475in}{1.371029in}}{\pgfqpoint{2.652034in}{1.390398in}}%
\pgfpathcurveto{\pgfqpoint{2.685593in}{1.409767in}}{\pgfqpoint{2.712567in}{1.434578in}}{\pgfqpoint{2.727014in}{1.459366in}}%
\pgfpathcurveto{\pgfqpoint{2.741462in}{1.484155in}}{\pgfqpoint{2.742204in}{1.506897in}}{\pgfqpoint{2.729077in}{1.522585in}}%
\pgfpathcurveto{\pgfqpoint{2.715949in}{1.538272in}}{\pgfqpoint{2.690025in}{1.545624in}}{\pgfqpoint{2.657013in}{1.543021in}}%
\pgfpathcurveto{\pgfqpoint{2.624001in}{1.540418in}}{\pgfqpoint{2.586596in}{1.528072in}}{\pgfqpoint{2.553037in}{1.508703in}}%
\pgfpathclose%
\pgfusepath{stroke}%
\end{pgfscope}%
\begin{pgfscope}%
\pgfpathrectangle{\pgfqpoint{0.644217in}{0.500309in}}{\pgfqpoint{2.405783in}{1.149691in}}%
\pgfusepath{clip}%
\pgfsetbuttcap%
\pgfsetmiterjoin%
\pgfsetlinewidth{1.003750pt}%
\definecolor{currentstroke}{rgb}{0.000000,0.000000,1.000000}%
\pgfsetstrokecolor{currentstroke}%
\pgfsetdash{}{0pt}%
\pgfpathmoveto{\pgfqpoint{2.676031in}{1.492995in}}%
\pgfpathcurveto{\pgfqpoint{2.642473in}{1.473626in}}{\pgfqpoint{2.615499in}{1.448816in}}{\pgfqpoint{2.601052in}{1.424027in}}%
\pgfpathcurveto{\pgfqpoint{2.586605in}{1.399239in}}{\pgfqpoint{2.585863in}{1.376497in}}{\pgfqpoint{2.598990in}{1.360810in}}%
\pgfpathcurveto{\pgfqpoint{2.612117in}{1.345123in}}{\pgfqpoint{2.638041in}{1.337771in}}{\pgfqpoint{2.671052in}{1.340374in}}%
\pgfpathcurveto{\pgfqpoint{2.704064in}{1.342977in}}{\pgfqpoint{2.741468in}{1.355323in}}{\pgfqpoint{2.775026in}{1.374691in}}%
\pgfpathcurveto{\pgfqpoint{2.808585in}{1.394060in}}{\pgfqpoint{2.835558in}{1.418871in}}{\pgfqpoint{2.850005in}{1.443659in}}%
\pgfpathcurveto{\pgfqpoint{2.864453in}{1.468447in}}{\pgfqpoint{2.865195in}{1.491189in}}{\pgfqpoint{2.852068in}{1.506876in}}%
\pgfpathcurveto{\pgfqpoint{2.838941in}{1.522563in}}{\pgfqpoint{2.813017in}{1.529915in}}{\pgfqpoint{2.780005in}{1.527312in}}%
\pgfpathcurveto{\pgfqpoint{2.746994in}{1.524709in}}{\pgfqpoint{2.709590in}{1.512363in}}{\pgfqpoint{2.676031in}{1.492995in}}%
\pgfpathclose%
\pgfusepath{stroke}%
\end{pgfscope}%
\begin{pgfscope}%
\pgfpathrectangle{\pgfqpoint{0.644217in}{0.500309in}}{\pgfqpoint{2.405783in}{1.149691in}}%
\pgfusepath{clip}%
\pgfsetbuttcap%
\pgfsetmiterjoin%
\pgfsetlinewidth{1.003750pt}%
\definecolor{currentstroke}{rgb}{0.000000,0.000000,1.000000}%
\pgfsetstrokecolor{currentstroke}%
\pgfsetdash{}{0pt}%
\pgfpathmoveto{\pgfqpoint{2.826980in}{1.559351in}}%
\pgfpathcurveto{\pgfqpoint{2.793421in}{1.539983in}}{\pgfqpoint{2.766448in}{1.515172in}}{\pgfqpoint{2.752001in}{1.490384in}}%
\pgfpathcurveto{\pgfqpoint{2.737554in}{1.465596in}}{\pgfqpoint{2.736812in}{1.442855in}}{\pgfqpoint{2.749939in}{1.427168in}}%
\pgfpathcurveto{\pgfqpoint{2.763066in}{1.411481in}}{\pgfqpoint{2.788990in}{1.404129in}}{\pgfqpoint{2.822001in}{1.406732in}}%
\pgfpathcurveto{\pgfqpoint{2.855012in}{1.409335in}}{\pgfqpoint{2.892416in}{1.421681in}}{\pgfqpoint{2.925974in}{1.441049in}}%
\pgfpathcurveto{\pgfqpoint{2.959532in}{1.460417in}}{\pgfqpoint{2.986505in}{1.485228in}}{\pgfqpoint{3.000952in}{1.510016in}}%
\pgfpathcurveto{\pgfqpoint{3.015399in}{1.534804in}}{\pgfqpoint{3.016141in}{1.557545in}}{\pgfqpoint{3.003014in}{1.573232in}}%
\pgfpathcurveto{\pgfqpoint{2.989887in}{1.588919in}}{\pgfqpoint{2.963964in}{1.596271in}}{\pgfqpoint{2.930952in}{1.593668in}}%
\pgfpathcurveto{\pgfqpoint{2.897941in}{1.591065in}}{\pgfqpoint{2.860538in}{1.578719in}}{\pgfqpoint{2.826980in}{1.559351in}}%
\pgfpathclose%
\pgfusepath{stroke}%
\end{pgfscope}%
\begin{pgfscope}%
\pgfsetbuttcap%
\pgfsetmiterjoin%
\definecolor{currentfill}{rgb}{1.000000,1.000000,1.000000}%
\pgfsetfillcolor{currentfill}%
\pgfsetfillopacity{0.800000}%
\pgfsetlinewidth{1.003750pt}%
\definecolor{currentstroke}{rgb}{0.800000,0.800000,0.800000}%
\pgfsetstrokecolor{currentstroke}%
\pgfsetstrokeopacity{0.800000}%
\pgfsetdash{}{0pt}%
\pgfpathmoveto{\pgfqpoint{1.880850in}{0.548503in}}%
\pgfpathlineto{\pgfqpoint{2.982528in}{0.548503in}}%
\pgfpathquadraticcurveto{\pgfqpoint{3.001806in}{0.548503in}}{\pgfqpoint{3.001806in}{0.567781in}}%
\pgfpathlineto{\pgfqpoint{3.001806in}{1.231212in}}%
\pgfpathquadraticcurveto{\pgfqpoint{3.001806in}{1.250489in}}{\pgfqpoint{2.982528in}{1.250489in}}%
\pgfpathlineto{\pgfqpoint{1.880850in}{1.250489in}}%
\pgfpathquadraticcurveto{\pgfqpoint{1.861573in}{1.250489in}}{\pgfqpoint{1.861573in}{1.231212in}}%
\pgfpathlineto{\pgfqpoint{1.861573in}{0.567781in}}%
\pgfpathquadraticcurveto{\pgfqpoint{1.861573in}{0.548503in}}{\pgfqpoint{1.880850in}{0.548503in}}%
\pgfpathclose%
\pgfusepath{stroke,fill}%
\end{pgfscope}%
\begin{pgfscope}%
\pgfsetbuttcap%
\pgfsetroundjoin%
\definecolor{currentfill}{rgb}{0.196078,0.803922,0.196078}%
\pgfsetfillcolor{currentfill}%
\pgfsetlinewidth{1.505625pt}%
\definecolor{currentstroke}{rgb}{0.196078,0.803922,0.196078}%
\pgfsetstrokecolor{currentstroke}%
\pgfsetdash{}{0pt}%
\pgfpathmoveto{\pgfqpoint{1.954850in}{1.128054in}}%
\pgfpathlineto{\pgfqpoint{2.038184in}{1.211387in}}%
\pgfpathmoveto{\pgfqpoint{1.954850in}{1.211387in}}%
\pgfpathlineto{\pgfqpoint{2.038184in}{1.128054in}}%
\pgfusepath{stroke,fill}%
\end{pgfscope}%
\begin{pgfscope}%
\definecolor{textcolor}{rgb}{0.000000,0.000000,0.000000}%
\pgfsetstrokecolor{textcolor}%
\pgfsetfillcolor{textcolor}%
\pgftext[x=2.170017in,y=1.144419in,left,base]{\color{textcolor}\rmfamily\fontsize{6.940000}{8.328000}\selectfont Sensor 1}%
\end{pgfscope}%
\begin{pgfscope}%
\pgfsetbuttcap%
\pgfsetroundjoin%
\definecolor{currentfill}{rgb}{0.392157,0.584314,0.929412}%
\pgfsetfillcolor{currentfill}%
\pgfsetlinewidth{1.505625pt}%
\definecolor{currentstroke}{rgb}{0.392157,0.584314,0.929412}%
\pgfsetstrokecolor{currentstroke}%
\pgfsetdash{}{0pt}%
\pgfpathmoveto{\pgfqpoint{1.954850in}{0.993440in}}%
\pgfpathlineto{\pgfqpoint{2.038184in}{1.076773in}}%
\pgfpathmoveto{\pgfqpoint{1.954850in}{1.076773in}}%
\pgfpathlineto{\pgfqpoint{2.038184in}{0.993440in}}%
\pgfusepath{stroke,fill}%
\end{pgfscope}%
\begin{pgfscope}%
\definecolor{textcolor}{rgb}{0.000000,0.000000,0.000000}%
\pgfsetstrokecolor{textcolor}%
\pgfsetfillcolor{textcolor}%
\pgftext[x=2.170017in,y=1.009805in,left,base]{\color{textcolor}\rmfamily\fontsize{6.940000}{8.328000}\selectfont Sensor 2}%
\end{pgfscope}%
\begin{pgfscope}%
\pgfsetbuttcap%
\pgfsetroundjoin%
\definecolor{currentfill}{rgb}{1.000000,0.647059,0.000000}%
\pgfsetfillcolor{currentfill}%
\pgfsetlinewidth{1.505625pt}%
\definecolor{currentstroke}{rgb}{1.000000,0.647059,0.000000}%
\pgfsetstrokecolor{currentstroke}%
\pgfsetdash{}{0pt}%
\pgfpathmoveto{\pgfqpoint{1.954850in}{0.858826in}}%
\pgfpathlineto{\pgfqpoint{2.038184in}{0.942160in}}%
\pgfpathmoveto{\pgfqpoint{1.954850in}{0.942160in}}%
\pgfpathlineto{\pgfqpoint{2.038184in}{0.858826in}}%
\pgfusepath{stroke,fill}%
\end{pgfscope}%
\begin{pgfscope}%
\definecolor{textcolor}{rgb}{0.000000,0.000000,0.000000}%
\pgfsetstrokecolor{textcolor}%
\pgfsetfillcolor{textcolor}%
\pgftext[x=2.170017in,y=0.875191in,left,base]{\color{textcolor}\rmfamily\fontsize{6.940000}{8.328000}\selectfont Sensor 3}%
\end{pgfscope}%
\begin{pgfscope}%
\pgfsetbuttcap%
\pgfsetroundjoin%
\definecolor{currentfill}{rgb}{1.000000,0.000000,0.000000}%
\pgfsetfillcolor{currentfill}%
\pgfsetlinewidth{1.003750pt}%
\definecolor{currentstroke}{rgb}{1.000000,0.000000,0.000000}%
\pgfsetstrokecolor{currentstroke}%
\pgfsetdash{}{0pt}%
\pgfpathmoveto{\pgfqpoint{1.996517in}{0.745046in}}%
\pgfpathcurveto{\pgfqpoint{2.002042in}{0.745046in}}{\pgfqpoint{2.007342in}{0.747241in}}{\pgfqpoint{2.011248in}{0.751148in}}%
\pgfpathcurveto{\pgfqpoint{2.015155in}{0.755054in}}{\pgfqpoint{2.017350in}{0.760354in}}{\pgfqpoint{2.017350in}{0.765879in}}%
\pgfpathcurveto{\pgfqpoint{2.017350in}{0.771404in}}{\pgfqpoint{2.015155in}{0.776704in}}{\pgfqpoint{2.011248in}{0.780610in}}%
\pgfpathcurveto{\pgfqpoint{2.007342in}{0.784517in}}{\pgfqpoint{2.002042in}{0.786712in}}{\pgfqpoint{1.996517in}{0.786712in}}%
\pgfpathcurveto{\pgfqpoint{1.990992in}{0.786712in}}{\pgfqpoint{1.985692in}{0.784517in}}{\pgfqpoint{1.981786in}{0.780610in}}%
\pgfpathcurveto{\pgfqpoint{1.977879in}{0.776704in}}{\pgfqpoint{1.975684in}{0.771404in}}{\pgfqpoint{1.975684in}{0.765879in}}%
\pgfpathcurveto{\pgfqpoint{1.975684in}{0.760354in}}{\pgfqpoint{1.977879in}{0.755054in}}{\pgfqpoint{1.981786in}{0.751148in}}%
\pgfpathcurveto{\pgfqpoint{1.985692in}{0.747241in}}{\pgfqpoint{1.990992in}{0.745046in}}{\pgfqpoint{1.996517in}{0.745046in}}%
\pgfpathclose%
\pgfusepath{stroke,fill}%
\end{pgfscope}%
\begin{pgfscope}%
\definecolor{textcolor}{rgb}{0.000000,0.000000,0.000000}%
\pgfsetstrokecolor{textcolor}%
\pgfsetfillcolor{textcolor}%
\pgftext[x=2.170017in,y=0.740577in,left,base]{\color{textcolor}\rmfamily\fontsize{6.940000}{8.328000}\selectfont FCI estimate}%
\end{pgfscope}%
\begin{pgfscope}%
\pgfsetbuttcap%
\pgfsetroundjoin%
\definecolor{currentfill}{rgb}{0.000000,0.000000,1.000000}%
\pgfsetfillcolor{currentfill}%
\pgfsetlinewidth{1.003750pt}%
\definecolor{currentstroke}{rgb}{0.000000,0.000000,1.000000}%
\pgfsetstrokecolor{currentstroke}%
\pgfsetdash{}{0pt}%
\pgfpathmoveto{\pgfqpoint{1.996517in}{0.610432in}}%
\pgfpathcurveto{\pgfqpoint{2.002042in}{0.610432in}}{\pgfqpoint{2.007342in}{0.612627in}}{\pgfqpoint{2.011248in}{0.616534in}}%
\pgfpathcurveto{\pgfqpoint{2.015155in}{0.620441in}}{\pgfqpoint{2.017350in}{0.625740in}}{\pgfqpoint{2.017350in}{0.631265in}}%
\pgfpathcurveto{\pgfqpoint{2.017350in}{0.636790in}}{\pgfqpoint{2.015155in}{0.642090in}}{\pgfqpoint{2.011248in}{0.645997in}}%
\pgfpathcurveto{\pgfqpoint{2.007342in}{0.649903in}}{\pgfqpoint{2.002042in}{0.652098in}}{\pgfqpoint{1.996517in}{0.652098in}}%
\pgfpathcurveto{\pgfqpoint{1.990992in}{0.652098in}}{\pgfqpoint{1.985692in}{0.649903in}}{\pgfqpoint{1.981786in}{0.645997in}}%
\pgfpathcurveto{\pgfqpoint{1.977879in}{0.642090in}}{\pgfqpoint{1.975684in}{0.636790in}}{\pgfqpoint{1.975684in}{0.631265in}}%
\pgfpathcurveto{\pgfqpoint{1.975684in}{0.625740in}}{\pgfqpoint{1.977879in}{0.620441in}}{\pgfqpoint{1.981786in}{0.616534in}}%
\pgfpathcurveto{\pgfqpoint{1.985692in}{0.612627in}}{\pgfqpoint{1.990992in}{0.610432in}}{\pgfqpoint{1.996517in}{0.610432in}}%
\pgfpathclose%
\pgfusepath{stroke,fill}%
\end{pgfscope}%
\begin{pgfscope}%
\definecolor{textcolor}{rgb}{0.000000,0.000000,0.000000}%
\pgfsetstrokecolor{textcolor}%
\pgfsetfillcolor{textcolor}%
\pgftext[x=2.170017in,y=0.605963in,left,base]{\color{textcolor}\rmfamily\fontsize{6.940000}{8.328000}\selectfont SecFCI estimate}%
\end{pgfscope}%
\end{pgfpicture}%
\makeatother%
\endgroup%

%DIFDELCMD <    %%%
\DIFdelendFL \DIFaddbeginFL \includegraphics{images/fci_secfci_cmp.pdf}
   \DIFaddendFL \end{center}
   \vspace{-15pt}
   \caption{Tracking simulation comparing SecFCI and FCI.}
   \vspace{-\baselineskip}
   \label{fig:fci_secfci_traj}
\end{figure}

To derive an upper bound on the accuracy difference between SecFCI and FCI, we note the two factors which introduce inconsistency between the two methods\DIFdelbegin \DIFdel{. Encoding }\DIFdelend \DIFaddbegin \DIFadd{: the encoding method }\DIFaddend from section \ref{subsec:complexity}, and the difference in fusion weights. Due to the possibility of choosing sufficiently large integer and fractional bit lengths $i$ and $f$, we will only consider the error caused by the difference in weights. We will treat this error as the distance between respective weight vectors
\begin{equation}
   \DIFdelbegin %DIFDELCMD < \begin{aligned}
%DIFDELCMD <       &\vec{\omega}_{SecFCI} = (\omega_{1,SecFCI},\dots,\omega_{n,SecFCI})\,, \\
%DIFDELCMD <       &\vec{\omega}_{FCI} = (\omega_{1,FCI},\dots,\omega_{n,FCI})
%DIFDELCMD <    \end{aligned}%%%
\DIFdelend \DIFaddbegin \begin{aligned}
      &\vec{\omega}_{SecFCI} = (\omega_{1,SecFCI},\dots,\omega_{n,SecFCI}) \\
      &\vec{\omega}_{FCI} = (\omega_{1,FCI},\dots,\omega_{n,FCI})\enspace,
   \end{aligned}\DIFaddend 
\end{equation}
where $\omega_{i,s}$ denotes weight $\omega_i$ from algorithm $s$. From section \ref{sec:secfci} we see that the largest difference $|\omega_{i,FCI} - \omega_{i,SecFCI}|$ is strictly bounded by $s/2$. \DIFdelbegin \DIFdel{Section \ref{sec:multi_secfci}shows that }\DIFdelend \DIFaddbegin \DIFadd{As shown in section \ref{sec:multi_secfci}, }\DIFaddend when more sensors are involved, a tighter bound on this difference is dependent on the value of $\vec{\omega}_{i,FCI}$, but will remain strictly bounded by $s/2$. Therefore, we can give a strict upper bound on the distance between weight vectors as
\begin{equation}
   |\vec{\omega}_{FCI} - \vec{\omega}_{SecFCI}| < 0.5\sqrt{ns^2}\DIFdelbegin \DIFdel{\,}\DIFdelend \DIFaddbegin \enspace\DIFaddend . \label{eqn:accuracy_error_bound}
\end{equation}

Finally, components of $\omega_{i,SecFCI}$, $\omega_{i,FCI}$ and the errors $|\vec{\omega}_{FCI} - \vec{\omega}_{SecFCI}|$, have been plotted over time in Fig. \ref{fig:fci_secfci_omegas}, and show the \DIFdelbegin \DIFdel{commputed }\DIFdelend \DIFaddbegin \DIFadd{computed }\DIFaddend error bound when $n=3$ and $s=0.1$.
\begin{figure}[tb]
   %\vspace{-5pt}
   %\vspace{-10pt}
   \begin{center}
      \DIFdelbeginFL %DIFDELCMD < %% Creator: Matplotlib, PGF backend
%%
%% To include the figure in your LaTeX document, write
%%   \input{<filename>.pgf}
%%
%% Make sure the required packages are loaded in your preamble
%%   \usepackage{pgf}
%%
%% Figures using additional raster images can only be included by \input if
%% they are in the same directory as the main LaTeX file. For loading figures
%% from other directories you can use the `import` package
%%   \usepackage{import}
%% and then include the figures with
%%   \import{<path to file>}{<filename>.pgf}
%%
%% Matplotlib used the following preamble
%%
\begingroup%
\makeatletter%
\begin{pgfpicture}%
\pgfpathrectangle{\pgfpointorigin}{\pgfqpoint{3.200000in}{1.550000in}}%
\pgfusepath{use as bounding box, clip}%
\begin{pgfscope}%
\pgfsetbuttcap%
\pgfsetmiterjoin%
\definecolor{currentfill}{rgb}{1.000000,1.000000,1.000000}%
\pgfsetfillcolor{currentfill}%
\pgfsetlinewidth{0.000000pt}%
\definecolor{currentstroke}{rgb}{1.000000,1.000000,1.000000}%
\pgfsetstrokecolor{currentstroke}%
\pgfsetdash{}{0pt}%
\pgfpathmoveto{\pgfqpoint{0.000000in}{0.000000in}}%
\pgfpathlineto{\pgfqpoint{3.200000in}{0.000000in}}%
\pgfpathlineto{\pgfqpoint{3.200000in}{1.550000in}}%
\pgfpathlineto{\pgfqpoint{0.000000in}{1.550000in}}%
\pgfpathclose%
\pgfusepath{fill}%
\end{pgfscope}%
\begin{pgfscope}%
\pgfsetbuttcap%
\pgfsetmiterjoin%
\definecolor{currentfill}{rgb}{1.000000,1.000000,1.000000}%
\pgfsetfillcolor{currentfill}%
\pgfsetlinewidth{0.000000pt}%
\definecolor{currentstroke}{rgb}{0.000000,0.000000,0.000000}%
\pgfsetstrokecolor{currentstroke}%
\pgfsetstrokeopacity{0.000000}%
\pgfsetdash{}{0pt}%
\pgfpathmoveto{\pgfqpoint{0.552394in}{0.500309in}}%
\pgfpathlineto{\pgfqpoint{3.050000in}{0.500309in}}%
\pgfpathlineto{\pgfqpoint{3.050000in}{1.400000in}}%
\pgfpathlineto{\pgfqpoint{0.552394in}{1.400000in}}%
\pgfpathclose%
\pgfusepath{fill}%
\end{pgfscope}%
\begin{pgfscope}%
\pgfsetbuttcap%
\pgfsetroundjoin%
\definecolor{currentfill}{rgb}{0.000000,0.000000,0.000000}%
\pgfsetfillcolor{currentfill}%
\pgfsetlinewidth{0.803000pt}%
\definecolor{currentstroke}{rgb}{0.000000,0.000000,0.000000}%
\pgfsetstrokecolor{currentstroke}%
\pgfsetdash{}{0pt}%
\pgfsys@defobject{currentmarker}{\pgfqpoint{0.000000in}{-0.048611in}}{\pgfqpoint{0.000000in}{0.000000in}}{%
\pgfpathmoveto{\pgfqpoint{0.000000in}{0.000000in}}%
\pgfpathlineto{\pgfqpoint{0.000000in}{-0.048611in}}%
\pgfusepath{stroke,fill}%
}%
\begin{pgfscope}%
\pgfsys@transformshift{0.665922in}{0.500309in}%
\pgfsys@useobject{currentmarker}{}%
\end{pgfscope}%
\end{pgfscope}%
\begin{pgfscope}%
\definecolor{textcolor}{rgb}{0.000000,0.000000,0.000000}%
\pgfsetstrokecolor{textcolor}%
\pgfsetfillcolor{textcolor}%
\pgftext[x=0.665922in,y=0.403087in,,top]{\color{textcolor}\rmfamily\fontsize{8.330000}{9.996000}\selectfont \(\displaystyle 0\)}%
\end{pgfscope}%
\begin{pgfscope}%
\pgfsetbuttcap%
\pgfsetroundjoin%
\definecolor{currentfill}{rgb}{0.000000,0.000000,0.000000}%
\pgfsetfillcolor{currentfill}%
\pgfsetlinewidth{0.803000pt}%
\definecolor{currentstroke}{rgb}{0.000000,0.000000,0.000000}%
\pgfsetstrokecolor{currentstroke}%
\pgfsetdash{}{0pt}%
\pgfsys@defobject{currentmarker}{\pgfqpoint{0.000000in}{-0.048611in}}{\pgfqpoint{0.000000in}{0.000000in}}{%
\pgfpathmoveto{\pgfqpoint{0.000000in}{0.000000in}}%
\pgfpathlineto{\pgfqpoint{0.000000in}{-0.048611in}}%
\pgfusepath{stroke,fill}%
}%
\begin{pgfscope}%
\pgfsys@transformshift{1.129300in}{0.500309in}%
\pgfsys@useobject{currentmarker}{}%
\end{pgfscope}%
\end{pgfscope}%
\begin{pgfscope}%
\definecolor{textcolor}{rgb}{0.000000,0.000000,0.000000}%
\pgfsetstrokecolor{textcolor}%
\pgfsetfillcolor{textcolor}%
\pgftext[x=1.129300in,y=0.403087in,,top]{\color{textcolor}\rmfamily\fontsize{8.330000}{9.996000}\selectfont \(\displaystyle 10\)}%
\end{pgfscope}%
\begin{pgfscope}%
\pgfsetbuttcap%
\pgfsetroundjoin%
\definecolor{currentfill}{rgb}{0.000000,0.000000,0.000000}%
\pgfsetfillcolor{currentfill}%
\pgfsetlinewidth{0.803000pt}%
\definecolor{currentstroke}{rgb}{0.000000,0.000000,0.000000}%
\pgfsetstrokecolor{currentstroke}%
\pgfsetdash{}{0pt}%
\pgfsys@defobject{currentmarker}{\pgfqpoint{0.000000in}{-0.048611in}}{\pgfqpoint{0.000000in}{0.000000in}}{%
\pgfpathmoveto{\pgfqpoint{0.000000in}{0.000000in}}%
\pgfpathlineto{\pgfqpoint{0.000000in}{-0.048611in}}%
\pgfusepath{stroke,fill}%
}%
\begin{pgfscope}%
\pgfsys@transformshift{1.592677in}{0.500309in}%
\pgfsys@useobject{currentmarker}{}%
\end{pgfscope}%
\end{pgfscope}%
\begin{pgfscope}%
\definecolor{textcolor}{rgb}{0.000000,0.000000,0.000000}%
\pgfsetstrokecolor{textcolor}%
\pgfsetfillcolor{textcolor}%
\pgftext[x=1.592677in,y=0.403087in,,top]{\color{textcolor}\rmfamily\fontsize{8.330000}{9.996000}\selectfont \(\displaystyle 20\)}%
\end{pgfscope}%
\begin{pgfscope}%
\pgfsetbuttcap%
\pgfsetroundjoin%
\definecolor{currentfill}{rgb}{0.000000,0.000000,0.000000}%
\pgfsetfillcolor{currentfill}%
\pgfsetlinewidth{0.803000pt}%
\definecolor{currentstroke}{rgb}{0.000000,0.000000,0.000000}%
\pgfsetstrokecolor{currentstroke}%
\pgfsetdash{}{0pt}%
\pgfsys@defobject{currentmarker}{\pgfqpoint{0.000000in}{-0.048611in}}{\pgfqpoint{0.000000in}{0.000000in}}{%
\pgfpathmoveto{\pgfqpoint{0.000000in}{0.000000in}}%
\pgfpathlineto{\pgfqpoint{0.000000in}{-0.048611in}}%
\pgfusepath{stroke,fill}%
}%
\begin{pgfscope}%
\pgfsys@transformshift{2.056055in}{0.500309in}%
\pgfsys@useobject{currentmarker}{}%
\end{pgfscope}%
\end{pgfscope}%
\begin{pgfscope}%
\definecolor{textcolor}{rgb}{0.000000,0.000000,0.000000}%
\pgfsetstrokecolor{textcolor}%
\pgfsetfillcolor{textcolor}%
\pgftext[x=2.056055in,y=0.403087in,,top]{\color{textcolor}\rmfamily\fontsize{8.330000}{9.996000}\selectfont \(\displaystyle 30\)}%
\end{pgfscope}%
\begin{pgfscope}%
\pgfsetbuttcap%
\pgfsetroundjoin%
\definecolor{currentfill}{rgb}{0.000000,0.000000,0.000000}%
\pgfsetfillcolor{currentfill}%
\pgfsetlinewidth{0.803000pt}%
\definecolor{currentstroke}{rgb}{0.000000,0.000000,0.000000}%
\pgfsetstrokecolor{currentstroke}%
\pgfsetdash{}{0pt}%
\pgfsys@defobject{currentmarker}{\pgfqpoint{0.000000in}{-0.048611in}}{\pgfqpoint{0.000000in}{0.000000in}}{%
\pgfpathmoveto{\pgfqpoint{0.000000in}{0.000000in}}%
\pgfpathlineto{\pgfqpoint{0.000000in}{-0.048611in}}%
\pgfusepath{stroke,fill}%
}%
\begin{pgfscope}%
\pgfsys@transformshift{2.519433in}{0.500309in}%
\pgfsys@useobject{currentmarker}{}%
\end{pgfscope}%
\end{pgfscope}%
\begin{pgfscope}%
\definecolor{textcolor}{rgb}{0.000000,0.000000,0.000000}%
\pgfsetstrokecolor{textcolor}%
\pgfsetfillcolor{textcolor}%
\pgftext[x=2.519433in,y=0.403087in,,top]{\color{textcolor}\rmfamily\fontsize{8.330000}{9.996000}\selectfont \(\displaystyle 40\)}%
\end{pgfscope}%
\begin{pgfscope}%
\pgfsetbuttcap%
\pgfsetroundjoin%
\definecolor{currentfill}{rgb}{0.000000,0.000000,0.000000}%
\pgfsetfillcolor{currentfill}%
\pgfsetlinewidth{0.803000pt}%
\definecolor{currentstroke}{rgb}{0.000000,0.000000,0.000000}%
\pgfsetstrokecolor{currentstroke}%
\pgfsetdash{}{0pt}%
\pgfsys@defobject{currentmarker}{\pgfqpoint{0.000000in}{-0.048611in}}{\pgfqpoint{0.000000in}{0.000000in}}{%
\pgfpathmoveto{\pgfqpoint{0.000000in}{0.000000in}}%
\pgfpathlineto{\pgfqpoint{0.000000in}{-0.048611in}}%
\pgfusepath{stroke,fill}%
}%
\begin{pgfscope}%
\pgfsys@transformshift{2.982810in}{0.500309in}%
\pgfsys@useobject{currentmarker}{}%
\end{pgfscope}%
\end{pgfscope}%
\begin{pgfscope}%
\definecolor{textcolor}{rgb}{0.000000,0.000000,0.000000}%
\pgfsetstrokecolor{textcolor}%
\pgfsetfillcolor{textcolor}%
\pgftext[x=2.982810in,y=0.403087in,,top]{\color{textcolor}\rmfamily\fontsize{8.330000}{9.996000}\selectfont \(\displaystyle 50\)}%
\end{pgfscope}%
\begin{pgfscope}%
\definecolor{textcolor}{rgb}{0.000000,0.000000,0.000000}%
\pgfsetstrokecolor{textcolor}%
\pgfsetfillcolor{textcolor}%
\pgftext[x=1.801197in,y=0.248766in,,top]{\color{textcolor}\rmfamily\fontsize{8.330000}{9.996000}\selectfont Time}%
\end{pgfscope}%
\begin{pgfscope}%
\pgfsetbuttcap%
\pgfsetroundjoin%
\definecolor{currentfill}{rgb}{0.000000,0.000000,0.000000}%
\pgfsetfillcolor{currentfill}%
\pgfsetlinewidth{0.803000pt}%
\definecolor{currentstroke}{rgb}{0.000000,0.000000,0.000000}%
\pgfsetstrokecolor{currentstroke}%
\pgfsetdash{}{0pt}%
\pgfsys@defobject{currentmarker}{\pgfqpoint{-0.048611in}{0.000000in}}{\pgfqpoint{0.000000in}{0.000000in}}{%
\pgfpathmoveto{\pgfqpoint{0.000000in}{0.000000in}}%
\pgfpathlineto{\pgfqpoint{-0.048611in}{0.000000in}}%
\pgfusepath{stroke,fill}%
}%
\begin{pgfscope}%
\pgfsys@transformshift{0.552394in}{0.894074in}%
\pgfsys@useobject{currentmarker}{}%
\end{pgfscope}%
\end{pgfscope}%
\begin{pgfscope}%
\definecolor{textcolor}{rgb}{0.000000,0.000000,0.000000}%
\pgfsetstrokecolor{textcolor}%
\pgfsetfillcolor{textcolor}%
\pgftext[x=0.304321in,y=0.855494in,left,base]{\color{textcolor}\rmfamily\fontsize{8.330000}{9.996000}\selectfont \(\displaystyle 0.2\)}%
\end{pgfscope}%
\begin{pgfscope}%
\pgfsetbuttcap%
\pgfsetroundjoin%
\definecolor{currentfill}{rgb}{0.000000,0.000000,0.000000}%
\pgfsetfillcolor{currentfill}%
\pgfsetlinewidth{0.803000pt}%
\definecolor{currentstroke}{rgb}{0.000000,0.000000,0.000000}%
\pgfsetstrokecolor{currentstroke}%
\pgfsetdash{}{0pt}%
\pgfsys@defobject{currentmarker}{\pgfqpoint{-0.048611in}{0.000000in}}{\pgfqpoint{0.000000in}{0.000000in}}{%
\pgfpathmoveto{\pgfqpoint{0.000000in}{0.000000in}}%
\pgfpathlineto{\pgfqpoint{-0.048611in}{0.000000in}}%
\pgfusepath{stroke,fill}%
}%
\begin{pgfscope}%
\pgfsys@transformshift{0.552394in}{1.354516in}%
\pgfsys@useobject{currentmarker}{}%
\end{pgfscope}%
\end{pgfscope}%
\begin{pgfscope}%
\definecolor{textcolor}{rgb}{0.000000,0.000000,0.000000}%
\pgfsetstrokecolor{textcolor}%
\pgfsetfillcolor{textcolor}%
\pgftext[x=0.304321in,y=1.315936in,left,base]{\color{textcolor}\rmfamily\fontsize{8.330000}{9.996000}\selectfont \(\displaystyle 0.4\)}%
\end{pgfscope}%
\begin{pgfscope}%
\definecolor{textcolor}{rgb}{0.000000,0.000000,0.000000}%
\pgfsetstrokecolor{textcolor}%
\pgfsetfillcolor{textcolor}%
\pgftext[x=0.248766in,y=0.950154in,,bottom,rotate=90.000000]{\color{textcolor}\rmfamily\fontsize{8.330000}{9.996000}\selectfont Values of \(\displaystyle \omega_i\)}%
\end{pgfscope}%
\begin{pgfscope}%
\pgfpathrectangle{\pgfqpoint{0.552394in}{0.500309in}}{\pgfqpoint{2.497606in}{0.899691in}}%
\pgfusepath{clip}%
\pgfsetrectcap%
\pgfsetroundjoin%
\pgfsetlinewidth{1.505625pt}%
\definecolor{currentstroke}{rgb}{0.900000,0.000000,0.000000}%
\pgfsetstrokecolor{currentstroke}%
\pgfsetdash{}{0pt}%
\pgfpathmoveto{\pgfqpoint{0.665922in}{1.194904in}}%
\pgfpathlineto{\pgfqpoint{0.712260in}{1.182082in}}%
\pgfpathlineto{\pgfqpoint{0.758597in}{1.173065in}}%
\pgfpathlineto{\pgfqpoint{0.804935in}{1.169011in}}%
\pgfpathlineto{\pgfqpoint{0.851273in}{1.166634in}}%
\pgfpathlineto{\pgfqpoint{0.897611in}{1.164757in}}%
\pgfpathlineto{\pgfqpoint{0.943948in}{1.163113in}}%
\pgfpathlineto{\pgfqpoint{0.990286in}{1.161669in}}%
\pgfpathlineto{\pgfqpoint{1.036624in}{1.160429in}}%
\pgfpathlineto{\pgfqpoint{1.082962in}{1.159390in}}%
\pgfpathlineto{\pgfqpoint{1.129300in}{1.158534in}}%
\pgfpathlineto{\pgfqpoint{1.175637in}{1.157841in}}%
\pgfpathlineto{\pgfqpoint{1.221975in}{1.157284in}}%
\pgfpathlineto{\pgfqpoint{1.268313in}{1.156842in}}%
\pgfpathlineto{\pgfqpoint{1.314651in}{1.156492in}}%
\pgfpathlineto{\pgfqpoint{1.360988in}{1.156217in}}%
\pgfpathlineto{\pgfqpoint{1.407326in}{1.156002in}}%
\pgfpathlineto{\pgfqpoint{1.453664in}{1.155835in}}%
\pgfpathlineto{\pgfqpoint{1.500002in}{1.155705in}}%
\pgfpathlineto{\pgfqpoint{1.546339in}{1.155605in}}%
\pgfpathlineto{\pgfqpoint{1.592677in}{1.155528in}}%
\pgfpathlineto{\pgfqpoint{1.639015in}{1.155470in}}%
\pgfpathlineto{\pgfqpoint{1.685353in}{1.155425in}}%
\pgfpathlineto{\pgfqpoint{1.731691in}{1.155392in}}%
\pgfpathlineto{\pgfqpoint{1.778028in}{1.155367in}}%
\pgfpathlineto{\pgfqpoint{1.824366in}{1.155348in}}%
\pgfpathlineto{\pgfqpoint{1.870704in}{1.155335in}}%
\pgfpathlineto{\pgfqpoint{1.917042in}{1.155326in}}%
\pgfpathlineto{\pgfqpoint{1.963379in}{1.155319in}}%
\pgfpathlineto{\pgfqpoint{2.009717in}{1.155315in}}%
\pgfpathlineto{\pgfqpoint{2.056055in}{1.155312in}}%
\pgfpathlineto{\pgfqpoint{2.102393in}{1.155311in}}%
\pgfpathlineto{\pgfqpoint{2.148730in}{1.155310in}}%
\pgfpathlineto{\pgfqpoint{2.195068in}{1.155310in}}%
\pgfpathlineto{\pgfqpoint{2.241406in}{1.155310in}}%
\pgfpathlineto{\pgfqpoint{2.287744in}{1.155311in}}%
\pgfpathlineto{\pgfqpoint{2.334081in}{1.155311in}}%
\pgfpathlineto{\pgfqpoint{2.380419in}{1.155312in}}%
\pgfpathlineto{\pgfqpoint{2.426757in}{1.155313in}}%
\pgfpathlineto{\pgfqpoint{2.473095in}{1.155314in}}%
\pgfpathlineto{\pgfqpoint{2.519433in}{1.155315in}}%
\pgfpathlineto{\pgfqpoint{2.565770in}{1.155316in}}%
\pgfpathlineto{\pgfqpoint{2.612108in}{1.155317in}}%
\pgfpathlineto{\pgfqpoint{2.658446in}{1.155317in}}%
\pgfpathlineto{\pgfqpoint{2.704784in}{1.155318in}}%
\pgfpathlineto{\pgfqpoint{2.751121in}{1.155319in}}%
\pgfpathlineto{\pgfqpoint{2.797459in}{1.155319in}}%
\pgfpathlineto{\pgfqpoint{2.843797in}{1.155319in}}%
\pgfpathlineto{\pgfqpoint{2.890135in}{1.155320in}}%
\pgfpathlineto{\pgfqpoint{2.936472in}{1.155320in}}%
\pgfusepath{stroke}%
\end{pgfscope}%
\begin{pgfscope}%
\pgfpathrectangle{\pgfqpoint{0.552394in}{0.500309in}}{\pgfqpoint{2.497606in}{0.899691in}}%
\pgfusepath{clip}%
\pgfsetbuttcap%
\pgfsetroundjoin%
\definecolor{currentfill}{rgb}{0.900000,0.000000,0.000000}%
\pgfsetfillcolor{currentfill}%
\pgfsetlinewidth{1.003750pt}%
\definecolor{currentstroke}{rgb}{0.900000,0.000000,0.000000}%
\pgfsetstrokecolor{currentstroke}%
\pgfsetdash{}{0pt}%
\pgfsys@defobject{currentmarker}{\pgfqpoint{-0.020833in}{-0.020833in}}{\pgfqpoint{0.020833in}{0.020833in}}{%
\pgfpathmoveto{\pgfqpoint{0.000000in}{-0.020833in}}%
\pgfpathcurveto{\pgfqpoint{0.005525in}{-0.020833in}}{\pgfqpoint{0.010825in}{-0.018638in}}{\pgfqpoint{0.014731in}{-0.014731in}}%
\pgfpathcurveto{\pgfqpoint{0.018638in}{-0.010825in}}{\pgfqpoint{0.020833in}{-0.005525in}}{\pgfqpoint{0.020833in}{0.000000in}}%
\pgfpathcurveto{\pgfqpoint{0.020833in}{0.005525in}}{\pgfqpoint{0.018638in}{0.010825in}}{\pgfqpoint{0.014731in}{0.014731in}}%
\pgfpathcurveto{\pgfqpoint{0.010825in}{0.018638in}}{\pgfqpoint{0.005525in}{0.020833in}}{\pgfqpoint{0.000000in}{0.020833in}}%
\pgfpathcurveto{\pgfqpoint{-0.005525in}{0.020833in}}{\pgfqpoint{-0.010825in}{0.018638in}}{\pgfqpoint{-0.014731in}{0.014731in}}%
\pgfpathcurveto{\pgfqpoint{-0.018638in}{0.010825in}}{\pgfqpoint{-0.020833in}{0.005525in}}{\pgfqpoint{-0.020833in}{0.000000in}}%
\pgfpathcurveto{\pgfqpoint{-0.020833in}{-0.005525in}}{\pgfqpoint{-0.018638in}{-0.010825in}}{\pgfqpoint{-0.014731in}{-0.014731in}}%
\pgfpathcurveto{\pgfqpoint{-0.010825in}{-0.018638in}}{\pgfqpoint{-0.005525in}{-0.020833in}}{\pgfqpoint{0.000000in}{-0.020833in}}%
\pgfpathclose%
\pgfusepath{stroke,fill}%
}%
\begin{pgfscope}%
\pgfsys@transformshift{0.665922in}{1.194904in}%
\pgfsys@useobject{currentmarker}{}%
\end{pgfscope}%
\begin{pgfscope}%
\pgfsys@transformshift{0.712260in}{1.182082in}%
\pgfsys@useobject{currentmarker}{}%
\end{pgfscope}%
\begin{pgfscope}%
\pgfsys@transformshift{0.758597in}{1.173065in}%
\pgfsys@useobject{currentmarker}{}%
\end{pgfscope}%
\begin{pgfscope}%
\pgfsys@transformshift{0.804935in}{1.169011in}%
\pgfsys@useobject{currentmarker}{}%
\end{pgfscope}%
\begin{pgfscope}%
\pgfsys@transformshift{0.851273in}{1.166634in}%
\pgfsys@useobject{currentmarker}{}%
\end{pgfscope}%
\begin{pgfscope}%
\pgfsys@transformshift{0.897611in}{1.164757in}%
\pgfsys@useobject{currentmarker}{}%
\end{pgfscope}%
\begin{pgfscope}%
\pgfsys@transformshift{0.943948in}{1.163113in}%
\pgfsys@useobject{currentmarker}{}%
\end{pgfscope}%
\begin{pgfscope}%
\pgfsys@transformshift{0.990286in}{1.161669in}%
\pgfsys@useobject{currentmarker}{}%
\end{pgfscope}%
\begin{pgfscope}%
\pgfsys@transformshift{1.036624in}{1.160429in}%
\pgfsys@useobject{currentmarker}{}%
\end{pgfscope}%
\begin{pgfscope}%
\pgfsys@transformshift{1.082962in}{1.159390in}%
\pgfsys@useobject{currentmarker}{}%
\end{pgfscope}%
\begin{pgfscope}%
\pgfsys@transformshift{1.129300in}{1.158534in}%
\pgfsys@useobject{currentmarker}{}%
\end{pgfscope}%
\begin{pgfscope}%
\pgfsys@transformshift{1.175637in}{1.157841in}%
\pgfsys@useobject{currentmarker}{}%
\end{pgfscope}%
\begin{pgfscope}%
\pgfsys@transformshift{1.221975in}{1.157284in}%
\pgfsys@useobject{currentmarker}{}%
\end{pgfscope}%
\begin{pgfscope}%
\pgfsys@transformshift{1.268313in}{1.156842in}%
\pgfsys@useobject{currentmarker}{}%
\end{pgfscope}%
\begin{pgfscope}%
\pgfsys@transformshift{1.314651in}{1.156492in}%
\pgfsys@useobject{currentmarker}{}%
\end{pgfscope}%
\begin{pgfscope}%
\pgfsys@transformshift{1.360988in}{1.156217in}%
\pgfsys@useobject{currentmarker}{}%
\end{pgfscope}%
\begin{pgfscope}%
\pgfsys@transformshift{1.407326in}{1.156002in}%
\pgfsys@useobject{currentmarker}{}%
\end{pgfscope}%
\begin{pgfscope}%
\pgfsys@transformshift{1.453664in}{1.155835in}%
\pgfsys@useobject{currentmarker}{}%
\end{pgfscope}%
\begin{pgfscope}%
\pgfsys@transformshift{1.500002in}{1.155705in}%
\pgfsys@useobject{currentmarker}{}%
\end{pgfscope}%
\begin{pgfscope}%
\pgfsys@transformshift{1.546339in}{1.155605in}%
\pgfsys@useobject{currentmarker}{}%
\end{pgfscope}%
\begin{pgfscope}%
\pgfsys@transformshift{1.592677in}{1.155528in}%
\pgfsys@useobject{currentmarker}{}%
\end{pgfscope}%
\begin{pgfscope}%
\pgfsys@transformshift{1.639015in}{1.155470in}%
\pgfsys@useobject{currentmarker}{}%
\end{pgfscope}%
\begin{pgfscope}%
\pgfsys@transformshift{1.685353in}{1.155425in}%
\pgfsys@useobject{currentmarker}{}%
\end{pgfscope}%
\begin{pgfscope}%
\pgfsys@transformshift{1.731691in}{1.155392in}%
\pgfsys@useobject{currentmarker}{}%
\end{pgfscope}%
\begin{pgfscope}%
\pgfsys@transformshift{1.778028in}{1.155367in}%
\pgfsys@useobject{currentmarker}{}%
\end{pgfscope}%
\begin{pgfscope}%
\pgfsys@transformshift{1.824366in}{1.155348in}%
\pgfsys@useobject{currentmarker}{}%
\end{pgfscope}%
\begin{pgfscope}%
\pgfsys@transformshift{1.870704in}{1.155335in}%
\pgfsys@useobject{currentmarker}{}%
\end{pgfscope}%
\begin{pgfscope}%
\pgfsys@transformshift{1.917042in}{1.155326in}%
\pgfsys@useobject{currentmarker}{}%
\end{pgfscope}%
\begin{pgfscope}%
\pgfsys@transformshift{1.963379in}{1.155319in}%
\pgfsys@useobject{currentmarker}{}%
\end{pgfscope}%
\begin{pgfscope}%
\pgfsys@transformshift{2.009717in}{1.155315in}%
\pgfsys@useobject{currentmarker}{}%
\end{pgfscope}%
\begin{pgfscope}%
\pgfsys@transformshift{2.056055in}{1.155312in}%
\pgfsys@useobject{currentmarker}{}%
\end{pgfscope}%
\begin{pgfscope}%
\pgfsys@transformshift{2.102393in}{1.155311in}%
\pgfsys@useobject{currentmarker}{}%
\end{pgfscope}%
\begin{pgfscope}%
\pgfsys@transformshift{2.148730in}{1.155310in}%
\pgfsys@useobject{currentmarker}{}%
\end{pgfscope}%
\begin{pgfscope}%
\pgfsys@transformshift{2.195068in}{1.155310in}%
\pgfsys@useobject{currentmarker}{}%
\end{pgfscope}%
\begin{pgfscope}%
\pgfsys@transformshift{2.241406in}{1.155310in}%
\pgfsys@useobject{currentmarker}{}%
\end{pgfscope}%
\begin{pgfscope}%
\pgfsys@transformshift{2.287744in}{1.155311in}%
\pgfsys@useobject{currentmarker}{}%
\end{pgfscope}%
\begin{pgfscope}%
\pgfsys@transformshift{2.334081in}{1.155311in}%
\pgfsys@useobject{currentmarker}{}%
\end{pgfscope}%
\begin{pgfscope}%
\pgfsys@transformshift{2.380419in}{1.155312in}%
\pgfsys@useobject{currentmarker}{}%
\end{pgfscope}%
\begin{pgfscope}%
\pgfsys@transformshift{2.426757in}{1.155313in}%
\pgfsys@useobject{currentmarker}{}%
\end{pgfscope}%
\begin{pgfscope}%
\pgfsys@transformshift{2.473095in}{1.155314in}%
\pgfsys@useobject{currentmarker}{}%
\end{pgfscope}%
\begin{pgfscope}%
\pgfsys@transformshift{2.519433in}{1.155315in}%
\pgfsys@useobject{currentmarker}{}%
\end{pgfscope}%
\begin{pgfscope}%
\pgfsys@transformshift{2.565770in}{1.155316in}%
\pgfsys@useobject{currentmarker}{}%
\end{pgfscope}%
\begin{pgfscope}%
\pgfsys@transformshift{2.612108in}{1.155317in}%
\pgfsys@useobject{currentmarker}{}%
\end{pgfscope}%
\begin{pgfscope}%
\pgfsys@transformshift{2.658446in}{1.155317in}%
\pgfsys@useobject{currentmarker}{}%
\end{pgfscope}%
\begin{pgfscope}%
\pgfsys@transformshift{2.704784in}{1.155318in}%
\pgfsys@useobject{currentmarker}{}%
\end{pgfscope}%
\begin{pgfscope}%
\pgfsys@transformshift{2.751121in}{1.155319in}%
\pgfsys@useobject{currentmarker}{}%
\end{pgfscope}%
\begin{pgfscope}%
\pgfsys@transformshift{2.797459in}{1.155319in}%
\pgfsys@useobject{currentmarker}{}%
\end{pgfscope}%
\begin{pgfscope}%
\pgfsys@transformshift{2.843797in}{1.155319in}%
\pgfsys@useobject{currentmarker}{}%
\end{pgfscope}%
\begin{pgfscope}%
\pgfsys@transformshift{2.890135in}{1.155320in}%
\pgfsys@useobject{currentmarker}{}%
\end{pgfscope}%
\begin{pgfscope}%
\pgfsys@transformshift{2.936472in}{1.155320in}%
\pgfsys@useobject{currentmarker}{}%
\end{pgfscope}%
\end{pgfscope}%
\begin{pgfscope}%
\pgfpathrectangle{\pgfqpoint{0.552394in}{0.500309in}}{\pgfqpoint{2.497606in}{0.899691in}}%
\pgfusepath{clip}%
\pgfsetrectcap%
\pgfsetroundjoin%
\pgfsetlinewidth{1.505625pt}%
\definecolor{currentstroke}{rgb}{0.900000,0.200000,0.200000}%
\pgfsetstrokecolor{currentstroke}%
\pgfsetdash{}{0pt}%
\pgfpathmoveto{\pgfqpoint{0.665922in}{1.188690in}}%
\pgfpathlineto{\pgfqpoint{0.712260in}{1.191554in}}%
\pgfpathlineto{\pgfqpoint{0.758597in}{1.194119in}}%
\pgfpathlineto{\pgfqpoint{0.804935in}{1.195378in}}%
\pgfpathlineto{\pgfqpoint{0.851273in}{1.196488in}}%
\pgfpathlineto{\pgfqpoint{0.897611in}{1.197691in}}%
\pgfpathlineto{\pgfqpoint{0.943948in}{1.198902in}}%
\pgfpathlineto{\pgfqpoint{0.990286in}{1.200018in}}%
\pgfpathlineto{\pgfqpoint{1.036624in}{1.200979in}}%
\pgfpathlineto{\pgfqpoint{1.082962in}{1.201765in}}%
\pgfpathlineto{\pgfqpoint{1.129300in}{1.202382in}}%
\pgfpathlineto{\pgfqpoint{1.175637in}{1.202850in}}%
\pgfpathlineto{\pgfqpoint{1.221975in}{1.203193in}}%
\pgfpathlineto{\pgfqpoint{1.268313in}{1.203435in}}%
\pgfpathlineto{\pgfqpoint{1.314651in}{1.203597in}}%
\pgfpathlineto{\pgfqpoint{1.360988in}{1.203699in}}%
\pgfpathlineto{\pgfqpoint{1.407326in}{1.203754in}}%
\pgfpathlineto{\pgfqpoint{1.453664in}{1.203777in}}%
\pgfpathlineto{\pgfqpoint{1.500002in}{1.203775in}}%
\pgfpathlineto{\pgfqpoint{1.546339in}{1.203758in}}%
\pgfpathlineto{\pgfqpoint{1.592677in}{1.203730in}}%
\pgfpathlineto{\pgfqpoint{1.639015in}{1.203695in}}%
\pgfpathlineto{\pgfqpoint{1.685353in}{1.203657in}}%
\pgfpathlineto{\pgfqpoint{1.731691in}{1.203617in}}%
\pgfpathlineto{\pgfqpoint{1.778028in}{1.203578in}}%
\pgfpathlineto{\pgfqpoint{1.824366in}{1.203541in}}%
\pgfpathlineto{\pgfqpoint{1.870704in}{1.203505in}}%
\pgfpathlineto{\pgfqpoint{1.917042in}{1.203473in}}%
\pgfpathlineto{\pgfqpoint{1.963379in}{1.203442in}}%
\pgfpathlineto{\pgfqpoint{2.009717in}{1.203415in}}%
\pgfpathlineto{\pgfqpoint{2.056055in}{1.203390in}}%
\pgfpathlineto{\pgfqpoint{2.102393in}{1.203368in}}%
\pgfpathlineto{\pgfqpoint{2.148730in}{1.203348in}}%
\pgfpathlineto{\pgfqpoint{2.195068in}{1.203330in}}%
\pgfpathlineto{\pgfqpoint{2.241406in}{1.203315in}}%
\pgfpathlineto{\pgfqpoint{2.287744in}{1.203301in}}%
\pgfpathlineto{\pgfqpoint{2.334081in}{1.203290in}}%
\pgfpathlineto{\pgfqpoint{2.380419in}{1.203279in}}%
\pgfpathlineto{\pgfqpoint{2.426757in}{1.203270in}}%
\pgfpathlineto{\pgfqpoint{2.473095in}{1.203262in}}%
\pgfpathlineto{\pgfqpoint{2.519433in}{1.203256in}}%
\pgfpathlineto{\pgfqpoint{2.565770in}{1.203250in}}%
\pgfpathlineto{\pgfqpoint{2.612108in}{1.203245in}}%
\pgfpathlineto{\pgfqpoint{2.658446in}{1.203241in}}%
\pgfpathlineto{\pgfqpoint{2.704784in}{1.203237in}}%
\pgfpathlineto{\pgfqpoint{2.751121in}{1.203234in}}%
\pgfpathlineto{\pgfqpoint{2.797459in}{1.203231in}}%
\pgfpathlineto{\pgfqpoint{2.843797in}{1.203229in}}%
\pgfpathlineto{\pgfqpoint{2.890135in}{1.203227in}}%
\pgfpathlineto{\pgfqpoint{2.936472in}{1.203225in}}%
\pgfusepath{stroke}%
\end{pgfscope}%
\begin{pgfscope}%
\pgfpathrectangle{\pgfqpoint{0.552394in}{0.500309in}}{\pgfqpoint{2.497606in}{0.899691in}}%
\pgfusepath{clip}%
\pgfsetbuttcap%
\pgfsetroundjoin%
\definecolor{currentfill}{rgb}{0.900000,0.200000,0.200000}%
\pgfsetfillcolor{currentfill}%
\pgfsetlinewidth{1.003750pt}%
\definecolor{currentstroke}{rgb}{0.900000,0.200000,0.200000}%
\pgfsetstrokecolor{currentstroke}%
\pgfsetdash{}{0pt}%
\pgfsys@defobject{currentmarker}{\pgfqpoint{-0.020833in}{-0.020833in}}{\pgfqpoint{0.020833in}{0.020833in}}{%
\pgfpathmoveto{\pgfqpoint{0.000000in}{-0.020833in}}%
\pgfpathcurveto{\pgfqpoint{0.005525in}{-0.020833in}}{\pgfqpoint{0.010825in}{-0.018638in}}{\pgfqpoint{0.014731in}{-0.014731in}}%
\pgfpathcurveto{\pgfqpoint{0.018638in}{-0.010825in}}{\pgfqpoint{0.020833in}{-0.005525in}}{\pgfqpoint{0.020833in}{0.000000in}}%
\pgfpathcurveto{\pgfqpoint{0.020833in}{0.005525in}}{\pgfqpoint{0.018638in}{0.010825in}}{\pgfqpoint{0.014731in}{0.014731in}}%
\pgfpathcurveto{\pgfqpoint{0.010825in}{0.018638in}}{\pgfqpoint{0.005525in}{0.020833in}}{\pgfqpoint{0.000000in}{0.020833in}}%
\pgfpathcurveto{\pgfqpoint{-0.005525in}{0.020833in}}{\pgfqpoint{-0.010825in}{0.018638in}}{\pgfqpoint{-0.014731in}{0.014731in}}%
\pgfpathcurveto{\pgfqpoint{-0.018638in}{0.010825in}}{\pgfqpoint{-0.020833in}{0.005525in}}{\pgfqpoint{-0.020833in}{0.000000in}}%
\pgfpathcurveto{\pgfqpoint{-0.020833in}{-0.005525in}}{\pgfqpoint{-0.018638in}{-0.010825in}}{\pgfqpoint{-0.014731in}{-0.014731in}}%
\pgfpathcurveto{\pgfqpoint{-0.010825in}{-0.018638in}}{\pgfqpoint{-0.005525in}{-0.020833in}}{\pgfqpoint{0.000000in}{-0.020833in}}%
\pgfpathclose%
\pgfusepath{stroke,fill}%
}%
\begin{pgfscope}%
\pgfsys@transformshift{0.665922in}{1.188690in}%
\pgfsys@useobject{currentmarker}{}%
\end{pgfscope}%
\begin{pgfscope}%
\pgfsys@transformshift{0.712260in}{1.191554in}%
\pgfsys@useobject{currentmarker}{}%
\end{pgfscope}%
\begin{pgfscope}%
\pgfsys@transformshift{0.758597in}{1.194119in}%
\pgfsys@useobject{currentmarker}{}%
\end{pgfscope}%
\begin{pgfscope}%
\pgfsys@transformshift{0.804935in}{1.195378in}%
\pgfsys@useobject{currentmarker}{}%
\end{pgfscope}%
\begin{pgfscope}%
\pgfsys@transformshift{0.851273in}{1.196488in}%
\pgfsys@useobject{currentmarker}{}%
\end{pgfscope}%
\begin{pgfscope}%
\pgfsys@transformshift{0.897611in}{1.197691in}%
\pgfsys@useobject{currentmarker}{}%
\end{pgfscope}%
\begin{pgfscope}%
\pgfsys@transformshift{0.943948in}{1.198902in}%
\pgfsys@useobject{currentmarker}{}%
\end{pgfscope}%
\begin{pgfscope}%
\pgfsys@transformshift{0.990286in}{1.200018in}%
\pgfsys@useobject{currentmarker}{}%
\end{pgfscope}%
\begin{pgfscope}%
\pgfsys@transformshift{1.036624in}{1.200979in}%
\pgfsys@useobject{currentmarker}{}%
\end{pgfscope}%
\begin{pgfscope}%
\pgfsys@transformshift{1.082962in}{1.201765in}%
\pgfsys@useobject{currentmarker}{}%
\end{pgfscope}%
\begin{pgfscope}%
\pgfsys@transformshift{1.129300in}{1.202382in}%
\pgfsys@useobject{currentmarker}{}%
\end{pgfscope}%
\begin{pgfscope}%
\pgfsys@transformshift{1.175637in}{1.202850in}%
\pgfsys@useobject{currentmarker}{}%
\end{pgfscope}%
\begin{pgfscope}%
\pgfsys@transformshift{1.221975in}{1.203193in}%
\pgfsys@useobject{currentmarker}{}%
\end{pgfscope}%
\begin{pgfscope}%
\pgfsys@transformshift{1.268313in}{1.203435in}%
\pgfsys@useobject{currentmarker}{}%
\end{pgfscope}%
\begin{pgfscope}%
\pgfsys@transformshift{1.314651in}{1.203597in}%
\pgfsys@useobject{currentmarker}{}%
\end{pgfscope}%
\begin{pgfscope}%
\pgfsys@transformshift{1.360988in}{1.203699in}%
\pgfsys@useobject{currentmarker}{}%
\end{pgfscope}%
\begin{pgfscope}%
\pgfsys@transformshift{1.407326in}{1.203754in}%
\pgfsys@useobject{currentmarker}{}%
\end{pgfscope}%
\begin{pgfscope}%
\pgfsys@transformshift{1.453664in}{1.203777in}%
\pgfsys@useobject{currentmarker}{}%
\end{pgfscope}%
\begin{pgfscope}%
\pgfsys@transformshift{1.500002in}{1.203775in}%
\pgfsys@useobject{currentmarker}{}%
\end{pgfscope}%
\begin{pgfscope}%
\pgfsys@transformshift{1.546339in}{1.203758in}%
\pgfsys@useobject{currentmarker}{}%
\end{pgfscope}%
\begin{pgfscope}%
\pgfsys@transformshift{1.592677in}{1.203730in}%
\pgfsys@useobject{currentmarker}{}%
\end{pgfscope}%
\begin{pgfscope}%
\pgfsys@transformshift{1.639015in}{1.203695in}%
\pgfsys@useobject{currentmarker}{}%
\end{pgfscope}%
\begin{pgfscope}%
\pgfsys@transformshift{1.685353in}{1.203657in}%
\pgfsys@useobject{currentmarker}{}%
\end{pgfscope}%
\begin{pgfscope}%
\pgfsys@transformshift{1.731691in}{1.203617in}%
\pgfsys@useobject{currentmarker}{}%
\end{pgfscope}%
\begin{pgfscope}%
\pgfsys@transformshift{1.778028in}{1.203578in}%
\pgfsys@useobject{currentmarker}{}%
\end{pgfscope}%
\begin{pgfscope}%
\pgfsys@transformshift{1.824366in}{1.203541in}%
\pgfsys@useobject{currentmarker}{}%
\end{pgfscope}%
\begin{pgfscope}%
\pgfsys@transformshift{1.870704in}{1.203505in}%
\pgfsys@useobject{currentmarker}{}%
\end{pgfscope}%
\begin{pgfscope}%
\pgfsys@transformshift{1.917042in}{1.203473in}%
\pgfsys@useobject{currentmarker}{}%
\end{pgfscope}%
\begin{pgfscope}%
\pgfsys@transformshift{1.963379in}{1.203442in}%
\pgfsys@useobject{currentmarker}{}%
\end{pgfscope}%
\begin{pgfscope}%
\pgfsys@transformshift{2.009717in}{1.203415in}%
\pgfsys@useobject{currentmarker}{}%
\end{pgfscope}%
\begin{pgfscope}%
\pgfsys@transformshift{2.056055in}{1.203390in}%
\pgfsys@useobject{currentmarker}{}%
\end{pgfscope}%
\begin{pgfscope}%
\pgfsys@transformshift{2.102393in}{1.203368in}%
\pgfsys@useobject{currentmarker}{}%
\end{pgfscope}%
\begin{pgfscope}%
\pgfsys@transformshift{2.148730in}{1.203348in}%
\pgfsys@useobject{currentmarker}{}%
\end{pgfscope}%
\begin{pgfscope}%
\pgfsys@transformshift{2.195068in}{1.203330in}%
\pgfsys@useobject{currentmarker}{}%
\end{pgfscope}%
\begin{pgfscope}%
\pgfsys@transformshift{2.241406in}{1.203315in}%
\pgfsys@useobject{currentmarker}{}%
\end{pgfscope}%
\begin{pgfscope}%
\pgfsys@transformshift{2.287744in}{1.203301in}%
\pgfsys@useobject{currentmarker}{}%
\end{pgfscope}%
\begin{pgfscope}%
\pgfsys@transformshift{2.334081in}{1.203290in}%
\pgfsys@useobject{currentmarker}{}%
\end{pgfscope}%
\begin{pgfscope}%
\pgfsys@transformshift{2.380419in}{1.203279in}%
\pgfsys@useobject{currentmarker}{}%
\end{pgfscope}%
\begin{pgfscope}%
\pgfsys@transformshift{2.426757in}{1.203270in}%
\pgfsys@useobject{currentmarker}{}%
\end{pgfscope}%
\begin{pgfscope}%
\pgfsys@transformshift{2.473095in}{1.203262in}%
\pgfsys@useobject{currentmarker}{}%
\end{pgfscope}%
\begin{pgfscope}%
\pgfsys@transformshift{2.519433in}{1.203256in}%
\pgfsys@useobject{currentmarker}{}%
\end{pgfscope}%
\begin{pgfscope}%
\pgfsys@transformshift{2.565770in}{1.203250in}%
\pgfsys@useobject{currentmarker}{}%
\end{pgfscope}%
\begin{pgfscope}%
\pgfsys@transformshift{2.612108in}{1.203245in}%
\pgfsys@useobject{currentmarker}{}%
\end{pgfscope}%
\begin{pgfscope}%
\pgfsys@transformshift{2.658446in}{1.203241in}%
\pgfsys@useobject{currentmarker}{}%
\end{pgfscope}%
\begin{pgfscope}%
\pgfsys@transformshift{2.704784in}{1.203237in}%
\pgfsys@useobject{currentmarker}{}%
\end{pgfscope}%
\begin{pgfscope}%
\pgfsys@transformshift{2.751121in}{1.203234in}%
\pgfsys@useobject{currentmarker}{}%
\end{pgfscope}%
\begin{pgfscope}%
\pgfsys@transformshift{2.797459in}{1.203231in}%
\pgfsys@useobject{currentmarker}{}%
\end{pgfscope}%
\begin{pgfscope}%
\pgfsys@transformshift{2.843797in}{1.203229in}%
\pgfsys@useobject{currentmarker}{}%
\end{pgfscope}%
\begin{pgfscope}%
\pgfsys@transformshift{2.890135in}{1.203227in}%
\pgfsys@useobject{currentmarker}{}%
\end{pgfscope}%
\begin{pgfscope}%
\pgfsys@transformshift{2.936472in}{1.203225in}%
\pgfsys@useobject{currentmarker}{}%
\end{pgfscope}%
\end{pgfscope}%
\begin{pgfscope}%
\pgfpathrectangle{\pgfqpoint{0.552394in}{0.500309in}}{\pgfqpoint{2.497606in}{0.899691in}}%
\pgfusepath{clip}%
\pgfsetrectcap%
\pgfsetroundjoin%
\pgfsetlinewidth{1.505625pt}%
\definecolor{currentstroke}{rgb}{0.900000,0.400000,0.400000}%
\pgfsetstrokecolor{currentstroke}%
\pgfsetdash{}{0pt}%
\pgfpathmoveto{\pgfqpoint{0.665922in}{1.219512in}}%
\pgfpathlineto{\pgfqpoint{0.712260in}{1.229469in}}%
\pgfpathlineto{\pgfqpoint{0.758597in}{1.235922in}}%
\pgfpathlineto{\pgfqpoint{0.804935in}{1.238718in}}%
\pgfpathlineto{\pgfqpoint{0.851273in}{1.239985in}}%
\pgfpathlineto{\pgfqpoint{0.897611in}{1.240658in}}%
\pgfpathlineto{\pgfqpoint{0.943948in}{1.241091in}}%
\pgfpathlineto{\pgfqpoint{0.990286in}{1.241419in}}%
\pgfpathlineto{\pgfqpoint{1.036624in}{1.241698in}}%
\pgfpathlineto{\pgfqpoint{1.082962in}{1.241952in}}%
\pgfpathlineto{\pgfqpoint{1.129300in}{1.242189in}}%
\pgfpathlineto{\pgfqpoint{1.175637in}{1.242415in}}%
\pgfpathlineto{\pgfqpoint{1.221975in}{1.242628in}}%
\pgfpathlineto{\pgfqpoint{1.268313in}{1.242829in}}%
\pgfpathlineto{\pgfqpoint{1.314651in}{1.243017in}}%
\pgfpathlineto{\pgfqpoint{1.360988in}{1.243190in}}%
\pgfpathlineto{\pgfqpoint{1.407326in}{1.243349in}}%
\pgfpathlineto{\pgfqpoint{1.453664in}{1.243494in}}%
\pgfpathlineto{\pgfqpoint{1.500002in}{1.243625in}}%
\pgfpathlineto{\pgfqpoint{1.546339in}{1.243743in}}%
\pgfpathlineto{\pgfqpoint{1.592677in}{1.243848in}}%
\pgfpathlineto{\pgfqpoint{1.639015in}{1.243942in}}%
\pgfpathlineto{\pgfqpoint{1.685353in}{1.244024in}}%
\pgfpathlineto{\pgfqpoint{1.731691in}{1.244097in}}%
\pgfpathlineto{\pgfqpoint{1.778028in}{1.244161in}}%
\pgfpathlineto{\pgfqpoint{1.824366in}{1.244217in}}%
\pgfpathlineto{\pgfqpoint{1.870704in}{1.244265in}}%
\pgfpathlineto{\pgfqpoint{1.917042in}{1.244308in}}%
\pgfpathlineto{\pgfqpoint{1.963379in}{1.244345in}}%
\pgfpathlineto{\pgfqpoint{2.009717in}{1.244376in}}%
\pgfpathlineto{\pgfqpoint{2.056055in}{1.244404in}}%
\pgfpathlineto{\pgfqpoint{2.102393in}{1.244428in}}%
\pgfpathlineto{\pgfqpoint{2.148730in}{1.244448in}}%
\pgfpathlineto{\pgfqpoint{2.195068in}{1.244466in}}%
\pgfpathlineto{\pgfqpoint{2.241406in}{1.244481in}}%
\pgfpathlineto{\pgfqpoint{2.287744in}{1.244494in}}%
\pgfpathlineto{\pgfqpoint{2.334081in}{1.244505in}}%
\pgfpathlineto{\pgfqpoint{2.380419in}{1.244514in}}%
\pgfpathlineto{\pgfqpoint{2.426757in}{1.244523in}}%
\pgfpathlineto{\pgfqpoint{2.473095in}{1.244529in}}%
\pgfpathlineto{\pgfqpoint{2.519433in}{1.244535in}}%
\pgfpathlineto{\pgfqpoint{2.565770in}{1.244540in}}%
\pgfpathlineto{\pgfqpoint{2.612108in}{1.244545in}}%
\pgfpathlineto{\pgfqpoint{2.658446in}{1.244548in}}%
\pgfpathlineto{\pgfqpoint{2.704784in}{1.244551in}}%
\pgfpathlineto{\pgfqpoint{2.751121in}{1.244554in}}%
\pgfpathlineto{\pgfqpoint{2.797459in}{1.244556in}}%
\pgfpathlineto{\pgfqpoint{2.843797in}{1.244558in}}%
\pgfpathlineto{\pgfqpoint{2.890135in}{1.244560in}}%
\pgfpathlineto{\pgfqpoint{2.936472in}{1.244561in}}%
\pgfusepath{stroke}%
\end{pgfscope}%
\begin{pgfscope}%
\pgfpathrectangle{\pgfqpoint{0.552394in}{0.500309in}}{\pgfqpoint{2.497606in}{0.899691in}}%
\pgfusepath{clip}%
\pgfsetbuttcap%
\pgfsetroundjoin%
\definecolor{currentfill}{rgb}{0.900000,0.400000,0.400000}%
\pgfsetfillcolor{currentfill}%
\pgfsetlinewidth{1.003750pt}%
\definecolor{currentstroke}{rgb}{0.900000,0.400000,0.400000}%
\pgfsetstrokecolor{currentstroke}%
\pgfsetdash{}{0pt}%
\pgfsys@defobject{currentmarker}{\pgfqpoint{-0.020833in}{-0.020833in}}{\pgfqpoint{0.020833in}{0.020833in}}{%
\pgfpathmoveto{\pgfqpoint{0.000000in}{-0.020833in}}%
\pgfpathcurveto{\pgfqpoint{0.005525in}{-0.020833in}}{\pgfqpoint{0.010825in}{-0.018638in}}{\pgfqpoint{0.014731in}{-0.014731in}}%
\pgfpathcurveto{\pgfqpoint{0.018638in}{-0.010825in}}{\pgfqpoint{0.020833in}{-0.005525in}}{\pgfqpoint{0.020833in}{0.000000in}}%
\pgfpathcurveto{\pgfqpoint{0.020833in}{0.005525in}}{\pgfqpoint{0.018638in}{0.010825in}}{\pgfqpoint{0.014731in}{0.014731in}}%
\pgfpathcurveto{\pgfqpoint{0.010825in}{0.018638in}}{\pgfqpoint{0.005525in}{0.020833in}}{\pgfqpoint{0.000000in}{0.020833in}}%
\pgfpathcurveto{\pgfqpoint{-0.005525in}{0.020833in}}{\pgfqpoint{-0.010825in}{0.018638in}}{\pgfqpoint{-0.014731in}{0.014731in}}%
\pgfpathcurveto{\pgfqpoint{-0.018638in}{0.010825in}}{\pgfqpoint{-0.020833in}{0.005525in}}{\pgfqpoint{-0.020833in}{0.000000in}}%
\pgfpathcurveto{\pgfqpoint{-0.020833in}{-0.005525in}}{\pgfqpoint{-0.018638in}{-0.010825in}}{\pgfqpoint{-0.014731in}{-0.014731in}}%
\pgfpathcurveto{\pgfqpoint{-0.010825in}{-0.018638in}}{\pgfqpoint{-0.005525in}{-0.020833in}}{\pgfqpoint{0.000000in}{-0.020833in}}%
\pgfpathclose%
\pgfusepath{stroke,fill}%
}%
\begin{pgfscope}%
\pgfsys@transformshift{0.665922in}{1.219512in}%
\pgfsys@useobject{currentmarker}{}%
\end{pgfscope}%
\begin{pgfscope}%
\pgfsys@transformshift{0.712260in}{1.229469in}%
\pgfsys@useobject{currentmarker}{}%
\end{pgfscope}%
\begin{pgfscope}%
\pgfsys@transformshift{0.758597in}{1.235922in}%
\pgfsys@useobject{currentmarker}{}%
\end{pgfscope}%
\begin{pgfscope}%
\pgfsys@transformshift{0.804935in}{1.238718in}%
\pgfsys@useobject{currentmarker}{}%
\end{pgfscope}%
\begin{pgfscope}%
\pgfsys@transformshift{0.851273in}{1.239985in}%
\pgfsys@useobject{currentmarker}{}%
\end{pgfscope}%
\begin{pgfscope}%
\pgfsys@transformshift{0.897611in}{1.240658in}%
\pgfsys@useobject{currentmarker}{}%
\end{pgfscope}%
\begin{pgfscope}%
\pgfsys@transformshift{0.943948in}{1.241091in}%
\pgfsys@useobject{currentmarker}{}%
\end{pgfscope}%
\begin{pgfscope}%
\pgfsys@transformshift{0.990286in}{1.241419in}%
\pgfsys@useobject{currentmarker}{}%
\end{pgfscope}%
\begin{pgfscope}%
\pgfsys@transformshift{1.036624in}{1.241698in}%
\pgfsys@useobject{currentmarker}{}%
\end{pgfscope}%
\begin{pgfscope}%
\pgfsys@transformshift{1.082962in}{1.241952in}%
\pgfsys@useobject{currentmarker}{}%
\end{pgfscope}%
\begin{pgfscope}%
\pgfsys@transformshift{1.129300in}{1.242189in}%
\pgfsys@useobject{currentmarker}{}%
\end{pgfscope}%
\begin{pgfscope}%
\pgfsys@transformshift{1.175637in}{1.242415in}%
\pgfsys@useobject{currentmarker}{}%
\end{pgfscope}%
\begin{pgfscope}%
\pgfsys@transformshift{1.221975in}{1.242628in}%
\pgfsys@useobject{currentmarker}{}%
\end{pgfscope}%
\begin{pgfscope}%
\pgfsys@transformshift{1.268313in}{1.242829in}%
\pgfsys@useobject{currentmarker}{}%
\end{pgfscope}%
\begin{pgfscope}%
\pgfsys@transformshift{1.314651in}{1.243017in}%
\pgfsys@useobject{currentmarker}{}%
\end{pgfscope}%
\begin{pgfscope}%
\pgfsys@transformshift{1.360988in}{1.243190in}%
\pgfsys@useobject{currentmarker}{}%
\end{pgfscope}%
\begin{pgfscope}%
\pgfsys@transformshift{1.407326in}{1.243349in}%
\pgfsys@useobject{currentmarker}{}%
\end{pgfscope}%
\begin{pgfscope}%
\pgfsys@transformshift{1.453664in}{1.243494in}%
\pgfsys@useobject{currentmarker}{}%
\end{pgfscope}%
\begin{pgfscope}%
\pgfsys@transformshift{1.500002in}{1.243625in}%
\pgfsys@useobject{currentmarker}{}%
\end{pgfscope}%
\begin{pgfscope}%
\pgfsys@transformshift{1.546339in}{1.243743in}%
\pgfsys@useobject{currentmarker}{}%
\end{pgfscope}%
\begin{pgfscope}%
\pgfsys@transformshift{1.592677in}{1.243848in}%
\pgfsys@useobject{currentmarker}{}%
\end{pgfscope}%
\begin{pgfscope}%
\pgfsys@transformshift{1.639015in}{1.243942in}%
\pgfsys@useobject{currentmarker}{}%
\end{pgfscope}%
\begin{pgfscope}%
\pgfsys@transformshift{1.685353in}{1.244024in}%
\pgfsys@useobject{currentmarker}{}%
\end{pgfscope}%
\begin{pgfscope}%
\pgfsys@transformshift{1.731691in}{1.244097in}%
\pgfsys@useobject{currentmarker}{}%
\end{pgfscope}%
\begin{pgfscope}%
\pgfsys@transformshift{1.778028in}{1.244161in}%
\pgfsys@useobject{currentmarker}{}%
\end{pgfscope}%
\begin{pgfscope}%
\pgfsys@transformshift{1.824366in}{1.244217in}%
\pgfsys@useobject{currentmarker}{}%
\end{pgfscope}%
\begin{pgfscope}%
\pgfsys@transformshift{1.870704in}{1.244265in}%
\pgfsys@useobject{currentmarker}{}%
\end{pgfscope}%
\begin{pgfscope}%
\pgfsys@transformshift{1.917042in}{1.244308in}%
\pgfsys@useobject{currentmarker}{}%
\end{pgfscope}%
\begin{pgfscope}%
\pgfsys@transformshift{1.963379in}{1.244345in}%
\pgfsys@useobject{currentmarker}{}%
\end{pgfscope}%
\begin{pgfscope}%
\pgfsys@transformshift{2.009717in}{1.244376in}%
\pgfsys@useobject{currentmarker}{}%
\end{pgfscope}%
\begin{pgfscope}%
\pgfsys@transformshift{2.056055in}{1.244404in}%
\pgfsys@useobject{currentmarker}{}%
\end{pgfscope}%
\begin{pgfscope}%
\pgfsys@transformshift{2.102393in}{1.244428in}%
\pgfsys@useobject{currentmarker}{}%
\end{pgfscope}%
\begin{pgfscope}%
\pgfsys@transformshift{2.148730in}{1.244448in}%
\pgfsys@useobject{currentmarker}{}%
\end{pgfscope}%
\begin{pgfscope}%
\pgfsys@transformshift{2.195068in}{1.244466in}%
\pgfsys@useobject{currentmarker}{}%
\end{pgfscope}%
\begin{pgfscope}%
\pgfsys@transformshift{2.241406in}{1.244481in}%
\pgfsys@useobject{currentmarker}{}%
\end{pgfscope}%
\begin{pgfscope}%
\pgfsys@transformshift{2.287744in}{1.244494in}%
\pgfsys@useobject{currentmarker}{}%
\end{pgfscope}%
\begin{pgfscope}%
\pgfsys@transformshift{2.334081in}{1.244505in}%
\pgfsys@useobject{currentmarker}{}%
\end{pgfscope}%
\begin{pgfscope}%
\pgfsys@transformshift{2.380419in}{1.244514in}%
\pgfsys@useobject{currentmarker}{}%
\end{pgfscope}%
\begin{pgfscope}%
\pgfsys@transformshift{2.426757in}{1.244523in}%
\pgfsys@useobject{currentmarker}{}%
\end{pgfscope}%
\begin{pgfscope}%
\pgfsys@transformshift{2.473095in}{1.244529in}%
\pgfsys@useobject{currentmarker}{}%
\end{pgfscope}%
\begin{pgfscope}%
\pgfsys@transformshift{2.519433in}{1.244535in}%
\pgfsys@useobject{currentmarker}{}%
\end{pgfscope}%
\begin{pgfscope}%
\pgfsys@transformshift{2.565770in}{1.244540in}%
\pgfsys@useobject{currentmarker}{}%
\end{pgfscope}%
\begin{pgfscope}%
\pgfsys@transformshift{2.612108in}{1.244545in}%
\pgfsys@useobject{currentmarker}{}%
\end{pgfscope}%
\begin{pgfscope}%
\pgfsys@transformshift{2.658446in}{1.244548in}%
\pgfsys@useobject{currentmarker}{}%
\end{pgfscope}%
\begin{pgfscope}%
\pgfsys@transformshift{2.704784in}{1.244551in}%
\pgfsys@useobject{currentmarker}{}%
\end{pgfscope}%
\begin{pgfscope}%
\pgfsys@transformshift{2.751121in}{1.244554in}%
\pgfsys@useobject{currentmarker}{}%
\end{pgfscope}%
\begin{pgfscope}%
\pgfsys@transformshift{2.797459in}{1.244556in}%
\pgfsys@useobject{currentmarker}{}%
\end{pgfscope}%
\begin{pgfscope}%
\pgfsys@transformshift{2.843797in}{1.244558in}%
\pgfsys@useobject{currentmarker}{}%
\end{pgfscope}%
\begin{pgfscope}%
\pgfsys@transformshift{2.890135in}{1.244560in}%
\pgfsys@useobject{currentmarker}{}%
\end{pgfscope}%
\begin{pgfscope}%
\pgfsys@transformshift{2.936472in}{1.244561in}%
\pgfsys@useobject{currentmarker}{}%
\end{pgfscope}%
\end{pgfscope}%
\begin{pgfscope}%
\pgfpathrectangle{\pgfqpoint{0.552394in}{0.500309in}}{\pgfqpoint{2.497606in}{0.899691in}}%
\pgfusepath{clip}%
\pgfsetrectcap%
\pgfsetroundjoin%
\pgfsetlinewidth{1.505625pt}%
\definecolor{currentstroke}{rgb}{0.000000,0.000000,0.900000}%
\pgfsetstrokecolor{currentstroke}%
\pgfsetdash{}{0pt}%
\pgfpathmoveto{\pgfqpoint{0.665922in}{1.250545in}}%
\pgfpathlineto{\pgfqpoint{0.712260in}{1.053164in}}%
\pgfpathlineto{\pgfqpoint{0.758597in}{1.053164in}}%
\pgfpathlineto{\pgfqpoint{0.804935in}{1.053164in}}%
\pgfpathlineto{\pgfqpoint{0.851273in}{1.053164in}}%
\pgfpathlineto{\pgfqpoint{0.897611in}{1.053164in}}%
\pgfpathlineto{\pgfqpoint{0.943948in}{1.053164in}}%
\pgfpathlineto{\pgfqpoint{0.990286in}{1.053164in}}%
\pgfpathlineto{\pgfqpoint{1.036624in}{1.053164in}}%
\pgfpathlineto{\pgfqpoint{1.082962in}{1.053164in}}%
\pgfpathlineto{\pgfqpoint{1.129300in}{1.053164in}}%
\pgfpathlineto{\pgfqpoint{1.175637in}{1.053164in}}%
\pgfpathlineto{\pgfqpoint{1.221975in}{1.053164in}}%
\pgfpathlineto{\pgfqpoint{1.268313in}{1.053164in}}%
\pgfpathlineto{\pgfqpoint{1.314651in}{1.053164in}}%
\pgfpathlineto{\pgfqpoint{1.360988in}{1.053164in}}%
\pgfpathlineto{\pgfqpoint{1.407326in}{1.053164in}}%
\pgfpathlineto{\pgfqpoint{1.453664in}{1.053164in}}%
\pgfpathlineto{\pgfqpoint{1.500002in}{1.053164in}}%
\pgfpathlineto{\pgfqpoint{1.546339in}{1.053164in}}%
\pgfpathlineto{\pgfqpoint{1.592677in}{1.053164in}}%
\pgfpathlineto{\pgfqpoint{1.639015in}{1.053164in}}%
\pgfpathlineto{\pgfqpoint{1.685353in}{1.053164in}}%
\pgfpathlineto{\pgfqpoint{1.731691in}{1.053164in}}%
\pgfpathlineto{\pgfqpoint{1.778028in}{1.053164in}}%
\pgfpathlineto{\pgfqpoint{1.824366in}{1.053164in}}%
\pgfpathlineto{\pgfqpoint{1.870704in}{1.053164in}}%
\pgfpathlineto{\pgfqpoint{1.917042in}{1.053164in}}%
\pgfpathlineto{\pgfqpoint{1.963379in}{1.053164in}}%
\pgfpathlineto{\pgfqpoint{2.009717in}{1.053164in}}%
\pgfpathlineto{\pgfqpoint{2.056055in}{1.053164in}}%
\pgfpathlineto{\pgfqpoint{2.102393in}{1.053164in}}%
\pgfpathlineto{\pgfqpoint{2.148730in}{1.053164in}}%
\pgfpathlineto{\pgfqpoint{2.195068in}{1.053164in}}%
\pgfpathlineto{\pgfqpoint{2.241406in}{1.053164in}}%
\pgfpathlineto{\pgfqpoint{2.287744in}{1.053164in}}%
\pgfpathlineto{\pgfqpoint{2.334081in}{1.053164in}}%
\pgfpathlineto{\pgfqpoint{2.380419in}{1.053164in}}%
\pgfpathlineto{\pgfqpoint{2.426757in}{1.053164in}}%
\pgfpathlineto{\pgfqpoint{2.473095in}{1.053164in}}%
\pgfpathlineto{\pgfqpoint{2.519433in}{1.053164in}}%
\pgfpathlineto{\pgfqpoint{2.565770in}{1.053164in}}%
\pgfpathlineto{\pgfqpoint{2.612108in}{1.053164in}}%
\pgfpathlineto{\pgfqpoint{2.658446in}{1.053164in}}%
\pgfpathlineto{\pgfqpoint{2.704784in}{1.053164in}}%
\pgfpathlineto{\pgfqpoint{2.751121in}{1.053164in}}%
\pgfpathlineto{\pgfqpoint{2.797459in}{1.053164in}}%
\pgfpathlineto{\pgfqpoint{2.843797in}{1.053164in}}%
\pgfpathlineto{\pgfqpoint{2.890135in}{1.053164in}}%
\pgfpathlineto{\pgfqpoint{2.936472in}{1.053164in}}%
\pgfusepath{stroke}%
\end{pgfscope}%
\begin{pgfscope}%
\pgfpathrectangle{\pgfqpoint{0.552394in}{0.500309in}}{\pgfqpoint{2.497606in}{0.899691in}}%
\pgfusepath{clip}%
\pgfsetbuttcap%
\pgfsetroundjoin%
\definecolor{currentfill}{rgb}{0.000000,0.000000,0.900000}%
\pgfsetfillcolor{currentfill}%
\pgfsetlinewidth{1.003750pt}%
\definecolor{currentstroke}{rgb}{0.000000,0.000000,0.900000}%
\pgfsetstrokecolor{currentstroke}%
\pgfsetdash{}{0pt}%
\pgfsys@defobject{currentmarker}{\pgfqpoint{-0.020833in}{-0.020833in}}{\pgfqpoint{0.020833in}{0.020833in}}{%
\pgfpathmoveto{\pgfqpoint{0.000000in}{-0.020833in}}%
\pgfpathcurveto{\pgfqpoint{0.005525in}{-0.020833in}}{\pgfqpoint{0.010825in}{-0.018638in}}{\pgfqpoint{0.014731in}{-0.014731in}}%
\pgfpathcurveto{\pgfqpoint{0.018638in}{-0.010825in}}{\pgfqpoint{0.020833in}{-0.005525in}}{\pgfqpoint{0.020833in}{0.000000in}}%
\pgfpathcurveto{\pgfqpoint{0.020833in}{0.005525in}}{\pgfqpoint{0.018638in}{0.010825in}}{\pgfqpoint{0.014731in}{0.014731in}}%
\pgfpathcurveto{\pgfqpoint{0.010825in}{0.018638in}}{\pgfqpoint{0.005525in}{0.020833in}}{\pgfqpoint{0.000000in}{0.020833in}}%
\pgfpathcurveto{\pgfqpoint{-0.005525in}{0.020833in}}{\pgfqpoint{-0.010825in}{0.018638in}}{\pgfqpoint{-0.014731in}{0.014731in}}%
\pgfpathcurveto{\pgfqpoint{-0.018638in}{0.010825in}}{\pgfqpoint{-0.020833in}{0.005525in}}{\pgfqpoint{-0.020833in}{0.000000in}}%
\pgfpathcurveto{\pgfqpoint{-0.020833in}{-0.005525in}}{\pgfqpoint{-0.018638in}{-0.010825in}}{\pgfqpoint{-0.014731in}{-0.014731in}}%
\pgfpathcurveto{\pgfqpoint{-0.010825in}{-0.018638in}}{\pgfqpoint{-0.005525in}{-0.020833in}}{\pgfqpoint{0.000000in}{-0.020833in}}%
\pgfpathclose%
\pgfusepath{stroke,fill}%
}%
\begin{pgfscope}%
\pgfsys@transformshift{0.665922in}{1.250545in}%
\pgfsys@useobject{currentmarker}{}%
\end{pgfscope}%
\begin{pgfscope}%
\pgfsys@transformshift{0.712260in}{1.053164in}%
\pgfsys@useobject{currentmarker}{}%
\end{pgfscope}%
\begin{pgfscope}%
\pgfsys@transformshift{0.758597in}{1.053164in}%
\pgfsys@useobject{currentmarker}{}%
\end{pgfscope}%
\begin{pgfscope}%
\pgfsys@transformshift{0.804935in}{1.053164in}%
\pgfsys@useobject{currentmarker}{}%
\end{pgfscope}%
\begin{pgfscope}%
\pgfsys@transformshift{0.851273in}{1.053164in}%
\pgfsys@useobject{currentmarker}{}%
\end{pgfscope}%
\begin{pgfscope}%
\pgfsys@transformshift{0.897611in}{1.053164in}%
\pgfsys@useobject{currentmarker}{}%
\end{pgfscope}%
\begin{pgfscope}%
\pgfsys@transformshift{0.943948in}{1.053164in}%
\pgfsys@useobject{currentmarker}{}%
\end{pgfscope}%
\begin{pgfscope}%
\pgfsys@transformshift{0.990286in}{1.053164in}%
\pgfsys@useobject{currentmarker}{}%
\end{pgfscope}%
\begin{pgfscope}%
\pgfsys@transformshift{1.036624in}{1.053164in}%
\pgfsys@useobject{currentmarker}{}%
\end{pgfscope}%
\begin{pgfscope}%
\pgfsys@transformshift{1.082962in}{1.053164in}%
\pgfsys@useobject{currentmarker}{}%
\end{pgfscope}%
\begin{pgfscope}%
\pgfsys@transformshift{1.129300in}{1.053164in}%
\pgfsys@useobject{currentmarker}{}%
\end{pgfscope}%
\begin{pgfscope}%
\pgfsys@transformshift{1.175637in}{1.053164in}%
\pgfsys@useobject{currentmarker}{}%
\end{pgfscope}%
\begin{pgfscope}%
\pgfsys@transformshift{1.221975in}{1.053164in}%
\pgfsys@useobject{currentmarker}{}%
\end{pgfscope}%
\begin{pgfscope}%
\pgfsys@transformshift{1.268313in}{1.053164in}%
\pgfsys@useobject{currentmarker}{}%
\end{pgfscope}%
\begin{pgfscope}%
\pgfsys@transformshift{1.314651in}{1.053164in}%
\pgfsys@useobject{currentmarker}{}%
\end{pgfscope}%
\begin{pgfscope}%
\pgfsys@transformshift{1.360988in}{1.053164in}%
\pgfsys@useobject{currentmarker}{}%
\end{pgfscope}%
\begin{pgfscope}%
\pgfsys@transformshift{1.407326in}{1.053164in}%
\pgfsys@useobject{currentmarker}{}%
\end{pgfscope}%
\begin{pgfscope}%
\pgfsys@transformshift{1.453664in}{1.053164in}%
\pgfsys@useobject{currentmarker}{}%
\end{pgfscope}%
\begin{pgfscope}%
\pgfsys@transformshift{1.500002in}{1.053164in}%
\pgfsys@useobject{currentmarker}{}%
\end{pgfscope}%
\begin{pgfscope}%
\pgfsys@transformshift{1.546339in}{1.053164in}%
\pgfsys@useobject{currentmarker}{}%
\end{pgfscope}%
\begin{pgfscope}%
\pgfsys@transformshift{1.592677in}{1.053164in}%
\pgfsys@useobject{currentmarker}{}%
\end{pgfscope}%
\begin{pgfscope}%
\pgfsys@transformshift{1.639015in}{1.053164in}%
\pgfsys@useobject{currentmarker}{}%
\end{pgfscope}%
\begin{pgfscope}%
\pgfsys@transformshift{1.685353in}{1.053164in}%
\pgfsys@useobject{currentmarker}{}%
\end{pgfscope}%
\begin{pgfscope}%
\pgfsys@transformshift{1.731691in}{1.053164in}%
\pgfsys@useobject{currentmarker}{}%
\end{pgfscope}%
\begin{pgfscope}%
\pgfsys@transformshift{1.778028in}{1.053164in}%
\pgfsys@useobject{currentmarker}{}%
\end{pgfscope}%
\begin{pgfscope}%
\pgfsys@transformshift{1.824366in}{1.053164in}%
\pgfsys@useobject{currentmarker}{}%
\end{pgfscope}%
\begin{pgfscope}%
\pgfsys@transformshift{1.870704in}{1.053164in}%
\pgfsys@useobject{currentmarker}{}%
\end{pgfscope}%
\begin{pgfscope}%
\pgfsys@transformshift{1.917042in}{1.053164in}%
\pgfsys@useobject{currentmarker}{}%
\end{pgfscope}%
\begin{pgfscope}%
\pgfsys@transformshift{1.963379in}{1.053164in}%
\pgfsys@useobject{currentmarker}{}%
\end{pgfscope}%
\begin{pgfscope}%
\pgfsys@transformshift{2.009717in}{1.053164in}%
\pgfsys@useobject{currentmarker}{}%
\end{pgfscope}%
\begin{pgfscope}%
\pgfsys@transformshift{2.056055in}{1.053164in}%
\pgfsys@useobject{currentmarker}{}%
\end{pgfscope}%
\begin{pgfscope}%
\pgfsys@transformshift{2.102393in}{1.053164in}%
\pgfsys@useobject{currentmarker}{}%
\end{pgfscope}%
\begin{pgfscope}%
\pgfsys@transformshift{2.148730in}{1.053164in}%
\pgfsys@useobject{currentmarker}{}%
\end{pgfscope}%
\begin{pgfscope}%
\pgfsys@transformshift{2.195068in}{1.053164in}%
\pgfsys@useobject{currentmarker}{}%
\end{pgfscope}%
\begin{pgfscope}%
\pgfsys@transformshift{2.241406in}{1.053164in}%
\pgfsys@useobject{currentmarker}{}%
\end{pgfscope}%
\begin{pgfscope}%
\pgfsys@transformshift{2.287744in}{1.053164in}%
\pgfsys@useobject{currentmarker}{}%
\end{pgfscope}%
\begin{pgfscope}%
\pgfsys@transformshift{2.334081in}{1.053164in}%
\pgfsys@useobject{currentmarker}{}%
\end{pgfscope}%
\begin{pgfscope}%
\pgfsys@transformshift{2.380419in}{1.053164in}%
\pgfsys@useobject{currentmarker}{}%
\end{pgfscope}%
\begin{pgfscope}%
\pgfsys@transformshift{2.426757in}{1.053164in}%
\pgfsys@useobject{currentmarker}{}%
\end{pgfscope}%
\begin{pgfscope}%
\pgfsys@transformshift{2.473095in}{1.053164in}%
\pgfsys@useobject{currentmarker}{}%
\end{pgfscope}%
\begin{pgfscope}%
\pgfsys@transformshift{2.519433in}{1.053164in}%
\pgfsys@useobject{currentmarker}{}%
\end{pgfscope}%
\begin{pgfscope}%
\pgfsys@transformshift{2.565770in}{1.053164in}%
\pgfsys@useobject{currentmarker}{}%
\end{pgfscope}%
\begin{pgfscope}%
\pgfsys@transformshift{2.612108in}{1.053164in}%
\pgfsys@useobject{currentmarker}{}%
\end{pgfscope}%
\begin{pgfscope}%
\pgfsys@transformshift{2.658446in}{1.053164in}%
\pgfsys@useobject{currentmarker}{}%
\end{pgfscope}%
\begin{pgfscope}%
\pgfsys@transformshift{2.704784in}{1.053164in}%
\pgfsys@useobject{currentmarker}{}%
\end{pgfscope}%
\begin{pgfscope}%
\pgfsys@transformshift{2.751121in}{1.053164in}%
\pgfsys@useobject{currentmarker}{}%
\end{pgfscope}%
\begin{pgfscope}%
\pgfsys@transformshift{2.797459in}{1.053164in}%
\pgfsys@useobject{currentmarker}{}%
\end{pgfscope}%
\begin{pgfscope}%
\pgfsys@transformshift{2.843797in}{1.053164in}%
\pgfsys@useobject{currentmarker}{}%
\end{pgfscope}%
\begin{pgfscope}%
\pgfsys@transformshift{2.890135in}{1.053164in}%
\pgfsys@useobject{currentmarker}{}%
\end{pgfscope}%
\begin{pgfscope}%
\pgfsys@transformshift{2.936472in}{1.053164in}%
\pgfsys@useobject{currentmarker}{}%
\end{pgfscope}%
\end{pgfscope}%
\begin{pgfscope}%
\pgfpathrectangle{\pgfqpoint{0.552394in}{0.500309in}}{\pgfqpoint{2.497606in}{0.899691in}}%
\pgfusepath{clip}%
\pgfsetrectcap%
\pgfsetroundjoin%
\pgfsetlinewidth{1.505625pt}%
\definecolor{currentstroke}{rgb}{0.200000,0.200000,0.900000}%
\pgfsetstrokecolor{currentstroke}%
\pgfsetdash{}{0pt}%
\pgfpathmoveto{\pgfqpoint{0.665922in}{1.102016in}}%
\pgfpathlineto{\pgfqpoint{0.712260in}{1.190837in}}%
\pgfpathlineto{\pgfqpoint{0.758597in}{1.190837in}}%
\pgfpathlineto{\pgfqpoint{0.804935in}{1.190837in}}%
\pgfpathlineto{\pgfqpoint{0.851273in}{1.190837in}}%
\pgfpathlineto{\pgfqpoint{0.897611in}{1.190837in}}%
\pgfpathlineto{\pgfqpoint{0.943948in}{1.190837in}}%
\pgfpathlineto{\pgfqpoint{0.990286in}{1.190837in}}%
\pgfpathlineto{\pgfqpoint{1.036624in}{1.190837in}}%
\pgfpathlineto{\pgfqpoint{1.082962in}{1.190837in}}%
\pgfpathlineto{\pgfqpoint{1.129300in}{1.190837in}}%
\pgfpathlineto{\pgfqpoint{1.175637in}{1.190837in}}%
\pgfpathlineto{\pgfqpoint{1.221975in}{1.190837in}}%
\pgfpathlineto{\pgfqpoint{1.268313in}{1.190837in}}%
\pgfpathlineto{\pgfqpoint{1.314651in}{1.190837in}}%
\pgfpathlineto{\pgfqpoint{1.360988in}{1.190837in}}%
\pgfpathlineto{\pgfqpoint{1.407326in}{1.190837in}}%
\pgfpathlineto{\pgfqpoint{1.453664in}{1.190837in}}%
\pgfpathlineto{\pgfqpoint{1.500002in}{1.190837in}}%
\pgfpathlineto{\pgfqpoint{1.546339in}{1.190837in}}%
\pgfpathlineto{\pgfqpoint{1.592677in}{1.190837in}}%
\pgfpathlineto{\pgfqpoint{1.639015in}{1.190837in}}%
\pgfpathlineto{\pgfqpoint{1.685353in}{1.190837in}}%
\pgfpathlineto{\pgfqpoint{1.731691in}{1.190837in}}%
\pgfpathlineto{\pgfqpoint{1.778028in}{1.190837in}}%
\pgfpathlineto{\pgfqpoint{1.824366in}{1.190837in}}%
\pgfpathlineto{\pgfqpoint{1.870704in}{1.190837in}}%
\pgfpathlineto{\pgfqpoint{1.917042in}{1.190837in}}%
\pgfpathlineto{\pgfqpoint{1.963379in}{1.190837in}}%
\pgfpathlineto{\pgfqpoint{2.009717in}{1.190837in}}%
\pgfpathlineto{\pgfqpoint{2.056055in}{1.190837in}}%
\pgfpathlineto{\pgfqpoint{2.102393in}{1.190837in}}%
\pgfpathlineto{\pgfqpoint{2.148730in}{1.190837in}}%
\pgfpathlineto{\pgfqpoint{2.195068in}{1.190837in}}%
\pgfpathlineto{\pgfqpoint{2.241406in}{1.190837in}}%
\pgfpathlineto{\pgfqpoint{2.287744in}{1.190837in}}%
\pgfpathlineto{\pgfqpoint{2.334081in}{1.190837in}}%
\pgfpathlineto{\pgfqpoint{2.380419in}{1.190837in}}%
\pgfpathlineto{\pgfqpoint{2.426757in}{1.190837in}}%
\pgfpathlineto{\pgfqpoint{2.473095in}{1.190837in}}%
\pgfpathlineto{\pgfqpoint{2.519433in}{1.190837in}}%
\pgfpathlineto{\pgfqpoint{2.565770in}{1.190837in}}%
\pgfpathlineto{\pgfqpoint{2.612108in}{1.190837in}}%
\pgfpathlineto{\pgfqpoint{2.658446in}{1.190837in}}%
\pgfpathlineto{\pgfqpoint{2.704784in}{1.190837in}}%
\pgfpathlineto{\pgfqpoint{2.751121in}{1.190837in}}%
\pgfpathlineto{\pgfqpoint{2.797459in}{1.190837in}}%
\pgfpathlineto{\pgfqpoint{2.843797in}{1.190837in}}%
\pgfpathlineto{\pgfqpoint{2.890135in}{1.190837in}}%
\pgfpathlineto{\pgfqpoint{2.936472in}{1.190837in}}%
\pgfusepath{stroke}%
\end{pgfscope}%
\begin{pgfscope}%
\pgfpathrectangle{\pgfqpoint{0.552394in}{0.500309in}}{\pgfqpoint{2.497606in}{0.899691in}}%
\pgfusepath{clip}%
\pgfsetbuttcap%
\pgfsetroundjoin%
\definecolor{currentfill}{rgb}{0.200000,0.200000,0.900000}%
\pgfsetfillcolor{currentfill}%
\pgfsetlinewidth{1.003750pt}%
\definecolor{currentstroke}{rgb}{0.200000,0.200000,0.900000}%
\pgfsetstrokecolor{currentstroke}%
\pgfsetdash{}{0pt}%
\pgfsys@defobject{currentmarker}{\pgfqpoint{-0.020833in}{-0.020833in}}{\pgfqpoint{0.020833in}{0.020833in}}{%
\pgfpathmoveto{\pgfqpoint{0.000000in}{-0.020833in}}%
\pgfpathcurveto{\pgfqpoint{0.005525in}{-0.020833in}}{\pgfqpoint{0.010825in}{-0.018638in}}{\pgfqpoint{0.014731in}{-0.014731in}}%
\pgfpathcurveto{\pgfqpoint{0.018638in}{-0.010825in}}{\pgfqpoint{0.020833in}{-0.005525in}}{\pgfqpoint{0.020833in}{0.000000in}}%
\pgfpathcurveto{\pgfqpoint{0.020833in}{0.005525in}}{\pgfqpoint{0.018638in}{0.010825in}}{\pgfqpoint{0.014731in}{0.014731in}}%
\pgfpathcurveto{\pgfqpoint{0.010825in}{0.018638in}}{\pgfqpoint{0.005525in}{0.020833in}}{\pgfqpoint{0.000000in}{0.020833in}}%
\pgfpathcurveto{\pgfqpoint{-0.005525in}{0.020833in}}{\pgfqpoint{-0.010825in}{0.018638in}}{\pgfqpoint{-0.014731in}{0.014731in}}%
\pgfpathcurveto{\pgfqpoint{-0.018638in}{0.010825in}}{\pgfqpoint{-0.020833in}{0.005525in}}{\pgfqpoint{-0.020833in}{0.000000in}}%
\pgfpathcurveto{\pgfqpoint{-0.020833in}{-0.005525in}}{\pgfqpoint{-0.018638in}{-0.010825in}}{\pgfqpoint{-0.014731in}{-0.014731in}}%
\pgfpathcurveto{\pgfqpoint{-0.010825in}{-0.018638in}}{\pgfqpoint{-0.005525in}{-0.020833in}}{\pgfqpoint{0.000000in}{-0.020833in}}%
\pgfpathclose%
\pgfusepath{stroke,fill}%
}%
\begin{pgfscope}%
\pgfsys@transformshift{0.665922in}{1.102016in}%
\pgfsys@useobject{currentmarker}{}%
\end{pgfscope}%
\begin{pgfscope}%
\pgfsys@transformshift{0.712260in}{1.190837in}%
\pgfsys@useobject{currentmarker}{}%
\end{pgfscope}%
\begin{pgfscope}%
\pgfsys@transformshift{0.758597in}{1.190837in}%
\pgfsys@useobject{currentmarker}{}%
\end{pgfscope}%
\begin{pgfscope}%
\pgfsys@transformshift{0.804935in}{1.190837in}%
\pgfsys@useobject{currentmarker}{}%
\end{pgfscope}%
\begin{pgfscope}%
\pgfsys@transformshift{0.851273in}{1.190837in}%
\pgfsys@useobject{currentmarker}{}%
\end{pgfscope}%
\begin{pgfscope}%
\pgfsys@transformshift{0.897611in}{1.190837in}%
\pgfsys@useobject{currentmarker}{}%
\end{pgfscope}%
\begin{pgfscope}%
\pgfsys@transformshift{0.943948in}{1.190837in}%
\pgfsys@useobject{currentmarker}{}%
\end{pgfscope}%
\begin{pgfscope}%
\pgfsys@transformshift{0.990286in}{1.190837in}%
\pgfsys@useobject{currentmarker}{}%
\end{pgfscope}%
\begin{pgfscope}%
\pgfsys@transformshift{1.036624in}{1.190837in}%
\pgfsys@useobject{currentmarker}{}%
\end{pgfscope}%
\begin{pgfscope}%
\pgfsys@transformshift{1.082962in}{1.190837in}%
\pgfsys@useobject{currentmarker}{}%
\end{pgfscope}%
\begin{pgfscope}%
\pgfsys@transformshift{1.129300in}{1.190837in}%
\pgfsys@useobject{currentmarker}{}%
\end{pgfscope}%
\begin{pgfscope}%
\pgfsys@transformshift{1.175637in}{1.190837in}%
\pgfsys@useobject{currentmarker}{}%
\end{pgfscope}%
\begin{pgfscope}%
\pgfsys@transformshift{1.221975in}{1.190837in}%
\pgfsys@useobject{currentmarker}{}%
\end{pgfscope}%
\begin{pgfscope}%
\pgfsys@transformshift{1.268313in}{1.190837in}%
\pgfsys@useobject{currentmarker}{}%
\end{pgfscope}%
\begin{pgfscope}%
\pgfsys@transformshift{1.314651in}{1.190837in}%
\pgfsys@useobject{currentmarker}{}%
\end{pgfscope}%
\begin{pgfscope}%
\pgfsys@transformshift{1.360988in}{1.190837in}%
\pgfsys@useobject{currentmarker}{}%
\end{pgfscope}%
\begin{pgfscope}%
\pgfsys@transformshift{1.407326in}{1.190837in}%
\pgfsys@useobject{currentmarker}{}%
\end{pgfscope}%
\begin{pgfscope}%
\pgfsys@transformshift{1.453664in}{1.190837in}%
\pgfsys@useobject{currentmarker}{}%
\end{pgfscope}%
\begin{pgfscope}%
\pgfsys@transformshift{1.500002in}{1.190837in}%
\pgfsys@useobject{currentmarker}{}%
\end{pgfscope}%
\begin{pgfscope}%
\pgfsys@transformshift{1.546339in}{1.190837in}%
\pgfsys@useobject{currentmarker}{}%
\end{pgfscope}%
\begin{pgfscope}%
\pgfsys@transformshift{1.592677in}{1.190837in}%
\pgfsys@useobject{currentmarker}{}%
\end{pgfscope}%
\begin{pgfscope}%
\pgfsys@transformshift{1.639015in}{1.190837in}%
\pgfsys@useobject{currentmarker}{}%
\end{pgfscope}%
\begin{pgfscope}%
\pgfsys@transformshift{1.685353in}{1.190837in}%
\pgfsys@useobject{currentmarker}{}%
\end{pgfscope}%
\begin{pgfscope}%
\pgfsys@transformshift{1.731691in}{1.190837in}%
\pgfsys@useobject{currentmarker}{}%
\end{pgfscope}%
\begin{pgfscope}%
\pgfsys@transformshift{1.778028in}{1.190837in}%
\pgfsys@useobject{currentmarker}{}%
\end{pgfscope}%
\begin{pgfscope}%
\pgfsys@transformshift{1.824366in}{1.190837in}%
\pgfsys@useobject{currentmarker}{}%
\end{pgfscope}%
\begin{pgfscope}%
\pgfsys@transformshift{1.870704in}{1.190837in}%
\pgfsys@useobject{currentmarker}{}%
\end{pgfscope}%
\begin{pgfscope}%
\pgfsys@transformshift{1.917042in}{1.190837in}%
\pgfsys@useobject{currentmarker}{}%
\end{pgfscope}%
\begin{pgfscope}%
\pgfsys@transformshift{1.963379in}{1.190837in}%
\pgfsys@useobject{currentmarker}{}%
\end{pgfscope}%
\begin{pgfscope}%
\pgfsys@transformshift{2.009717in}{1.190837in}%
\pgfsys@useobject{currentmarker}{}%
\end{pgfscope}%
\begin{pgfscope}%
\pgfsys@transformshift{2.056055in}{1.190837in}%
\pgfsys@useobject{currentmarker}{}%
\end{pgfscope}%
\begin{pgfscope}%
\pgfsys@transformshift{2.102393in}{1.190837in}%
\pgfsys@useobject{currentmarker}{}%
\end{pgfscope}%
\begin{pgfscope}%
\pgfsys@transformshift{2.148730in}{1.190837in}%
\pgfsys@useobject{currentmarker}{}%
\end{pgfscope}%
\begin{pgfscope}%
\pgfsys@transformshift{2.195068in}{1.190837in}%
\pgfsys@useobject{currentmarker}{}%
\end{pgfscope}%
\begin{pgfscope}%
\pgfsys@transformshift{2.241406in}{1.190837in}%
\pgfsys@useobject{currentmarker}{}%
\end{pgfscope}%
\begin{pgfscope}%
\pgfsys@transformshift{2.287744in}{1.190837in}%
\pgfsys@useobject{currentmarker}{}%
\end{pgfscope}%
\begin{pgfscope}%
\pgfsys@transformshift{2.334081in}{1.190837in}%
\pgfsys@useobject{currentmarker}{}%
\end{pgfscope}%
\begin{pgfscope}%
\pgfsys@transformshift{2.380419in}{1.190837in}%
\pgfsys@useobject{currentmarker}{}%
\end{pgfscope}%
\begin{pgfscope}%
\pgfsys@transformshift{2.426757in}{1.190837in}%
\pgfsys@useobject{currentmarker}{}%
\end{pgfscope}%
\begin{pgfscope}%
\pgfsys@transformshift{2.473095in}{1.190837in}%
\pgfsys@useobject{currentmarker}{}%
\end{pgfscope}%
\begin{pgfscope}%
\pgfsys@transformshift{2.519433in}{1.190837in}%
\pgfsys@useobject{currentmarker}{}%
\end{pgfscope}%
\begin{pgfscope}%
\pgfsys@transformshift{2.565770in}{1.190837in}%
\pgfsys@useobject{currentmarker}{}%
\end{pgfscope}%
\begin{pgfscope}%
\pgfsys@transformshift{2.612108in}{1.190837in}%
\pgfsys@useobject{currentmarker}{}%
\end{pgfscope}%
\begin{pgfscope}%
\pgfsys@transformshift{2.658446in}{1.190837in}%
\pgfsys@useobject{currentmarker}{}%
\end{pgfscope}%
\begin{pgfscope}%
\pgfsys@transformshift{2.704784in}{1.190837in}%
\pgfsys@useobject{currentmarker}{}%
\end{pgfscope}%
\begin{pgfscope}%
\pgfsys@transformshift{2.751121in}{1.190837in}%
\pgfsys@useobject{currentmarker}{}%
\end{pgfscope}%
\begin{pgfscope}%
\pgfsys@transformshift{2.797459in}{1.190837in}%
\pgfsys@useobject{currentmarker}{}%
\end{pgfscope}%
\begin{pgfscope}%
\pgfsys@transformshift{2.843797in}{1.190837in}%
\pgfsys@useobject{currentmarker}{}%
\end{pgfscope}%
\begin{pgfscope}%
\pgfsys@transformshift{2.890135in}{1.190837in}%
\pgfsys@useobject{currentmarker}{}%
\end{pgfscope}%
\begin{pgfscope}%
\pgfsys@transformshift{2.936472in}{1.190837in}%
\pgfsys@useobject{currentmarker}{}%
\end{pgfscope}%
\end{pgfscope}%
\begin{pgfscope}%
\pgfpathrectangle{\pgfqpoint{0.552394in}{0.500309in}}{\pgfqpoint{2.497606in}{0.899691in}}%
\pgfusepath{clip}%
\pgfsetrectcap%
\pgfsetroundjoin%
\pgfsetlinewidth{1.505625pt}%
\definecolor{currentstroke}{rgb}{0.400000,0.400000,0.900000}%
\pgfsetstrokecolor{currentstroke}%
\pgfsetdash{}{0pt}%
\pgfpathmoveto{\pgfqpoint{0.665922in}{1.250545in}}%
\pgfpathlineto{\pgfqpoint{0.712260in}{1.359105in}}%
\pgfpathlineto{\pgfqpoint{0.758597in}{1.359105in}}%
\pgfpathlineto{\pgfqpoint{0.804935in}{1.359105in}}%
\pgfpathlineto{\pgfqpoint{0.851273in}{1.359105in}}%
\pgfpathlineto{\pgfqpoint{0.897611in}{1.359105in}}%
\pgfpathlineto{\pgfqpoint{0.943948in}{1.359105in}}%
\pgfpathlineto{\pgfqpoint{0.990286in}{1.359105in}}%
\pgfpathlineto{\pgfqpoint{1.036624in}{1.359105in}}%
\pgfpathlineto{\pgfqpoint{1.082962in}{1.359105in}}%
\pgfpathlineto{\pgfqpoint{1.129300in}{1.359105in}}%
\pgfpathlineto{\pgfqpoint{1.175637in}{1.359105in}}%
\pgfpathlineto{\pgfqpoint{1.221975in}{1.359105in}}%
\pgfpathlineto{\pgfqpoint{1.268313in}{1.359105in}}%
\pgfpathlineto{\pgfqpoint{1.314651in}{1.359105in}}%
\pgfpathlineto{\pgfqpoint{1.360988in}{1.359105in}}%
\pgfpathlineto{\pgfqpoint{1.407326in}{1.359105in}}%
\pgfpathlineto{\pgfqpoint{1.453664in}{1.359105in}}%
\pgfpathlineto{\pgfqpoint{1.500002in}{1.359105in}}%
\pgfpathlineto{\pgfqpoint{1.546339in}{1.359105in}}%
\pgfpathlineto{\pgfqpoint{1.592677in}{1.359105in}}%
\pgfpathlineto{\pgfqpoint{1.639015in}{1.359105in}}%
\pgfpathlineto{\pgfqpoint{1.685353in}{1.359105in}}%
\pgfpathlineto{\pgfqpoint{1.731691in}{1.359105in}}%
\pgfpathlineto{\pgfqpoint{1.778028in}{1.359105in}}%
\pgfpathlineto{\pgfqpoint{1.824366in}{1.359105in}}%
\pgfpathlineto{\pgfqpoint{1.870704in}{1.359105in}}%
\pgfpathlineto{\pgfqpoint{1.917042in}{1.359105in}}%
\pgfpathlineto{\pgfqpoint{1.963379in}{1.359105in}}%
\pgfpathlineto{\pgfqpoint{2.009717in}{1.359105in}}%
\pgfpathlineto{\pgfqpoint{2.056055in}{1.359105in}}%
\pgfpathlineto{\pgfqpoint{2.102393in}{1.359105in}}%
\pgfpathlineto{\pgfqpoint{2.148730in}{1.359105in}}%
\pgfpathlineto{\pgfqpoint{2.195068in}{1.359105in}}%
\pgfpathlineto{\pgfqpoint{2.241406in}{1.359105in}}%
\pgfpathlineto{\pgfqpoint{2.287744in}{1.359105in}}%
\pgfpathlineto{\pgfqpoint{2.334081in}{1.359105in}}%
\pgfpathlineto{\pgfqpoint{2.380419in}{1.359105in}}%
\pgfpathlineto{\pgfqpoint{2.426757in}{1.359105in}}%
\pgfpathlineto{\pgfqpoint{2.473095in}{1.359105in}}%
\pgfpathlineto{\pgfqpoint{2.519433in}{1.359105in}}%
\pgfpathlineto{\pgfqpoint{2.565770in}{1.359105in}}%
\pgfpathlineto{\pgfqpoint{2.612108in}{1.359105in}}%
\pgfpathlineto{\pgfqpoint{2.658446in}{1.359105in}}%
\pgfpathlineto{\pgfqpoint{2.704784in}{1.359105in}}%
\pgfpathlineto{\pgfqpoint{2.751121in}{1.359105in}}%
\pgfpathlineto{\pgfqpoint{2.797459in}{1.359105in}}%
\pgfpathlineto{\pgfqpoint{2.843797in}{1.359105in}}%
\pgfpathlineto{\pgfqpoint{2.890135in}{1.359105in}}%
\pgfpathlineto{\pgfqpoint{2.936472in}{1.359105in}}%
\pgfusepath{stroke}%
\end{pgfscope}%
\begin{pgfscope}%
\pgfpathrectangle{\pgfqpoint{0.552394in}{0.500309in}}{\pgfqpoint{2.497606in}{0.899691in}}%
\pgfusepath{clip}%
\pgfsetbuttcap%
\pgfsetroundjoin%
\definecolor{currentfill}{rgb}{0.400000,0.400000,0.900000}%
\pgfsetfillcolor{currentfill}%
\pgfsetlinewidth{1.003750pt}%
\definecolor{currentstroke}{rgb}{0.400000,0.400000,0.900000}%
\pgfsetstrokecolor{currentstroke}%
\pgfsetdash{}{0pt}%
\pgfsys@defobject{currentmarker}{\pgfqpoint{-0.020833in}{-0.020833in}}{\pgfqpoint{0.020833in}{0.020833in}}{%
\pgfpathmoveto{\pgfqpoint{0.000000in}{-0.020833in}}%
\pgfpathcurveto{\pgfqpoint{0.005525in}{-0.020833in}}{\pgfqpoint{0.010825in}{-0.018638in}}{\pgfqpoint{0.014731in}{-0.014731in}}%
\pgfpathcurveto{\pgfqpoint{0.018638in}{-0.010825in}}{\pgfqpoint{0.020833in}{-0.005525in}}{\pgfqpoint{0.020833in}{0.000000in}}%
\pgfpathcurveto{\pgfqpoint{0.020833in}{0.005525in}}{\pgfqpoint{0.018638in}{0.010825in}}{\pgfqpoint{0.014731in}{0.014731in}}%
\pgfpathcurveto{\pgfqpoint{0.010825in}{0.018638in}}{\pgfqpoint{0.005525in}{0.020833in}}{\pgfqpoint{0.000000in}{0.020833in}}%
\pgfpathcurveto{\pgfqpoint{-0.005525in}{0.020833in}}{\pgfqpoint{-0.010825in}{0.018638in}}{\pgfqpoint{-0.014731in}{0.014731in}}%
\pgfpathcurveto{\pgfqpoint{-0.018638in}{0.010825in}}{\pgfqpoint{-0.020833in}{0.005525in}}{\pgfqpoint{-0.020833in}{0.000000in}}%
\pgfpathcurveto{\pgfqpoint{-0.020833in}{-0.005525in}}{\pgfqpoint{-0.018638in}{-0.010825in}}{\pgfqpoint{-0.014731in}{-0.014731in}}%
\pgfpathcurveto{\pgfqpoint{-0.010825in}{-0.018638in}}{\pgfqpoint{-0.005525in}{-0.020833in}}{\pgfqpoint{0.000000in}{-0.020833in}}%
\pgfpathclose%
\pgfusepath{stroke,fill}%
}%
\begin{pgfscope}%
\pgfsys@transformshift{0.665922in}{1.250545in}%
\pgfsys@useobject{currentmarker}{}%
\end{pgfscope}%
\begin{pgfscope}%
\pgfsys@transformshift{0.712260in}{1.359105in}%
\pgfsys@useobject{currentmarker}{}%
\end{pgfscope}%
\begin{pgfscope}%
\pgfsys@transformshift{0.758597in}{1.359105in}%
\pgfsys@useobject{currentmarker}{}%
\end{pgfscope}%
\begin{pgfscope}%
\pgfsys@transformshift{0.804935in}{1.359105in}%
\pgfsys@useobject{currentmarker}{}%
\end{pgfscope}%
\begin{pgfscope}%
\pgfsys@transformshift{0.851273in}{1.359105in}%
\pgfsys@useobject{currentmarker}{}%
\end{pgfscope}%
\begin{pgfscope}%
\pgfsys@transformshift{0.897611in}{1.359105in}%
\pgfsys@useobject{currentmarker}{}%
\end{pgfscope}%
\begin{pgfscope}%
\pgfsys@transformshift{0.943948in}{1.359105in}%
\pgfsys@useobject{currentmarker}{}%
\end{pgfscope}%
\begin{pgfscope}%
\pgfsys@transformshift{0.990286in}{1.359105in}%
\pgfsys@useobject{currentmarker}{}%
\end{pgfscope}%
\begin{pgfscope}%
\pgfsys@transformshift{1.036624in}{1.359105in}%
\pgfsys@useobject{currentmarker}{}%
\end{pgfscope}%
\begin{pgfscope}%
\pgfsys@transformshift{1.082962in}{1.359105in}%
\pgfsys@useobject{currentmarker}{}%
\end{pgfscope}%
\begin{pgfscope}%
\pgfsys@transformshift{1.129300in}{1.359105in}%
\pgfsys@useobject{currentmarker}{}%
\end{pgfscope}%
\begin{pgfscope}%
\pgfsys@transformshift{1.175637in}{1.359105in}%
\pgfsys@useobject{currentmarker}{}%
\end{pgfscope}%
\begin{pgfscope}%
\pgfsys@transformshift{1.221975in}{1.359105in}%
\pgfsys@useobject{currentmarker}{}%
\end{pgfscope}%
\begin{pgfscope}%
\pgfsys@transformshift{1.268313in}{1.359105in}%
\pgfsys@useobject{currentmarker}{}%
\end{pgfscope}%
\begin{pgfscope}%
\pgfsys@transformshift{1.314651in}{1.359105in}%
\pgfsys@useobject{currentmarker}{}%
\end{pgfscope}%
\begin{pgfscope}%
\pgfsys@transformshift{1.360988in}{1.359105in}%
\pgfsys@useobject{currentmarker}{}%
\end{pgfscope}%
\begin{pgfscope}%
\pgfsys@transformshift{1.407326in}{1.359105in}%
\pgfsys@useobject{currentmarker}{}%
\end{pgfscope}%
\begin{pgfscope}%
\pgfsys@transformshift{1.453664in}{1.359105in}%
\pgfsys@useobject{currentmarker}{}%
\end{pgfscope}%
\begin{pgfscope}%
\pgfsys@transformshift{1.500002in}{1.359105in}%
\pgfsys@useobject{currentmarker}{}%
\end{pgfscope}%
\begin{pgfscope}%
\pgfsys@transformshift{1.546339in}{1.359105in}%
\pgfsys@useobject{currentmarker}{}%
\end{pgfscope}%
\begin{pgfscope}%
\pgfsys@transformshift{1.592677in}{1.359105in}%
\pgfsys@useobject{currentmarker}{}%
\end{pgfscope}%
\begin{pgfscope}%
\pgfsys@transformshift{1.639015in}{1.359105in}%
\pgfsys@useobject{currentmarker}{}%
\end{pgfscope}%
\begin{pgfscope}%
\pgfsys@transformshift{1.685353in}{1.359105in}%
\pgfsys@useobject{currentmarker}{}%
\end{pgfscope}%
\begin{pgfscope}%
\pgfsys@transformshift{1.731691in}{1.359105in}%
\pgfsys@useobject{currentmarker}{}%
\end{pgfscope}%
\begin{pgfscope}%
\pgfsys@transformshift{1.778028in}{1.359105in}%
\pgfsys@useobject{currentmarker}{}%
\end{pgfscope}%
\begin{pgfscope}%
\pgfsys@transformshift{1.824366in}{1.359105in}%
\pgfsys@useobject{currentmarker}{}%
\end{pgfscope}%
\begin{pgfscope}%
\pgfsys@transformshift{1.870704in}{1.359105in}%
\pgfsys@useobject{currentmarker}{}%
\end{pgfscope}%
\begin{pgfscope}%
\pgfsys@transformshift{1.917042in}{1.359105in}%
\pgfsys@useobject{currentmarker}{}%
\end{pgfscope}%
\begin{pgfscope}%
\pgfsys@transformshift{1.963379in}{1.359105in}%
\pgfsys@useobject{currentmarker}{}%
\end{pgfscope}%
\begin{pgfscope}%
\pgfsys@transformshift{2.009717in}{1.359105in}%
\pgfsys@useobject{currentmarker}{}%
\end{pgfscope}%
\begin{pgfscope}%
\pgfsys@transformshift{2.056055in}{1.359105in}%
\pgfsys@useobject{currentmarker}{}%
\end{pgfscope}%
\begin{pgfscope}%
\pgfsys@transformshift{2.102393in}{1.359105in}%
\pgfsys@useobject{currentmarker}{}%
\end{pgfscope}%
\begin{pgfscope}%
\pgfsys@transformshift{2.148730in}{1.359105in}%
\pgfsys@useobject{currentmarker}{}%
\end{pgfscope}%
\begin{pgfscope}%
\pgfsys@transformshift{2.195068in}{1.359105in}%
\pgfsys@useobject{currentmarker}{}%
\end{pgfscope}%
\begin{pgfscope}%
\pgfsys@transformshift{2.241406in}{1.359105in}%
\pgfsys@useobject{currentmarker}{}%
\end{pgfscope}%
\begin{pgfscope}%
\pgfsys@transformshift{2.287744in}{1.359105in}%
\pgfsys@useobject{currentmarker}{}%
\end{pgfscope}%
\begin{pgfscope}%
\pgfsys@transformshift{2.334081in}{1.359105in}%
\pgfsys@useobject{currentmarker}{}%
\end{pgfscope}%
\begin{pgfscope}%
\pgfsys@transformshift{2.380419in}{1.359105in}%
\pgfsys@useobject{currentmarker}{}%
\end{pgfscope}%
\begin{pgfscope}%
\pgfsys@transformshift{2.426757in}{1.359105in}%
\pgfsys@useobject{currentmarker}{}%
\end{pgfscope}%
\begin{pgfscope}%
\pgfsys@transformshift{2.473095in}{1.359105in}%
\pgfsys@useobject{currentmarker}{}%
\end{pgfscope}%
\begin{pgfscope}%
\pgfsys@transformshift{2.519433in}{1.359105in}%
\pgfsys@useobject{currentmarker}{}%
\end{pgfscope}%
\begin{pgfscope}%
\pgfsys@transformshift{2.565770in}{1.359105in}%
\pgfsys@useobject{currentmarker}{}%
\end{pgfscope}%
\begin{pgfscope}%
\pgfsys@transformshift{2.612108in}{1.359105in}%
\pgfsys@useobject{currentmarker}{}%
\end{pgfscope}%
\begin{pgfscope}%
\pgfsys@transformshift{2.658446in}{1.359105in}%
\pgfsys@useobject{currentmarker}{}%
\end{pgfscope}%
\begin{pgfscope}%
\pgfsys@transformshift{2.704784in}{1.359105in}%
\pgfsys@useobject{currentmarker}{}%
\end{pgfscope}%
\begin{pgfscope}%
\pgfsys@transformshift{2.751121in}{1.359105in}%
\pgfsys@useobject{currentmarker}{}%
\end{pgfscope}%
\begin{pgfscope}%
\pgfsys@transformshift{2.797459in}{1.359105in}%
\pgfsys@useobject{currentmarker}{}%
\end{pgfscope}%
\begin{pgfscope}%
\pgfsys@transformshift{2.843797in}{1.359105in}%
\pgfsys@useobject{currentmarker}{}%
\end{pgfscope}%
\begin{pgfscope}%
\pgfsys@transformshift{2.890135in}{1.359105in}%
\pgfsys@useobject{currentmarker}{}%
\end{pgfscope}%
\begin{pgfscope}%
\pgfsys@transformshift{2.936472in}{1.359105in}%
\pgfsys@useobject{currentmarker}{}%
\end{pgfscope}%
\end{pgfscope}%
\begin{pgfscope}%
\pgfpathrectangle{\pgfqpoint{0.552394in}{0.500309in}}{\pgfqpoint{2.497606in}{0.899691in}}%
\pgfusepath{clip}%
\pgfsetrectcap%
\pgfsetroundjoin%
\pgfsetlinewidth{1.505625pt}%
\definecolor{currentstroke}{rgb}{0.501961,0.501961,0.501961}%
\pgfsetstrokecolor{currentstroke}%
\pgfsetdash{}{0pt}%
\pgfpathmoveto{\pgfqpoint{0.665922in}{0.541204in}}%
\pgfpathlineto{\pgfqpoint{0.712260in}{0.616460in}}%
\pgfpathlineto{\pgfqpoint{0.758597in}{0.605566in}}%
\pgfpathlineto{\pgfqpoint{0.804935in}{0.600768in}}%
\pgfpathlineto{\pgfqpoint{0.851273in}{0.598244in}}%
\pgfpathlineto{\pgfqpoint{0.897611in}{0.596512in}}%
\pgfpathlineto{\pgfqpoint{0.943948in}{0.595130in}}%
\pgfpathlineto{\pgfqpoint{0.990286in}{0.593969in}}%
\pgfpathlineto{\pgfqpoint{1.036624in}{0.592985in}}%
\pgfpathlineto{\pgfqpoint{1.082962in}{0.592152in}}%
\pgfpathlineto{\pgfqpoint{1.129300in}{0.591448in}}%
\pgfpathlineto{\pgfqpoint{1.175637in}{0.590853in}}%
\pgfpathlineto{\pgfqpoint{1.221975in}{0.590351in}}%
\pgfpathlineto{\pgfqpoint{1.268313in}{0.589927in}}%
\pgfpathlineto{\pgfqpoint{1.314651in}{0.589569in}}%
\pgfpathlineto{\pgfqpoint{1.360988in}{0.589266in}}%
\pgfpathlineto{\pgfqpoint{1.407326in}{0.589010in}}%
\pgfpathlineto{\pgfqpoint{1.453664in}{0.588793in}}%
\pgfpathlineto{\pgfqpoint{1.500002in}{0.588609in}}%
\pgfpathlineto{\pgfqpoint{1.546339in}{0.588454in}}%
\pgfpathlineto{\pgfqpoint{1.592677in}{0.588322in}}%
\pgfpathlineto{\pgfqpoint{1.639015in}{0.588211in}}%
\pgfpathlineto{\pgfqpoint{1.685353in}{0.588117in}}%
\pgfpathlineto{\pgfqpoint{1.731691in}{0.588037in}}%
\pgfpathlineto{\pgfqpoint{1.778028in}{0.587970in}}%
\pgfpathlineto{\pgfqpoint{1.824366in}{0.587913in}}%
\pgfpathlineto{\pgfqpoint{1.870704in}{0.587865in}}%
\pgfpathlineto{\pgfqpoint{1.917042in}{0.587825in}}%
\pgfpathlineto{\pgfqpoint{1.963379in}{0.587791in}}%
\pgfpathlineto{\pgfqpoint{2.009717in}{0.587762in}}%
\pgfpathlineto{\pgfqpoint{2.056055in}{0.587737in}}%
\pgfpathlineto{\pgfqpoint{2.102393in}{0.587717in}}%
\pgfpathlineto{\pgfqpoint{2.148730in}{0.587700in}}%
\pgfpathlineto{\pgfqpoint{2.195068in}{0.587685in}}%
\pgfpathlineto{\pgfqpoint{2.241406in}{0.587673in}}%
\pgfpathlineto{\pgfqpoint{2.287744in}{0.587662in}}%
\pgfpathlineto{\pgfqpoint{2.334081in}{0.587654in}}%
\pgfpathlineto{\pgfqpoint{2.380419in}{0.587646in}}%
\pgfpathlineto{\pgfqpoint{2.426757in}{0.587640in}}%
\pgfpathlineto{\pgfqpoint{2.473095in}{0.587635in}}%
\pgfpathlineto{\pgfqpoint{2.519433in}{0.587631in}}%
\pgfpathlineto{\pgfqpoint{2.565770in}{0.587627in}}%
\pgfpathlineto{\pgfqpoint{2.612108in}{0.587624in}}%
\pgfpathlineto{\pgfqpoint{2.658446in}{0.587621in}}%
\pgfpathlineto{\pgfqpoint{2.704784in}{0.587619in}}%
\pgfpathlineto{\pgfqpoint{2.751121in}{0.587617in}}%
\pgfpathlineto{\pgfqpoint{2.797459in}{0.587616in}}%
\pgfpathlineto{\pgfqpoint{2.843797in}{0.587615in}}%
\pgfpathlineto{\pgfqpoint{2.890135in}{0.587613in}}%
\pgfpathlineto{\pgfqpoint{2.936472in}{0.587613in}}%
\pgfusepath{stroke}%
\end{pgfscope}%
\begin{pgfscope}%
\pgfpathrectangle{\pgfqpoint{0.552394in}{0.500309in}}{\pgfqpoint{2.497606in}{0.899691in}}%
\pgfusepath{clip}%
\pgfsetbuttcap%
\pgfsetroundjoin%
\definecolor{currentfill}{rgb}{0.501961,0.501961,0.501961}%
\pgfsetfillcolor{currentfill}%
\pgfsetlinewidth{1.003750pt}%
\definecolor{currentstroke}{rgb}{0.501961,0.501961,0.501961}%
\pgfsetstrokecolor{currentstroke}%
\pgfsetdash{}{0pt}%
\pgfsys@defobject{currentmarker}{\pgfqpoint{-0.020833in}{-0.020833in}}{\pgfqpoint{0.020833in}{0.020833in}}{%
\pgfpathmoveto{\pgfqpoint{0.000000in}{-0.020833in}}%
\pgfpathcurveto{\pgfqpoint{0.005525in}{-0.020833in}}{\pgfqpoint{0.010825in}{-0.018638in}}{\pgfqpoint{0.014731in}{-0.014731in}}%
\pgfpathcurveto{\pgfqpoint{0.018638in}{-0.010825in}}{\pgfqpoint{0.020833in}{-0.005525in}}{\pgfqpoint{0.020833in}{0.000000in}}%
\pgfpathcurveto{\pgfqpoint{0.020833in}{0.005525in}}{\pgfqpoint{0.018638in}{0.010825in}}{\pgfqpoint{0.014731in}{0.014731in}}%
\pgfpathcurveto{\pgfqpoint{0.010825in}{0.018638in}}{\pgfqpoint{0.005525in}{0.020833in}}{\pgfqpoint{0.000000in}{0.020833in}}%
\pgfpathcurveto{\pgfqpoint{-0.005525in}{0.020833in}}{\pgfqpoint{-0.010825in}{0.018638in}}{\pgfqpoint{-0.014731in}{0.014731in}}%
\pgfpathcurveto{\pgfqpoint{-0.018638in}{0.010825in}}{\pgfqpoint{-0.020833in}{0.005525in}}{\pgfqpoint{-0.020833in}{0.000000in}}%
\pgfpathcurveto{\pgfqpoint{-0.020833in}{-0.005525in}}{\pgfqpoint{-0.018638in}{-0.010825in}}{\pgfqpoint{-0.014731in}{-0.014731in}}%
\pgfpathcurveto{\pgfqpoint{-0.010825in}{-0.018638in}}{\pgfqpoint{-0.005525in}{-0.020833in}}{\pgfqpoint{0.000000in}{-0.020833in}}%
\pgfpathclose%
\pgfusepath{stroke,fill}%
}%
\begin{pgfscope}%
\pgfsys@transformshift{0.665922in}{0.541204in}%
\pgfsys@useobject{currentmarker}{}%
\end{pgfscope}%
\begin{pgfscope}%
\pgfsys@transformshift{0.712260in}{0.616460in}%
\pgfsys@useobject{currentmarker}{}%
\end{pgfscope}%
\begin{pgfscope}%
\pgfsys@transformshift{0.758597in}{0.605566in}%
\pgfsys@useobject{currentmarker}{}%
\end{pgfscope}%
\begin{pgfscope}%
\pgfsys@transformshift{0.804935in}{0.600768in}%
\pgfsys@useobject{currentmarker}{}%
\end{pgfscope}%
\begin{pgfscope}%
\pgfsys@transformshift{0.851273in}{0.598244in}%
\pgfsys@useobject{currentmarker}{}%
\end{pgfscope}%
\begin{pgfscope}%
\pgfsys@transformshift{0.897611in}{0.596512in}%
\pgfsys@useobject{currentmarker}{}%
\end{pgfscope}%
\begin{pgfscope}%
\pgfsys@transformshift{0.943948in}{0.595130in}%
\pgfsys@useobject{currentmarker}{}%
\end{pgfscope}%
\begin{pgfscope}%
\pgfsys@transformshift{0.990286in}{0.593969in}%
\pgfsys@useobject{currentmarker}{}%
\end{pgfscope}%
\begin{pgfscope}%
\pgfsys@transformshift{1.036624in}{0.592985in}%
\pgfsys@useobject{currentmarker}{}%
\end{pgfscope}%
\begin{pgfscope}%
\pgfsys@transformshift{1.082962in}{0.592152in}%
\pgfsys@useobject{currentmarker}{}%
\end{pgfscope}%
\begin{pgfscope}%
\pgfsys@transformshift{1.129300in}{0.591448in}%
\pgfsys@useobject{currentmarker}{}%
\end{pgfscope}%
\begin{pgfscope}%
\pgfsys@transformshift{1.175637in}{0.590853in}%
\pgfsys@useobject{currentmarker}{}%
\end{pgfscope}%
\begin{pgfscope}%
\pgfsys@transformshift{1.221975in}{0.590351in}%
\pgfsys@useobject{currentmarker}{}%
\end{pgfscope}%
\begin{pgfscope}%
\pgfsys@transformshift{1.268313in}{0.589927in}%
\pgfsys@useobject{currentmarker}{}%
\end{pgfscope}%
\begin{pgfscope}%
\pgfsys@transformshift{1.314651in}{0.589569in}%
\pgfsys@useobject{currentmarker}{}%
\end{pgfscope}%
\begin{pgfscope}%
\pgfsys@transformshift{1.360988in}{0.589266in}%
\pgfsys@useobject{currentmarker}{}%
\end{pgfscope}%
\begin{pgfscope}%
\pgfsys@transformshift{1.407326in}{0.589010in}%
\pgfsys@useobject{currentmarker}{}%
\end{pgfscope}%
\begin{pgfscope}%
\pgfsys@transformshift{1.453664in}{0.588793in}%
\pgfsys@useobject{currentmarker}{}%
\end{pgfscope}%
\begin{pgfscope}%
\pgfsys@transformshift{1.500002in}{0.588609in}%
\pgfsys@useobject{currentmarker}{}%
\end{pgfscope}%
\begin{pgfscope}%
\pgfsys@transformshift{1.546339in}{0.588454in}%
\pgfsys@useobject{currentmarker}{}%
\end{pgfscope}%
\begin{pgfscope}%
\pgfsys@transformshift{1.592677in}{0.588322in}%
\pgfsys@useobject{currentmarker}{}%
\end{pgfscope}%
\begin{pgfscope}%
\pgfsys@transformshift{1.639015in}{0.588211in}%
\pgfsys@useobject{currentmarker}{}%
\end{pgfscope}%
\begin{pgfscope}%
\pgfsys@transformshift{1.685353in}{0.588117in}%
\pgfsys@useobject{currentmarker}{}%
\end{pgfscope}%
\begin{pgfscope}%
\pgfsys@transformshift{1.731691in}{0.588037in}%
\pgfsys@useobject{currentmarker}{}%
\end{pgfscope}%
\begin{pgfscope}%
\pgfsys@transformshift{1.778028in}{0.587970in}%
\pgfsys@useobject{currentmarker}{}%
\end{pgfscope}%
\begin{pgfscope}%
\pgfsys@transformshift{1.824366in}{0.587913in}%
\pgfsys@useobject{currentmarker}{}%
\end{pgfscope}%
\begin{pgfscope}%
\pgfsys@transformshift{1.870704in}{0.587865in}%
\pgfsys@useobject{currentmarker}{}%
\end{pgfscope}%
\begin{pgfscope}%
\pgfsys@transformshift{1.917042in}{0.587825in}%
\pgfsys@useobject{currentmarker}{}%
\end{pgfscope}%
\begin{pgfscope}%
\pgfsys@transformshift{1.963379in}{0.587791in}%
\pgfsys@useobject{currentmarker}{}%
\end{pgfscope}%
\begin{pgfscope}%
\pgfsys@transformshift{2.009717in}{0.587762in}%
\pgfsys@useobject{currentmarker}{}%
\end{pgfscope}%
\begin{pgfscope}%
\pgfsys@transformshift{2.056055in}{0.587737in}%
\pgfsys@useobject{currentmarker}{}%
\end{pgfscope}%
\begin{pgfscope}%
\pgfsys@transformshift{2.102393in}{0.587717in}%
\pgfsys@useobject{currentmarker}{}%
\end{pgfscope}%
\begin{pgfscope}%
\pgfsys@transformshift{2.148730in}{0.587700in}%
\pgfsys@useobject{currentmarker}{}%
\end{pgfscope}%
\begin{pgfscope}%
\pgfsys@transformshift{2.195068in}{0.587685in}%
\pgfsys@useobject{currentmarker}{}%
\end{pgfscope}%
\begin{pgfscope}%
\pgfsys@transformshift{2.241406in}{0.587673in}%
\pgfsys@useobject{currentmarker}{}%
\end{pgfscope}%
\begin{pgfscope}%
\pgfsys@transformshift{2.287744in}{0.587662in}%
\pgfsys@useobject{currentmarker}{}%
\end{pgfscope}%
\begin{pgfscope}%
\pgfsys@transformshift{2.334081in}{0.587654in}%
\pgfsys@useobject{currentmarker}{}%
\end{pgfscope}%
\begin{pgfscope}%
\pgfsys@transformshift{2.380419in}{0.587646in}%
\pgfsys@useobject{currentmarker}{}%
\end{pgfscope}%
\begin{pgfscope}%
\pgfsys@transformshift{2.426757in}{0.587640in}%
\pgfsys@useobject{currentmarker}{}%
\end{pgfscope}%
\begin{pgfscope}%
\pgfsys@transformshift{2.473095in}{0.587635in}%
\pgfsys@useobject{currentmarker}{}%
\end{pgfscope}%
\begin{pgfscope}%
\pgfsys@transformshift{2.519433in}{0.587631in}%
\pgfsys@useobject{currentmarker}{}%
\end{pgfscope}%
\begin{pgfscope}%
\pgfsys@transformshift{2.565770in}{0.587627in}%
\pgfsys@useobject{currentmarker}{}%
\end{pgfscope}%
\begin{pgfscope}%
\pgfsys@transformshift{2.612108in}{0.587624in}%
\pgfsys@useobject{currentmarker}{}%
\end{pgfscope}%
\begin{pgfscope}%
\pgfsys@transformshift{2.658446in}{0.587621in}%
\pgfsys@useobject{currentmarker}{}%
\end{pgfscope}%
\begin{pgfscope}%
\pgfsys@transformshift{2.704784in}{0.587619in}%
\pgfsys@useobject{currentmarker}{}%
\end{pgfscope}%
\begin{pgfscope}%
\pgfsys@transformshift{2.751121in}{0.587617in}%
\pgfsys@useobject{currentmarker}{}%
\end{pgfscope}%
\begin{pgfscope}%
\pgfsys@transformshift{2.797459in}{0.587616in}%
\pgfsys@useobject{currentmarker}{}%
\end{pgfscope}%
\begin{pgfscope}%
\pgfsys@transformshift{2.843797in}{0.587615in}%
\pgfsys@useobject{currentmarker}{}%
\end{pgfscope}%
\begin{pgfscope}%
\pgfsys@transformshift{2.890135in}{0.587613in}%
\pgfsys@useobject{currentmarker}{}%
\end{pgfscope}%
\begin{pgfscope}%
\pgfsys@transformshift{2.936472in}{0.587613in}%
\pgfsys@useobject{currentmarker}{}%
\end{pgfscope}%
\end{pgfscope}%
\begin{pgfscope}%
\pgfpathrectangle{\pgfqpoint{0.552394in}{0.500309in}}{\pgfqpoint{2.497606in}{0.899691in}}%
\pgfusepath{clip}%
\pgfsetbuttcap%
\pgfsetroundjoin%
\pgfsetlinewidth{1.505625pt}%
\definecolor{currentstroke}{rgb}{0.827451,0.827451,0.827451}%
\pgfsetstrokecolor{currentstroke}%
\pgfsetdash{{5.550000pt}{2.400000pt}}{0.000000pt}%
\pgfpathmoveto{\pgfqpoint{0.665922in}{0.633010in}}%
\pgfpathlineto{\pgfqpoint{0.712260in}{0.633010in}}%
\pgfpathlineto{\pgfqpoint{0.758597in}{0.633010in}}%
\pgfpathlineto{\pgfqpoint{0.804935in}{0.633010in}}%
\pgfpathlineto{\pgfqpoint{0.851273in}{0.633010in}}%
\pgfpathlineto{\pgfqpoint{0.897611in}{0.633010in}}%
\pgfpathlineto{\pgfqpoint{0.943948in}{0.633010in}}%
\pgfpathlineto{\pgfqpoint{0.990286in}{0.633010in}}%
\pgfpathlineto{\pgfqpoint{1.036624in}{0.633010in}}%
\pgfpathlineto{\pgfqpoint{1.082962in}{0.633010in}}%
\pgfpathlineto{\pgfqpoint{1.129300in}{0.633010in}}%
\pgfpathlineto{\pgfqpoint{1.175637in}{0.633010in}}%
\pgfpathlineto{\pgfqpoint{1.221975in}{0.633010in}}%
\pgfpathlineto{\pgfqpoint{1.268313in}{0.633010in}}%
\pgfpathlineto{\pgfqpoint{1.314651in}{0.633010in}}%
\pgfpathlineto{\pgfqpoint{1.360988in}{0.633010in}}%
\pgfpathlineto{\pgfqpoint{1.407326in}{0.633010in}}%
\pgfpathlineto{\pgfqpoint{1.453664in}{0.633010in}}%
\pgfpathlineto{\pgfqpoint{1.500002in}{0.633010in}}%
\pgfpathlineto{\pgfqpoint{1.546339in}{0.633010in}}%
\pgfpathlineto{\pgfqpoint{1.592677in}{0.633010in}}%
\pgfpathlineto{\pgfqpoint{1.639015in}{0.633010in}}%
\pgfpathlineto{\pgfqpoint{1.685353in}{0.633010in}}%
\pgfpathlineto{\pgfqpoint{1.731691in}{0.633010in}}%
\pgfpathlineto{\pgfqpoint{1.778028in}{0.633010in}}%
\pgfpathlineto{\pgfqpoint{1.824366in}{0.633010in}}%
\pgfpathlineto{\pgfqpoint{1.870704in}{0.633010in}}%
\pgfpathlineto{\pgfqpoint{1.917042in}{0.633010in}}%
\pgfpathlineto{\pgfqpoint{1.963379in}{0.633010in}}%
\pgfpathlineto{\pgfqpoint{2.009717in}{0.633010in}}%
\pgfpathlineto{\pgfqpoint{2.056055in}{0.633010in}}%
\pgfpathlineto{\pgfqpoint{2.102393in}{0.633010in}}%
\pgfpathlineto{\pgfqpoint{2.148730in}{0.633010in}}%
\pgfpathlineto{\pgfqpoint{2.195068in}{0.633010in}}%
\pgfpathlineto{\pgfqpoint{2.241406in}{0.633010in}}%
\pgfpathlineto{\pgfqpoint{2.287744in}{0.633010in}}%
\pgfpathlineto{\pgfqpoint{2.334081in}{0.633010in}}%
\pgfpathlineto{\pgfqpoint{2.380419in}{0.633010in}}%
\pgfpathlineto{\pgfqpoint{2.426757in}{0.633010in}}%
\pgfpathlineto{\pgfqpoint{2.473095in}{0.633010in}}%
\pgfpathlineto{\pgfqpoint{2.519433in}{0.633010in}}%
\pgfpathlineto{\pgfqpoint{2.565770in}{0.633010in}}%
\pgfpathlineto{\pgfqpoint{2.612108in}{0.633010in}}%
\pgfpathlineto{\pgfqpoint{2.658446in}{0.633010in}}%
\pgfpathlineto{\pgfqpoint{2.704784in}{0.633010in}}%
\pgfpathlineto{\pgfqpoint{2.751121in}{0.633010in}}%
\pgfpathlineto{\pgfqpoint{2.797459in}{0.633010in}}%
\pgfpathlineto{\pgfqpoint{2.843797in}{0.633010in}}%
\pgfpathlineto{\pgfqpoint{2.890135in}{0.633010in}}%
\pgfpathlineto{\pgfqpoint{2.936472in}{0.633010in}}%
\pgfusepath{stroke}%
\end{pgfscope}%
\begin{pgfscope}%
\pgfsetrectcap%
\pgfsetmiterjoin%
\pgfsetlinewidth{0.803000pt}%
\definecolor{currentstroke}{rgb}{0.000000,0.000000,0.000000}%
\pgfsetstrokecolor{currentstroke}%
\pgfsetdash{}{0pt}%
\pgfpathmoveto{\pgfqpoint{0.552394in}{0.500309in}}%
\pgfpathlineto{\pgfqpoint{0.552394in}{1.400000in}}%
\pgfusepath{stroke}%
\end{pgfscope}%
\begin{pgfscope}%
\pgfsetrectcap%
\pgfsetmiterjoin%
\pgfsetlinewidth{0.803000pt}%
\definecolor{currentstroke}{rgb}{0.000000,0.000000,0.000000}%
\pgfsetstrokecolor{currentstroke}%
\pgfsetdash{}{0pt}%
\pgfpathmoveto{\pgfqpoint{3.050000in}{0.500309in}}%
\pgfpathlineto{\pgfqpoint{3.050000in}{1.400000in}}%
\pgfusepath{stroke}%
\end{pgfscope}%
\begin{pgfscope}%
\pgfsetrectcap%
\pgfsetmiterjoin%
\pgfsetlinewidth{0.803000pt}%
\definecolor{currentstroke}{rgb}{0.000000,0.000000,0.000000}%
\pgfsetstrokecolor{currentstroke}%
\pgfsetdash{}{0pt}%
\pgfpathmoveto{\pgfqpoint{0.552394in}{0.500309in}}%
\pgfpathlineto{\pgfqpoint{3.050000in}{0.500309in}}%
\pgfusepath{stroke}%
\end{pgfscope}%
\begin{pgfscope}%
\pgfsetrectcap%
\pgfsetmiterjoin%
\pgfsetlinewidth{0.803000pt}%
\definecolor{currentstroke}{rgb}{0.000000,0.000000,0.000000}%
\pgfsetstrokecolor{currentstroke}%
\pgfsetdash{}{0pt}%
\pgfpathmoveto{\pgfqpoint{0.552394in}{1.400000in}}%
\pgfpathlineto{\pgfqpoint{3.050000in}{1.400000in}}%
\pgfusepath{stroke}%
\end{pgfscope}%
\begin{pgfscope}%
\pgfsetbuttcap%
\pgfsetmiterjoin%
\definecolor{currentfill}{rgb}{1.000000,1.000000,1.000000}%
\pgfsetfillcolor{currentfill}%
\pgfsetfillopacity{0.800000}%
\pgfsetlinewidth{1.003750pt}%
\definecolor{currentstroke}{rgb}{0.800000,0.800000,0.800000}%
\pgfsetstrokecolor{currentstroke}%
\pgfsetstrokeopacity{0.800000}%
\pgfsetdash{}{0pt}%
\pgfpathmoveto{\pgfqpoint{1.637925in}{0.623685in}}%
\pgfpathlineto{\pgfqpoint{2.969014in}{0.623685in}}%
\pgfpathquadraticcurveto{\pgfqpoint{2.992153in}{0.623685in}}{\pgfqpoint{2.992153in}{0.646824in}}%
\pgfpathlineto{\pgfqpoint{2.992153in}{1.319014in}}%
\pgfpathquadraticcurveto{\pgfqpoint{2.992153in}{1.342153in}}{\pgfqpoint{2.969014in}{1.342153in}}%
\pgfpathlineto{\pgfqpoint{1.637925in}{1.342153in}}%
\pgfpathquadraticcurveto{\pgfqpoint{1.614787in}{1.342153in}}{\pgfqpoint{1.614787in}{1.319014in}}%
\pgfpathlineto{\pgfqpoint{1.614787in}{0.646824in}}%
\pgfpathquadraticcurveto{\pgfqpoint{1.614787in}{0.623685in}}{\pgfqpoint{1.637925in}{0.623685in}}%
\pgfpathclose%
\pgfusepath{stroke,fill}%
\end{pgfscope}%
\begin{pgfscope}%
\pgfsetrectcap%
\pgfsetroundjoin%
\pgfsetlinewidth{1.505625pt}%
\definecolor{currentstroke}{rgb}{0.000000,0.000000,0.900000}%
\pgfsetstrokecolor{currentstroke}%
\pgfsetdash{}{0pt}%
\pgfpathmoveto{\pgfqpoint{1.661064in}{1.255382in}}%
\pgfpathlineto{\pgfqpoint{1.707342in}{1.255382in}}%
\pgfusepath{stroke}%
\end{pgfscope}%
\begin{pgfscope}%
\pgfsetbuttcap%
\pgfsetroundjoin%
\definecolor{currentfill}{rgb}{0.000000,0.000000,0.900000}%
\pgfsetfillcolor{currentfill}%
\pgfsetlinewidth{1.003750pt}%
\definecolor{currentstroke}{rgb}{0.000000,0.000000,0.900000}%
\pgfsetstrokecolor{currentstroke}%
\pgfsetdash{}{0pt}%
\pgfsys@defobject{currentmarker}{\pgfqpoint{-0.020833in}{-0.020833in}}{\pgfqpoint{0.020833in}{0.020833in}}{%
\pgfpathmoveto{\pgfqpoint{0.000000in}{-0.020833in}}%
\pgfpathcurveto{\pgfqpoint{0.005525in}{-0.020833in}}{\pgfqpoint{0.010825in}{-0.018638in}}{\pgfqpoint{0.014731in}{-0.014731in}}%
\pgfpathcurveto{\pgfqpoint{0.018638in}{-0.010825in}}{\pgfqpoint{0.020833in}{-0.005525in}}{\pgfqpoint{0.020833in}{0.000000in}}%
\pgfpathcurveto{\pgfqpoint{0.020833in}{0.005525in}}{\pgfqpoint{0.018638in}{0.010825in}}{\pgfqpoint{0.014731in}{0.014731in}}%
\pgfpathcurveto{\pgfqpoint{0.010825in}{0.018638in}}{\pgfqpoint{0.005525in}{0.020833in}}{\pgfqpoint{0.000000in}{0.020833in}}%
\pgfpathcurveto{\pgfqpoint{-0.005525in}{0.020833in}}{\pgfqpoint{-0.010825in}{0.018638in}}{\pgfqpoint{-0.014731in}{0.014731in}}%
\pgfpathcurveto{\pgfqpoint{-0.018638in}{0.010825in}}{\pgfqpoint{-0.020833in}{0.005525in}}{\pgfqpoint{-0.020833in}{0.000000in}}%
\pgfpathcurveto{\pgfqpoint{-0.020833in}{-0.005525in}}{\pgfqpoint{-0.018638in}{-0.010825in}}{\pgfqpoint{-0.014731in}{-0.014731in}}%
\pgfpathcurveto{\pgfqpoint{-0.010825in}{-0.018638in}}{\pgfqpoint{-0.005525in}{-0.020833in}}{\pgfqpoint{0.000000in}{-0.020833in}}%
\pgfpathclose%
\pgfusepath{stroke,fill}%
}%
\begin{pgfscope}%
\pgfsys@transformshift{1.684203in}{1.255382in}%
\pgfsys@useobject{currentmarker}{}%
\end{pgfscope}%
\end{pgfscope}%
\begin{pgfscope}%
\pgfsetrectcap%
\pgfsetroundjoin%
\pgfsetlinewidth{1.505625pt}%
\definecolor{currentstroke}{rgb}{0.200000,0.200000,0.900000}%
\pgfsetstrokecolor{currentstroke}%
\pgfsetdash{}{0pt}%
\pgfpathmoveto{\pgfqpoint{1.753620in}{1.255382in}}%
\pgfpathlineto{\pgfqpoint{1.799898in}{1.255382in}}%
\pgfusepath{stroke}%
\end{pgfscope}%
\begin{pgfscope}%
\pgfsetbuttcap%
\pgfsetroundjoin%
\definecolor{currentfill}{rgb}{0.200000,0.200000,0.900000}%
\pgfsetfillcolor{currentfill}%
\pgfsetlinewidth{1.003750pt}%
\definecolor{currentstroke}{rgb}{0.200000,0.200000,0.900000}%
\pgfsetstrokecolor{currentstroke}%
\pgfsetdash{}{0pt}%
\pgfsys@defobject{currentmarker}{\pgfqpoint{-0.020833in}{-0.020833in}}{\pgfqpoint{0.020833in}{0.020833in}}{%
\pgfpathmoveto{\pgfqpoint{0.000000in}{-0.020833in}}%
\pgfpathcurveto{\pgfqpoint{0.005525in}{-0.020833in}}{\pgfqpoint{0.010825in}{-0.018638in}}{\pgfqpoint{0.014731in}{-0.014731in}}%
\pgfpathcurveto{\pgfqpoint{0.018638in}{-0.010825in}}{\pgfqpoint{0.020833in}{-0.005525in}}{\pgfqpoint{0.020833in}{0.000000in}}%
\pgfpathcurveto{\pgfqpoint{0.020833in}{0.005525in}}{\pgfqpoint{0.018638in}{0.010825in}}{\pgfqpoint{0.014731in}{0.014731in}}%
\pgfpathcurveto{\pgfqpoint{0.010825in}{0.018638in}}{\pgfqpoint{0.005525in}{0.020833in}}{\pgfqpoint{0.000000in}{0.020833in}}%
\pgfpathcurveto{\pgfqpoint{-0.005525in}{0.020833in}}{\pgfqpoint{-0.010825in}{0.018638in}}{\pgfqpoint{-0.014731in}{0.014731in}}%
\pgfpathcurveto{\pgfqpoint{-0.018638in}{0.010825in}}{\pgfqpoint{-0.020833in}{0.005525in}}{\pgfqpoint{-0.020833in}{0.000000in}}%
\pgfpathcurveto{\pgfqpoint{-0.020833in}{-0.005525in}}{\pgfqpoint{-0.018638in}{-0.010825in}}{\pgfqpoint{-0.014731in}{-0.014731in}}%
\pgfpathcurveto{\pgfqpoint{-0.010825in}{-0.018638in}}{\pgfqpoint{-0.005525in}{-0.020833in}}{\pgfqpoint{0.000000in}{-0.020833in}}%
\pgfpathclose%
\pgfusepath{stroke,fill}%
}%
\begin{pgfscope}%
\pgfsys@transformshift{1.776759in}{1.255382in}%
\pgfsys@useobject{currentmarker}{}%
\end{pgfscope}%
\end{pgfscope}%
\begin{pgfscope}%
\pgfsetrectcap%
\pgfsetroundjoin%
\pgfsetlinewidth{1.505625pt}%
\definecolor{currentstroke}{rgb}{0.400000,0.400000,0.900000}%
\pgfsetstrokecolor{currentstroke}%
\pgfsetdash{}{0pt}%
\pgfpathmoveto{\pgfqpoint{1.846175in}{1.255382in}}%
\pgfpathlineto{\pgfqpoint{1.892453in}{1.255382in}}%
\pgfusepath{stroke}%
\end{pgfscope}%
\begin{pgfscope}%
\pgfsetbuttcap%
\pgfsetroundjoin%
\definecolor{currentfill}{rgb}{0.400000,0.400000,0.900000}%
\pgfsetfillcolor{currentfill}%
\pgfsetlinewidth{1.003750pt}%
\definecolor{currentstroke}{rgb}{0.400000,0.400000,0.900000}%
\pgfsetstrokecolor{currentstroke}%
\pgfsetdash{}{0pt}%
\pgfsys@defobject{currentmarker}{\pgfqpoint{-0.020833in}{-0.020833in}}{\pgfqpoint{0.020833in}{0.020833in}}{%
\pgfpathmoveto{\pgfqpoint{0.000000in}{-0.020833in}}%
\pgfpathcurveto{\pgfqpoint{0.005525in}{-0.020833in}}{\pgfqpoint{0.010825in}{-0.018638in}}{\pgfqpoint{0.014731in}{-0.014731in}}%
\pgfpathcurveto{\pgfqpoint{0.018638in}{-0.010825in}}{\pgfqpoint{0.020833in}{-0.005525in}}{\pgfqpoint{0.020833in}{0.000000in}}%
\pgfpathcurveto{\pgfqpoint{0.020833in}{0.005525in}}{\pgfqpoint{0.018638in}{0.010825in}}{\pgfqpoint{0.014731in}{0.014731in}}%
\pgfpathcurveto{\pgfqpoint{0.010825in}{0.018638in}}{\pgfqpoint{0.005525in}{0.020833in}}{\pgfqpoint{0.000000in}{0.020833in}}%
\pgfpathcurveto{\pgfqpoint{-0.005525in}{0.020833in}}{\pgfqpoint{-0.010825in}{0.018638in}}{\pgfqpoint{-0.014731in}{0.014731in}}%
\pgfpathcurveto{\pgfqpoint{-0.018638in}{0.010825in}}{\pgfqpoint{-0.020833in}{0.005525in}}{\pgfqpoint{-0.020833in}{0.000000in}}%
\pgfpathcurveto{\pgfqpoint{-0.020833in}{-0.005525in}}{\pgfqpoint{-0.018638in}{-0.010825in}}{\pgfqpoint{-0.014731in}{-0.014731in}}%
\pgfpathcurveto{\pgfqpoint{-0.010825in}{-0.018638in}}{\pgfqpoint{-0.005525in}{-0.020833in}}{\pgfqpoint{0.000000in}{-0.020833in}}%
\pgfpathclose%
\pgfusepath{stroke,fill}%
}%
\begin{pgfscope}%
\pgfsys@transformshift{1.869314in}{1.255382in}%
\pgfsys@useobject{currentmarker}{}%
\end{pgfscope}%
\end{pgfscope}%
\begin{pgfscope}%
\definecolor{textcolor}{rgb}{0.000000,0.000000,0.000000}%
\pgfsetstrokecolor{textcolor}%
\pgfsetfillcolor{textcolor}%
\pgftext[x=1.985009in,y=1.214889in,left,base]{\color{textcolor}\rmfamily\fontsize{8.330000}{9.996000}\selectfont \(\displaystyle \omega_{i,SecFCI}\)}%
\end{pgfscope}%
\begin{pgfscope}%
\pgfsetrectcap%
\pgfsetroundjoin%
\pgfsetlinewidth{1.505625pt}%
\definecolor{currentstroke}{rgb}{0.900000,0.000000,0.000000}%
\pgfsetstrokecolor{currentstroke}%
\pgfsetdash{}{0pt}%
\pgfpathmoveto{\pgfqpoint{1.661064in}{1.081672in}}%
\pgfpathlineto{\pgfqpoint{1.707342in}{1.081672in}}%
\pgfusepath{stroke}%
\end{pgfscope}%
\begin{pgfscope}%
\pgfsetbuttcap%
\pgfsetroundjoin%
\definecolor{currentfill}{rgb}{0.900000,0.000000,0.000000}%
\pgfsetfillcolor{currentfill}%
\pgfsetlinewidth{1.003750pt}%
\definecolor{currentstroke}{rgb}{0.900000,0.000000,0.000000}%
\pgfsetstrokecolor{currentstroke}%
\pgfsetdash{}{0pt}%
\pgfsys@defobject{currentmarker}{\pgfqpoint{-0.020833in}{-0.020833in}}{\pgfqpoint{0.020833in}{0.020833in}}{%
\pgfpathmoveto{\pgfqpoint{0.000000in}{-0.020833in}}%
\pgfpathcurveto{\pgfqpoint{0.005525in}{-0.020833in}}{\pgfqpoint{0.010825in}{-0.018638in}}{\pgfqpoint{0.014731in}{-0.014731in}}%
\pgfpathcurveto{\pgfqpoint{0.018638in}{-0.010825in}}{\pgfqpoint{0.020833in}{-0.005525in}}{\pgfqpoint{0.020833in}{0.000000in}}%
\pgfpathcurveto{\pgfqpoint{0.020833in}{0.005525in}}{\pgfqpoint{0.018638in}{0.010825in}}{\pgfqpoint{0.014731in}{0.014731in}}%
\pgfpathcurveto{\pgfqpoint{0.010825in}{0.018638in}}{\pgfqpoint{0.005525in}{0.020833in}}{\pgfqpoint{0.000000in}{0.020833in}}%
\pgfpathcurveto{\pgfqpoint{-0.005525in}{0.020833in}}{\pgfqpoint{-0.010825in}{0.018638in}}{\pgfqpoint{-0.014731in}{0.014731in}}%
\pgfpathcurveto{\pgfqpoint{-0.018638in}{0.010825in}}{\pgfqpoint{-0.020833in}{0.005525in}}{\pgfqpoint{-0.020833in}{0.000000in}}%
\pgfpathcurveto{\pgfqpoint{-0.020833in}{-0.005525in}}{\pgfqpoint{-0.018638in}{-0.010825in}}{\pgfqpoint{-0.014731in}{-0.014731in}}%
\pgfpathcurveto{\pgfqpoint{-0.010825in}{-0.018638in}}{\pgfqpoint{-0.005525in}{-0.020833in}}{\pgfqpoint{0.000000in}{-0.020833in}}%
\pgfpathclose%
\pgfusepath{stroke,fill}%
}%
\begin{pgfscope}%
\pgfsys@transformshift{1.684203in}{1.081672in}%
\pgfsys@useobject{currentmarker}{}%
\end{pgfscope}%
\end{pgfscope}%
\begin{pgfscope}%
\pgfsetrectcap%
\pgfsetroundjoin%
\pgfsetlinewidth{1.505625pt}%
\definecolor{currentstroke}{rgb}{0.900000,0.200000,0.200000}%
\pgfsetstrokecolor{currentstroke}%
\pgfsetdash{}{0pt}%
\pgfpathmoveto{\pgfqpoint{1.753620in}{1.081672in}}%
\pgfpathlineto{\pgfqpoint{1.799898in}{1.081672in}}%
\pgfusepath{stroke}%
\end{pgfscope}%
\begin{pgfscope}%
\pgfsetbuttcap%
\pgfsetroundjoin%
\definecolor{currentfill}{rgb}{0.900000,0.200000,0.200000}%
\pgfsetfillcolor{currentfill}%
\pgfsetlinewidth{1.003750pt}%
\definecolor{currentstroke}{rgb}{0.900000,0.200000,0.200000}%
\pgfsetstrokecolor{currentstroke}%
\pgfsetdash{}{0pt}%
\pgfsys@defobject{currentmarker}{\pgfqpoint{-0.020833in}{-0.020833in}}{\pgfqpoint{0.020833in}{0.020833in}}{%
\pgfpathmoveto{\pgfqpoint{0.000000in}{-0.020833in}}%
\pgfpathcurveto{\pgfqpoint{0.005525in}{-0.020833in}}{\pgfqpoint{0.010825in}{-0.018638in}}{\pgfqpoint{0.014731in}{-0.014731in}}%
\pgfpathcurveto{\pgfqpoint{0.018638in}{-0.010825in}}{\pgfqpoint{0.020833in}{-0.005525in}}{\pgfqpoint{0.020833in}{0.000000in}}%
\pgfpathcurveto{\pgfqpoint{0.020833in}{0.005525in}}{\pgfqpoint{0.018638in}{0.010825in}}{\pgfqpoint{0.014731in}{0.014731in}}%
\pgfpathcurveto{\pgfqpoint{0.010825in}{0.018638in}}{\pgfqpoint{0.005525in}{0.020833in}}{\pgfqpoint{0.000000in}{0.020833in}}%
\pgfpathcurveto{\pgfqpoint{-0.005525in}{0.020833in}}{\pgfqpoint{-0.010825in}{0.018638in}}{\pgfqpoint{-0.014731in}{0.014731in}}%
\pgfpathcurveto{\pgfqpoint{-0.018638in}{0.010825in}}{\pgfqpoint{-0.020833in}{0.005525in}}{\pgfqpoint{-0.020833in}{0.000000in}}%
\pgfpathcurveto{\pgfqpoint{-0.020833in}{-0.005525in}}{\pgfqpoint{-0.018638in}{-0.010825in}}{\pgfqpoint{-0.014731in}{-0.014731in}}%
\pgfpathcurveto{\pgfqpoint{-0.010825in}{-0.018638in}}{\pgfqpoint{-0.005525in}{-0.020833in}}{\pgfqpoint{0.000000in}{-0.020833in}}%
\pgfpathclose%
\pgfusepath{stroke,fill}%
}%
\begin{pgfscope}%
\pgfsys@transformshift{1.776759in}{1.081672in}%
\pgfsys@useobject{currentmarker}{}%
\end{pgfscope}%
\end{pgfscope}%
\begin{pgfscope}%
\pgfsetrectcap%
\pgfsetroundjoin%
\pgfsetlinewidth{1.505625pt}%
\definecolor{currentstroke}{rgb}{0.900000,0.400000,0.400000}%
\pgfsetstrokecolor{currentstroke}%
\pgfsetdash{}{0pt}%
\pgfpathmoveto{\pgfqpoint{1.846175in}{1.081672in}}%
\pgfpathlineto{\pgfqpoint{1.892453in}{1.081672in}}%
\pgfusepath{stroke}%
\end{pgfscope}%
\begin{pgfscope}%
\pgfsetbuttcap%
\pgfsetroundjoin%
\definecolor{currentfill}{rgb}{0.900000,0.400000,0.400000}%
\pgfsetfillcolor{currentfill}%
\pgfsetlinewidth{1.003750pt}%
\definecolor{currentstroke}{rgb}{0.900000,0.400000,0.400000}%
\pgfsetstrokecolor{currentstroke}%
\pgfsetdash{}{0pt}%
\pgfsys@defobject{currentmarker}{\pgfqpoint{-0.020833in}{-0.020833in}}{\pgfqpoint{0.020833in}{0.020833in}}{%
\pgfpathmoveto{\pgfqpoint{0.000000in}{-0.020833in}}%
\pgfpathcurveto{\pgfqpoint{0.005525in}{-0.020833in}}{\pgfqpoint{0.010825in}{-0.018638in}}{\pgfqpoint{0.014731in}{-0.014731in}}%
\pgfpathcurveto{\pgfqpoint{0.018638in}{-0.010825in}}{\pgfqpoint{0.020833in}{-0.005525in}}{\pgfqpoint{0.020833in}{0.000000in}}%
\pgfpathcurveto{\pgfqpoint{0.020833in}{0.005525in}}{\pgfqpoint{0.018638in}{0.010825in}}{\pgfqpoint{0.014731in}{0.014731in}}%
\pgfpathcurveto{\pgfqpoint{0.010825in}{0.018638in}}{\pgfqpoint{0.005525in}{0.020833in}}{\pgfqpoint{0.000000in}{0.020833in}}%
\pgfpathcurveto{\pgfqpoint{-0.005525in}{0.020833in}}{\pgfqpoint{-0.010825in}{0.018638in}}{\pgfqpoint{-0.014731in}{0.014731in}}%
\pgfpathcurveto{\pgfqpoint{-0.018638in}{0.010825in}}{\pgfqpoint{-0.020833in}{0.005525in}}{\pgfqpoint{-0.020833in}{0.000000in}}%
\pgfpathcurveto{\pgfqpoint{-0.020833in}{-0.005525in}}{\pgfqpoint{-0.018638in}{-0.010825in}}{\pgfqpoint{-0.014731in}{-0.014731in}}%
\pgfpathcurveto{\pgfqpoint{-0.010825in}{-0.018638in}}{\pgfqpoint{-0.005525in}{-0.020833in}}{\pgfqpoint{0.000000in}{-0.020833in}}%
\pgfpathclose%
\pgfusepath{stroke,fill}%
}%
\begin{pgfscope}%
\pgfsys@transformshift{1.869314in}{1.081672in}%
\pgfsys@useobject{currentmarker}{}%
\end{pgfscope}%
\end{pgfscope}%
\begin{pgfscope}%
\definecolor{textcolor}{rgb}{0.000000,0.000000,0.000000}%
\pgfsetstrokecolor{textcolor}%
\pgfsetfillcolor{textcolor}%
\pgftext[x=1.985009in,y=1.041179in,left,base]{\color{textcolor}\rmfamily\fontsize{8.330000}{9.996000}\selectfont \(\displaystyle \omega_{i,FCI}\)}%
\end{pgfscope}%
\begin{pgfscope}%
\pgfsetrectcap%
\pgfsetroundjoin%
\pgfsetlinewidth{1.505625pt}%
\definecolor{currentstroke}{rgb}{0.501961,0.501961,0.501961}%
\pgfsetstrokecolor{currentstroke}%
\pgfsetdash{}{0pt}%
\pgfpathmoveto{\pgfqpoint{1.661064in}{0.905615in}}%
\pgfpathlineto{\pgfqpoint{1.892453in}{0.905615in}}%
\pgfusepath{stroke}%
\end{pgfscope}%
\begin{pgfscope}%
\pgfsetbuttcap%
\pgfsetroundjoin%
\definecolor{currentfill}{rgb}{0.501961,0.501961,0.501961}%
\pgfsetfillcolor{currentfill}%
\pgfsetlinewidth{1.003750pt}%
\definecolor{currentstroke}{rgb}{0.501961,0.501961,0.501961}%
\pgfsetstrokecolor{currentstroke}%
\pgfsetdash{}{0pt}%
\pgfsys@defobject{currentmarker}{\pgfqpoint{-0.020833in}{-0.020833in}}{\pgfqpoint{0.020833in}{0.020833in}}{%
\pgfpathmoveto{\pgfqpoint{0.000000in}{-0.020833in}}%
\pgfpathcurveto{\pgfqpoint{0.005525in}{-0.020833in}}{\pgfqpoint{0.010825in}{-0.018638in}}{\pgfqpoint{0.014731in}{-0.014731in}}%
\pgfpathcurveto{\pgfqpoint{0.018638in}{-0.010825in}}{\pgfqpoint{0.020833in}{-0.005525in}}{\pgfqpoint{0.020833in}{0.000000in}}%
\pgfpathcurveto{\pgfqpoint{0.020833in}{0.005525in}}{\pgfqpoint{0.018638in}{0.010825in}}{\pgfqpoint{0.014731in}{0.014731in}}%
\pgfpathcurveto{\pgfqpoint{0.010825in}{0.018638in}}{\pgfqpoint{0.005525in}{0.020833in}}{\pgfqpoint{0.000000in}{0.020833in}}%
\pgfpathcurveto{\pgfqpoint{-0.005525in}{0.020833in}}{\pgfqpoint{-0.010825in}{0.018638in}}{\pgfqpoint{-0.014731in}{0.014731in}}%
\pgfpathcurveto{\pgfqpoint{-0.018638in}{0.010825in}}{\pgfqpoint{-0.020833in}{0.005525in}}{\pgfqpoint{-0.020833in}{0.000000in}}%
\pgfpathcurveto{\pgfqpoint{-0.020833in}{-0.005525in}}{\pgfqpoint{-0.018638in}{-0.010825in}}{\pgfqpoint{-0.014731in}{-0.014731in}}%
\pgfpathcurveto{\pgfqpoint{-0.010825in}{-0.018638in}}{\pgfqpoint{-0.005525in}{-0.020833in}}{\pgfqpoint{0.000000in}{-0.020833in}}%
\pgfpathclose%
\pgfusepath{stroke,fill}%
}%
\begin{pgfscope}%
\pgfsys@transformshift{1.776759in}{0.905615in}%
\pgfsys@useobject{currentmarker}{}%
\end{pgfscope}%
\end{pgfscope}%
\begin{pgfscope}%
\definecolor{textcolor}{rgb}{0.000000,0.000000,0.000000}%
\pgfsetstrokecolor{textcolor}%
\pgfsetfillcolor{textcolor}%
\pgftext[x=1.985009in,y=0.865122in,left,base]{\color{textcolor}\rmfamily\fontsize{8.330000}{9.996000}\selectfont \(\displaystyle |\underline{\omega}_{FCI}-\underline{\omega}_{SecFCI}|\)}%
\end{pgfscope}%
\begin{pgfscope}%
\pgfsetbuttcap%
\pgfsetroundjoin%
\pgfsetlinewidth{1.505625pt}%
\definecolor{currentstroke}{rgb}{0.827451,0.827451,0.827451}%
\pgfsetstrokecolor{currentstroke}%
\pgfsetdash{{5.550000pt}{2.400000pt}}{0.000000pt}%
\pgfpathmoveto{\pgfqpoint{1.661064in}{0.732061in}}%
\pgfpathlineto{\pgfqpoint{1.892453in}{0.732061in}}%
\pgfusepath{stroke}%
\end{pgfscope}%
\begin{pgfscope}%
\definecolor{textcolor}{rgb}{0.000000,0.000000,0.000000}%
\pgfsetstrokecolor{textcolor}%
\pgfsetfillcolor{textcolor}%
\pgftext[x=1.985009in,y=0.691568in,left,base]{\color{textcolor}\rmfamily\fontsize{8.330000}{9.996000}\selectfont Error Bound}%
\end{pgfscope}%
\end{pgfpicture}%
\makeatother%
\endgroup%

%DIFDELCMD <    %%%
\DIFdelendFL \DIFaddbeginFL \includegraphics{images/omegas_cmp.pdf}
   \DIFaddendFL \end{center}
   \vspace{-15pt}
   \caption{$\vec{\omega}_{SecFCI}$ and $\vec{\omega}_{FCI}$ components.}
   \vspace{-\baselineskip}
   \label{fig:fci_secfci_omegas}
\end{figure}

%  .d8888b.
% d88P  Y88b
% 888    888
% 888         .d88b.  88888b.   .d8888b
% 888        d88""88b 888 "88b d88P"
% 888    888 888  888 888  888 888
% Y88b  d88P Y88..88P 888  888 Y88b.    d8b
%  "Y8888P"   "Y88P"  888  888  "Y8888P Y8P



\section{Conclusion} \label{sec:conclusion}
FCI is a commonly used, and efficiently computable, approximation to the CI optimization problem that requires the sharing of local sensor estimates to compute their fusion. We propose a secure approximation to FCI, SecFCI, to compute the fused estimate homomorphically. The novel encrypted fusion approach may find uses in various security-critical applications or over untrusted networks \DIFaddbegin \DIFadd{subject to eavesdroppers and malicious participants}\DIFaddend . Possible future work includes run-time comparisons with FHE implementations, giving a computational bound for its practicality, and quantification of fusion weight leakages via formal security proofs.

\bibliographystyle{IEEEtran}
\bibliography{IEEEtranBST/IEEEabrv,BibTeX/IEEEBibControls,BibTeX/secure_FCI_cdc2020}

\end{document}
